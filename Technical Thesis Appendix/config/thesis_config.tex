% Define custom colors
\definecolor{hyperColor}{HTML}{0B3EE3}
\definecolor{tableGray}{HTML}{F5F5F7}
\definecolor{veryPeri}{HTML}{6667ab}

% Set chapter font size
\chapterfont{\fontsize{24}{22}\selectfont}

% Set line height
\linespread{1.5}

% Custom hyphenation rules
\hyphenation{
    data-base
    al-go-rithms
    soft-ware
}

\begin{filecontents*}{\jobname.xmpdata}
  \Title{Designing an accessibility learning toolkit: bridging the gap between guidelines and implementation}
  \Author{Gabriel Rovesti} 
  \Language{en-EN}
  \Subject{Designing an accessible learning toolkit realized in React Native to help and guide developers in the creation of accessible mobile applications}
  \Keywords{Accessibility \sep React Native \sep Guidelines \sep Accessibility testing \sep WCAG \sep MCAG \sep Developer education}
\end{filecontents*} 

% Images path
\graphicspath{{img/}}

% Force page color
\pagecolor{white}

% Better spacing for footnotes
\setlength{\skip\footins}{5mm}
\setlength{\footnotesep}{5mm}

% Fix header height warning
\setlength{\headheight}{27.11652pt}
\addtolength{\topmargin}{-12.61652pt}

% Add glossary subscript
\newcommand{\glox}{\textsubscript{\textbf{\textit{G}}}}

% Paragraph formatting
\titleformat{\paragraph}
{\normalfont\normalsize\bfseries}{\theparagraph}{1em}{}
\titlespacing*{\paragraph}
{0pt}{3.25ex plus 1ex minus .2ex}{1.5ex plus .2ex}

% Keep chapter format with "Chapter X" but modify spacing
\titleformat{\chapter}[display]
    {\normalfont\huge\bfseries}
    {\chaptertitlename\ \thechapter}
    {10pt}   % Reduced space between "Chapter X" and the title
    {\Huge}

% Reduce spacing before and after the chapter
\titlespacing*{\chapter}
    {0pt}    % left margin
    {20pt}   % space before title (reduced from default 50pt)
    {20pt}   % space after title (reduced from default 40pt)

\newcommand{\chapterintroline}[1]{%
    \begin{quote}
    \normalfont
    #1
    \end{quote}
    \centerline{\rule{0.5\textwidth}{0.5pt}}  % Adds a horizontal line
}

% Page styling
\pagestyle{fancy}
\fancyhf{}
\fancyhead[L]{\leftmark}
\fancyfoot[C]{\thepage}

% Utility commands
\newcommand\blankpage{
\clearpage
    \begingroup
    \null
    \thispagestyle{empty}
    \hypersetup{pageanchor=false}
    \clearpage
\endgroup
}

\newcommand\blankpagewithnumber{
  \clearpage
  \thispagestyle{plain}
  \null
  \clearpage
}

\newcommand\increasepagenumbering{
    \addtocounter{page}{+1}
}

% Load glossaries
% Acronyms
\newacronym{ui}{UI}{User Interface}
\newacronym{ux}{UX}{User Experience}

% Glossary
\newglossaryentry{uig}{
    name={UI},
    text={User Interface},
    sort=ui,
    description={The User Interface refers to the space where interactions between humans and machines occur. It includes the design and arrangement of graphical elements (such as buttons, icons, and menus) that enable users to interact with software or hardware systems. The goal of a UI is to make the user's interaction simple and efficient in accomplishing tasks within a system.}
}

\newglossaryentry{uxg}{
    name={UX},
    text={User Experience},
    sort=ux,
    description={User Experience encompasses the overall experience a user has while interacting with a product or service. It includes not only usability and interface design but also the emotional response, satisfaction, and ease of use a person feels while using a system. UX design focuses on optimizing a product’s interaction to provide meaningful and relevant experiences to users, ensuring that the system is intuitive, efficient, and enjoyable to use.}
}
\makeglossaries
\renewcommand*{\glstextformat}[1]{\textit{#1}\glox}
\glsaddall

% Bibliography setup
\addbibresource{references/bibliography.bib}
\defbibheading{bibliography}{
    \cleardoublepage
    \phantomsection
    \addcontentsline{toc}{chapter}{\bibname}
    \chapter*{\bibname\markboth{\bibname}{\bibname}}
}

% Add this to suppress split bibliography warning
\BiblatexSplitbibDefernumbersWarningOff

% Hyperref setup
\hypersetup{
    colorlinks=true,
    linktocpage=true,
    pdfstartpage=1,
    pdfstartview=,
    breaklinks=true,
    pdfpagemode=UseNone,
    pageanchor=true,
    pdfpagemode=UseOutlines,
    plainpages=false,
    bookmarksnumbered,
    bookmarksopen=true,
    bookmarksopenlevel=1,
    hypertexnames=true,
    pdfhighlight=/O,
    allcolors = hyperColor
}

% Caption setup
\captionsetup{
    tableposition=top,
    figureposition=bottom,
    font=small,
    format=hang,
    labelfont=bf
}

% Adjust the way tables are handled
\setlength{\tabcolsep}{4pt}  % Default is 6pt, reducing gives more space in tables

% For longtable format
\setlength\LTleft{0pt}
\setlength\LTright{0pt}

% Adjust the way floats are handled to avoid excessive whitespace
\renewcommand{\topfraction}{0.9}        % Default 0.7
\renewcommand{\bottomfraction}{0.9}     % Default 0.3
\renewcommand{\textfraction}{0.1}       % Default 0.2
\renewcommand{\floatpagefraction}{0.85} % Default 0.5
\setcounter{topnumber}{3}              % Default 2
\setcounter{bottomnumber}{3}           % Default 1
\setcounter{totalnumber}{5}            % Default 3

% Code listings configuration
\definecolor{codeBackground}{HTML}{F5F5F7}
\definecolor{codeKeyword}{HTML}{CF222E}
\definecolor{codeString}{HTML}{0550AE}
\definecolor{codeComment}{HTML}{6E7781}

\lstset{
    basicstyle=\ttfamily\small,
    backgroundcolor=\color{codeBackground},
    commentstyle=\color{codeComment},
    keywordstyle=\color{codeKeyword}\bfseries,
    stringstyle=\color{codeString},
    numbers=left,
    numberstyle=\tiny\color{codeComment},
    numbersep=10pt,
    frame=single,
    framesep=2mm,
    xleftmargin=25pt,
    breaklines=true,
    breakatwhitespace=true,
    showstringspaces=false,
    tabsize=2,
    captionpos=b,
    float=htbp
}

% Language definition for React Native
\lstdefinelanguage{ReactNative}{
  keywords={
    import, export, default, function, return, const, let, var, 
    if, else, for, while, do, switch, case, break, continue, 
    class, extends, new, try, catch, finally, throw, async, await,
    useState, useEffect, useContext, useRef
  },
  sensitive=true,
  comment=[l]{//},
  morecomment=[s]{/*}{*/},
  morestring=[b]',
  morestring=[b]",
  morestring=[b]`
}

% Then define a style that uses this language
\lstdefinestyle{ReactNativeStyle}{
  language=ReactNative,
  basicstyle=\ttfamily\footnotesize,
  backgroundcolor=\color{codeBackground},
  commentstyle=\color{codeComment},
  keywordstyle=\color{codeKeyword}\bfseries,
  stringstyle=\color{codeString},
  numbers=left,
  numberstyle=\tiny\color{codeComment},
  numbersep=10pt,
  frame=single,
  framesep=2mm,
  xleftmargin=25pt,
  breaklines=true,
  breakatwhitespace=true,
  showstringspaces=false,
  tabsize=2,
  captionpos=b,
  float=htbp
}

\lstdefinelanguage{Dart}{
  keywords={
    import, export, part, library, class, extends, implements, mixin, with, 
    static, final, const, var, void, dynamic, if, else, for, while, do, switch, 
    case, break, continue, return, try, catch, finally, throw, assert, 
    this, super, new, async, await, yield, true, false, null
  },
  sensitive=true,
  comment=[l]{//},
  morecomment=[s]{/*}{*/},
  morestring=[b]',
  morestring=[b]",
  morestring=[b]'''
}

\lstdefinestyle{DartStyle}{
  language=Dart,
  basicstyle=\ttfamily\footnotesize,
  backgroundcolor=\color{codeBackground},
  commentstyle=\color{codeComment},
  keywordstyle=\color{codeKeyword}\bfseries,
  stringstyle=\color{codeString},
  numbers=left,
  numberstyle=\tiny\color{codeComment},
  numbersep=10pt,
  frame=single,
  framesep=2mm,
  xleftmargin=25pt,
  breaklines=true,
  breakatwhitespace=true,
  showstringspaces=false,
  tabsize=2,
  captionpos=b,
  float=htbp
}