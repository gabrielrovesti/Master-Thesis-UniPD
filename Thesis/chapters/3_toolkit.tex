\chapter{AccessibleHub: Transforming mobile accessibility guidelines into code}
\label{chap:accessibility-toolkit}

\chapterintroline{
    This chapter presents an accessibility-focused learning toolkit, which is an all-encompassing guide to mobile application developers. It extends Budai's research into Flutter accessibility and gives a more focused approach to orient the developers themselves in how to actually implement an accessible mobile application. Here \textit{AccessibleHub} is introduced, an interactive learning toolkit built using \gls{reactnative}, which aims to enhance accessibility implementation through hands-on examples, component-level guidance, and comparative insights between React Native and \gls{flutter}. By providing a structured educational approach grounded in \gls{wcagg} principles and mobile-specific considerations, AccessibleHub empowers developers to bridge the gap between accessibility guidelines and real-world implementation. 
}

\section{Introduction}
\label{sec:intro}

\subsection{Challenges in implementing accessibility guidelines}

The importance of mobile app accessibility extends beyond mere compliance with legal regulations.. Ensuring equal access to digital content and services is not only an ethical obligation but also a smart business decision. By prioritizing accessibility, app developers and companies can tap into a larger user base, improve user satisfaction, and demonstrate their commitment to social responsibility.
Despite the clear benefits and moral imperatives of mobile app accessibility, many developers still struggle to effectively implement accessibility guidelines in their projects. The \acrshort{wcagacr}, developed by the \gls{w3cg}, serve as the international standard for digital accessibility. However, translating these guidelines into practical implementation can be a challenging task, particularly starting from pure formal guidelines into everyday code. \\

One of the primary challenges lies in the complexity of the guidelines themselves. WCAG encompasses a wide range of \textit{success criteria}, organized under four main general \textit{principles}: perceivable, operable, understandable, and robust. Each principle contains multiple guidelines, and each guideline has several success criteria at different levels of \textit{conformance} (A, AA, AAA). Navigating this intricate web of requirements and understanding how to apply them to specific mobile app components can be overwhelming for developers, especially those new to accessibility. 
Moreover, the practical implementation of accessibility guidelines often varies across different platforms and frameworks. iOS and Android, the two dominant mobile operating systems, have their own unique accessibility APIs, tools, and best practices. Cross-platform frameworks like React Native and Flutter add another layer of complexity, as developers must ensure that their accessibility implementations are compatible with the underlying platform-specific mechanisms. \\ 

Furthermore, there is often a lack of clear, practical examples and guidance on how to implement accessibility features in real-world mobile app projects. While the WCAG provides a solid foundation, it is primarily focused on web content and may not always directly address the unique challenges and interaction patterns of mobile apps. Developers often struggle to bridge the gap between the theoretical guidelines and the specific implementation details required for their projects.

\subsection{The need for practical developer education}

To address these challenges and bridge the gap between accessibility guidelines and practical implementation, there is a pressing need for developer education resources that focus on real-world, hands-on learning experiences. Traditional documentation and guidelines, while valuable, often fall short in providing the level of detail and interactivity needed to effectively guide developers through the accessibility implementation process.
This is where the concept of an \textit{accessibility learning toolkit} comes into play. An accessibility toolkit is designed to serve as a comprehensive, interactive resource that empowers developers to create accessible mobile applications by providing:

\begin{enumerate}
    \item Clear explanations of \acrshort{wcagacr} guidelines and their applicability to mobile apps
    \item Step-by-step implementation guidance for common mobile app components and interaction patterns
    \item Practical code examples and tutorials that demonstrate best practices
    \item Hands-on exercises and challenges to reinforce learning and build confidence
    \item Tools and techniques for testing and validating the accessibility of mobile apps
\end{enumerate}

The primary goal of an accessibility learning toolkit is to bridge the gap between the theoretical knowledge of accessibility guidelines and the practical skills needed to implement them effectively in real-world projects. 
The toolkit should cater to developers at various levels of expertise, from beginners who are new to accessibility concepts to experienced professionals seeking to deepen their knowledge and stay up-to-date with the latest best practices. By providing a comprehensive, hands-on learning resource, the accessibility toolkit can play a crucial role in promoting a culture of inclusive design and development within the mobile app industry. \\

Current research, including Budai's work on Flutter accessibility testing, has primarily focused on end-user validation and testing methodologies. However, developers need practical, implementation-focused guidance that bridges multiple frameworks and platforms.
Despite widespread accessibility guidelines and standard, mobile application developers face significant challenges in translating theoretical requirements into practical implementations. This gap between guidelines and implementation is particularly evident in mobile development, where different platforms, screen sizes, and interaction models add complexity to accessibility implementation. Some of the most common challenges include:

\begin{itemize}
    \item Complex testing requirements - developers must validate across multiple devices, \gls{screenreaderg}, and interaction modes
    \item Framework-specific implementations - each platform has unique accessibility \gls{apig}s and requirements
    \item Limited practical examples - most documentation focuses on theoretical guidelines rather than concrete implementation patterns
    \item Performance considerations - accessibility features must be implemented without compromising app performance
\end{itemize}

Effective developer education in accessibility requires a solid grounding in learning theories that emphasize hands-on, interactive approaches. By integrating established learning theories with technical education principles, it's possible to justify the interactive and practical approach adopted in this toolkit. In doing so, we draw on constructivist and experiential learning models, which have been widely recognized as effective frameworks in technical and developer education.

Constructivist learning theories, pioneered by Piaget \cite{piaget1970science} and Vygotsky \cite{vygotsky1978mind}, posit that learning is an active process in which individuals construct knowledge based on their prior experiences and interactions with the environment. In the context of developer education, this suggests that hands-on learning is more effective than passive instruction \cite{savery2006overview}. By engaging with real-world accessibility challenges and actively experimenting with code implementations, developers can build a deeper understanding of accessibility guidelines and best practices, by having a tool at their disposal easy to use and to navigate. \\

Kolb's \textit{Experiential Learning Theory} \cite{kolb1984experiential} further supports this approach by describing learning as a four-stage cycle: concrete experience, reflective observation, abstract conceptualization, and active experimentation. For developers learning about accessibility, this cycle might involve encountering accessibility issues in their projects, analyzing existing solutions and guidelines, synthesizing their understanding of \acrshort{wcagacr} principles, and applying these principles to their own code. AccessibleHub facilitates this learning cycle by providing a structured, interactive environment for developers to engage with accessibility concepts and implementations being organized into different core sections. By aligning with these proven pedagogical approaches, AccessibleHub aims to provide an effective and engaging learning experience for developers. Moreover, by fostering a community of practice around accessibility while providing easier access to learning resources, this project encourages ongoing learning and knowledge sharing among developers, promoting the continuous improvement and dissemination of accessibility best practices.

\subsection{Research objectives and methodology}

Building upon previous research into mobile accessibility, this work aims to provide a comprehensive understanding of accessibility implementation across major cross-platform frameworks. While existing research indeed set grounds for both guidelines on accessibility and testing methodologies, there is a critical need to understand how these guidelines translate into practice for developers. 

This research addresses three fundamental questions about accessibility implementation in mobile development frameworks (referring to these ones as \textit{research questions}, following the work in \cite{perinello2024accessibility}:

\begin{itemize}
    \item First, we investigate whether components and widgets provided by frameworks are \textit{accessible by default}, without requiring additional developer intervention. This analysis is crucial for understanding the baseline accessibility support provided by each framework and identifying areas where additional implementation effort may be required.
    \item Second, we examine the \textit{feasibility of making non-accessible components accessible} through additional development effort. This involves analyzing the technical capabilities of each framework and identifying the necessary modifications to achieve accessibility compliance.
    \item Third, we quantify the \textit{development overhead required to implement accessibility features} when they are not provided by default. This includes measuring additional code requirements, analyzing complexity increases, and evaluating the impact on development workflows.
\end{itemize}

These questions is addressed via the usage of a systematic methodology aiming to address in detail accessibility support in React Native and Flutter, focusing on component implementation patterns and native platform integration. The implementation is comparative, allowing developers to directly implement accessible code examples with different degrees of implementation complexity measured quantitatively (including lines of code, required properties, and additional components needed for accessibility support). Comprehensive testing of implementations is also done using screen readers and other assistive technologies to verify accessibility compliance.

The \textit{goal} is to create an accessible application that serves three key purposes:
\begin{enumerate}
    \item To provide developers with practical, interactive examples of accessibility implementation, able to be copied easily and ported inside of other projects;
    \item To compare and contrast accessibility approaches between the main cross-development mobile frameworks in the current mobile landscape;
    \item To establish a reusable pattern library that demonstrates engine architecture, widget systems, and native platform integration, while ensuring compliance with current accessibility guidelines and legal requirements.
\end{enumerate}

The following sections will detail the development of AccessibleHub, an application developed in React Native designed to serve as a practical manual for implementing accessibility features. While the technical aspects of cross-platform frameworks will be discussed later, the focus remains on providing developers with actionable implementation patterns and comparative insights for building accessible applications.

\section{Accessibility guidelines and standards}
\label{sec:guidelines}

Accessibility guidelines and standards form the foundation upon which inclusive mobile app development practices are built. They provide a shared framework for understanding and addressing the diverse needs of users with disabilities, ensuring that mobile apps are perceivable, operable, understandable, and robust. This section explores the key accessibility guidelines and standards relevant to mobile app development, with a particular focus on the Web Content Accessibility Guidelines (WCAG) and mobile-specific considerations.

\subsection{Web Content Accessibility Guidelines (WCAG)}

The \gls{wcagg}, developed by the \gls{w3cg}, serve as the international standard for digital accessibility. Although originally designed for web content, the WCAG principles and guidelines are equally applicable to mobile app development. The WCAG is organized around four main principles:

\begin{itemize}
    \item \textit{Perceivable}: Information and user interface components must be presentable to users in ways they can perceive. This includes providing text alternatives for non-text content, creating content that can be presented in different ways without losing meaning, and making it easier for users to see and hear content.
    \item \textit{Operable}: User interface components and navigation must be operable. This means that all functionality should be available from a keyboard, users should have enough time to read and use the content, and content should not cause seizures or physical reactions.
    \item \textit{Understandable}: Information and the operation of the user interface must be understandable. This involves making text content readable and understandable, making content appear and operate in predictable ways, and helping users avoid and correct mistakes.
    \item \textit{Robust}: Content must be robust enough that it can be interpreted by a wide variety of user agents, including assistive technologies. This requires maximizing compatibility with current and future user agents.
\end{itemize}

Under each principle, the WCAG provides specific guidelines and success criteria at three levels of conformance (A, AA, and AAA). These success criteria are testable statements that help developers determine whether their app meets the accessibility requirements. By understanding and applying the WCAG principles and guidelines, mobile app developers can create more inclusive and accessible experiences for their users.

\subsection{Mobile Content Accessibility Guidelines (MCAG)}

While \acrshort{wcagacr} offers a comprehensive foundation, mobile platforms introduce additional complexities that may not be fully addressed by web-centric guidelines. The \gls{mcagg} build upon the previous ones by focusing on the specific interaction patterns, form factors, and environmental contexts unique to mobile devices. For example, MCAG emphasizes:

\begin{itemize}
    \item \textit{Touch interaction and gestures}: Ensuring that tap targets, swipe gestures, and multi-finger interactions are usable for individuals with varying motor skills;
    
    \item \textit{Limited screen real estate}: Designing content that remains clear and functional on smaller displays, including proper zooming and reflow behavior;

    \item \textit{Diverse mardware and os versions}: Accounting for a wide range of device capabilities, operating system versions, and hardware configurations that can affect accessibility;

    \item \textit{Contextual usage scenarios}: Recognizing that mobile apps are often used in changing lighting conditions, noisy environments, or while users are on the move.
    
\end{itemize}

In practice, \acrshort{mcagacr} complements \acrshort{wcagacr} by providing more granular, mobile-oriented guidance considering specific factors. Developers who follow these guidelines in addition to \acrshort{wcagacr} are better equipped to deliver an inclusive experience that accounts for real-world mobile usage. As part of this work, AccessibleHub integrates both sets of guidelines to ensure comprehensive coverage of accessibility requirements across platforms.

\subsection{Mobile-specific accessibility considerations}

While the previous guidelines provide by themselves a solid foundation for digital accessibility, mobile apps present unique challenges and considerations that require additional attention. Some of the key mobile-specific accessibility factors include:

\begin{itemize}
    \item \textit{Touch interaction}: Mobile devices rely heavily on touch-based interactions, such as tapping, swiping, and multi-finger gestures. Developers must ensure that all interactive elements are large enough to be easily tapped, provide alternative input methods for complex gestures, and offer appropriate haptic and visual feedback;
    
    \item \textit{Small screens}: The limited screen real estate on mobile devices can pose challenges for users with visual impairments. Developers should provide sufficient contrast, use clear and legible fonts, and ensure that content can be easily zoomed or resized without losing functionality;
    
    \item \textit{Screen reader compatibility}: Mobile screen readers, such as VoiceOver on iOS and TalkBack on Android, require proper labeling and semantic structure to effectively convey content and functionality to users with visual impairments. Developers must use appropriate accessibility APIs and ensure that all elements are properly labeled and navigable;
    
    \item \textit{Device fragmentation}: The wide range of mobile devices, screen sizes, and operating system versions can complicate accessibility testing and implementation. Developers should test their apps on a diverse range of devices and ensure that accessibility features function consistently across different configurations;
    
    \item \textit{Mobile context}: Mobile apps are often used in a variety of contexts, such as outdoors, in low-light conditions, or in noisy environments. Developers should consider these contexts and provide appropriate accommodations, such as high-contrast modes or subtitles for audio content.
\end{itemize}

By understanding and addressing these mobile-specific accessibility considerations, developers can create apps that are more inclusive and usable for a wider range of users.

\section{AccessibleHub: An Interactive Learning Toolkit}
\label{sec:accessiblehub}

\subsection{Core architecture and design principles}

AccessibleHub is a React Native application designed to serve as an interactive manual for implementing accessibility features in mobile development. Unlike traditional documentation or testing frameworks, the application provides developers with hands-on examples and implementation patterns that can be directly applied to their projects.

The application is structured around four conceptual main sections:
\begin{enumerate}
    \item \textit{Component examples}: Interactive demonstrations of common \acrshort{ui} elements with proper accessibility implementations, including buttons, forms, media content, and navigation patterns. This allows developers to clearly see the implementation of an accessible component and easily copy the code to their convenience;
    
    \item \textit{Framework comparison}: A detailed analysis of accessibility implementation approaches between React Native and Flutter, highlighting differences in component structure, properties, and required code;
    
    \item \textit{Testing tools}: Built-in utilities for validating accessibility features, allowing developers to understand how screen readers and other assistive technologies interact with their implementations;
    
    \item \textit{Implementation guidelines}: Technical documentation that connects WCAG requirements to practical code examples, providing clear paths for meeting accessibility standards.
\end{enumerate}

Each component presented serves dual purposes: demonstrating proper accessibility implementation while providing reusable code patterns. The application emphasizes practical implementation over theoretical guidelines, showing developers not just what to implement effectively. By focusing on developer experience, AccessibleHub bridges the gap between accessibility requirements and actual implementation, providing a resource that can be directly integrated into the development workflow. \\

The \textit{design} philosophy of AccessibleHub is founded on principles that bridge theoretical accessibility guidelines with practical implementation needs. While analyzing the current landscape of mobile development frameworks and accessibility implementation presented in \ref{chap:accessibility-literature}, a clear pattern emerges: developers need more practical, implementation-focused guidance that directly addresses the complexity of building accessible applications. To address this need, AccessibleHub adopts three fundamental architectural principles:

\begin{enumerate}
    \item The usage of a \textit{component-first architecture}, where each UI element exists as an independent, self-contained unit demonstrating both implementation patterns and accessibility features. In other words, each one of them is being constructed within an \textit{accessibility-first} experience which ensures that usage of screen readers and other assistive technologies is kept as a priority. This modular approach provides two advantages: it first allows developers to comprehend and apply accessibility features in isolation, hence reducing cognitive load and implementation complexity, and enables systematic testing and validation of accessibility features of every component. Also, this means accessibility patterns can be studied, implemented, and verified in isolation from added complexity brought in by interactions among those components. 

    \item \textit{Progressive enhancement} as a core design methodology. Instead of presenting accessibility as big challenge from the start, components are structured in increasing levels of complexity. This starts with basic elements like buttons and text inputs where basic accessibility patterns can be established. As developers master these foundational components, the application introduces more complex patterns such as forms, navigation systems, and gesture-based interactions. This helps into guiding the development towards more complicated scenarios.

    \item Focus on \textit{framework-agnostic patterns}, not depending on a specific framework while providing concrete code implementations. Even though AccessibleHub has been implemented in React Native, all the patterns and principles explained are designed to transcend into specific framework implementations. The approach wants to give importance to the compatibility and reusability in the framework on the mobile development side. It will compare the implementations, mainly between React Native and Flutter, to show how developers can port accessibility patterns across different frameworks and understand core accessibility concepts in an easy-to-implement manner within professional projects. 
    
\end{enumerate}

Through these principles, AccessibleHub aims to transform accessibility from an afterthought into an \textit{accessibility-by-design}. The application serves not just as a reference implementation, but as an educational tool that guides developers through the process of building truly accessible applications. This approach recognizes that effective accessibility implementation requires both theoretical understanding and practical experience, providing developers with the tools they need to create more inclusive mobile applications.

\subsection{Educational framework design}

AccessibleHub's educational framework is designed to provide a structured, incremental learning experience that progressively builds accessibility knowledge and skills. The content is organized into different \textit{learning modules}, each focusing on a key aspect of mobile accessibility. This is structured incrementally, so to help a developer gather a general idea on what needs to be implemented following a practical roadmap of steps: this allows to focus on different aspects of mobile accessibility, selecting each time the most relevant ones.

The core of the application is divided into different main sections, following:

\begin{enumerate}
    \item The \textit{Home} screen - The entry point for the AccessibleHub application. It provides an overview of the main sections and guides users on where to start their accessibility learning journey. The Home screen is designed to be intuitive and user-friendly, with clear call-to-action towards the accessible components section, allowing a developer or a user navigate to the desired section from the Home screen, comprehensive of comparison between the main mobile frameworks, learn about best practices in mobile accessibility and access testing tools documentation. There is also present a compliance dashboard provides an overview of an app's accessibility compliance status, based on the \acrshort{wcagacr} and \acrshort{mcagacr} guidelines. Developers can use this information to prioritize their accessibility efforts and focus on the areas that need the most attention;

    \item The \textit{Accessible Components} screen - Developers can learn how to implement accessible UI components in their mobile applications. This section is divided into four subscreens, each focusing on a specific category of components:

    \begin{itemize}
        \item \textit{Buttons and Touchables}: It covers the implementation of accessible buttons and touchable elements. It provides code examples and best practices for ensuring that these interactive elements are perceivable, operable, and understandable by all users, including those with disabilities;

        \item \textit{Forms}: The subscreen focuses on creating accessible input forms, including text fields, checkboxes, radio buttons, and date/time pickers. It demonstrates how to properly label form elements, provide instructions and feedback, and ensure that forms can be navigated and completed using various input methods, such as keyboards and screen readers;

        \item \textit{Media}: In the Media subscreen, developers learn how to make media content, such as images, videos, and audio, accessible to users with visual or auditory impairments. This includes providing alternative text for images, captions for videos, and transcripts for audio content;

        \item \textit{Dialogs}: It covers the creation of accessible modal dialogs, popups, and alerts. It provides guidance on how to ensure that these elements are properly announced by screen readers, can be easily dismissed, and do not interfere with the user's ability to navigate the application, maintaining focus management and ensuring clear exit strategies;

       \item \textit{Advanced}: This particular subscreen covers elements like alerts, sliders, progress bars and tab navigation, analyzing how accessibility may regard different animated or interactive components for more complex gesture interactions used everyday by users.
    \end{itemize}

Throughout the Components section, code implementations are provided, which developers can easily copy to their clipboard and integrate into their own projects. This hands-on approach allows developers to quickly apply the accessibility principles they learn and see the results in action.

\item The \textit{Best Practices} screen - Designed to give developers a general understanding of the overarching principles and guidelines for creating accessible mobile applications. It is divided into five subscreens, each addressing a key aspect of mobile accessibility:

    \begin{itemize}
        \item \textit{Gestures Tutorial}: This subscreen provides an overview of the various gesture interactions used in mobile applications and how to make them accessible to users with motor impairments or those relying on assistive technologies. It covers best practices for implementing alternative input methods and providing clear instructions and feedback. These gestures are general, tested to be used universally, both by everyday users and screen reader ones;

        \item \textit{Semantics Structure}: Here, developers learn about the importance of using semantic \textit{HTML} and \gls{ariag} roles to convey the structure and meaning of the application's content. This helps screen readers and other assistive technologies better understand and navigate the application;

        \item \textit{Navigation}: This one focuses on creating accessible navigation patterns, such as menus, tabs, and breadcrumbs. It provides guidance on how to ensure that navigation elements are properly labeled, can be operated using various input methods, and provide clear feedback to the user, jumping directly to the main context of a screen and bringing the attention to an element on-screen without distracting him from the action to be completed;

        \item \textit{Screen Reader Support}: This subscreen covers the specific considerations for making mobile applications compatible with screen readers, such as \gls{voiceover} on \textit{iOS} and \gls{talkback} on \textit{Android}. It includes best practices for labeling elements, providing alternative text, and ensuring that the application's content and functionality can be fully accessed and understood using a screen reader;

        \item \textit{Accessibility Guidelines}: The Accessibility Guidelines subscreen provides an overview of the key accessibility standards to be followed and a general list of principles to incorporate into a project, seeing how they apply to mobile application development. It helps developers understand the different levels of conformance and how to assess their application's accessibility against these guidelines.
    \end{itemize}

\item The \textit{Framework Comparison} screen - It provides a side-by-side comparison of the accessibility features and implementation differences between popular mobile development frameworks, such as React Native and Flutter. This section helps developers understand how accessibility is handled in each framework and provides guidance on leveraging the specific accessibility APIs and tools available in each one. By highlighting the similarities and differences between frameworks, developers can make informed decisions about which framework to use for their accessibility needs and how to optimize their implementations for each platform;

\item The \textit{Tools} screen - It serves as a central hub for accessing various accessibility-related tools and resources. This includes links to official documentation, such as the React Native Accessibility \acrshort{api} reference and the \textit{Flutter Accessibility package} documentation. It also provides quick access to popular accessibility testing tools, such as \textit{Accessibility Scanner} for \textit{Android} and \textit{Accessibility Inspector} for \textit{iOS}. By consolidating these resources in one place, the Tools screen makes it easy for developers to find the information and tools they need to ensure their applications are fully accessible; 

\item The \textit{Settings} screen - It allows users to customize various aspects of the AccessibleHub application to suit their individual learning needs and preferences. This includes options for adjusting the font size, color contrast (including options for gray scale and dark mode), reduced motion settings and others to help users and ensure the application itself is accessible to a wide range of users. It also provides information on how to configure the accessibility settings on the user's device, such as enabling screen readers or adjusting the display settings. By offering these customization options and guidance, the Settings page reinforces the importance of accessibility as an everyday tool, meant to accompany practical user needs in an easy and quick way;

\item The \textit{Instruction and Community} screen - It provides a collaborative learning environment that extends beyond technical implementation. This section offers developers an opportunity to dive deeper into accessibility knowledge through curated resources and community engagement allowing for easier exploration towards other online resources. By providing a platform for continuous learning and collaboration, this screen reinforces the importance of accessibility as a collective effort and a fundamental aspect of modern mobile application development.
    
\end{enumerate}

\subsection{From guidelines to implementation: a screen-based methodology}

Accessibility guidelines and standards - most notably the \acrshort{wcagacr}  and related mobile-specific considerations—establish the formal foundation for inclusive digital design. These criteria are essential but inherently abstract and can be challenging to implement directly in code. Building on Budai's approach focusing solely on post-implementation testing, the methodology presented embeds accessibility into the development process. We do this by analyzing each screen of the application through a structured framework that connects theoretical requirements with practical implementation strategies. The approach to be considered is built following these layers:

\begin{enumerate}
    \item \textit{Theoretical foundation} – This layer encompasses the abstract principles and success criteria defined by WCAG/\acrshort{mcagacr}. For example, WCAG’s four core principles require that content be presented in ways users can perceive, interact with, and understand. These criteria serve as the benchmark for our analysis;

    \item \textit{Implementation pattern} – Here, we translate the abstract requirements into concrete code structures within a mobile development context. In AccessibleHub, this involves the systematic use of React Native properties (such as \textit{accessibilityLabel}, \textit{accessibilityRole}, etc.) to ensure that \acrshort{ui} components satisfy the established guidelines;

    \item \textit{User interaction flow} – Finally, we consider how end users interact with these components. This includes the behavior of assistive technologies (like screen readers), proper focus management, and the overall usability of the component within its real-world context.
\end{enumerate}

To illustrate this methodology in practice, we first map the UI elements of a representative screen to their corresponding semantic roles. Next, we link each component to the relevant \acrshort{wcagacr} and \acrshort{mcagacr} criteria presented in the previous subsections, noting both the minimum compliance requirements and potential enhancements. Finally, we describe the technical solution—specifically, how React Native accessibility code properties are applied to meet and exceed these standards. This structured approach not only bridges the gap between abstract guidelines and real-world coding tasks but also sets the stage for the more detailed, screen-by-screen analyses presented in the next section.

\section{Accessibility Implementation Guidelines}
\label{sec:implementation-guidelines}

**TODO COMMENT** This section should apply into practice, going into formal details of each screen, how to apply guidelines which should be present formally, but in a more readable way, not like Matteo's. I would proceed like this. For each screen/section presented:

\begin{enumerate}
    \item Foundational accessibility principles
    \item WCAG/MCAG guidelines
    \item Practical React Native implementation discussed formally
\end{enumerate}

\subsection{Perceivable}

\subsubsection{Text alternatives}

\subsubsection{Captions and audio descriptions}

\subsubsection{Color contrast}

\subsection{Operable}

\subsubsection{Keyboard navigation}

\subsubsection{Touchscreen gestures}

\subsubsection{Screen reader compatibility}

\subsection{Understandable}

\subsubsection{Predictable interactions}

\subsubsection{Input assistance}

\subsubsection{Error handling and feedback}

\subsection{Robustness}

\subsubsection{Compatibility with assistive technologies}

\section{Fostering an accessibility-focused developer community}
\label{sec:community}

\subsection{Knowledge sharing and collaboration}

\subsection{Contributor recognition and rewards}

\subsection{Continuous learning and improvement}

\newpage

