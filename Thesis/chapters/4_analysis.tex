\chapter{Accessibility analysis: framework comparison and implementation patterns} 
\label{chap:implementation}

\chapterintroline{
    This chapter presents a systematic comparative analysis between React Native and Flutter frameworks regarding their accessibility implementations, examining how each framework addresses accessibility requirements in practice. Through empirical evaluation of equivalent components, we investigate three key research questions: whether components are accessible by default, if non-accessible components can be made accessible, and the development overhead required for accessibility implementation. Building upon the educational toolkit introduced previously, this analysis combines quantitative metrics with qualitative assessment of developer experience, providing insights into how each framework supports the creation of accessible mobile applications.
}

\section {Research methodology}

\subsection {Research questions and objectives}

\subsection {Testing approach and criteria}

\subsection {Evaluation metrics}

\section {React Native Approach}

\subsection {Component accessibility analysis}

\subsection {Native platform integration}

\subsection {Implementation patterns and solutions}

\section {Flutter Approach}

\subsection{Framework overview}

\subsection {Component accessibility patterns}

\subsection {Platform-specific considerations}

\subsection {Common implementation challenges}

\section {Comparative analysis}

\subsection {Default accessibility support}

\subsection {Implementation complexity}

\subsection {Developer experience}

\subsection {Performance implications}

\section {Framework-specific best practices}

\subsection{React Native optimization patterns}

\subsection{Flutter accessibility patterns}

\subsection{Cross-platform considerations}

\newpage