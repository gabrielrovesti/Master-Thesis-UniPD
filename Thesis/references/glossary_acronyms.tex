% Acronyms
\newacronym{ui}{UI}{User Interface}
\newacronym{ux}{UX}{User Experience}
\newacronym{w3c}{W3C}{World Wide Web Consortium}
\newacronym{aria}{ARIA}{Accessible Rich Internet Applications}

% Glossary
\newglossaryentry{uig}{
    name={User Interface},
    text={User Interface},
    sort=ui,
    description={The User Interface refers to the space where interactions between humans and machines occur. It includes the design and arrangement of graphical elements (such as buttons, icons, and menus) that enable users to interact with software or hardware systems. The goal of a UI is to make the user's interaction simple and efficient in accomplishing tasks within a system.}
}

\newglossaryentry{uxg}{
    name={User Experience},
    text={User Experience},
    sort=ux,
    description={User Experience encompasses the overall experience a user has while interacting with a product or service. It includes not only usability and interface design but also the emotional response, satisfaction, and ease of use a person feels while using a system. UX design focuses on optimizing a product’s interaction to provide meaningful and relevant experiences to users, ensuring that the system is intuitive, efficient, and enjoyable to use.}
}

\newglossaryentry{grayliteraturereview}{
    name={Gray Literature Review},
    text={Gray Literature Review},
    sort=grayliteraturereview,
    description={A structured method of collecting and analyzing non-traditional published literature, much of which is published outside conventional academic channels. This research methodology concerns conducting a review of gray literature, such as technical reports, blog postings, professional forums, and industry documentation, to gain insight from practical experience. Gray literature reviews apply most to software engineering research as they represent real practices, challenges, and solutions that have taken place during implementation that may not have been captured or documented in the academic literature. This methodology acts like a bridge that closes the gap between theoretical research and its industry application.}
}

\newglossaryentry{wcag}{ 
 name={WCAG},
 text={WCAG},
 sort=wcag,
 description={The Web Content Accessibility Guidelines (WCAG) are a set of recommendations for making web content more accessible to people with disabilities. They provide a wide range of recommendations for making web content more accessible, including guidelines for text, images, sound, and more.}
}

\newglossaryentry{ariag}{ 
 name={ARIA},
 text={ARIA},
 sort=aria,
 description={Accessible Rich Internet Applications (ARIA) is a set of attributes that define ways to make web content and web applications more accessible to people with disabilities. ARIA roles, states, and properties help assistive technologies understand and interact with dynamic content and complex user interface controls.}
}

\newglossaryentry{voiceover}{ 
 name={VoiceOver},
 text={VoiceOver},
 sort=voiceover,
 description={VoiceOver is a screen reader built into Apple's macOS and iOS operating systems. It provides spoken descriptions of on-screen elements and allows users to navigate and interact with their devices using gestures and keyboard commands.}
}

\newglossaryentry{talkback}{ 
 name={TalkBack},
 text={TalkBack},
 sort=talkback,
 description={TalkBack is a screen reader developed by Google for Android devices. It provides spoken feedback and vibration to help visually impaired users navigate their devices and interact with apps.}
}