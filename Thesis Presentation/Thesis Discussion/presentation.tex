\documentclass[10pt,aspectratio=169]{beamer}

% Essential packages
\usepackage[T1]{fontenc}
\usepackage[utf8]{inputenc}
\usepackage[english]{babel}
\usepackage{lmodern}
\usepackage{microtype}
\usepackage{graphicx}
\usepackage{booktabs}
\usepackage{listings}
\usepackage{xcolor}
\usepackage{amsmath}
\usepackage{amsfonts}
\usepackage{amssymb}

% Use official University of Padua theme
\usetheme[pageofpages=of]{Unipd}

% Title and author information
\title{AccessibleHub - Designing an accessibility learning toolkit}
\subtitle{Bridging the gap between guidelines and implementation}
\author[Gabriel Rovesti]{\textbf{Graduate}: Gabriel Rovesti}
\institute[Dept. Mathematics - UniPD]{Department of Mathematics ``Tullio Levi-Civita'' - Master Degree in Computer Science}
\date{July 2025}

% Supervisor information
\newcommand{\supervisor}{Prof. Ombretta Gaggi}
\newcommand{\academicyear}{Academic Year 2024-2025 - Graduation Session: 25/07/2025}

\begin{document}

% Title slide
\begin{frame}[plain]
    \usebeamercolor[fg]{title page}
    \begin{center}
        \vspace{0.3cm}
        {\Huge \textcolor{white}{\textbf{\inserttitle}}}
        
        \vspace{0.3cm}
        {\Large \textcolor{white}{\insertsubtitle}}
        
        \vspace{1cm}
        {\large \textcolor{white} \insertauthor}
        
        \vspace{0.3cm}
        {\large \textcolor{white}{\textbf{Supervisor:} \supervisor}}

        \vspace{1cm}
        {\normalsize \textcolor{white}{\insertinstitute}}
        
        \vspace{0.3cm}
        {\normalsize \textcolor{white}{\academicyear}}
    \end{center}
\end{frame}

% Table of contents
\begin{frame}{Table of Contents}
    \tableofcontents
\end{frame}

% Section 1: Context and Motivation
\section{Research Context \& Motivation}

\begin{frame}{Research Context \& Motivation}
    \begin{columns}[c]
        \begin{column}{0.6\textwidth}
            \textbf{The Scale}
            \begin{itemize}
                \item 1+ billion people with disabilities globally
                \item 7 billion mobile device users
                \item Explosive growth in mobile applications
            \end{itemize}
            
            \vspace{0.5cm}
            
            \textbf{The Problem}
            \begin{itemize}
                \item Abstract WCAG guidelines exist
                \item Limited practical implementation guidance
                \item Gap between theory and mobile development
            \end{itemize}
        \end{column}
        \begin{column}{0.35\textwidth}
            \begin{block}{Evidence}
                \begin{itemize}
                    \item 22/57 EU public apps fail accessibility
                    \item 30\% of Android apps have accessibility issues
                    \item Developers struggle with implementation
                \end{itemize}
            \end{block}
        \end{column}
    \end{columns}
    
    \vspace{0.3cm}
    \begin{alertblock}{Core Challenge}
        How do we bridge the gap between abstract accessibility guidelines and concrete mobile implementation?
    \end{alertblock}
\end{frame}

% Section 2: Research Questions & Contributions
\section{Research Questions \& Contributions}

\begin{frame}{Research Questions \& Contributions}
    \begin{columns}[c]
        \begin{column}{0.6\textwidth}
            \textbf{Primary Research Questions}
            \begin{enumerate}
                \item How can we systematically evaluate accessibility implementation across frameworks?
                \item What design patterns optimize both accessibility and developer experience?
                \item How effective are theory-informed educational approaches?
            \end{enumerate}
            
            \vspace{0.5cm}
            
            \textbf{Research Contributions}
            \begin{itemize}
                \item Novel evaluation framework with 6 formal metrics
                \item First systematic quantitative comparison of React Native vs Flutter
                \item AccessibleHub: Open-source learning toolkit
            \end{itemize}
        \end{column}
        \begin{column}{0.35\textwidth}
            \begin{block}{Innovation}
                \textbf{Theory-Practice Bridge}
                \begin{itemize}
                    \item WCAG2Mobile mapping
                    \item Quantitative metrics
                    \item Educational design theory
                \end{itemize}
            \end{block}
        \end{column}
    \end{columns}
\end{frame}

% Section 3: Methodological Innovation
\section{Methodological Innovation}

\begin{frame}{Methodological Innovation}
    \begin{columns}[c]
        \begin{column}{0.65\textwidth}
            \textbf{Six Formal Accessibility Metrics}
            \begin{enumerate}
                \item \textbf{CAS}: Component Accessibility Score
                \item \textbf{WCR}: WCAG Compliance Rate
                \item \textbf{SRSS}: Screen Reader Support Score
                \item \textbf{IMO}: Implementation Overhead
                \item \textbf{API}: Accessibility API Coverage
                \item \textbf{DTE}: Development Time Efficiency
            \end{enumerate}
            
            \vspace{0.3cm}
            
            \textbf{WCAG2Mobile Integration}
            \begin{itemize}
                \item Direct mapping: WCAG 2.1 AA success criteria $\to$ mobile components
                \item Weighted scoring based on impact and frequency
                \item Cross-platform consistency validation
            \end{itemize}
        \end{column}
        \begin{column}{0.3\textwidth}
            \begin{block}{Formula}
                \small
                $CAS = \frac{\sum_{i=1}^{n} w_i \cdot c_i}{\sum_{i=1}^{n} w_i}$
                
                \vspace{0.2cm}
                
                $WCR = \frac{|S_p|}{|S_t|} \times 100\%$
                
                where $S_p =$ passed criteria, $S_t =$ total criteria
            \end{block}
        \end{column}
    \end{columns}
\end{frame}

% Section 4: AccessibleHub Research Vehicle
\section{AccessibleHub: Research Vehicle}

\begin{frame}{AccessibleHub: Research Vehicle}
    \begin{columns}[c]
        \begin{column}{0.55\textwidth}
            \textbf{Educational Design Principles}
            \begin{itemize}
                \item \textbf{Scaffolded Learning}: Progressive complexity
                \item \textbf{Theory-Practice Integration}: WCAG $\leftrightarrow$ Code
                \item \textbf{Multi-Modal}: Visual + Audio + Hands-on
                \item \textbf{Community-Centered}: Social learning
            \end{itemize}
            
            \vspace{0.4cm}
            
            \textbf{Technical Architecture}
            \begin{itemize}
                \item React Native foundation
                \item Expo development workflow
                \item Cross-platform deployment
                \item Open-source MIT license
            \end{itemize}
        \end{column}
        \begin{column}{0.4\textwidth}
            \begin{block}{Core Features}
                \begin{itemize}
                    \item 20+ Interactive components
                    \item WCAG 2.1 AA compliance
                    \item Screen reader optimized
                    \item Real-time testing
                    \item Community contributions
                \end{itemize}
            \end{block}
            
            \vspace{0.3cm}
            
            \textbf{Availability}
            \begin{itemize}
                \item GitHub: gabrielrovesti/AccessibleHub
                \item Documentation \& guides
                \item Video tutorials
            \end{itemize}
        \end{column}
    \end{columns}
\end{frame}

% Section 5: Implementation Examples
\section{Implementation Examples}

\begin{frame}[fragile]{Implementation Examples \& Patterns}
    \begin{columns}[t]
        \begin{column}{0.48\textwidth}
            \textbf{React Native Button}
            \begin{lstlisting}[basicstyle=\tiny\ttfamily]
<TouchableOpacity
  accessibilityRole="button"
  accessibilityLabel="Submit form"
  accessibilityHint="Validates and submits the current form"
  onPress={handleSubmit}>
  <Text>Submit</Text>
</TouchableOpacity>
            \end{lstlisting}
            
            \vspace{0.3cm}
            
            \textbf{Key Patterns}
            \begin{itemize}
                \item Explicit semantic roles
                \item Descriptive labels
                \item Action hints
                \item Focus management
            \end{itemize}
        \end{column}
        \begin{column}{0.48\textwidth}
            \textbf{Flutter Button}
            \begin{lstlisting}[basicstyle=\tiny\ttfamily]
Semantics(
  label: 'Submit form',
  button: true,
  onTap: () => handleSubmit(),
  child: ElevatedButton(
    onPressed: handleSubmit,
    child: Text('Submit'),
  ),
)
            \end{lstlisting}
            
            \vspace{0.3cm}
            
            \textbf{Architecture Differences}
            \begin{itemize}
                \item Property-based vs Widget-based
                \item Implicit vs Explicit semantics
                \item Platform integration approaches
            \end{itemize}
        \end{column}
    \end{columns}
\end{frame}

% Section 6: Quantitative Results
\section{Quantitative Results}

\begin{frame}{Quantitative Results}
    \begin{columns}[c]
        \begin{column}{0.6\textwidth}
            \textbf{AccessibleHub Performance}
            \begin{itemize}
                \item \textbf{CAS}: 100\% component implementation (20/20)
                \item \textbf{WCR}: 88\% WCAG 2.1 AA compliance
                \item \textbf{SRSS}: 4.3/5.0 average (VoiceOver + TalkBack)
                \item \textbf{IMO}: 23.3\% average implementation overhead
            \end{itemize}
            
            \vspace{0.5cm}
            
            \textbf{Framework Comparison Results}
            \begin{itemize}
                \item React Native: 38\% default accessible components
                \item Flutter: 32\% default accessible components
                \item React Native: 45\% less implementation code
                \item Screen reader consistency: React Native advantage
            \end{itemize}
        \end{column}
        \begin{column}{0.35\textwidth}
            \begin{block}{Key Metrics}
                \begin{tabular}{|l|c|}
                    \hline
                    \textbf{Metric} & \textbf{Score} \\
                    \hline
                    CAS & 100\% \\
                    WCR & 88\% \\
                    SRSS & 4.3/5.0 \\
                    IMO & 23.3\% \\
                    \hline
                \end{tabular}
            \end{block}
        \end{column}
    \end{columns}
\end{frame}

% Section 7: Conclusions
\section{Conclusions}

\begin{frame}{Conclusions}
    \begin{columns}[c]
        \begin{column}{0.6\textwidth}
            \textbf{Research Contributions}
            \begin{enumerate}
                \item \textbf{Methodological}: Novel evaluation framework with 6 formal metrics
                \item \textbf{Practical}: AccessibleHub toolkit with theory-informed design
                \item \textbf{Empirical}: First systematic quantitative framework comparison
            \end{enumerate}
            
            \vspace{0.5cm}
            
            \textbf{Future Research Directions}
            \begin{itemize}
                \item Framework expansion: SwiftUI, Jetpack Compose
                \item User studies: Developer effectiveness measurement
                \item Automation: CI/CD pipeline integration
                \item Community: Open source ecosystem development
            \end{itemize}
        \end{column}
        \begin{column}{0.35\textwidth}
            \begin{alertblock}{Impact}
                \textbf{Theory-Practice Bridge}
                \begin{itemize}
                    \item Quantitative accessibility metrics
                    \item Open-source educational toolkit
                    \item Evidence-based framework decisions
                \end{itemize}
            \end{alertblock}
        \end{column}
    \end{columns}
\end{frame}

% Final slide
\begin{frame}[plain]
    \begin{center}
        \Huge Thank You
        
        \vspace{1cm}
        
        \Large Questions \& Discussion
        
        \vspace{1cm}
        
        \normalsize
        Gabriel Rovesti \\
        gabriel.rovesti@studenti.unipd.it \\
        GitHub: gabrielrovesti/AccessibleHub
    \end{center}
\end{frame}

\end{document}