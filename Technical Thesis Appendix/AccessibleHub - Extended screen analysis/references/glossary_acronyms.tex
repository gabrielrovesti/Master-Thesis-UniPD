% Acronyms
\newacronym{ui}{UI}{User Interface}
\newacronym{ux}{UX}{User Experience}
\newacronym{w3c}{W3C}{World Wide Web Consortium}
\newacronym{aria}{ARIA}{Accessible Rich Internet Applications}
\newacronym{wcagacr}{WCAG}{Web Content Accessibility Guidelines}
\newacronym{mcagacr}{MCAG}{Mobile Content Accessibility Guidelines}
\newacronym{api}{API}{Application Programming Interface}
\newacronym{loc}{LOC}{Lines of Code}

% Glossary
\newglossaryentry{uig}{
    name={User Interface},
    text={User Interface},
    sort=ui,
    description={The User Interface refers to the space where interactions between humans and machines occur. It includes the design and arrangement of graphical elements (such as buttons, icons, and menus) that enable users to interact with software or hardware systems. The goal of a UI is to make the user's interaction simple and efficient in accomplishing tasks within a system}
}

\newglossaryentry{apig}{ 
 name={Application Programming Interface},
 text={API},
 sort=api,
 description={An Application Programming Interface (API) is a set of protocols, routines, and tools for building software applications. It specifies how software components should interact, allowing different software systems to communicate with each other. APIs define the methods and data structures that developers can use to interact with a system, service, or library without needing to understand the underlying implementation. They serve as a contract between different software components, enabling developers to integrate different systems, access web-based services, and create more complex and interconnected software solutions}
}

\newglossaryentry{uxg}{
    name={User Experience},
    text={User Experience},
    sort=ux,
    description={User Experience encompasses the overall experience a user has while interacting with a product or service. It includes not only usability and interface design but also the emotional response, satisfaction, and ease of use a person feels while using a system. UX design focuses on optimizing a product’s interaction to provide meaningful and relevant experiences to users, ensuring that the system is intuitive, efficient, and enjoyable to use}
}

\newglossaryentry{grayliteraturereview}{
    name={Gray Literature Review},
    text={Gray Literature Review},
    sort=grayliteraturereview,
    description={A structured method of collecting and analyzing non-traditional published literature, much of which is published outside conventional academic channels. This research methodology concerns conducting a review of gray literature, such as technical reports, blog postings, professional forums, and industry documentation, to gain insight from practical experience. Gray literature reviews apply most to software engineering research as they represent real practices, challenges, and solutions that have taken place during implementation that may not have been captured or documented in the academic literature. This methodology acts like a bridge that closes the gap between theoretical research and its industry application}
}

\newglossaryentry{wcagg}{ 
 name={WCAG},
 text={WCAG},
 sort=wcag,
 description={The Web Content Accessibility Guidelines (WCAG) are a set of recommendations for making web content more accessible to people with disabilities. They provide a wide range of recommendations for making web content more accessible, including guidelines for text, images, sound, and more}
}

\newglossaryentry{ariag}{ 
 name={ARIA},
 text={ARIA},
 sort=aria,
 description={Accessible Rich Internet Applications (ARIA) is a set of attributes that define ways to make web content and web applications more accessible to people with disabilities. ARIA roles, states, and properties help assistive technologies understand and interact with dynamic content and complex user interface controls}
}

\newglossaryentry{voiceover}{ 
 name={VoiceOver},
 text={VoiceOver},
 sort=voiceover,
 description={VoiceOver is a screen reader built into Apple's macOS and iOS operating systems. It provides spoken descriptions of on-screen elements and allows users to navigate and interact with their devices using gestures and keyboard commands}
}

\newglossaryentry{talkback}{ 
 name={TalkBack},
 text={TalkBack},
 sort=talkback,
 description={TalkBack is a screen reader developed by Google for Android devices. It provides spoken feedback and vibration to help visually impaired users navigate their devices and interact with apps}
}

\newglossaryentry{mcagg}{ 
 name={MCAG},
 text={MCAG},
 sort=mcag,
 description={Mobile Content Accessibility Guidelines (MCAG) are a specialized set of accessibility recommendations specifically tailored to mobile application and mobile web content. While building upon the foundational principles of WCAG, MCAG addresses unique challenges of mobile interfaces, such as touch interactions, small screen sizes, diverse input methods, and mobile-specific assistive technologies. These guidelines provide specific considerations for creating accessible content and interfaces on smartphones, tablets, and other mobile devices, taking into account the distinct interaction patterns and technological constraints of mobile platforms}
}

\newglossaryentry{w3cg}{ 
 name={W3C},
 text={W3C},
 sort=w3c,
 description={The World Wide Web Consortium (W3C) is an international community that develops open standards to ensure the long-term growth and evolution of the web. Founded by Tim Berners-Lee in 1994, the W3C works to create universal web standards that promote interoperability and accessibility across different platforms, browsers, and devices. This non-profit organization brings together technology experts, researchers, and industry leaders to develop guidelines and protocols that form the fundamental architecture of the World Wide Web. Key contributions include HTML, CSS, accessibility guidelines (WCAG), and web standards that ensure a consistent, inclusive, and innovative web experience for users worldwide}
}

\newglossaryentry{flutter}{ 
 name={Flutter},
 text={Flutter},
 sort=flutter,
 description={Flutter is an open-source UI software development kit created by Google, designed for building natively compiled applications for mobile, web, and desktop platforms from a single codebase. Launched in 2017, Flutter uses the Dart programming language and provides a comprehensive framework for creating high-performance, visually attractive applications with a focus on smooth, responsive user interfaces. Unlike traditional cross-platform frameworks that use web view rendering, Flutter compiles directly to native code, enabling near-native performance. Its key features include a rich set of pre-designed widgets, hot reload for rapid development, extensive customization capabilities, and a robust ecosystem that supports complex application development across multiple platforms}
}

\newglossaryentry{reactnative}{ 
 name={React Native},
 text={React Native},
 sort=reactnative,
 description={React Native is an open-source mobile application development framework created by Facebook (now Meta) that allows developers to build mobile applications using JavaScript and React. Introduced in 2015, React Native enables developers to create native mobile apps for both iOS and Android platforms using a single codebase, leveraging the popular React web development library. Unlike hybrid app frameworks, React Native renders components using actual native platform UI elements, providing a more authentic user experience and better performance. The framework bridges the gap between web and mobile development, allowing web developers to create mobile applications using familiar JavaScript and React paradigms, while still achieving near-native performance and user interface responsiveness}
}

\newglossaryentry{screenreaderg}{ 
 name={Screen Reader},
 text={Screen Reader},
 sort=screenreader,
 description={A screen reader is an assistive technology software that enables people with visual impairments or reading disabilities to interact with digital devices by converting on-screen text and elements into synthesized speech or Braille output. Screen readers navigate through user interfaces, reading text, describing graphical elements, and providing auditory feedback about the computer or mobile device's content and functionality. They interpret and verbalize user interface elements, buttons, menus, and other interactive components, allowing visually impaired users to understand and interact with digital content. Popular screen readers include VoiceOver for Apple devices, TalkBack for Android, and NVDA and JAWS for desktop computers}
}

\newglossaryentry{locg}{
    name={Lines of Code},
    text={LOC},
    sort=loc,
    description={A metric used to quantify the size or complexity of a software program by counting the number of lines in its source code. LOC is often used as an indicator of development effort, code maintenance, and project scale.}
}