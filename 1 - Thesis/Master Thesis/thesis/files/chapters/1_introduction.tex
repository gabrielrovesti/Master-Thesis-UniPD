\chapter{Introduction}
\label{chap:intro}

% \begin{figure}[H]
%     \centering
%     \includegraphics[alt={Testo alternativo dell'immagine}, width=1\columnwidth]{img/quantum_entanglement.jpeg}
%     \caption{Lorem}
%     \label{fig:entanglement}
% \end{figure}
% 
% Introduzione al contesto applicativo.
% 
% Lorem Figure \ref{fig:entanglement}
% 
% Esempio di utilizzo di un termine nel glossario \gls{api}.
% 
% Esempio di citazione direttamente nel testo \cite{site:agile-manifesto}.
% 
% Esempio di citazione nel piè di pagina \footcite{womak:lean-thinking}.
% Introduzione all'idea dello stage\footcite{article:spooky}.

\section{Background and motivations}
\label{chap:intro-background}

In an era where digital technology permeates every aspect of our lives, mobile devices have emerged as the primary gateway to the digital world, allowing a lot of new people to be connected at any given time, no matter the condition. An estimated number of circa 7 billions \cite{article:number-of-users}, representing a dramatic increase from just one billion users in 2013. This explosive growth has not only changed how we communicate and access information but has also created a massive market for different needs and introduced new categories of users, with different habits and cultures into a truly global market.

As mobile applications become increasingly central to daily life, ensuring their accessibility to all users, regardless of their abilities or disabilities, has become a critical imperative, since not only technology should be able to connect, but also to unite seamlessly people with different capabilities. Accessibility refers to the design and development practices enabling all users, regardless of their abilities or disabilities, to perceive, understand and navigate with digital content effectively. Not only the quantity of media increased, but also the quantity of different media which allow to access information definitely increased; finding appropriate measurements to establish a good level of understanding and usability is important and finding appropriate levels of measurements is non-trivial. \\

An estimated portion of over one billion people lives globally with some forms of disability \cite{article:who-disability}. Inaccessible mobile applications can, therefore, present considerable barriers to participation in that large and growing part of modern life that involves education, employment, social interaction, and even basic services. Accessibility is not about a majority giving special dispensation to a minority but rather about providing equal access and opportunities to very big and diverse user bases.

Accessibility encompasses a wide range of considerations, including but not limited to:
\begin{enumerate}
    \item \textit{Visual accessibility}

    \item \textit{Auditory accessibility}

    \item \textit{Motor accessibility}

    \item \textit{Cognitive accessibility}
\end{enumerate}

\section{Related works}
\label{chap:intro-related-works}

\section{Thesis structure}
\label{chap:intro-structure} 

% \begin{description}
%     \item[{\hyperref[chap:processi-metodologie]{The second chapter}}] describes ...
%     
%     \item[{\hyperref[chap:descrizione-stage]{The third chapter}}] describes ...
%     
%     \item[{\hyperref[chap:analisi-requisiti]{The fourth chapter}}] describes ...
%     
%     \item[{\hyperref[chap:progettazione-codifica]{The fifth chapter}}] describes ...
%     
%     \item[{\hyperref[chap:verifica-validazione]{The sixth chapter}}] describes ...
%     
%     \item[{\hyperref[chap:conclusions]{The last chapter}}] describes ...
% \end{description}
% 
% Regarding the text composition, the following typographical conventions have been adopted for this document:
% \begin{itemize}
% 	\item acronyms, abbreviations, and ambiguous or uncommon terms mentioned are defined in the glossary, located at the end of this document;
% 	\item For the first occurrence of terms listed in the glossary, the following nomenclature is used: \gls{apig};
% 	\item foreign language terms or those belonging to technical jargon are highlighted in \textit{italic} characters.
% \end{itemize}
% 
% \begin{listing}[H]
% \begin{minted}{c}
% #include <stdio.h>
% int main() {
%     print("Hello, world!");
%     return 0;
% }
% \end{minted}
% \caption{Example of code}
% \label{listing:a}
% \end{listing}

\newpage