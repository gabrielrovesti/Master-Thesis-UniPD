\chapter{Accessibility analysis: framework comparison and implementation patterns} 
\label{chap:implementation}

\chapterintroline{
   This chapter offers a systematic, comparative analysis of accessibility implementation in React Native and Flutter. Through empirical evaluation of equivalent components, we address three core questions: the default accessibility of components, the feasibility of implementing accessibility for non-accessible components, and the development effort required for these implementations. Combining quantitative metrics with qualitative assessments of developer experience, this analysis provides practical insights into how each framework facilitates the creation of accessible mobile applications.
}

\section {Research methodology}

\subsection {Research questions and objectives}

\subsection {Testing approach and criteria}

\subsection {Evaluation metrics}

\section {React Native Approach}

\subsection {Component accessibility analysis}

\subsection {Native platform integration}

\subsection {Implementation patterns and solutions}

\section {Flutter Approach}

\subsection{Framework overview}

\subsection {Component accessibility patterns}

\subsection {Platform-specific considerations}

\subsection {Common implementation challenges}

\section {Comparative analysis}

\subsection {Default accessibility support}

\subsection {Implementation complexity}

\subsection {Developer experience}

\subsection {Performance implications}

\section {Framework-specific best practices}

\subsection{React Native optimization patterns}

\subsection{Flutter accessibility patterns}

\subsection{Cross-platform considerations}

\newpage