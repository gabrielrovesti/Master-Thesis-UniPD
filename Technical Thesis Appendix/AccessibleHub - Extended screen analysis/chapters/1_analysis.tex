\chapter{Bridging the gap between implementation and accessibility guidelines - Full app analysis}
\label{chap:accessibility-developer-manual}

\chapterintroline{
    This manual provides comprehensive practical implementation guidance for mobile developers seeking to create accessible applications. Building upon the research presented in Chapter 3 of the Master Thesis present \href{https://github.com/gabrielrovesti/Master-Thesis-UniPD/blob/main/Thesis/Thesis.pdf}{here}, it offers a detailed, screen-by-screen analysis of accessibility considerations in mobile interfaces. Through concrete implementation examples, WCAG compliance mappings, and platform-specific considerations, this manual transforms theoretical accessibility guidelines into practical code patterns.
}

\section{Introduction}
\label{sec:dev-intro}

Despite the widespread availability of accessibility guidelines, mobile developers often struggle to translate these abstract principles into concrete implementation practices. While theoretical understanding of the Web Content Accessibility Guidelines (WCAG) and Mobile Content Accessibility Guidelines (MCAG) is increasingly common, practical examples and implementation patterns remain scarce. This manual addresses this implementation gap by providing comprehensive, code-focused guidance for integrating accessibility features across different mobile interface patterns.

\subsection{Purpose and scope}
\label{subsec:dev-purpose}

This developer manual serves several complementary purposes:

\begin{itemize}
    \item Providing concrete implementation examples for accessibility features in mobile applications, with real-world code patterns that can be directly applied;
    
    \item Demonstrating how abstract WCAG and MCAG guidelines translate into practical code, creating a direct bridge between theory and implementation;
    
    \item Quantifying the implementation overhead of accessibility features to assist in project planning and resource allocation;
    
    \item Highlighting mobile-specific accessibility considerations that extend beyond standard web guidelines, addressing the unique challenges of touch interfaces;
    
    \item Establishing reusable patterns that developers can adapt to their own projects, regardless of specific application domain.
\end{itemize}

Rather than offering general recommendations, this manual takes a methodical screen-by-screen approach. Each screen analysis follows a consistent framework that includes:

\begin{enumerate}
    \item \textbf{Component inventory and WCAG/MCAG mapping}: Formal identification of UI elements with their semantic roles and relationships to accessibility guidelines;
    
    \item \textbf{Technical implementation analysis}: Detailed code examples with annotations highlighting key accessibility properties and implementation techniques;
    
    \item \textbf{Screen reader support analysis}: Results from empirical testing with VoiceOver (iOS) and TalkBack (Android), documenting actual behavior for screen reader users;
    
    \item \textbf{Implementation overhead analysis}: Quantification of the additional code required to implement accessibility features, with detailed breakdowns by feature type;
    
    \item \textbf{Mobile-specific considerations}: Adaptations necessary for touch interfaces and mobile contexts that extend beyond standard WCAG requirements.
\end{enumerate}

This structured approach ensures that developers gain not just theoretical understanding but practical knowledge about implementing accessibility in mobile contexts.

\subsection{Relationship to \textit{AccessibleHub}}
\label{subsec:dev-relation}

\textit{AccessibleHub} is a React Native application designed to serve as an interactive manual for implementing accessibility features in mobile development. This developer manual documents the technical implementation details of \textit{AccessibleHub} itself, analyzing in full detail how its various screens address accessibility requirements through specific code patterns.

The application structure follows the educational framework described in Section §4.3 of the Master Thesis, with screens organized into logical educational sections:

\begin{itemize}
    \item \textbf{Accessible components section}: Providing practical implementations of common UI elements with varying complexity levels, offering copyable code examples that developers can incorporate directly into their projects;
    
    \item \textbf{Best practices main screen and section}: Offering guidance on overarching accessibility principles, from semantic structure to screen reader optimization, with concrete implementation examples;
    
    \item \textbf{Tools screen}: Cataloging resources for testing and implementing accessibility, connecting theoretical knowledge with practical evaluation techniques;
    
    \item \textbf{Instruction and community screen}: Extending beyond technical implementation to connect developers with broader learning resources and community projects;
    
    \item \textbf{Settings screen}: Demonstrating accessibility customization options that adapt the interface to diverse user needs, serving both as a functional feature and an educational example.
\end{itemize}

By analyzing the accessibility implementation of \textit{AccessibleHub} itself, this manual creates a recursive learning experience where the medium demonstrates the message—showing accessibility implementation through an accessible application.

\subsection{How to use this manual}
\label{subsec:dev-usage}

This manual can be approached in several ways depending on the developer's specific needs and learning objectives:

\begin{itemize}
    \item \textbf{Component-focused learning}: Developers working on specific interface elements can go directly to the relevant component sections (e.g., buttons, forms, dialogs) to find implementation patterns for their immediate tasks;
    
    \item \textbf{Screen-type guidance}: Those designing particular screen types (e.g., settings screens, home screens) can reference the corresponding section for holistic accessibility approaches that address full-screen concerns;
    
    \item \textbf{Code pattern library}: The annotated code samples throughout the manual serve as a pattern library for implementing specific accessibility features, with explanations of why particular techniques are used;
    
    \item \textbf{Compliance mapping reference}: The WCAG/MCAG mapping tables provide a reference for understanding which implementation techniques address specific accessibility requirements, helping with formal compliance documentation;
    
    \item \textbf{Implementation planning}: The overhead analyses help developers and project managers understand the additional resources required to implement comprehensive accessibility, facilitating realistic project planning.
\end{itemize}

Throughout the manual, the focus remains on practical implementation while grounding recommendations in formal accessibility standards and empirical testing results. The examples are presented in React Native code but include notes on Flutter implementation differences where relevant, making the guidance valuable across multiple frameworks.

\section{Theoretical foundation}
\label{sec:dev-theoretical-foundation}

Before diving into specific screen implementations, it's important to establish the theoretical foundation that guides our accessibility approach. This foundation consists of three interconnected dimensions: formal guidelines, mobile-specific adaptations, and implementation principles that shape how accessibility is realized in code.

\subsection{Accessibility standards and guidelines}
\label{subsec:dev-standards}

Our implementation patterns are guided by several formal accessibility standards that provide the framework for ensuring digital inclusivity:

\begin{itemize}
    \item \textbf{WCAG 2.2}: Web Content Accessibility Guidelines provide the core success criteria for digital accessibility, organized under four principles: Perceivable (ensuring users can perceive content), Operable (ensuring users can navigate and interact), Understandable (ensuring content is clear and predictable), and Robust (ensuring compatibility with assistive technologies);
    
    \item \textbf{MCAG}: Mobile Content Accessibility Guidelines extend WCAG principles with mobile-specific considerations for touch interfaces, limited screen size, and varied usage contexts. These considerations include touch target sizes, gesture alternatives, and orientation adaptations that are particularly relevant in mobile environments;
    
    \item \textbf{Platform-specific guidelines}: Both Apple (iOS) and Google (Android) maintain specific accessibility guidelines for their respective platforms. These address platform-specific implementation details like VoiceOver gestures, TalkBack focus behavior, and native component accessibility properties.
\end{itemize}

Throughout this manual, implementations are mapped to specific WCAG and MCAG criteria, demonstrating how abstract guidelines translate into concrete code patterns. The analysis prioritizes meeting Level AA requirements (the generally accepted standard for accessibility compliance) while implementing Level AAA enhancements where they provide significant user benefits without excessive implementation burden.

\subsection{Mobile-specific considerations}
\label{subsec:mobile-considerations}

Mobile platforms present unique accessibility challenges that extend beyond traditional web accessibility concerns. The implementation patterns in this manual address several mobile-specific considerations:

\begin{itemize}
    \item \textbf{Touch interaction}: Unlike keyboard and mouse interfaces, touch interactions lack precision and require larger interaction targets. Mobile accessibility implementation must account for varying finger sizes, motor control limitations, and the inability to "hover" before selecting;
    
    \item \textbf{Limited screen real estate}: Mobile screens provide significantly less space than desktop interfaces, creating tension between content density and accessibility. Implementation patterns must balance information presentation with readability and touch target size;
    
    \item \textbf{Screen reader gesture conflicts}: Mobile screen readers use touch gestures (swipes, taps) that may conflict with application gestures. Accessible implementations must provide alternative interaction methods when screen readers are active;
    
    \item \textbf{Variable contexts}: Mobile devices are used in diverse environmental conditions (bright sunlight, moving vehicles, one-handed operation) that impact accessibility. Implementations must adapt to these varying contexts through customization options;
    
    \item \textbf{Platform differences}: iOS and Android differ significantly in their accessibility APIs, screen reader behaviors, and gesture conventions. Cross-platform implementations must address these differences while maintaining consistent accessibility.
\end{itemize}

These mobile-specific considerations inform the implementation patterns documented throughout this manual, extending accessibility implementation beyond what would be sufficient for web interfaces.

\subsection{Implementation principles}
\label{subsec:dev-principles}

Throughout this manual, several core implementation principles guide our approach to accessibility. These principles connect abstract guidelines to concrete implementation decisions:

\begin{itemize}
    \item \textbf{Semantic integrity}: UI elements communicate their purpose and role explicitly to assistive technologies through appropriate \texttt{accessibilityRole} assignments and descriptive labels. This principle directly addresses WCAG 4.1.2 Name, Role, Value (A) by ensuring assistive technologies can interpret interface elements correctly;
    
    \item \textbf{Focus management}: Applications maintain logical focus order and explicitly manage focus during dynamic interactions like opening dialogs or navigating between screens. This implementation principle supports WCAG 2.4.3 Focus Order (A) by ensuring predictable and logical navigation for assistive technology users;
    
    \item \textbf{State communication}: Interactive elements communicate their states (selected, disabled, etc.) clearly to all users through both visual cues and explicit \texttt{accessibilityState} properties. This supports WCAG 4.1.2 Name, Role, Value (A) by making dynamic state changes perceivable to assistive technology users;
    
    \item \textbf{Multi-sensory feedback}: Actions and changes are communicated through multiple sensory channels (visual, auditory, haptic) to ensure perceivability regardless of user capabilities. This principle addresses WCAG 1.3.1 Info and Relationships (A) by providing redundant information paths;
    
    \item \textbf{Adaptability}: Interfaces adapt to user preferences and needs, providing options for personalization including text size, contrast, motion reduction, and other adaptations. This implementation principle supports WCAG 1.4.4 Resize Text (AA) and 1.4.8 Visual Presentation (AAA);
    
    \item \textbf{Touch optimization}: Interactive elements are sized and positioned for optimal touch interaction, exceeding minimum size requirements to accommodate users with motor control limitations. This directly addresses WCAG 2.5.8 Target Size (AA);
    
    \item \textbf{Screen reader efficiency}: Implementation minimizes unnecessary interactions for screen reader users by hiding decorative elements and providing comprehensive contextual labels. This principle particularly addresses mobile-specific concerns about "swipe fatigue" during screen reader navigation.
\end{itemize}

These principles inform the specific code patterns and implementations documented throughout this manual, creating a coherent approach to accessibility that extends across different screen types and component categories. Each principle connects directly to specific WCAG success criteria while addressing the practical realities of mobile implementation.

By grounding our implementation patterns in both formal guidelines and practical mobile considerations, we establish a foundation for the screen-by-screen analyses that follow. This approach ensures that accessibility is implemented systematically rather than as an afterthought, resulting in truly inclusive mobile experiences.

\section{Accessibility implementation guidelines}
\label{sec:implementation-guidelines}

Each of the following subsections highlights the key \textit{success criteria} addressed, references relevant \textit{mobile-specific considerations}, and demonstrates practical solutions in React Native building upon the insights from Gaggi and Perinello's approach \cite{budai2024mobile}. The following part is intended to complete the screen analysis considered in the Master Thesis, with the goal of supplying a complete developer resource to be used and applied in professional projects.

\subsection{WCAG2Mobile integration framework}
\label{sec:wcag2mobile-framework}

While the AccessibleHub toolkit was developed based on established WCAG 2.2 guidelines, the W3C has recently published "Guidance on applying WCAG 2.2 to mobile applications" (WCAG2Mobile) \cite{w3c-wcag2mobile}, which provides authoritative interpretations of accessibility success criteria specifically tailored for mobile contexts. This section outlines how WCAG2Mobile principles have been integrated into the \textit{AccessibleHub} framework to ensure alignment with the latest mobile accessibility standards.

\subsubsection{Understanding WCAG2Mobile}

WCAG2Mobile represents an important evolution in accessibility guidelines, adapting the web-oriented WCAG 2.2 criteria to address the unique challenges of mobile application interfaces. Unlike traditional web content, mobile applications present distinct interaction patterns, viewport limitations, and platform-specific considerations that require specialized interpretation of accessibility principles.

Key aspects of WCAG2Mobile include:

\begin{itemize}
    \item Mobile-specific terminology adaptations (e.g., "screens" instead of "pages", "views" instead of "web content");
    
    \item Reinterpretation of success criteria for touch-based interactions rather than keyboard/mouse paradigms;
    
    \item Platform-specific implementation guidance for iOS and Android accessibility frameworks;
    
    \item Considerations for limited screen real estate and its impact on perceivability;
    
    \item Guidelines for handling mobile-specific interaction patterns such as gestures, notifications, and platform transitions.
\end{itemize}

By integrating WCAG2Mobile principles into \textit{AccessibleHub}, we ensure that the toolkit not only addresses general accessibility requirements but also provides mobile-specific guidance that reflects the authoritative W3C interpretation of how WCAG 2.2 applies to mobile application development.

\subsubsection{WCAG2Mobile mapping to component categories}

Table~\ref{tab:wcag2mobile_mapping} provides a formal mapping between \textit{AccessibleHub} component categories and the relevant WCAG2Mobile success criteria, highlighting mobile-specific considerations for each component type.

\begin{longtable}[c]{|C{2.5cm}|C{3cm}|C{4cm}|C{5cm}|}
\caption{AccessibleHub component categories mapped to WCAG2Mobile success criteria}
\label{tab:wcag2mobile_mapping}\\
\hline
\textbf{Component Category} & \textbf{Primary WCAG2Mobile Success Criteria} & \textbf{Mobile-Specific Considerations} & \textbf{Implementation Focus} \\
\hline
\endfirsthead
\multicolumn{4}{c}%
{{\bfseries Table \thetable\ -- continued from previous page}} \\
\hline
\textbf{Component Category} & \textbf{Primary WCAG2Mobile Success Criteria} & \textbf{Mobile-Specific Considerations} & \textbf{Implementation Focus} \\
\hline
\endhead
\hline
\multicolumn{4}{r}{{Continued on next page}} \\
\endfoot
\hline
\endlastfoot
Buttons \& Touchables & 2.5.8 Target Size (AA), 4.1.2 Name, Role, Value (A), 2.1.1 Keyboard (A) & Touch target optimization, Screen reader focus, One-handed operation & Minimum touch target of 44×44dp, Precise \texttt{accessibility \ Role} assignment, Gesture alternatives \\
\hline
Forms & 3.3.1 Error Identification (A), 3.3.2 Labels or Instructions (A), 1.3.1 Info and Relationships (A) & Virtual keyboard management, Input validation, Form control grouping & Input labeling techniques, Error messaging, Keyboard type adaptation \\
\hline
Media & 1.1.1 Non-text Content (A), 1.2.2 Captions (A), 1.4.2 Audio Control (A) & Mobile bandwidth considerations, Small-screen playback controls, Device orientation adaptation & Efficient alt text, Responsive media controls, Bandwidth-aware loading \\
\hline
Dialogs & 2.4.3 Focus Order (A), 4.1.2 Name, Role, Value (A), 3.2.1 On Focus (A) & Overlay management, Modal dismissal gestures, Screen reader trapping prevention & Focus management, Dismissal methods, Proper dialog role assignment \\
\hline
Advanced Controls & 1.3.1 Info and Relationships (A), 4.1.2 Name, Role, Value (A), 2.5.1 Pointer Gestures (A) & Complex gesture alternatives, State communication, Interactive element grouping & Slider implementations, Tab navigation patterns, Platform-specific role mappings \\
\end{longtable}

\FloatBarrier

This mapping forms the foundation for integrating WCAG2Mobile throughout the \textit{AccessibleHub} toolkit, guiding both the implementation details in code examples and the educational content presented to developers. Each component category has been analyzed through the lens of WCAG2Mobile to identify mobile-specific considerations that extend beyond standard WCAG 2.2 interpretations.

\subsubsection{Integration approach}

Rather than treating WCAG2Mobile as a separate framework, this implementation integrates these mobile-specific interpretations directly into the existing structure of \textit{AccessibleHub}. This integration occurs at multiple levels:

\begin{enumerate}
    \item \textbf{Terminology alignment} - Throughout the toolkit, terminology has been updated to match WCAG2Mobile conventions (e.g., "screens" instead of "pages") to reinforce mobile-specific contexts;
    
    \item \textbf{Success criteria mapping} - Each component example explicitly references relevant WCAG2Mobile interpretations of success criteria, highlighting where mobile implementations differ from web approaches;
    
    \item \textbf{Implementation patterns} - Code examples demonstrate WCAG2Mobile-compliant implementation techniques, with particular attention to mobile-specific challenges like touch target sizing and screen reader optimization;
    
    \item \textbf{Testing methodologies} - Screen reader testing procedures align with WCAG2Mobile guidance on platform-specific assistive technology testing;
    
    \item \textbf{Educational content} - Best practices sections incorporate WCAG2Mobile insights on mobile-specific accessibility considerations.
\end{enumerate}

This integrated approach ensures that developers using \textit{AccessibleHub} receive guidance that is both fundamentally sound (based on core WCAG 2.2 principles) and contextually appropriate for mobile development (through WCAG2Mobile interpretations).

Detailed implementation examples illustrating WCAG2Mobile principles are provided in the following sections for each component category, with cross-references to the formal success criteria and mobile-specific interpretations documented by the W3C.

\subsection{Cross-screen accessibility implementation analysis}
\label{subsec:cross-screen-analysis}

This section provides a consolidated analysis of accessibility implementation across the various screens of \textit{AccessibleHub}, highlighting key patterns, implementation overhead, and compliance levels across all of the application. This is made in order to help the reader link the complete implementation of the remaining screens present here and the analysis made in detail in thesis.

\subsubsection{Implementation overhead comparison}

A comparative analysis of accessibility implementation overhead across different screen types reveals patterns related to screen purpose and interaction complexity. Table~\ref{tab:consolidated_implementation_overhead} presents a comprehensive overview.

\begin{longtable}[c]{|C{3.5cm}|C{2cm}|C{2.5cm}|C{2.5cm}|C{4cm}|}
\caption{Consolidated accessibility implementation overhead across screen types}
\label{tab:consolidated_implementation_overhead}\\
\hline
\textbf{Screen Type} & \textbf{Lines of Code} & \textbf{Percentage Overhead} & \textbf{Complexity Impact} & \textbf{Primary Contributors} \\
\hline
\endfirsthead
\multicolumn{5}{c}%
{{\bfseries Table \thetable\ -- continued from previous page}} \\
\hline
\textbf{Screen Type} & \textbf{Lines of Code} & \textbf{Percentage Overhead} & \textbf{Complexity Impact} & \textbf{Primary Contributors} \\
\hline
\endhead
\hline
\multicolumn{5}{r}{{Continued on next page}} \\
\endfoot
\hline
\endlastfoot
Components Overview & 206 & 36.3\% & Medium-High & Breadcrumb Navigation, Drawer Accessibility \\
\hline
Buttons & 60 & 13.3\% & Low & Semantic Roles, Descriptive Labels \\
\hline
Forms & 153 & 21.5\% & Medium & State Management, Error Identification \\
\hline
Dialogs & 94 & 16.2\% & Medium & Focus Management, Modal Context \\
\hline
Media & 68 & 12.7\% & Low & Alternative Text, Navigation Controls \\
\hline
Advanced Components & 183 & 22.7\% & High & Slider Controls, State Announcements \\
\hline
Best Practices & 134 & 24.1\% & Medium & Element Hiding, Descriptive Labels \\
\hline
WCAG Guidelines & 48 & 8.7\% & Low & Element Hiding \\
\hline
Gestures Tutorial & 104 & 24.4\% & Medium-High & Adaptive Logic, Accessibility Actions \\
\hline
Logical Navigation & 72 & 18.3\% & Medium & Focus Management, Skip Links \\
\hline
Tools & 112 & 19.2\% & Medium & Element Hiding, Expandable Content \\
\hline
Instruction \& Community & 156 & 20.2\% & Medium & Descriptive Labels, Collapsible Content \\
\hline
Settings & 102 & 18.0\% & Medium & Dynamic Styling, Element Hiding \\
\hline
\end{longtable}

Several important patterns emerge from this analysis:

\begin{enumerate}
    \item \textbf{Content complexity correlation}: Predominantly informational screens (WCAG Guidelines) have the lowest overhead (8.7\%), while navigational hubs (Components Overview) have significantly higher overhead (36.3\%);
    
    \item \textbf{Interaction complexity impact}: Screens with complex interactions (Advanced Components, Gestures Tutorial) require substantially more accessibility code than simpler interaction models (Buttons, Media);
    
    \item \textbf{Focus management burden}: Screens requiring explicit focus management (Dialogs, Logical Navigation) show medium implementation overhead even with relatively simple content;
    
    \item \textbf{Element hiding consistency}: Almost all screens require significant element hiding implementation to reduce "garbage interactions" for screen reader users.
\end{enumerate}

The average accessibility implementation overhead across all screen types is 19.6\%, with educational screens slightly lower (17.9\%) than component demonstration screens (21.3\%). This quantification helps development teams plan appropriate resources for accessibility implementation in mobile applications.

\subsubsection{WCAG compliance comparison}

Table~\ref{tab:consolidated_wcag_compliance} presents a comparative analysis of WCAG 2.2 compliance across screen types, organized by the four core principles.

\begin{longtable}[c]{|C{3.0cm}|C{2.5cm}|C{2.5cm}|C{2.5cm}|C{2.5cm}|C{1.5cm}|}
\caption{WCAG compliance percentage by principle across screen types}
\label{tab:consolidated_wcag_compliance}\\
\hline
\textbf{Screen Type} & \textbf{Perceivable} & \textbf{Operable} & \textbf{Understandable} & \textbf{Robust} & \textbf{Overall} \\
\hline
\endfirsthead
\multicolumn{6}{c}%
{{\bfseries Table \thetable\ -- continued from previous page}} \\
\hline
\textbf{Screen Type} & \textbf{Perceivable} & \textbf{Operable} & \textbf{Understandable} & \textbf{Robust} & \textbf{Overall} \\
\hline
\endhead
\hline
\multicolumn{6}{r}{{Continued on next page}} \\
\endfoot
\hline
\endlastfoot
Components Overview & 92.8\% & 100\% & 100\% & 100\% & 96.8\% \\
\hline
Component Screens (Avg) & 87.5\% & 88.3\% & 76.7\% & 100\% & 85.2\% \\
\hline
Best Practices & 92\% & 88\% & 80\% & 100\% & 88.8\% \\
\hline
Best Practices Screens (Avg) & 84.6\% & 82.4\% & 72.5\% & 100\% & 82.3\% \\
\hline
Tools & 85.7\% & 82.4\% & 75\% & 100\% & 83.6\% \\
\hline
Instruction \& Community & 78.6\% & 76.5\% & 70\% & 100\% & 78.6\% \\
\hline
Settings & 100\% & 88\% & 100\% & 100\% & 96.5\% \\
\hline
\textbf{Average} & \textbf{88.7\%} & \textbf{86.5\%} & \textbf{82.0\%} & \textbf{100\%} & \textbf{87.4\%} \\
\hline
\end{longtable}

This analysis reveals several key insights:

\begin{enumerate}
    \item \textbf{Robust principle universality}: All screens achieve 100\% compliance with the Robust principle, reflecting the consistent implementation of proper semantic roles and values;
    
    \item \textbf{Understandable principle challenges}: The Understandable principle consistently shows the lowest compliance percentages, particularly in screens with complex interaction patterns;
    
    \item \textbf{Settings screen excellence}: The Settings screen achieves the highest overall compliance (96.5\%), reflecting its critical role in providing accessibility customization options;
    
    \item \textbf{Component overview effectiveness}: The main Components screen achieves high compliance (96.8\%), demonstrating that navigational hubs can effectively implement accessibility principles despite complex structure.
\end{enumerate}

The overall WCAG compliance rate of 87.4\% across all screens demonstrates substantial accessibility achievement, particularly considering the implementation of numerous AAA success criteria that exceed minimum requirements.

\subsubsection{Implementation patterns across screen types}

Table~\ref{tab:implementation_patterns} compares the frequency and complexity of common accessibility implementation patterns across different screen categories.

\begin{longtable}[c]{|C{3.2cm}|C{2.8cm}|C{2.8cm}|C{2.8cm}|C{2.8cm}|}
\caption{Accessibility implementation patterns across screen categories}
\label{tab:implementation_patterns}\\
\hline
\textbf{Implementation Pattern} & \textbf{Component Screens} & \textbf{Best Practices Screens} & \textbf{Tool \& Community Screens} & \textbf{Settings Screen} \\
\hline
\endfirsthead
\multicolumn{5}{c}%
{{\bfseries Table \thetable\ -- continued from previous page}} \\
\hline
\textbf{Implementation Pattern} & \textbf{Component Screens} & \textbf{Best Practices Screens} & \textbf{Tool \& Community Screens} & \textbf{Settings Screen} \\
\hline
\endhead
\hline
\multicolumn{5}{r}{{Continued on next page}} \\
\endfoot
\hline
\endlastfoot
Semantic Role Assignment & High (5/5) & High (5/5) & High (5/5) & High (5/5) \\
\hline
Comprehensive Labeling & High (5/5) & High (5/5) & High (5/5) & High (5/5) \\
\hline
Element Hiding & High (5/5) & High (5/5) & High (5/5) & High (5/5) \\
\hline
Focus Management & Medium (3/5) & Medium (3/5) & Low (2/5) & Low (1/5) \\
\hline
State Communication & High (5/5) & Low (2/5) & Medium (3/5) & High (5/5) \\
\hline
Status Announcements & High (5/5) & Medium (3/5) & Medium (3/5) & High (5/5) \\
\hline
Alternative Interactions & Medium (3/5) & Low (2/5) & Low (1/5) & Low (1/5) \\
\hline
Screen Reader Adaptation & Low (2/5) & High (4/5) & Low (2/5) & Medium (3/5) \\
\hline
Expandable Content & Low (1/5) & Low (2/5) & High (5/5) & Low (1/5) \\
\hline
Touch Target Optimization & High (5/5) & Medium (3/5) & Medium (3/5) & High (5/5) \\
\hline
\end{longtable}

The rating indicates implementation frequency within the category, with High (5/5) meaning the pattern appears in all screens of that category, Medium (3/5) meaning it appears in most screens, and Low (1-2/5) meaning it appears in few screens.

This analysis highlights several key implementation patterns:

\begin{enumerate}
    \item \textbf{Universal patterns}: Three patterns (Semantic Role Assignment, Comprehensive Labeling, and Element Hiding) appear universally across all screen types, forming the foundation of accessibility implementation;
    
    \item \textbf{Component-specific patterns}: Focus Management and Alternative Interactions are more prominent in Component screens, reflecting their interactive nature;
    
    \item \textbf{Educational screen specialization}: Best Practices screens emphasize Screen Reader Adaptation, aligning with their educational purpose;
    
    \item \textbf{Tool screen uniqueness}: Expandable Content patterns are heavily emphasized in Tool and Community screens, reflecting their information-dense nature requiring progressive disclosure.
\end{enumerate}

The variations in implementation patterns across screen categories demonstrate the importance of adapting accessibility approaches to the specific purpose and interaction model of each screen type.

\subsubsection{Key framework implementation differences}

Table~\ref{tab:framework_difference_summary} presents a condensed overview of key accessibility implementation differences between React Native and Flutter, highlighting structural approaches and syntax variations.

\begin{longtable}[c]{|C{3cm}|C{5.5cm}|C{5.5cm}|}
\caption{Key accessibility implementation differences between frameworks}
\label{tab:framework_difference_summary}\\
\hline
\textbf{Feature} & \textbf{React Native Pattern} & \textbf{Flutter Pattern} \\
\hline
\endfirsthead
\multicolumn{3}{c}%
{{\bfseries Table \thetable\ -- continued from previous page}} \\
\hline
\textbf{Feature} & \textbf{React Native Pattern} & \textbf{Flutter Pattern} \\
\hline
\endhead
\hline
\multicolumn{3}{r}{{Continued on next page}} \\
\endfoot
\hline
\endlastfoot
Implementation Approach & Property-based: Accessibility properties added to existing components & Widget-based: \texttt{Semantics} widget wraps existing widgets \\
\hline
Role Assignment & String-based: \texttt{accessibilityRole="button"} & Boolean flags: \texttt{Semantics(button: true)} \\
\hline
Focus Management & Direct API: \texttt{AccessibilityInfo.\ setAccessibilityFocus \ (reactTag)} & Widget-based: \texttt{Focus} widget with \texttt{FocusNode} \\
\hline
Element Hiding & Multiple options: \texttt{accessibilityElements \ Hidden}, \texttt{importantForAccessibility} & Single widget: \texttt{ExcludeSemantics} \\
\hline
Code Organization & Properties integrated into component JSX & Explicit wrapper widgets creating deeper nesting \\
\hline
Implementation Overhead & Lower: Averaging 19.6\% LOC increase & Higher: Averaging 24.3\% LOC increase \\
\hline
Testing Approach & Manual testing focus with limited built-in tools & Robust testing framework with Semantics testing tools \\
\hline
\end{longtable}

Overall analysis indicates that while both frameworks provide comprehensive accessibility support, React Native typically requires less code overhead for basic accessibility features, while Flutter offers more robust built-in testing capabilities. The choice between frameworks for accessibility-focused development should consider these tradeoffs alongside other project requirements.

Detailed implementation code comparisons for specific components are available in the extended technical appendix referenced at the beginning of this section.

\subsubsection{Implementation insights across screen types}

Based on the comprehensive analysis of all screens, several key implementation insights emerge that can guide developers implementing accessibility in mobile applications:

\begin{enumerate}
    \item \textbf{Implementation complexity correlates with interaction complexity}: More complex interaction patterns require proportionally more sophisticated accessibility implementations, with state-heavy components like forms and advanced controls requiring the highest implementation overhead;
    
    \item \textbf{Focus management is critical for non-linear interactions}: Components that create new interaction contexts (dialogs, modals) or complex navigation patterns (tabs, multi-step processes) require explicit focus management to maintain user orientation;
    
    \item \textbf{Alternative interaction mechanisms are essential for inherently visual controls}: Components with primarily visual interaction models (sliders, drag interfaces) require additional interaction mechanisms to ensure operability by screen reader users;
    
    \item \textbf{Explicit state communication significantly improves usability}: All interactive components benefit from explicit state communication via \texttt{accessibilityState} and announcements, with the greatest impact on selection-based controls (toggles, checkboxes, radio buttons);
    
    \item \textbf{Element hiding optimization dramatically improves screen reader efficiency}: Proper implementation of element hiding for decorative and redundant content can reduce screen reader navigation time by 40-60\%, representing one of the highest impact-to-effort ratios in accessibility implementation;
    
    \item \textbf{Platform-specific adaptations remain necessary}: Despite cross-platform frameworks, some components (especially date pickers, custom inputs, and gesture handlers) benefit from platform-specific adaptations to leverage native accessibility features.
\end{enumerate}

For detailed screen-by-screen analysis, including comprehensive code samples, developers should refer to the extended technical appendix available at \href{https://github.com/gabrielrovesti/AccessibleHub/blob/main/Technical\%20Thesis\%20Appendix/AccessibleHub\%20-\%20Extended\%20screen\%20analysis.pdf}{AccessibleHub Extended Screen Analysis}.

\subsection{Accessible components section}
\label{subsec:accessible-components}

This section provides a formal analysis of the various screens within the Accessible Components section of \textit{AccessibleHub}. As the core educational element of the application, these screens demonstrate practical implementation patterns for accessibility across commonly used mobile interface elements. 

\subsubsection{Common implementation patterns}
\label{subsubsec:common-patterns}

Across all component screens in this section, several foundational accessibility implementation patterns are consistently applied:

\begin{enumerate}
    \item \textbf{Semantic role assignment}: All components use appropriate \texttt{accessibilityRole} properties to identify their purpose to assistive technologies;
    
    \item \textbf{Comprehensive labeling}: Components use \texttt{accessibilityLabel} to provide both identification and action context and, when in need of additional information to feedforward user action, the usage of \texttt{accessibilityHint};
    
    \item \textbf{Explicit state communication}: Interactive components use \texttt{accessibilityState} to communicate selection, completion, or disabled states;
    
    \item \textbf{Decorative element hiding}: Non-essential visual elements use \\\texttt{accessibilityElementsHidden} to streamline screen reader navigation;
    
    \item \textbf{Status announcements}: State changes are explicitly announced via \\\texttt{AccessibilityInfo.announceForAccessibility};
    
    \item \textbf{Enhanced touch targets}: All interactive elements maintain minimum dimensions of 44×44dp, exceeding WCAG 2.5.8 requirements.
\end{enumerate}

Each component screen also implements a consistent visual structure that reinforces the educational purpose:

\begin{itemize}
    \item A demonstration area with interactive examples;
    \item A code example section with syntax-highlighted implementation;
    \item A features section highlighting key accessibility properties;
    \item A platform considerations section addressing iOS and Android differences.
\end{itemize}

\subsubsection{Component implementation comparative analysis}
\label{subsec:comparative-analysis}

Analyzing accessibility implementations across different component types reveals important patterns in implementation complexity, WCAG compliance, and platform-specific adaptations.

\paragraph{WCAG criteria implementation}

Table~\ref{tab:comparative_wcag_implementation_summary} compares WCAG 2.2 success criteria implementation across component types, including conformance levels.

\begin{table}[ht]
\caption{WCAG criteria implementation by component type}
\label{tab:comparative_wcag_implementation_summary}
\centering
\begin{tabular}[c]{|C{3.5cm}|c|c|c|c|c|}
\hline
\textbf{WCAG Success Criteria} & \textbf{Buttons} & \textbf{Forms} & \textbf{Dialogs} & \textbf{Media} & \textbf{Advanced} \\
\hline
1.1.1 Non-text Content (A) & {\color{green}\ding{51}} & {\color{green}\ding{51}} & {\color{green}\ding{51}} & {\color{green}\ding{51}} & {\color{green}\ding{51}} \\
\hline
1.3.1 Info and Relationships (A) & {\color{green}\ding{51}} & {\color{green}\ding{51}} & {\color{green}\ding{51}} & {\color{green}\ding{51}} & {\color{green}\ding{51}} \\
\hline
1.4.3 Contrast (AA) & {\color{blue}\ding{51}} & {\color{blue}\ding{51}} & {\color{blue}\ding{51}} & {\color{blue}\ding{51}} & {\color{blue}\ding{51}} \\
\hline
2.4.3 Focus Order (A) & {\color{red}\ding{55}} & {\color{green}\ding{51}} & {\color{green}\ding{51}} & {\color{red}\ding{55}} & {\color{green}\ding{51}} \\
\hline
2.4.6 Headings (AA) & {\color{blue}\ding{51}} & {\color{blue}\ding{51}} & {\color{blue}\ding{51}} & {\color{blue}\ding{51}} & {\color{blue}\ding{51}} \\
\hline
2.5.5 Target Size (Enhanced) (AAA) & {\color{purple}\ding{51}} & {\color{purple}\ding{51}} & {\color{purple}\ding{51}} & {\color{purple}\ding{55}} & {\color{purple}\ding{51}} \\
\hline
2.5.8 Target Size (AA) & {\color{blue}\ding{51}} & {\color{blue}\ding{51}} & {\color{blue}\ding{51}} & {\color{blue}\ding{51}} & {\color{blue}\ding{51}} \\
\hline
3.2.5 Change on Request (AAA) & {\color{purple}\ding{55}} & {\color{purple}\ding{51}} & {\color{purple}\ding{51}} & {\color{purple}\ding{51}} & {\color{purple}\ding{51}} \\
\hline
3.3.1 Error Identification (A) & {\color{red}\ding{55}} & {\color{green}\ding{51}} & {\color{red}\ding{55}} & {\color{red}\ding{55}} & {\color{red}\ding{55}} \\
\hline
3.3.5 Help (AAA) & {\color{purple}\ding{55}} & {\color{purple}\ding{51}} & {\color{purple}\ding{55}} & {\color{purple}\ding{55}} & {\color{purple}\ding{55}} \\
\hline
3.3.6 Error Prevention (AAA) & {\color{purple}\ding{55}} & {\color{purple}\ding{51}} & {\color{purple}\ding{55}} & {\color{purple}\ding{55}} & {\color{purple}\ding{55}} \\
\hline
4.1.2 Name, Role, Value (A) & {\color{green}\ding{51}} & {\color{green}\ding{51}} & {\color{green}\ding{51}} & {\color{green}\ding{51}} & {\color{green}\ding{51}} \\
\hline
4.1.3 Status Messages (AA) & {\color{blue}\ding{51}} & {\color{blue}\ding{51}} & {\color{blue}\ding{51}} & {\color{blue}\ding{51}} & {\color{blue}\ding{51}} \\
\hline
\textbf{Total A/AA Implementation} & \textbf{7/9} & \textbf{9/9} & \textbf{8/9} & \textbf{7/9} & \textbf{8/9} \\
\hline
\textbf{Total AAA Implementation} & \textbf{1/3} & \textbf{3/3} & \textbf{2/3} & \textbf{1/3} & \textbf{2/3} \\
\hline
\end{tabular}
\end{table}
\FloatBarrier

\begin{table}[ht]
\caption{Legend for WCAG criteria implementation colors}
\label{tab:wcag_legend}
\centering
\begin{tabular}{|C{3cm}|C{8cm}|}
\hline
\textbf{Color} & \textbf{Meaning} \\
\hline
{\color{green}\ding{51}} & A-level criteria implemented \\
\hline
{\color{blue}\ding{51}} & AA-level criteria implemented \\
\hline
{\color{purple}\ding{51}} & AAA-level criteria implemented \\
\hline
{\color{red}\ding{55}} & Criteria not implemented \\
\hline
\end{tabular}
\end{table}
\FloatBarrier
\FloatBarrier

This analysis reveals several key patterns:

\begin{enumerate}
    \item \textbf{Universal criteria}: Three criteria (1.1.1 Non-text Content, 1.3.1 Info and Relationships, and 4.1.2 Name, Role, Value) are implemented across all component types, forming the core of mobile accessibility requirements;
    
    \item \textbf{Component-specific criteria}: Some criteria are relevant only to specific component types, such as 3.3.1 Error Identification for forms;
    
    \item \textbf{Interaction complexity correlation}: More complex interaction patterns (Forms, Dialogs, Advanced) implement more criteria, particularly those related to focus management and state communication;
    
    \item \textbf{AAA criteria implementation}: The forms screen achieves the highest level of AAA criteria implementation, with complete coverage of applicable AAA criteria (2.5.5 Target Size Enhanced, 3.2.5 Change on Request, 3.3.5 Help, and 3.3.6 Error Prevention). The provision of contextual help through \texttt{accessibilityHint} contributes to meeting 3.3.5, while validation with clear error prevention mechanisms addresses 3.3.6.
\end{enumerate}

\paragraph{Implementation overhead comparison}

Table~\ref{tab:comparative_overhead} compares the implementation overhead across component types.

\begin{table}[ht]
\caption{Accessibility implementation overhead by component type}
\label{tab:comparative_overhead}
\centering
\begin{tabular}[c]{|C{2.5cm}|C{2.5cm}|C{2.5cm}|C{3cm}|C{2.5cm}|}
\hline
\textbf{Component Type} & \textbf{Lines of Code} & \textbf{Percentage Overhead} & \textbf{Complexity Impact} & \textbf{Primary Contributors} \\
\hline
Buttons & 60 & 13.3\% & Low & Labels, Roles \\
\hline
Forms & 153 & 21.5\% & Medium & State, Labels, Errors \\
\hline
Dialogs & 94 & 16.2\% & Medium & Focus Management \\
\hline
Media & 68 & 12.7\% & Low & Alt Text, Controls \\
\hline
Advanced & 183 & 22.7\% & High & Slider Controls, Announcements \\
\hline
\end{tabular}
\end{table}
\FloatBarrier

This comparison reveals a direct correlation between interaction complexity and accessibility implementation overhead. Simple components like buttons and media have the lowest overhead (12-13\%), while complex components with state management and alternative interaction patterns have significantly higher overhead (21-23\%).

\paragraph{Key implementation differences across component types}

Each component type presents unique accessibility challenges requiring specialized implementation approaches:

\begin{enumerate}
    \item \textbf{Forms}: Require explicit error identification and validation feedback using \\ \texttt{accessibilityRole="alert"} to ensure compliance with WCAG 3.3.1 (Error Identification). They also implement complex state communication for selection controls like radio buttons and checkboxes via \texttt{accessibilityState=\{\{checked: selected\}\}}. The form screen addresses multiple AAA criteria through contextual help (\texttt{accessibilityHint}), error prevention through validation, and ensuring all changes happen on user request;
    
    \item \textbf{Dialogs}: Focus management represents the critical accessibility challenge, requiring explicit tracking of focus position and restoration when the dialog closes to comply with WCAG 2.4.3 (Focus Order). The implementation of \texttt{accessibilityViewIsModal=true} and proper focus control addresses AAA criterion 3.2.5 (Change on Request);
    
    \item \textbf{Media}: Alternative text implementation forms the core accessibility requirement, with proper \texttt{accessibilityLabel} values describing non-text content as per WCAG 1.1.1. The current implementation might benefit from additional enhancements to meet AAA criterion 2.5.5 (Target Size Enhanced) for media controls;
    
    \item \textbf{Advanced components}: Require the most sophisticated implementations, particularly for inherently visual controls like sliders, which implement alternative interaction mechanisms (buttons, presets) for screen reader users. These alternative controls address AAA criterion 2.5.5 (Target Size Enhanced) and 3.2.5 (Change on Request).
\end{enumerate}

\paragraph{Screen reader compatibility patterns}

Empirical testing with VoiceOver (iOS) and TalkBack (Android) reveals consistent patterns across component types:

\begin{enumerate}
    \item Both screen readers correctly identify components with properly assigned \\ \texttt{accessibilityRole} values;
    
    \item State changes communicated via \texttt{accessibilityState} are properly announced;
    
    \item Status messages delivered via \texttt{AccessibilityInfo.announceForAccessibility} are consistently reported to users;
    
    \item Focus management implementation in dialogs works reliably on both platforms, with some minor timing differences;
    
    \item Elements hidden with \texttt{accessibilityElementsHidden} are consistently excluded from the accessibility tree on both platforms.
\end{enumerate}

These findings confirm that the accessibility implementation patterns used throughout the component screens provide consistent and reliable behavior across both major mobile platforms when proper accessibility properties are applied.

\subsubsection{Buttons and touchables screen}
\label{subsubsec:buttons-touchables}

The Buttons and Touchables screen demonstrates fundamental accessibility implementations for the most common interactive elements in mobile applications. It provides implementation examples for accessible touch targets with proper sizing, meaningful labels, and appropriate feedback mechanisms. Figure~\ref{fig:button_screens_sidebyside} shows the main interface of this screen.

\begin{figure}[ht]
    \centering
    \begin{subfigure}[b]{0.48\textwidth}
        \centering
        \includegraphics[width=\linewidth, alt={First part of the Buttons and touchables screen}]{img/button1.png}
        \caption{Button screen - Part 1}
        \label{fig:button-left}
    \end{subfigure}
    \hfill
    \begin{subfigure}[b]{0.48\textwidth}
        \centering
        \includegraphics[width=\linewidth, alt={Second part of the Buttons and touchables screen}]{img/button2.png}
        \caption{Button screen - Part 2}
        \label{fig:button-right}
    \end{subfigure}
    \caption{Side-by-side view of the two Button and Touchables screen parts}
    \label{fig:button_screens_sidebyside}
\end{figure}
\FloatBarrier

\paragraph{Component inventory and WCAG/MCAG/WCAG2Mobile mapping}

Table~\ref{tab:buttons_component_mapping} provides a formal mapping between the UI components, their semantic roles, the specific WCAG 2.2 and MCAG criteria they address, WCAG2Mobile considerations, and their React Native implementation properties.

\begin{longtable}[c]{|C{2.5cm}|C{2cm}|C{2.8cm}|C{2.8cm}|C{4.7cm}|}
\caption{Buttons screen component-criteria mapping with WCAG2Mobile considerations}
\label{tab:buttons_component_mapping}\\
\hline
\textbf{Component and Location} & \textbf{Semantic Role} & \textbf{WCAG 2.2 Criteria} & \textbf{WCAG2Mobile Considerations} & \textbf{Implementation Properties} \\
\hline
\endfirsthead
\multicolumn{5}{c}%
{{\bfseries Table \thetable\ -- continued from previous page}} \\
\hline
\textbf{Component} & \textbf{Semantic Role} & \textbf{WCAG 2.2 Criteria} & \textbf{WCAG2Mobile Considerations} & \textbf{Implementation Properties} \\
\hline
\endhead
\hline
\multicolumn{5}{r}{{Continued on next page}} \\
\endfoot
\hline
\endlastfoot
Hero Title (top: "Interactive Example") & heading & 1.4.3 Contrast (AA)\newline 2.4.6 Headings (AA) & Mobile-specific view context; Screen title differentiation & \texttt{accessibility \ Role="header"} \\
\hline
Demo Button (blue "Submit" button in demo section) & button & 1.4.3 Contrast (AA)\newline 2.5.8 Target Size (AA)\newline 2.5.5 Target Size (Enhanced) (AAA)\newline 4.1.2 Name, Role, Value (A) & Touch target optimization; Haptic feedback; Platform-specific button behavior & \texttt{accessibility \ Role="button"},\newline \texttt{accessibility \ Label="Submit form"},\newline \texttt{minHeight: 44} \\
\hline
Code Snippet (dark code blocks showing implementation) & text & 1.3.1 Info and Relationships (A) & Content structure preservation in mobile context & \texttt{accessibility \ Role="text"},\newline \texttt{accessibility \ Label= "Button implementation code"} \\
\hline
Copy Button (top right of code blocks) & button & 1.4.3 Contrast (AA)\newline 4.1.3 Status Messages (AA) & Touch feedback for mobile interactions; Mobile platform notifications & \texttt{accessibility \ Role="button"},\newline \texttt{accessibility \ Label="\{copied ? "Code copied" : "Copy code example"\}"} \\
\hline
Success Modal (popup after submitting button) & alertdialog & 4.1.3 Status Messages (AA) & Mobile-specific notification pattern; Mobile platform focus management & \texttt{accessibility \ ViewIsModal},\newline \texttt{accessibility \ LiveRegion="polite"} \\
\hline
Feature Cards (bottom: Minimum Touch Target, etc.) & none & 1.3.1 Info and Relationships (A) & Mobile-specific grouping for screen context & \texttt{accessibility \ Role="text"} \\
\hline
Feature Icons (icons within feature cards) & none & 1.1.1 Non-text Content (A) & Reduction of unnecessary focus stops to optimize mobile navigation & \texttt{accessibility \ Elements \ Hidden=true},\newline \texttt{importantFor \ Accessibility="no-hide \ -descendants"} \\
\end{longtable}
\FloatBarrier

\paragraph{Technical implementation analysis}

The Buttons and Touchables screen exemplifies proper accessibility implementation for interactive elements. The core demo button showcases three fundamental accessibility considerations: proper role assignment, descriptive labeling, and sufficient touch target size. Listing~\ref{lst:buttons-accessibility} highlights the key implementation aspects.

\begin{lstlisting}[
  style=ReactNativeStyle,
  caption={Key implementation for accessible button component},
  label={lst:buttons-accessibility},
  basicstyle=\ttfamily\footnotesize,
  numbers=left,
]
<TouchableOpacity
  style={[styles.demoButton, { backgroundColor: colors.primary }]}
  accessibilityRole="button"
  accessibilityLabel="Submit form"
  onPress={() => {
    setShowSuccess(true);
    AccessibilityInfo.announceForAccessibility('Button pressed successfully');
    setTimeout(() => setShowSuccess(false), 2000);
  }}
>
  <Text style={[styles.buttonText, {
    color: '#FFFFFF'
  }]}>
    Submit
  </Text>
</TouchableOpacity>
\end{lstlisting}
\FloatBarrier

Several key accessibility considerations are implemented in this example:

\begin{enumerate}
    \item \textbf{Proper semantic role}: The implementation explicitly assigns the button role using \texttt{accessibilityRole="button"}, ensuring screen readers correctly identify the component's purpose;
    
    \item \textbf{Descriptive accessibility labels}: The button includes an \texttt{accessibilityLabel} that identifies its function, while \texttt{accessibilityHint} provides additional context about the outcome of interaction, offering comprehensive context for screen reader users;
    
    \item \textbf{Adequate touch target size}: The button implements the enhanced touch target size recommendation from WCAG 2.5.8 (Target Size) by using a minimum height of 44px, and approaches the WCAG 2.5.5 (Target Size Enhanced) AAA criterion which recommends 44×44 pixels;
    
    \item \textbf{Status feedback}: When pressed, the button announces its state change via \\\texttt{AccessibilityInfo.announceForAccessibility}, proactively notifying screen reader users of the action result;
    
    \item \textbf{Visual feedback}: The success modal provides visual confirmation of the button press, with appropriate \texttt{accessibilityLiveRegion="polite"} to ensure screen readers announce the status change.
\end{enumerate}

\paragraph{Screen reader support analysis}

Table~\ref{tab:buttons_screen_reader_analysis} presents results from systematic testing of the Buttons screen with screen readers on both iOS and Android platforms, with specific attention to WCAG2Mobile's guidance on platform-specific accessibility services.

\begin{longtable}[c]{|C{2.8cm}|C{3.5cm}|C{3.5cm}|C{4cm}|}
\caption{Buttons screen screen reader testing results with WCAG2Mobile considerations}
\label{tab:buttons_screen_reader_analysis}\\
\hline
\textbf{Test Case} & \textbf{VoiceOver (iOS)} & \textbf{TalkBack (Android)} & \textbf{WCAG2Mobile Considerations} \\
\hline
\endfirsthead
\multicolumn{4}{c}%
{{\bfseries Table \thetable\ -- continued from previous page}} \\
\hline
\textbf{Test Case} & \textbf{VoiceOver (iOS)} & \textbf{TalkBack (Android)} & \textbf{WCAG2Mobile Considerations} \\
\hline
\endhead
\hline
\multicolumn{4}{r}{{Continued on next page}} \\
\endfoot
\hline
\endlastfoot
Hero Title & {\color{green}\ding{51}} Announces ``Buttons \& Touchables - Interactive Example, heading'' & {\color{green}\ding{51}} Announces ``Buttons \& Touchables - Interactive Example, heading'' & SC 2.4.6 interpreted for screen contexts rather than web pages \\
\hline
Demo Button & {\color{green}\ding{51}} Announces ``Submit form, activates form submission, button'' & {\color{green}\ding{51}} Announces ``Submit form, button'' with hint & SC 4.1.2 implementation accounts for platform-specific announcement patterns \\
\hline
Button Press & {\color{green}\ding{51}} Announces ``Button pressed successfully'' & {\color{green}\ding{51}} Announces ``Button pressed successfully'' & SC 4.1.3 implementation for mobile notification context \\
\hline
Success Modal & {\color{green}\ding{51}} Automatically announces content & {\color{green}\ding{51}} Automatically announces content & Live region implementation aligned with WCAG2Mobile's approach to status messages \\
\hline
Feature Cards & {\color{green}\ding{51}} Announces content with proper grouping & {\color{green}\ding{51}} Announces content with proper grouping & SC 1.3.1 implementation for mobile content structure \\
\hline
Feature Icons & {\color{green}\ding{51}} Skipped in navigation & {\color{green}\ding{51}} Skipped in navigation & Element hiding optimizes navigation sequence in mobile context \\
\hline
Copy Button State & {\color{green}\ding{51}} Announces state change & {\color{green}\ding{51}} Announces state change & SC 4.1.3 implementation for mobile button state changes \\
\end{longtable}
\FloatBarrier

The implementation addresses several WCAG2Mobile-specific considerations:

\begin{enumerate}
    \item \textbf{Screen-based role assignment}: The implementation uses semantic roles appropriate for screen contexts rather than web pages, following WCAG2Mobile's guidance on interpreting web terminology in a mobile context;
    
    \item \textbf{Platform-specific behavior}: The button implementation accounts for differences in how VoiceOver and TalkBack handle hints and labels, aligning with WCAG2Mobile's recognition of platform-specific accessibility services;
    
    \item \textbf{Swipe optimization}: Decorative elements are properly hidden from the accessibility tree, implementing WCAG2Mobile's emphasis on reducing unnecessary focus stops for more efficient touch-based screen reader navigation;
    
    \item \textbf{Mobile-specific status messages}: The implementation uses \texttt{accessibilityLiveRegion} and announces state changes explicitly, following WCAG2Mobile's guidance on providing clear feedback in mobile contexts where visual cues might be less prominent.
\end{enumerate}

\paragraph{Implementation overhead analysis}

Table~\ref{tab:buttons_implementation_overhead} quantifies the additional code required to implement accessibility features in the Buttons and touchables screen.

\begin{longtable}[c]{|C{3.8cm}|C{2.3cm}|C{2.8cm}|C{2.8cm}|}
\caption{Buttons screen accessibility implementation overhead}
\label{tab:buttons_implementation_overhead}\\
\hline
\textbf{Accessibility Feature} & \textbf{Lines of Code} & \textbf{Percentage of Total} & \textbf{Complexity Impact} \\
\hline
\endfirsthead
\multicolumn{4}{c}%
{{\bfseries Table \thetable\ -- continued from previous page}} \\
\hline
\textbf{Accessibility Feature} & \textbf{Lines of Code} & \textbf{Percentage of Total} & \textbf{Complexity Impact} \\
\hline
\endhead
\hline
\multicolumn{4}{r}{{Continued on next page}} \\
\endfoot
\hline
\endlastfoot
Semantic Roles & 10 LOC & 2.2\% & Low \\
\hline
Descriptive Labels & 14 LOC & 3.1\% & Low \\
\hline
Element Hiding & 12 LOC & 2.7\% & Low \\
\hline
Status Announcements & 8 LOC & 1.8\% & Low \\
\hline
Touch Target Sizing & 6 LOC & 1.3\% & Low \\
\hline
Modal Accessibility & 10 LOC & 2.2\% & Medium \\
\hline
\textbf{Total} & \textbf{60 LOC} & \textbf{13.3\%} & \textbf{Low} \\
\end{longtable}
\FloatBarrier

This analysis reveals that implementing comprehensive button accessibility features adds approximately 13.3\% to the code base, representing a relatively low overhead for significantly improved user experience. Notably, this overhead is lower than other component types due to the fundamental nature of button components, where accessibility considerations can be more directly integrated with minimal complexity impact.

\subsubsection{Form screen}
\label{subsubsec:forms-screen}

The Form screen demonstrates complex accessibility patterns for capturing user input. Unlike the simpler Buttons screen, form elements present additional challenges related to input association, validation feedback, and state communication. Figure~\ref{fig:form_screens_sidebyside} shows the main interface of this screen.

\begin{figure}[ht]
    \centering
    \begin{subfigure}[b]{0.48\textwidth}
        \centering
        \includegraphics[width=\linewidth, alt={First part of the Form creen}]{img/form1.png}
        \caption{Form screen - Part 1}
        \label{fig:form-left}
    \end{subfigure}
    \hfill
    \begin{subfigure}[b]{0.48\textwidth}
        \centering
        \includegraphics[width=\linewidth, alt={Second part of the Form screen}]{img/form2.png}
        \caption{Form screen - Part 2}
        \label{fig:form-right}
    \end{subfigure}
    \caption{Side-by-side view of the two Form screen parts}
    \label{fig:form_screens_sidebyside}
\end{figure}
\FloatBarrier

\paragraph{Component inventory and WCAG/MCAG/WCAG2Mobile mapping}

Table~\ref{tab:form_component_mapping} provides a comprehensive mapping between the Form screen components, their semantic roles, the WCAG 2.2 criteria they address, WCAG2Mobile considerations, and their implementation properties.

\begin{longtable}[c]{|C{2.5cm}|C{2cm}|C{2.8cm}|C{3.5cm}|C{4.7cm}|}
\caption{Form screen component-criteria mapping with WCAG2Mobile considerations}
\label{tab:form_component_mapping}\\
\hline
\textbf{Component and Location} & \textbf{Semantic Role} & \textbf{WCAG 2.2 Criteria} & \textbf{WCAG2Mobile Considerations} & \textbf{Implementation Properties} \\
\hline
\endfirsthead
\multicolumn{4}{c}%
{{\bfseries Table \thetable\ -- continued from previous page}} \\
\hline
\textbf{Component} & \textbf{Semantic Role} & \textbf{WCAG 2.2 Criteria} & \textbf{WCAG2Mobile Considerations} & \textbf{Implementation Properties} \\
\hline
\endhead
\hline
\multicolumn{5}{r}{{Continued on next page}} \\
\endfoot
\hline
\endlastfoot
Text Input (Name and Email fields) & textbox & 1.3.1 Info and Relationships (A)\newline 3.3.2 Labels or Instructions (A)\newline 3.3.5 Help (AAA) & Mobile input context; Mobile keyboard adaptation & \texttt{accessibility \ Label="Enter your name"} \\
\hline
Radio Buttons (Male/Female gender options) & radio & 1.3.1 Info and Relationships (A)\newline 4.1.2 Name, Role, Value (A) & Mobile touch selection patterns; Platform-specific state behavior & \texttt{accessibility \ Role="radio"},\newline \texttt{accessibility \ State=\{\{checked: selected\}\}} \\
\hline
Radio Group (Gender selection container) & radiogroup & 1.3.1 Info and Relationships (A) & Mobile-specific control grouping & \texttt{accessibility \ Role="radiogroup"} \\
\hline
Checkboxes (Morning/Afternoon/time options) & checkbox & 1.3.1 Info and Relationships (A)\newline 4.1.2 Name, Role, Value (A) & Mobile touch selection patterns & \texttt{accessibility \ Role="checkbox"},\newline \texttt{accessibility \ State=\{\{checked: selected\}\}} \\
\hline
Date Picker (Birth Date field) & button & 3.2.2 On Input (A)\newline 3.3.2 Labels or Instructions (A) & Native mobile date picker integration & \texttt{accessibility \ Role="button"},\newline \texttt{accessibility \ Label="Select birth date"} \\
\hline
Error Messages (feedback when forms incomplete) & alert & 3.3.1 Error Identification (A)\newline 3.3.3 Error Suggestion (AA) & Mobile-specific error notification patterns & \texttt{accessibility \ Role="alert"} \\
\hline
Submit Button (form submission at bottom) & button & 3.3.4 Error Prevention (AA)\newline 2.5.8 Target Size (AA) & Mobile form submission patterns; Mobile touch target sizing & \texttt{accessibility \ Role="button"},\newline \texttt{accessibility \ State=\{\{disabled: !formDataComplete\}\}} \\
\end{longtable}
\FloatBarrier

\paragraph{Key accessibility considerations}

The Form screen addresses several critical accessibility patterns beyond basic labeling, with specific attention to WCAG2Mobile interpretations:

\begin{enumerate}
    \item \textbf{Input association}: Clear association between labels and input fields using semantic grouping, addressing WCAG2Mobile's interpretation of Success Criterion 1.3.1 (Info and Relationships) for mobile form context;
    
    \item \textbf{Error identification}: Proper error messaging with \texttt{accessibilityRole="alert"} for validation feedback, implementing WCAG2Mobile's guidance on Success Criterion 3.3.1 (Error Identification) for mobile notification patterns;
    
    \item \textbf{State communication}: Selection state for radio buttons and checkboxes with \\ \texttt{accessibilityState=\{\{checked: selected\}\}}, aligning with WCAG2Mobile's interpretation of Success Criterion 4.1.2 (Name, Role, Value) for mobile interaction patterns;
    
    \item \textbf{Native picker integration}: Leveraging platform-native date pickers for optimal accessibility, following WCAG2Mobile's emphasis on using platform-specific accessibility services;
    
    \item \textbf{Help information}: Implementation of contextual \texttt{accessibilityHint} values that provide forward-looking information about how to interact with fields, addressing WCAG2Mobile's guidance on providing additional context in mobile environments;
    
    \item \textbf{Error prevention}: Comprehensive validation with clear warnings and confirmations, implementing WCAG2Mobile's interpretation of Success Criterion 3.3.4 (Error Prevention) for mobile form submission.
\end{enumerate}

Listing~\ref{lst:form_implementation} demonstrates the implementation of accessible form controls with proper state management and WCAG2Mobile considerations.

\begin{lstlisting}[
  style=ReactNativeStyle,
  caption={Accessible radio button implementation with state management for mobile context},
  label={lst:form_implementation},
  basicstyle=\ttfamily\footnotesize,
  numbers=left,
]
<View accessibilityRole="radiogroup">
  {['Male', 'Female'].map((option) => (
    <TouchableOpacity
      key={option}
      style={styles.radioItem}
      onPress={() => setFormData((prev) => ({ ...prev, gender: option }))}
      accessibilityRole="radio"
      accessibilityState={{ checked: formData.gender === option }}
      accessibilityLabel={`Select ${option}`}
    >
      <View
        style={[
          styles.radioButton,
          { borderColor: colors.primary },
          formData.gender === option && { backgroundColor: colors.primary },
        ]}
      />
      <Text style={[styles.radioLabel, { color: colors.text }]}>
        {option}
      </Text>
    </TouchableOpacity>
  ))}
</View>
\end{lstlisting}
\FloatBarrier

\paragraph{Screen reader support analysis}

Table~\ref{tab:form_screen_reader_analysis} presents results from systematic testing of the Form screen with screen readers on both iOS and Android platforms, with specific attention to WCAG2Mobile's guidance on platform-specific accessibility services.

\begin{longtable}[c]{|C{2.8cm}|C{3.5cm}|C{3.5cm}|C{4cm}|}
\caption{Form screen screen reader testing results with WCAG2Mobile considerations}
\label{tab:form_screen_reader_analysis}\\
\hline
\textbf{Test Case} & \textbf{VoiceOver (iOS)} & \textbf{TalkBack (Android)} & \textbf{WCAG2Mobile Considerations} \\
\hline
\endfirsthead
\multicolumn{4}{c}%
{{\bfseries Table \thetable\ -- continued from previous page}} \\
\hline
\textbf{Test Case} & \textbf{VoiceOver (iOS)} & \textbf{TalkBack (Android)} & \textbf{WCAG2Mobile Considerations} \\
\hline
\endhead
\hline
\multicolumn{4}{r}{{Continued on next page}} \\
\endfoot
\hline
\endlastfoot
Text Input & {\color{green}\ding{51}} Announces label and hint; activates keyboard & {\color{green}\ding{51}} Announces label and hint; activates keyboard & SC 3.3.2 implementation for mobile input contexts \\
\hline
Radio Buttons & {\color{green}\ding{51}} Announces option and selected state & {\color{green}\ding{51}} Announces option and selected state & SC 4.1.2 implementation for mobile selection patterns \\
\hline
Date Picker & {\color{green}\ding{51}} Opens native iOS picker & {\color{green}\ding{51}} Opens native Android picker & Platform-specific native control integration \\
\hline
Error Messages & {\color{green}\ding{51}} Automatically announces errors & {\color{green}\ding{51}} Automatically announces errors & SC 3.3.1 implementation for mobile notification patterns \\
\hline
Form Navigation & {\color{green}\ding{51}} Logical navigation between fields & {\color{green}\ding{51}} Logical navigation between fields & SC 2.4.3 implementation for mobile screen context \\
\hline
Submit Button State & {\color{green}\ding{51}} Announces disabled state when appropriate & {\color{green}\ding{51}} Announces disabled state & SC 4.1.2 implementation for mobile button state communication \\
\hline
Success Notification & {\color{green}\ding{51}} Announces success message & {\color{green}\ding{51}} Announces success message & SC 4.1.3 implementation for mobile status notifications \\
\end{longtable}
\FloatBarrier

The implementation addresses several WCAG2Mobile-specific considerations:

\begin{enumerate}
    \item \textbf{Platform-native controls}: The date picker implementation leverages platform-native controls for optimal accessibility, following WCAG2Mobile's emphasis on utilizing platform accessibility services;
    
    \item \textbf{Mobile keyboard adaptation}: Text inputs are properly configured with appropriate keyboard types (email, numeric), implementing WCAG2Mobile's guidance on optimizing input methods for mobile contexts;
    
    \item \textbf{Touch-optimized selection controls}: Radio buttons and checkboxes implement larger touch targets and clear visual states, addressing WCAG2Mobile's interpretation of Success Criterion 2.5.8 (Target Size) for mobile touch interactions;
    
    \item \textbf{Mobile-specific error patterns}: Error messages use \texttt{accessibilityRole="alert"} to ensure proper announcement on mobile screen readers, implementing WCAG2Mobile's guidance on error identification in mobile contexts.
\end{enumerate}

\paragraph{Implementation overhead analysis}

Table~\ref{tab:form_implementation_overhead} quantifies the additional code required to implement accessibility features in the Form screen, with specific attention to WCAG2Mobile considerations.

\begin{longtable}[c]{|C{3.8cm}|C{2.3cm}|C{2.8cm}|C{2.8cm}|}
\caption{Form screen accessibility implementation overhead with WCAG2Mobile considerations}
\label{tab:form_implementation_overhead}\\
\hline
\textbf{Accessibility Feature} & \textbf{Lines of Code} & \textbf{Percentage of Total} & \textbf{Complexity Impact} \\
\hline
\endfirsthead
\multicolumn{4}{c}%
{{\bfseries Table \thetable\ -- continued from previous page}} \\
\hline
\textbf{Accessibility Feature} & \textbf{Lines of Code} & \textbf{Percentage of Total} & \textbf{Complexity Impact} \\
\hline
\endhead
\hline
\multicolumn{4}{r}{{Continued on next page}} \\
\endfoot
\hline
\endlastfoot
Semantic Roles & 28 LOC & 3.9\% & Medium \\
\hline
Descriptive Labels & 35 LOC & 4.9\% & Medium \\
\hline
Element State Management & 32 LOC & 4.5\% & Medium \\
\hline
Error Announcements & 18 LOC & 2.5\% & Medium \\
\hline
Native Picker Integration & 25 LOC & 3.5\% & High \\
\hline
Platform-specific Adaptations & 15 LOC & 2.1\% & Medium \\
\hline
WCAG2Mobile-specific Implementations & 18 LOC & 2.5\% & Medium \\
\hline
\textbf{Total} & \textbf{171 LOC} & \textbf{24.0\%} & \textbf{Medium-High} \\
\end{longtable}
\FloatBarrier

Forms have a high accessibility implementation overhead (24.0\%) among single component types, reflecting the complexity of making multi-part input systems fully accessible on mobile platforms. The integration of WCAG2Mobile considerations adds an additional 2.5\% overhead, primarily focused on platform-specific adaptations and mobile-specific input patterns. The primary contributors to this overhead are state communication mechanisms, validation feedback systems, and native picker integrations that are essential for an accessible mobile form experience.

\subsubsection{Dialog screen}
\label{subsubsec:dialogs-screen}

The Dialog screen addresses one of the most challenging accessibility patterns in mobile applications: modal content that must trap and manage focus while providing clear context and exit mechanisms. WCAG2Mobile provides specific guidance on handling dialogs in a mobile context where screens rather than pages form the interaction unit. Figure~\ref{fig:dialog_screens_sidebyside} shows the main interface of this screen.

\begin{figure}[ht]
    \centering
    \begin{subfigure}[b]{0.48\textwidth}
        \centering
        \includegraphics[width=\linewidth, alt={First part of the Dialog screen}]{img/dialog1.png}
        \caption{Dialog screen - Part 1}
        \label{fig:dialog-left}
    \end{subfigure}
    \hfill
    \begin{subfigure}[b]{0.48\textwidth}
        \centering
        \includegraphics[width=\linewidth, alt={Second part of the Dialog screen}]{img/dialog2.png}
        \caption{Dialog screen - Part 2}
        \label{fig:dialog-right}
    \end{subfigure}
    \caption{Side-by-side view of the two Dialog screen parts}
    \label{fig:dialog_screens_sidebyside}
\end{figure}
\FloatBarrier

\paragraph{Component inventory and WCAG/MCAG/WCAG2Mobile mapping}

Table~\ref{tab:dialog_component_mapping} provides a comprehensive mapping between the Dialog screen components, their semantic roles, the WCAG 2.2 criteria they address, WCAG2Mobile considerations, and their implementation properties.

\begin{longtable}[c]{|C{2.5cm}|C{2cm}|C{2.8cm}|C{2.8cm}|C{4.7cm}|}
\caption{Dialog screen component-criteria mapping with WCAG2Mobile considerations}
\label{tab:dialog_component_mapping}\\
\hline
\textbf{Component and Location} & \textbf{Semantic Role} & \textbf{WCAG 2.2 Criteria} & \textbf{WCAG2Mobile Considerations} & \textbf{Implementation Properties} \\
\hline
\endfirsthead
\multicolumn{4}{c}%
{{\bfseries Table \thetable\ -- continued from previous page}} \\
\hline
\textbf{Component} & \textbf{Semantic Role} & \textbf{WCAG 2.2 Criteria} & \textbf{WCAG2Mobile Considerations} & \textbf{Implementation Properties} \\
\hline
\endhead
\hline
\multicolumn{5}{r}{{Continued on next page}} \\
\endfoot
\hline
\endlastfoot
Dialog Trigger Button (blue "Open Dialog" button) & button & 2.5.8 Target Size (AA)\newline 4.1.2 Name, Role, Value (A) & Mobile touch target optimization; Mobile action labeling & \texttt{accessibility \ Role="button"},\newline \texttt{accessibility \ Label="Open example dialog"},\newline \texttt{minHeight: 44} \\
\hline
Modal Dialog (popup overlay when triggered) & dialog & 2.4.3 Focus Order (A)\newline 4.1.2 Name, Role, Value (A) & Mobile view modal context; Mobile screen reader focus management & \texttt{accessibility \ ViewIsModal=true},\newline \texttt{accessibility \ LiveRegion="polite"} \\
\hline
Dialog Title (header within modal popup) & header & 1.3.1 Info and Relationships (A)\newline 2.4.6 Headings and Labels (AA) & Mobile dialog context structuring & \texttt{accessibility \ Role="header"} \\
\hline
Close Button (X button in modal header) & button & 2.1.2 No Keyboard Trap (A)\newline 2.5.8 Target Size (AA) & Mobile escape mechanism; Touch target sizing & \texttt{accessibility \ Role="button"},\newline \texttt{accessibility \ Label="Close dialog"} \\
\hline
Dialog Actions (action buttons within modal) & group & 1.3.1 Info and Relationships (A) & Mobile action button grouping & \texttt{accessibility \ Role="group"},\newline \texttt{accessibility \ Label="Dialog actions"} \\
\hline
Success Notification (confirmation after dialog actions) & alert & 4.1.3 Status Messages (AA) & Mobile notification pattern & \texttt{accessibility \ Role="alert"} \\
\end{longtable}
\FloatBarrier

\paragraph{Focus management implementation}

The key accessibility challenge for dialogs is proper focus management, particularly in mobile contexts where WCAG2Mobile provides specific guidance on handling focus in screen-based rather than page-based interfaces. Listing~\ref{lst:dialog_implementation} illustrates this implementation.

\begin{lstlisting}[
  style=ReactNativeStyle,
  caption={Dialog implementation with focus management for mobile context},
  label={lst:dialog_implementation},
  basicstyle=\ttfamily\footnotesize,
  numbers=left,
]
// References for focus management
const dialogRef = useRef(null);
const openButtonRef = useRef(null);

// Focus management useEffect hook
useEffect(() => {
  if (showDialog) {
    AccessibilityInfo.announceForAccessibility(
      'Example dialog opened. This dialog contains information about accessibility features.'
    );
    // Brief timeout to ensure dialog is fully rendered
    setTimeout(() => {
      dialogRef.current?.focus();
    }, 100);
  } else {
    // Return focus to open button when dialog closes
    openButtonRef.current?.focus();
  }
}, [showDialog]);
\end{lstlisting}
\FloatBarrier

The dialog implementation addresses several critical accessibility requirements with specific attention to WCAG2Mobile interpretations:

\begin{enumerate}
    \item \textbf{Modal context}: Setting \texttt{accessibilityViewIsModal=true} to establish a focused interaction context, implementing WCAG2Mobile's guidance on modal dialogs in a screen-based mobile context;
    
    \item \textbf{Focus trapping}: Managing focus to prevent interaction with background content, addressing WCAG2Mobile's interpretation of Success Criterion 2.4.3 (Focus Order) for mobile interfaces;
    
    \item \textbf{Return focus}: Explicitly returning focus to the triggering element when the dialog closes, following WCAG2Mobile's guidance on maintaining focus context in mobile interfaces;
    
    \item \textbf{Status announcements}: Using \texttt{AccessibilityInfo.announceForAccessibility} to provide context about dialog opening and closing, implementing WCAG2Mobile's interpretation of Success Criterion 4.1.3 (Status Messages) for mobile screen readers;
    
    \item \textbf{Accessibility escape}: Implementing \texttt{onAccessibilityEscape} to provide a native escape mechanism for screen reader users, addressing WCAG2Mobile's guidance on keyboard trap prevention in mobile contexts.
\end{enumerate}

\paragraph{Screen reader support analysis}

Table~\ref{tab:dialog_screen_reader_analysis} presents results from systematic testing of the Dialog screen with screen readers on both iOS and Android platforms, with specific attention to WCAG2Mobile's guidance on platform-specific accessibility services.

\begin{longtable}[c]{|C{2.8cm}|C{3.5cm}|C{3.5cm}|C{4cm}|}
\caption{Dialog screen screen reader testing results with WCAG2Mobile considerations}
\label{tab:dialog_screen_reader_analysis}\\
\hline
\textbf{Test Case} & \textbf{VoiceOver (iOS)} & \textbf{TalkBack (Android)} & \textbf{WCAG2Mobile Considerations} \\
\hline
\endfirsthead
\multicolumn{4}{c}%
{{\bfseries Table \thetable\ -- continued from previous page}} \\
\hline
\textbf{Test Case} & \textbf{VoiceOver (iOS)} & \textbf{TalkBack (Android)} & \textbf{WCAG2Mobile Considerations} \\
\hline
\endhead
\hline
\multicolumn{4}{r}{{Continued on next page}} \\
\endfoot
\hline
\endlastfoot
Dialog Opening & {\color{green}\ding{51}} Announces dialog opening; focus moves to dialog & {\color{green}\ding{51}} Announces dialog opening; focus moves to dialog & Implementation of mobile-specific dialog announcement pattern \\
\hline
Dialog Focus Trapping & {\color{green}\ding{51}} Focus remains within dialog & {\color{green}\ding{51}} Focus remains within dialog & Mobile implementation of SC 2.1.2 for view contexts \\
\hline
Dialog Title & {\color{green}\ding{51}} Announces as heading & {\color{green}\ding{51}} Announces as heading & Implementation of SC 2.4.6 for mobile dialog context \\
\hline
Dialog Actions & {\color{green}\ding{51}} Announces grouped actions & {\color{green}\ding{51}} Announces grouped actions & Implementation of SC 1.3.1 for mobile button grouping \\
\hline
Escape Gesture & {\color{green}\ding{51}} Responds to escape gesture & {\color{green}\ding{51}} Responds to back button & Platform-specific implementation of SC 2.1.2 \\
\hline
Dialog Closing & {\color{green}\ding{51}} Announces dialog closing; focus returns to trigger & {\color{green}\ding{51}} Announces dialog closing; focus returns to trigger & Mobile implementation of focus management for screen context \\
\hline
Success Notification & {\color{green}\ding{51}} Automatically announces & {\color{green}\ding{51}} Automatically announces & Mobile implementation of SC 4.1.3 for notification patterns \\
\end{longtable}
\FloatBarrier

The implementation addresses several WCAG2Mobile-specific considerations:

\begin{enumerate}
    \item \textbf{Mobile-specific focus management}: The implementation carefully manages focus transitions between screens and dialogs, following WCAG2Mobile's guidance on focus order in mobile contexts;
    
    \item \textbf{Platform-specific escape mechanisms}: The dialog supports both iOS VoiceOver's escape gesture and Android's back button, implementing WCAG2Mobile's emphasis on platform-specific accessibility services;
    
    \item \textbf{Mobile notification patterns}: Status announcements use mobile-specific notification patterns through \texttt{AccessibilityInfo.announceForAccessibility}, addressing WCAG2Mobile's guidance on mobile status message patterns;
    
    \item \textbf{Limited viewport context}: The implementation provides strong contextual cues to compensate for the limited viewport in mobile interfaces, following WCAG2Mobile's recognition of the challenges of screen-based rather than page-based interfaces.
\end{enumerate}

\paragraph{Mobile-specific considerations}

Dialog implementation on mobile platforms presents unique accessibility challenges that WCAG2Mobile specifically addresses:

\begin{itemize}
    \item \textbf{Limited viewport context}: Unlike desktop interfaces, mobile screens cannot show both dialog and background content simultaneously, requiring stronger contextual cues, addressing WCAG2Mobile's interpretation of Success Criterion 2.4.3 (Focus Order) for mobile screens;
    
    \item \textbf{Touch dismissal patterns}: Implementation of touch-friendly dismissal actions with adequate target sizes, following WCAG2Mobile's guidance on Success Criterion 2.5.8 (Target Size) for mobile touch targets;
    
    \item \textbf{Platform convention alignment}: Following platform-specific dialog patterns for consistent user experience, implementing WCAG2Mobile's emphasis on platform-specific accessibility services;
    
    \item \textbf{Mobile escape mechanisms}: Support for platform-specific escape mechanisms (VoiceOver escape gesture, Android back button), addressing WCAG2Mobile's interpretation of Success Criterion 2.1.2 (No Keyboard Trap) for mobile contexts.
\end{itemize}

\subsubsection{Media screen}
\label{subsubsec:media-screen}

The Media screen demonstrates accessibility techniques for non-text content—one of the most fundamental aspects of digital accessibility. WCAG2Mobile provides specific guidance on handling media in a mobile context, emphasizing alternative text implementations and proper semantic roles for mobile interfaces. Figure~\ref{fig:media_screens_sidebyside} shows the main interface of this screen.

\begin{figure}[ht]
    \centering
    \begin{subfigure}[b]{0.48\textwidth}
        \centering
        \includegraphics[width=\linewidth, alt={First part of the Media screen}]{img/media1.png}
        \caption{Media screen - Part 1}
        \label{fig:media-left}
    \end{subfigure}
    \hfill
    \begin{subfigure}[b]{0.48\textwidth}
        \centering
        \includegraphics[width=\linewidth, alt={Second part of the Media screen}]{img/media2.png}
        \caption{Media screen - Part 2}
        \label{fig:media-right}
    \end{subfigure}
    \caption{Side-by-side view of the two Media screen parts}
    \label{fig:media_screens_sidebyside}
\end{figure}
\FloatBarrier

\paragraph{Component inventory and WCAG/MCAG/WCAG2Mobile mapping}

Table~\ref{tab:media_component_mapping} provides a comprehensive mapping between the Media screen components, their semantic roles, the WCAG 2.2 criteria they address, WCAG2Mobile considerations, and their implementation properties.

\begin{longtable}[c]{|C{2.5cm}|C{2cm}|C{2.8cm}|C{2.8cm}|C{4.7cm}|}
\caption{Media screen component-criteria mapping with WCAG2Mobile considerations}
\label{tab:media_component_mapping}\\
\hline
\textbf{Component} & \textbf{Semantic Role} & \textbf{WCAG 2.2 Criteria} & \textbf{WCAG2Mobile Considerations} & \textbf{Implementation Properties} \\
\hline
\endfirsthead
\multicolumn{4}{c}%
{{\bfseries Table \thetable\ -- continued from previous page}} \\
\hline
\textbf{Component and Location} & \textbf{Semantic Role} & \textbf{WCAG 2.2 Criteria} & \textbf{WCAG2Mobile Considerations} & \textbf{Implementation Properties} \\
\hline
\endhead
\hline
\multicolumn{5}{r}{{Continued on next page}} \\
\endfoot
\hline
\endlastfoot
Image (media display with checkmarks and interface elements) & image & 1.1.1 Non-text Content (A) & Mobile-specific image description; Platform-specific image role & \texttt{accessibility \ Role="image"},\newline \texttt{accessibility \ Label="A placeholder image (first example)"} \\
\hline
Navigation Controls (left/right arrow buttons for image navigation) & button & 2.1.1 Keyboard (A)\newline 2.5.8 Target Size (AA) & Mobile touch target sizing; Mobile navigation patterns & \texttt{accessibility \ Role="button"},\newline \texttt{accessibility \ Label="Previous image"},\newline \texttt{accessibility \ State=\{\{disabled: currentImage === 1\}\}} \\
\hline
Alt Text Button (blue "Hide Alt Text" button below image) & button & 2.5.8 Target Size (AA)\newline 3.2.2 On Input (A) & Mobile button sizing; Mobile state change notification & \texttt{accessibility \ Role="button"},\newline \texttt{accessibility \ Label="Show alternative text"} \\
\hline
Alt Text Display (text description below the alt text button) & text & 1.1.1 Non-text Content (A)\newline 1.3.1 Info and Relationships (A) & Mobile-specific text display; Mobile content structuring & \texttt{style= \ themedStyles.altText \ Container} \\
\hline
Position Announcements (screen reader announcements during navigation) & none & 2.4.3 Focus Order (A)\newline 4.1.3 Status Messages (AA) & Mobile navigation sequence; Mobile screen reader announcements & \texttt{AccessibilityInfo. \ announce \ ForAccessibility(`Next image. Now showing image \${newIndex} of \${images.length}`)} \\
\end{longtable}
\FloatBarrier

\paragraph{Alternative text implementation}

Listing~\ref{lst:media_implementation} shows the core pattern for accessible image implementation with proper alternative text, following WCAG2Mobile's guidance on implementing Success Criterion 1.1.1 (Non-text Content) in mobile contexts.

\begin{lstlisting}[
  style=ReactNativeStyle,
  caption={Accessible image implementation with alternative text for mobile context},
  label={lst:media_implementation},
  basicstyle=\ttfamily\footnotesize,
  numbers=left,
]
<Image
  source={images[currentImage - 1].uri}
  style={themedStyles.demoImage}
  accessibilityLabel={images[currentImage - 1].alt}
  accessible={true}
  accessibilityRole="image"
/>
\end{lstlisting}
\FloatBarrier

The Media screen demonstrates additional accessibility features beyond basic alternative text, with specific attention to WCAG2Mobile considerations:

\begin{enumerate}
    \item \textbf{Navigation controls}: Accessible previous/next buttons with clear labeling and state indication, implementing WCAG2Mobile's guidance on touch-based navigation controls;
    
    \item \textbf{Interactive alt text}: Toggle mechanism to show/hide alternative text as an educational feature, addressing WCAG2Mobile's emphasis on providing explicit access to alternative content;
    
    \item \textbf{Position context}: Announcements that communicate current position within a gallery (e.g., "Image 2 of 5"), implementing WCAG2Mobile's guidance on providing context in limited viewport environments;
    
    \item \textbf{Responsive sizing}: Image container adapts to screen size, addressing WCAG2Mobile's interpretation of Success Criterion 1.4.10 (Reflow) for responsive mobile interfaces.
\end{enumerate}

\paragraph{Screen reader support analysis}

Table~\ref{tab:media_screen_reader_analysis} presents results from systematic testing of the Media screen with screen readers on both iOS and Android platforms, with specific attention to WCAG2Mobile's guidance on platform-specific accessibility services.

\begin{longtable}[c]{|C{2.8cm}|C{3.5cm}|C{3.5cm}|C{4cm}|}
\caption{Media screen screen reader testing results with WCAG2Mobile considerations}
\label{tab:media_screen_reader_analysis}\\
\hline
\textbf{Test Case} & \textbf{VoiceOver (iOS)} & \textbf{TalkBack (Android)} & \textbf{WCAG2Mobile Considerations} \\
\hline
\endfirsthead
\multicolumn{4}{c}%
{{\bfseries Table \thetable\ -- continued from previous page}} \\
\hline
\textbf{Test Case} & \textbf{VoiceOver (iOS)} & \textbf{TalkBack (Android)} & \textbf{WCAG2Mobile Considerations} \\
\hline
\endhead
\hline
\multicolumn{4}{r}{{Continued on next page}} \\
\endfoot
\hline
\endlastfoot
Image Alt Text & {\color{green}\ding{51}} Announces full alternative text & {\color{green}\ding{51}} Announces full alternative text & Implementation of SC 1.1.1 for mobile screen readers \\
\hline
Navigation Controls & {\color{green}\ding{51}} Announces button label and state & {\color{green}\ding{51}} Announces button label and state & Implementation of SC 4.1.2 for mobile navigation controls \\
\hline
Position Announcement & {\color{green}\ding{51}} Announces position update & {\color{green}\ding{51}} Announces position update & Implementation of SC 4.1.3 for mobile gallery context \\
\hline
Alt Text Toggle & {\color{green}\ding{51}} Announces toggle state & {\color{green}\ding{51}} Announces toggle state & Implementation of SC 3.2.2 for mobile interactive features \\
\hline
Responsive Sizing & {\color{green}\ding{51}} Image properly scales & {\color{green}\ding{51}} Image properly scales & Implementation of SC 1.4.10 for mobile viewports \\
\hline
Touch Target Sizing & {\color{green}\ding{51}} Navigation controls are easily activated & {\color{green}\ding{51}} Navigation controls are easily activated & Implementation of SC 2.5.8 for mobile touch targets \\
\end{longtable}
\FloatBarrier

The implementation addresses several WCAG2Mobile-specific considerations:

\begin{enumerate}
    \item \textbf{Platform-specific image roles}: The implementation uses proper \texttt{accessibilityRole="image"} to ensure consistent identification across mobile platforms, following WCAG2Mobile's guidance on semantic roles in mobile contexts;
    
    \item \textbf{Mobile gallery navigation}: Navigation controls implement clear state communication and position announcements, addressing WCAG2Mobile's interpretation of Success Criterion 4.1.3 (Status Messages) for sequential content in mobile interfaces;
    
    \item \textbf{Responsive image sizing}: Images adapt to different screen sizes and orientations, implementing WCAG2Mobile's guidance on Success Criterion 1.4.10 (Reflow) for mobile viewports;
    
    \item \textbf{Mobile-specific gesture support}: The implementation supports both touch and screen reader gestures for navigation, addressing WCAG2Mobile's emphasis on supporting multiple interaction methods in mobile interfaces.
\end{enumerate}

\paragraph{Implementation overhead analysis}

Media components have a relatively low accessibility implementation overhead (14.2\%) among component types, with WCAG2Mobile considerations adding approximately 2.3\% to this total. The primary requirement—alternative text—is implemented through straightforward property assignment, while the majority of the overhead comes from implementing accessible navigation controls and responsive layout adaptations for mobile contexts.

\begin{longtable}[c]{|C{3.8cm}|C{2.3cm}|C{2.8cm}|C{2.8cm}|}
\caption{Media screen accessibility implementation overhead with WCAG2Mobile considerations}
\label{tab:media_implementation_overhead}\\
\hline
\textbf{Accessibility Feature} & \textbf{Lines of Code} & \textbf{Percentage of Total} & \textbf{Complexity Impact} \\
\hline
\endfirsthead
\multicolumn{4}{c}%
{{\bfseries Table \thetable\ -- continued from previous page}} \\
\hline
\textbf{Accessibility Feature} & \textbf{Lines of Code} & \textbf{Percentage of Total} & \textbf{Complexity Impact} \\
\hline
\endhead
\hline
\multicolumn{4}{r}{{Continued on next page}} \\
\endfoot
\hline
\endlastfoot
Alternative Text & 8 LOC & 1.5\% & Low \\
\hline
Navigation Controls & 24 LOC & 4.5\% & Medium \\
\hline
Status Announcements & 18 LOC & 3.4\% & Low \\
\hline
Alt Text Display & 12 LOC & 2.2\% & Low \\
\hline
Responsive Layout & 15 LOC & 2.8\% & Medium \\
\hline
WCAG2Mobile-specific Adaptations & 12 LOC & 2.3\% & Low \\
\hline
\textbf{Total} & \textbf{75 LOC} & \textbf{14.2\%} & \textbf{Low-Medium} \\
\end{longtable}
\FloatBarrier

\subsubsection{Advanced components screen}
\label{subsubsec:advanced-screen}

The Advanced components screen demonstrates accessibility implementations for more complex UI patterns including tabs, progress indicators, alerts, and sliders. WCAG2Mobile provides specific guidance on implementing these complex patterns in mobile contexts, with emphasis on touch-based interactions and mobile screen reader support. Figure~\ref{fig:advanced_screens_sidebyside1} and ~\ref{fig:advanced_screens_sidebyside2}  shows the two parts of the main interface of this screen.

\begin{figure}[ht]
    \centering
    \begin{subfigure}[b]{0.48\textwidth}
        \centering
        \includegraphics[width=\linewidth, alt={First part of the Advanced screen}]{img/advanced1.png}
        \caption{Advanced screen - Part 1}
        \label{fig:advanced-left1}
    \end{subfigure}
    \hfill
    \begin{subfigure}[b]{0.48\textwidth}
        \centering
        \includegraphics[width=\linewidth, alt={Second part of the Advanced screen}]{img/advanced2.png}
        \caption{Advanced screen - Part 2}
        \label{fig:advanced-right1}
    \end{subfigure}
    \caption{Side-by-side view of the first two Advanced screen parts}
    \label{fig:advanced_screens_sidebyside1}
\end{figure}
\FloatBarrier

\begin{figure}[ht]
    \centering
    \begin{subfigure}[b]{0.48\textwidth}
        \centering
        \includegraphics[width=\linewidth, alt={Third part of the Advanced screen}]{img/advanced3.png}
        \caption{Advanced screen - Part 3}
        \label{fig:advanced-left2}
    \end{subfigure}
    \hfill
    \begin{subfigure}[b]{0.48\textwidth}
        \centering
        \includegraphics[width=\linewidth, alt={Fourth part of the Advanced screen}]{img/advanced4.png}
        \caption{Advanced screen - Part 4}
        \label{fig:advanced-right2}
    \end{subfigure}
    \caption{Side-by-side view of the second two Advanced screen parts}
    \label{fig:advanced_screens_sidebyside2}
\end{figure}
\FloatBarrier

\paragraph{Component inventory and WCAG/MCAG/WCAG2Mobile mapping}

Table~\ref{tab:advanced_component_mapping} provides a comprehensive mapping between the Advanced components screen elements, their semantic roles, the WCAG 2.2 criteria they address, WCAG2Mobile considerations, and their implementation properties.

\begin{longtable}[c]{|C{2.5cm}|C{2cm}|C{2.8cm}|C{2.8cm}|C{4.7cm}|}
\caption{Advanced screen component-criteria mapping with WCAG2Mobile considerations}
\label{tab:advanced_component_mapping}\\
\hline
\textbf{Component} & \textbf{Semantic Role} & \textbf{WCAG 2.2 Criteria} & \textbf{WCAG2Mobile Considerations} & \textbf{Implementation Properties} \\
\hline
\endfirsthead
\multicolumn{4}{c}%
{{\bfseries Table \thetable\ -- continued from previous page}} \\
\hline
\textbf{Component and Location} & \textbf{Semantic Role} & \textbf{WCAG 2.2 Criteria} & \textbf{WCAG2Mobile Considerations} & \textbf{Implementation Properties} \\
\hline
\endhead
\hline
\multicolumn{5}{r}{{Continued on next page}} \\
\endfoot
\hline
\endlastfoot
Tab Container (Section with three tabs) & tablist & 1.3.1 Info and Relationships (A)\newline 2.4.3 Focus Order (A) & Mobile tab navigation patterns; Mobile content grouping & \texttt{accessibility \ Role="tablist"},\newline \texttt{accessibility \ Label="Navigation tabs"} \\
\hline
Tab Item (Tab One, Tab Two, Tab Three buttons) & tab & 4.1.2 Name, Role, Value (A)\newline 3.2.1 On Focus (A) & Mobile tab selection; Mobile touch interaction & \texttt{accessibility \ Role="tab"},\newline \texttt{accessibility \ Label="Select \${tab}"},\newline \texttt{accessibility \ State=\{\{selected: isSelected\}\}} \\
\hline
Progress Bar (Progress Indicators section with values) & progressbar & 1.3.1 Info and Relationships (A)\newline 4.1.2 Name, Role, Value (A) & Mobile progress indication; Mobile value reporting & \texttt{accessibility \ Role="progressbar"},\newline \texttt{accessibility \ Label="Progress indicator"},\newline \texttt{accessibility \ Value=\{\{min: 0, max: 100, now: progress\}\}} \\
\hline
Alert Message (status notifications during interactions) & alert & 4.1.3 Status Messages (AA) & Mobile notification pattern; Mobile screen reader priority & \texttt{accessibility \ Role="alert"},\newline \texttt{accessibility \ LiveRegion="assertive"} \\
\hline
Slider Container (slider component grouping) & none & 1.3.1 Info and Relationships (A) & Mobile slider context grouping & \texttt{accessible=true},\newline \texttt{accessibility \ Label="Slider control, current value \${sliderValue} percent"} \\
\hline
Slider Controls (buttons for increasing/decreasing slider values) & button & 2.1.1 Keyboard (A)\newline 2.5.8 Target Size (AA) & Mobile alternative controls; Enhanced touch targets & \texttt{accessibility \ Role="button"},\newline \texttt{accessibility \ Label="Decrease value"},\newline \texttt{accessibility \ Hint="Decreases slider value by 5 percent"} \\
\hline
Slider Presets (preset value buttons for quick selection) & button & 2.5.1 Pointer Gestures (A)\newline 2.5.7 Dragging Movements (AA) & Mobile alternative to dragging; Touch-based presets & \texttt{accessibility \ Role="button"},\newline \texttt{accessibility \ Label="Set value to \${value} percent"},\newline \texttt{accessibility \ State=\{\{selected: sliderValue === value\}\}} \\
\hline
Visual Slider (graphical slider interface) & none & 2.5.1 Pointer Gestures (A)\newline 2.5.7 Dragging Movements (AA) & Hidden from screen readers; Alternative controls provided & \texttt{importantFor \ Accessibility="no- \ hide-descendants"} \\
\end{longtable}
\FloatBarrier

\paragraph{Complex interaction patterns}

Advanced components present unique accessibility challenges requiring specialized implementations, with specific attention to WCAG2Mobile's guidance on complex mobile interaction patterns:

\begin{enumerate}
    \item \textbf{Tab navigation}: Proper role assignment with \texttt{accessibilityRole="tablist"} for containers and \texttt{accessibilityRole="tab"} for individual tabs, with selection state communicated through \texttt{accessibilityState}, implementing WCAG2Mobile's interpretation of Success Criterion 1.3.1 (Info and Relationships) for mobile tab patterns;
    
    \item \textbf{Progress indicators}: Value communication through \texttt{accessibilityValue} properties with min/max/current parameters, addressing WCAG2Mobile's guidance on communicating progress in mobile contexts;
    
    \item \textbf{Alerts and toasts}: Implementation of \texttt{accessibilityLiveRegion="assertive"} for time-sensitive notifications, following WCAG2Mobile's interpretation of Success Criterion 4.1.3 (Status Messages) for mobile notification patterns;
    
    \item \textbf{Slider alternatives}: Provision of button-based alternatives for precise slider control by screen reader users, implementing WCAG2Mobile's guidance on Success Criterion 2.5.7 (Dragging Movements) for mobile interfaces;
    
    \item \textbf{Enhanced target sizes}: All controls implement sizes that meet WCAG 2.5.8 (Target Size) criterion with 44×44dp dimensions, addressing WCAG2Mobile's emphasis on larger touch targets for mobile interfaces.
\end{enumerate}

\paragraph{Slider accessibility pattern}

The slider implementation (shown in Figure~\ref{fig:advanced-right2}) demonstrates a particularly important accessibility pattern for mobile interfaces: providing alternative interaction mechanisms for inherently visual controls, with specific attention to WCAG2Mobile's guidance on Success Criterion 2.5.1 (Pointer Gestures) and 2.5.7 (Dragging Movements). Listing~\ref{lst:slider_implementation} shows this implementation.

\begin{lstlisting}[
  style=ReactNativeStyle,
  caption={Accessible slider implementation with alternative controls for mobile context},
  label={lst:slider_implementation},
  basicstyle=\ttfamily\footnotesize,
  numbers=left,
]
<View
  accessible={true}
  accessibilityLabel={`Slider control, current value ${sliderValue} percent`}
  style={{ marginVertical: 12 }}
>
  {/* Button controls for screen reader users */}
  <View style={{
    flexDirection: 'row',
    justifyContent: 'space-between',
    alignItems: 'center',
    marginBottom: 14
  }}>
    <TouchableOpacity
      style={{
        padding: 10,
        backgroundColor: colors.primary,
        borderRadius: 25,
        width: 50,
        height: 50,
        alignItems: 'center',
        justifyContent: 'center'
      }}
      onPress={() => {
        const newValue = Math.max(0, sliderValue - 5);
        setSliderValue(newValue);
        AccessibilityInfo.announceForAccessibility(`Value set to ${newValue} percent`);
      }}
      accessibilityRole="button"
      accessibilityLabel="Decrease value"
      accessibilityHint="Decreases slider value by 5 percent"
    >
      <Ionicons name="remove" size={24} color="#fff" />
    </TouchableOpacity>
    
    {/* Value display with current percentage */}
    <Text style={{color: colors.text, fontSize: textSizes.large, fontWeight: 'bold'}}>
      {sliderValue}%
    </Text>
    
    {/* Increase button with proper accessibility props */}
    <TouchableOpacity
      style={{...}} // Same styling as decrease button
      onPress={() => {
        const newValue = Math.min(100, sliderValue + 5);
        setSliderValue(newValue);
        AccessibilityInfo.announceForAccessibility(`Value set to ${newValue} percent`);
      }}
      accessibilityRole="button"
      accessibilityLabel="Increase value"
      accessibilityHint="Increases slider value by 5 percent"
    >
      <Ionicons name="add" size={24} color="#fff" />
    </TouchableOpacity>
  </View>
  
  {/* Visual Slider hidden from screen readers */}
  <Slider
    value={sliderValue}
    minimumValue={0}
    maximumValue={100}
    step={1}
    onSlidingComplete={(val) => {
      const newValue = Math.round(val);
      setSliderValue(newValue);
      AccessibilityInfo.announceForAccessibility(`Value set to ${newValue} percent`);
    }}
    importantForAccessibility="no-hide-descendants"
  />
</View>
\end{lstlisting}
\FloatBarrier

This pattern includes several key elements for mobile accessibility:

\begin{itemize}
    \item Button controls for incremental adjustments with enhanced touch targets, implementing WCAG2Mobile's guidance on Success Criterion 2.5.8 (Target Size) for mobile touch targets;
    \item Preset value buttons for common settings, providing alternatives to dragging as required by WCAG2Mobile's interpretation of Success Criterion 2.5.7 (Dragging Movements);
    \item Value announcements with appropriate throttling to prevent excessive feedback, addressing WCAG2Mobile's guidance on status message frequency in mobile contexts;
    \item Visual feedback synchronized with announced values, implementing WCAG2Mobile's emphasis on consistent feedback across visual and auditory channels in mobile interfaces.
\end{itemize}

\paragraph{Screen reader support analysis}

Table~\ref{tab:advanced_screen_reader_analysis} presents results from systematic testing of the Advanced components screen with screen readers on both iOS and Android platforms, with specific attention to WCAG2Mobile's guidance on platform-specific accessibility services.

\begin{longtable}[c]{|C{2.8cm}|C{3.5cm}|C{3.5cm}|C{4cm}|}
\caption{Advanced screen screen reader testing results with WCAG2Mobile considerations}
\label{tab:advanced_screen_reader_analysis}\\
\hline
\textbf{Test Case} & \textbf{VoiceOver (iOS)} & \textbf{TalkBack (Android)} & \textbf{WCAG2Mobile Considerations} \\
\hline
\endfirsthead
\multicolumn{4}{c}%
{{\bfseries Table \thetable\ -- continued from previous page}} \\
\hline
\textbf{Test Case} & \textbf{VoiceOver (iOS)} & \textbf{TalkBack (Android)} & \textbf{WCAG2Mobile Considerations} \\
\hline
\endhead
\hline
\multicolumn{4}{r}{{Continued on next page}} \\
\endfoot
\hline
\endlastfoot
Tab Navigation & {\color{green}\ding{51}} Announces tab selection with state & {\color{green}\ding{51}} Announces tab selection with state & Implementation of SC 4.1.2 for mobile tab patterns \\
\hline
Progress Bar & {\color{green}\ding{51}} Announces progress value changes & {\color{green}\ding{51}} Announces progress value changes & Implementation of SC 4.1.3 for mobile progress indication \\
\hline
Alert Notification & {\color{green}\ding{51}} Interrupts to announce alert & {\color{green}\ding{51}} Interrupts to announce alert & Implementation of SC 4.1.3 for mobile notification priority \\
\hline
Slider Controls & {\color{green}\ding{51}} Announces buttons and values & {\color{green}\ding{51}} Announces buttons and values & Implementation of SC 2.5.1 and 2.5.7 for mobile slider alternatives \\
\hline
Slider Presets & {\color{green}\ding{51}} Announces preset values & {\color{green}\ding{51}} Announces preset values & Implementation of SC 2.5.7 for mobile slider alternatives \\
\hline
Touch Target Sizing & {\color{green}\ding{51}} Controls are easily activated & {\color{green}\ding{51}} Controls are easily activated & Implementation of SC 2.5.8 for mobile touch targets \\
\end{longtable}
\FloatBarrier

The implementation addresses several WCAG2Mobile-specific considerations:

\begin{enumerate}
    \item \textbf{Mobile-specific role implementation}: The implementation uses appropriate semantic roles for mobile screen contexts, following WCAG2Mobile's guidance on interpreting web roles for mobile interfaces;
    
    \item \textbf{Touch-based alternatives}: Alternative interaction mechanisms are provided for drag-based controls, implementing WCAG2Mobile's interpretation of Success Criterion 2.5.7 (Dragging Movements) for mobile contexts;
    
    \item \textbf{Mobile-specific focus management}: Tab activation properly manages focus in mobile contexts, addressing WCAG2Mobile's guidance on Success Criterion 2.4.3 (Focus Order) for mobile screen navigation;
    
    \item \textbf{Platform-specific announcement patterns}: Status messages use platform-appropriate announcement methods, following WCAG2Mobile's emphasis on platform-specific accessibility services.
\end{enumerate}

\paragraph{Implementation overhead analysis}

Advanced components have the highest implementation overhead (25.2\%) among component types, reflecting the additional complexity required to make inherently visual controls accessible through alternative interaction mechanisms on mobile devices. The integration of WCAG2Mobile-specific considerations adds approximately 3.4\% to this total, focused primarily on providing touch-based alternatives to dragging and implementing proper mobile notification patterns.

\begin{longtable}[c]{|C{3.8cm}|C{2.3cm}|C{2.8cm}|C{2.8cm}|}
\caption{Advanced screen accessibility implementation overhead with WCAG2Mobile considerations}
\label{tab:advanced_implementation_overhead}\\
\hline
\textbf{Accessibility Feature} & \textbf{Lines of Code} & \textbf{Percentage of Total} & \textbf{Complexity Impact} \\
\hline
\endfirsthead
\multicolumn{4}{c}%
{{\bfseries Table \thetable\ -- continued from previous page}} \\
\hline
\textbf{Accessibility Feature} & \textbf{Lines of Code} & \textbf{Percentage of Total} & \textbf{Complexity Impact} \\
\hline
\endhead
\hline
\multicolumn{4}{r}{{Continued on next page}} \\
\endfoot
\hline
\endlastfoot
Tab Role Implementation & 18 LOC & 2.2\% & Medium \\
\hline
Progress Bar Accessibility & 12 LOC & 1.5\% & Medium \\
\hline
Alert Notifications & 15 LOC & 1.9\% & Medium \\
\hline
Slider Alternative Controls & 64 LOC & 7.9\% & High \\
\hline
State Announcements & 26 LOC & 3.2\% & Medium \\
\hline
Touch Target Enhancement & 18 LOC & 2.2\% & Low \\
\hline
Element Hiding & 24 LOC & 3.0\% & Low \\
\hline
WCAG2Mobile-specific Adaptations & 27 LOC & 3.4\% & Medium \\
\hline
\textbf{Total} & \textbf{204 LOC} & \textbf{25.2\%} & \textbf{Medium-High} \\
\end{longtable}
\FloatBarrier

\subsubsection{Key insights from component implementation}
\label{subsubsec:component-insights}

The analysis of multiple component implementations reveals several critical insights for developers implementing accessibility in mobile applications:

\begin{enumerate}
    \item \textbf{Implementation complexity correlates with interaction complexity}: More complex interaction patterns require more sophisticated accessibility implementations, with forms and advanced components requiring the highest implementation overhead;
    
    \item \textbf{Focus management is critical for non-linear interactions}: Components that create new interaction contexts (dialogs) or complex navigation patterns (tabs) require explicit focus management to maintain user orientation;
    
    \item \textbf{Alternative interaction mechanisms are essential for inherently visual controls}: Components like sliders require additional interaction mechanisms to ensure operability by screen reader users;
    
    \item \textbf{Explicit state communication improves usability}: All interactive components benefit from explicit state communication via \texttt{accessibilityState} and announcements, but this is particularly critical for selection-based controls;
    
    \item \textbf{Platform-specific adaptations may be necessary}: While React Native provides a unified accessibility API, some components (particularly date pickers and complex inputs) benefit from platform-specific adaptations to leverage native accessibility features;
    
    \item \textbf{AAA criteria implementation is achievable with careful design}: Several components successfully implement AAA criteria with relatively modest code overhead, particularly around enhanced target sizes (2.5.5), change on request (3.2.5), help information (3.3.5), and error prevention (3.3.6).
\end{enumerate}

These insights provide developers with a framework for prioritizing accessibility implementation efforts, focusing on the components and patterns that present the greatest challenges and require the most sophisticated approaches to ensure equal access for all users.

\subsection{Best practices main screen}

The Best practices screen serves as a comprehensive educational resource within the \textit{AccessibleHub} application. It provides developers with access to essential guidelines, patterns, and interactive resources for implementing accessibility in mobile applications. The screen organizes accessibility knowledge into five key categories: \textit{WCAG Guidelines, Semantic Structure, Gesture Tutorial, Screen Reader Support, and Logical Focus Order}. An example of the interface is shown in Figure~\ref{fig:best_practices_screens_sidebyside}.

\begin{figure}[ht]
    \centering
    \begin{subfigure}[b]{0.48\textwidth}
        \centering
        \includegraphics[width=\linewidth, alt={First part of the Best practices screen}]{img/practices1.png}
        \caption{Best practices screen - Top section}
        \label{fig:best-practices-top}
    \end{subfigure}
    \hfill
    \begin{subfigure}[b]{0.48\textwidth}
        \centering
        \includegraphics[width=\linewidth, alt={Second part of the Best practices screen}]{img/practices2.png}
        \caption{Best practices screen - Bottom section}
        \label{fig:best-practices-bottom}
    \end{subfigure}
    \caption{Side-by-side view of the Best practices screen sections, showing accessibility guideline categories}
    \label{fig:best_practices_screens_sidebyside}
\end{figure}
\FloatBarrier

\subsubsection{Component inventory and WCAG/MCAG/WCAG2Mobile mapping}

Table~\ref{tab:best_practices_wcag2mobile_mapping} provides a formal mapping between the UI components, their semantic roles, the specific WCAG 2.2 criteria they address, their WCAG2Mobile considerations, and their React Native implementation properties.

\begin{longtable}[c]{|C{2.5cm}|C{1.8cm}|C{2.8cm}|C{3.5cm}|C{4.7cm}|}
\caption{Best practices screen component-criteria mapping with WCAG2Mobile considerations}
\label{tab:best_practices_wcag2mobile_mapping}\\
\hline
\textbf{Component and Location} & \textbf{Semantic Role} & \textbf{WCAG 2.2 Criteria} & \textbf{WCAG2Mobile Considerations} & \textbf{Implementation Properties} \\
\hline
\endfirsthead
\multicolumn{5}{c}%
{{\bfseries Table \thetable\ -- continued from previous page}} \\
\hline
\textbf{Component} & \textbf{Semantic Role} & \textbf{WCAG 2.2 Criteria} & \textbf{WCAG2Mobile Considerations} & \textbf{Implementation Properties} \\
\hline
\endhead
\hline
\multicolumn{5}{r}{{Continued on next page}} \\
\endfoot
\hline
\endlastfoot
Hero Title (center: "Mobile Accessibility Best Practices") & heading & 1.4.3 Contrast (AA)\newline 2.4.6 Headings (AA) & SC 2.4.6 applied to screens rather than pages; Proper semantic structure in view context & \texttt{accessibility \ Role="header"} \\
\hline
Practice Cards (grid: WCAG Guidelines, Semantic Structure, etc.) & button & 1.4.3 Contrast (AA)\newline 2.5.8 Target Size (AA)\newline 2.5.5 Target Size (Enhanced) (AAA)\newline 4.1.2 Name, Role, Value (A) & SC 2.5.8 for mobile touch targets (min 44×44dp); SC 4.1.2 adapted for mobile platform accessibility services & \texttt{accessibility \ Role="button"},\newline \texttt{accessibility \ Label="[Practice]. [Description]"} \\
\hline
Category Icons (within cards: document, hand, eye icons) & none & 1.1.1 Non-text Content (A) & SC 1.1.1 applied to reduce unnecessary focus stops; Efficiency in screen reader navigation sequence & \texttt{accessibility \ ElementsHidden=true} \\
\hline
Badges (top-right labels: Documentation, Interactive Guide, etc.) & text & 1.4.3 Contrast (AA)\newline 1.3.1 Info and Relationships (A) & SC 1.3.1 applied to mobile views rather than pages; Descriptive labeling within view context & Inherited from parent button's \texttt{accessibility \ Label} \\
\hline
Feature Items (bulleted lists within cards) & text & 1.3.1 Info and Relationships (A) & SC 1.3.1 adapted for view-specific information grouping; Semantic structure mapping for screen components & All related information inherited from parent element\\
\hline
Chevron Icons (right arrows in cards) & none & 1.1.1 Non-text Content (A) & SC 1.1.1 applied to reduce swipes in mobile screen reader navigation & \texttt{accessibility \ ElementsHidden=true},\newline \texttt{importantFor \ Accessibility= \ "no-hide-descendants"} \\
\hline
Screen Announcements (navigation feedback) & status & 4.1.3 Status Messages (AA) & SC 4.1.3 adapted for screen transitions in mobile context; Status changes announced through platform-specific services & \texttt{AccessibilityInfo. \ announceFor \ Accessibility} \\
\end{longtable}
\FloatBarrier

\subsubsection{Technical implementation analysis}

The code sample in Listing~\ref{lst:best-practices-screen-accessibility} demonstrates the key accessibility properties implemented in the Best practices screen.

\begin{lstlisting}[
  style=ReactNativeStyle,
  caption={Annotated code sample demonstrating Best practices screen accessibility properties},
  label={lst:best-practices-screen-accessibility},
  basicstyle=\ttfamily\footnotesize,
  numbers=left,
]
  {/* 1. Practice card with accessibility label */}
  <TouchableOpacity
    style={themedStyles.card}
    onPress={() => {
      router.push('/practices-screens/guidelines');
      AccessibilityInfo.announceForAccessibility('Opening WCAG Guidelines');
    }}
    accessibilityRole="button"
    accessibilityLabel="WCAG Guidelines"
  >
    {/* 2. Icon with accessibility hiding to prevent redundant focus */}
    <View style={[themedStyles.iconWrapper, { backgroundColor: iconColors.wcag.bg }]}>
      <Ionicons
        name="document-text-outline"
        size={24}
        color={iconColors.wcag.icon}
        accessibilityElementsHidden
      />
    </View>

    <View style={themedStyles.cardContent}>
      <View style={themedStyles.titleRow}>
        <Text style={themedStyles.practiceTitle}>WCAG Guidelines</Text>
        <View style={themedStyles.badgeContainer}>
          <View style={themedStyles.badge}>
            <Text style={themedStyles.badgeText}>Documentation
            </Text>
          </View>
        </View>
      </View>

      {/* 3. Feature list with hidden decorative icons */}
      <View style={themedStyles.featureList}>
        <View style={themedStyles.featureItem}>
          <Ionicons
            name="checkmark-circle"
            accessibilityElementsHidden
            importantForAccessibility="no-hide-descendants"
          />
        </View>
      </View>
    </View>
  </TouchableOpacity>
\end{lstlisting}
\FloatBarrier

The implementation of the Best practices screen addresses several important accessibility considerations:

\begin{enumerate}
    \item \textbf{Elimination of garbage interactions}: Decorative elements (icons, chevrons) are properly hidden from screen readers using both \texttt{accessibilityElementsHidden} and \\ \texttt{importantForAccessibility="no-hide-descendants"} to eliminate unnecessary swipes, which directly addresses feedback received during accessibility testing;
    
    \item \textbf{Comprehensive card labels}: Each practice card provides detailed accessibility labels that include the category name and description, ensuring screen reader users get complete context without needing to navigate through sub-elements;
    
    \item \textbf{Navigation announcements}: The implementation uses \\ \texttt{AccessibilityInfo.announceForAccessibility} to proactively inform users about screen transitions when navigating to specific practice guides;
    
    \item \textbf{Touch target optimization}: All interactive elements maintain sufficient touch target sizes to accommodate various user needs, with cards providing ample tapping area that meets the WCAG 2.5.5 (Target Size Enhanced) AAA criterion.
\end{enumerate}

\subsubsection{Screen reader support analysis}

Table~\ref{tab:best_practices_screen_reader_wcag2mobile} presents results from systematic testing of the Best practices screen with screen readers on both iOS and Android platforms, with a specific focus on WCAG2Mobile interpretations.

\begin{longtable}[c]{|C{2.8cm}|C{3.5cm}|C{3.5cm}|C{4cm}|}
\caption{Best practices screen screen reader testing with WCAG2Mobile considerations}
\label{tab:best_practices_screen_reader_wcag2mobile}\\
\hline
\textbf{Test Case} & \textbf{VoiceOver (iOS 16)} & \textbf{TalkBack (Android 14-15)} & \textbf{WCAG2Mobile Considerations} \\
\hline
\endfirsthead
\multicolumn{4}{c}%
{{\bfseries Table \thetable\ -- continued from previous page}} \\
\hline
\textbf{Test Case} & \textbf{VoiceOver (iOS 16)} & \textbf{TalkBack (Android 14-15)} & \textbf{WCAG2Mobile Considerations} \\
\hline
\endhead
\hline
\multicolumn{4}{r}{{Continued on next page}} \\
\endfoot
\hline
\endlastfoot
Hero Title & {\ding{51}} Announces ``Mobile Accessibility Best Practices, heading'' & {\ding{51}} Announces ``Mobile Accessibility Best Practices, heading'' & SC 1.3.1 and 2.4.6 interpreted for screens rather than web pages \\
\hline
Practice Card & {\ding{51}} Announces complete label with category description and purpose & {\ding{51}} Announces complete label with category description and purpose & SC 4.1.2 adapted for mobile accessibility services for comprehensive label communication \\
\hline
Category Icons & {\ding{51}} Not focused or announced & {\ding{51}} Not focused or announced & SC 1.1.1 applied to reduce swipes in mobile screen reader navigation \\
\hline
Feature Items & {\ding{51}} Not individually announced, part of card description & {\ding{51}} Not individually announced, part of card description & SC 1.3.1 adapted for view-specific information grouping and navigation efficiency \\
\hline
Screen Transitions & {\ding{51}} Announces destination screen & {\ding{51}} Announces destination screen & SC 4.1.3 adapted for screen transitions in mobile context \\
\hline
Touch Target Testing & {\ding{51}} Components easily activated with touch & {\ding{51}} Components easily activated with touch & SC 2.5.8 implemented with mobile-specific target sizes exceeding 44×44dp \\
\end{longtable}
\FloatBarrier

The implementation addresses several key MCAG and WCAG2Mobile considerations specific to mobile platforms:
\begin{enumerate}
    \item \textbf{Swipe efficiency optimization}: The screen implements a carefully designed focus order with decorative and non-essential elements hidden from screen readers, significantly reducing the number of swipes required to navigate the content—a critical consideration for mobile screen reader users that improves navigation efficiency by approximately 60\% compared to a non-optimized implementation;
    
    \item \textbf{Contextual navigation announcements}: Screen transitions are explicitly announced using \\ \texttt{AccessibilityInfo.announceForAccessibility}, providing critical context during navigation between different practice guides—addressing a key mobile accessibility challenge where context can be easily lost during transitions on smaller screens;
    
    \item \textbf{Visual hierarchy reinforcement}: The implementation uses a consistent visual system of icons, badges, and categorized cards that reinforces the information hierarchy, helping users with cognitive disabilities understand content organization on smaller screens;
    
    \item \textbf{Touch-optimized interaction targets}: All interactive elements exceed the minimum recommended dimensions of 44×44dp, implementing mobile accessibility best practices for touch interactions that accommodate users with various motor control capabilities and meeting the WCAG 2.5.5 (Target Size Enhanced) AAA criterion;
    
    \item \textbf{Single-hand operation zones}: Practice cards are positioned to be easily reachable within the natural thumb zone for one-handed operation, implementing a mobile ergonomic principle not explicitly covered in WCAG but crucial for mobile accessibility.
\end{enumerate}

\subsubsection{Implementation overhead analysis}

Table~\ref{tab:best_practices_implementation_overhead} quantifies the additional code required to implement accessibility features in the Best practices screen.

\begin{longtable}[c]{|C{3.8cm}|C{2.3cm}|C{2.8cm}|C{2.8cm}|}
\caption{Best practices screen accessibility implementation overhead}
\label{tab:best_practices_implementation_overhead}\\
\hline
\textbf{Accessibility Feature} & \textbf{Lines of Code} & \textbf{Percentage of Total} & \textbf{Complexity Impact} \\
\hline
\endfirsthead
\multicolumn{4}{c}%
{{\bfseries Table \thetable\ -- continued from previous page}} \\
\hline
\textbf{Accessibility Feature} & \textbf{Lines of Code} & \textbf{Percentage of Total} & \textbf{Complexity Impact} \\
\hline
\endhead
\hline
\multicolumn{4}{r}{{Continued on next page}} \\
\endfoot
\hline
\endlastfoot
Semantic Roles & 14 LOC & 2.5\% & Low \\
\hline
Descriptive Labels & 25 LOC & 4.5\% & Medium \\
\hline
Element Hiding & 30 LOC & 5.4\% & Low \\
\hline
Screen Announcements & 15 LOC & 2.7\% & Low \\
\hline
Contrast Handling & 18 LOC & 3.2\% & Medium \\
\hline
Gradient Background & 12 LOC & 2.2\% & Low \\
\hline
Touch Target Sizing & 20 LOC & 3.6\% & Medium \\
\hline
\textbf{Total} & \textbf{134 LOC} & \textbf{24.1\%} & \textbf{Medium} \\
\end{longtable}
\FloatBarrier

This analysis reveals that implementing comprehensive accessibility adds approximately 24.1\% to the code base of the Best practices screen. This represents a slightly lower overhead compared to the Home screen (28.0\%) and Components screen (32.8\%), primarily due to the more straightforward structure of this screen that emphasizes categorization and navigation rather than complex interactive elements. The implementation overhead is justified by the improved user experience for people with disabilities and the educational value for developers learning to implement accessibility in their own applications.

\subsubsection{WCAG conformance by principle and level}

Table~\ref{tab:best_practices_wcag2mobile_by_principle} provides a detailed analysis of WCAG2Mobile compliance by principle:

\begin{longtable}[c]{|C{2.5cm}|C{3.8cm}|C{3.2cm}|C{5.2cm}|}
\caption{Best practices screen WCAG2Mobile compliance analysis by principle}
\label{tab:best_practices_wcag2mobile_by_principle}\\
\hline
\textbf{Principle} & \textbf{Description} & \textbf{Implementation Level} & \textbf{Key Success Criteria with Mobile Context} \\
\hline
\endfirsthead
\multicolumn{4}{c}%
{{\bfseries Table \thetable\ -- continued from previous page}} \\
\hline
\textbf{Principle} & \textbf{Description} & \textbf{Implementation Level} & \textbf{Key Success Criteria with Mobile Context} \\
\hline
\endhead
\hline
\multicolumn{4}{r}{{Continued on next page}} \\
\endfoot
\hline
\endlastfoot
1. Perceivable & Information and UI components must be presentable to users in ways they can perceive & 12/13 (92\%) & 1.1.1 Non-text Content (A) - Mobile focus on reducing unnecessary navigation elements\newline 1.3.1 Info and Relationships (A) - Adapted for screens rather than pages\newline 1.4.3 Contrast (Minimum) (AA) - Adapted for mobile viewing conditions \\
\hline
2. Operable & UI components and navigation must be operable & 15/17 (88\%) & 2.4.3 Focus Order (A) - Optimized for swipe navigation\newline 2.4.6 Headings and Labels (AA) - Adapted for screen context\newline 2.5.8 Target Size (Minimum) (AA) - Mobile-specific touch target dimensions\newline 2.5.5 Target Size (Enhanced) (AAA) - Implemented for mobile touch optimization \\
\hline
3. Understandable & Information and operation of UI must be understandable & 8/10 (80\%) & 3.2.1 On Focus (A) - Mobile-specific focus handling\newline 3.2.4 Consistent Identification (AA) - Consistent across screens\newline 3.3.2 Labels or Instructions (A) - Mobile-optimized labeling patterns \\
\hline
4. Robust & Content must be robust enough to be interpreted by a wide variety of user agents & 3/3 (100\%) & 4.1.2 Name, Role, Value (A) - Adapted for mobile platform accessibility services\newline 4.1.3 Status Messages (AA) - Implemented for mobile screen transitions \\
\end{longtable}
\FloatBarrier

The Best Practices screen shows strong WCAG2Mobile compliance with effective mobile adaptations. Robust (100\%) demonstrates perfect platform integration for screen transitions and accessibility services. Perceivable (92\%) reflects successful mobile-specific optimizations like reduced navigation elements and mobile viewing contrast considerations. Operable (88\%) shows good touch target implementation and swipe navigation optimization, with minor gaps in complex navigation patterns. Understandable (80\%) indicates room for improvement in mobile-optimized labeling and instruction clarity, typical for educational content where comprehensive information must fit mobile constraints.

Table~\ref{tab:best_practices_wcag_by_level} categorizes the implementation by WCAG conformance levels:

\begin{table}[ht]
\caption{Best practices screen WCAG implementation by conformance level}
\label{tab:best_practices_wcag_by_level}
\centering
\begin{tabular}[c]{|C{2.8cm}|C{4cm}|C{3.5cm}|C{5cm}|}
\hline
\textbf{Conformance Level} & \textbf{Description} & \textbf{Implementation Rate} & \textbf{Notable Implementations} \\
\hline
A (Level A) & Basic accessibility requirements that must be satisfied & 15/15 (100\%) & 1.1.1 Non-text Content\newline 1.3.1 Info and Relationships\newline 4.1.2 Name, Role, Value \\
\hline
AA (Level AA) & Advanced requirements beyond Level A & 13/13 (100\%) & 1.4.3 Contrast (Minimum)\newline 2.4.6 Headings and Labels\newline 4.1.3 Status Messages \\
\hline
AAA (Level AAA) & Highest level of accessibility & 2/5 (40\%) & 2.5.5 Target Size (Enhanced)\newline 3.2.5 Change on Request \\
\hline
\end{tabular}
\end{table}
\FloatBarrier

The implementation achieves complete compliance with Level A and AA requirements, ensuring the Best practices screen meets the baseline and intermediate accessibility standards. For Level AAA, two key criteria are implemented: 2.5.5 Target Size (Enhanced) through the large touch targets of practice cards, and 3.2.5 Change on Request by ensuring all changes occur only in response to explicit user actions.

\subsubsection{Category-specific accessibility analysis}

Each category card within the Best practices screen implements specific accessibility considerations relevant to its content domain:

\paragraph{WCAG guidelines card}

The WCAG Guidelines card connects abstract guidelines with concrete mobile implementation techniques, addressing:

\begin{enumerate}
    \item \textbf{Semantic role communication}: The card properly communicates its role as a button leading to detailed guidelines via \texttt{accessibilityRole="button"};
    
    \item \textbf{Purpose clarity}: The description provides clear context about the destination content, addressing WCAG 2.4.4 Link Purpose (In Context);
    
    \item \textbf{Navigation announcement}: When activated, it announces the screen transition using \\ \texttt{AccessibilityInfo.announceForAccessibility('Opening WCAG Guidelines')}, providing critical context for screen reader users.
\end{enumerate}

\paragraph{Gesture tutorial card}

The Gesture Tutorial card implements specific considerations for educational interactive content:

\begin{enumerate}
    \item \textbf{Self-descriptive labeling}: The card's label identifies it as an interactive guide specifically for learning gestures, setting appropriate expectations;
    
    \item \textbf{Associated feature items}: The feature items ("Gesture Patterns", "Interactive Demo") provide additional context about the tutorial's content structure;
    
    \item \textbf{Enhanced visual cues}: The hand icon provides a clear visual cue about gesture content, while remaining properly hidden from screen readers to avoid redundancy.
\end{enumerate}

\paragraph{Screen reader support card}

The Screen reader support card serves as a gateway to platform-specific accessibility guidance:

\begin{enumerate}
    \item \textbf{Platform-specific indication}: The card includes feature items that indicate platform-specific guidance will be provided, setting appropriate user expectations;
    
    \item \textbf{Adaptive technology focus}: The eye icon and explicit naming communicate direct relevance to screen reader users, making this card particularly important for developers creating applications for users with visual impairments;
    
    \item \textbf{Clear purpose communication}: The description "Optimizing your app for VoiceOver and TalkBack" provides specific platform references that assist developers in understanding the content's relevance to their development context.
\end{enumerate}

\subsubsection{Mobile-specific considerations}

The Best practices screen implementation addresses several mobile-specific accessibility considerations beyond standard WCAG requirements:

\begin{enumerate}
    \item \textbf{Card-based information architecture}: The implementation uses a card-based design pattern that maintains clear boundaries between content categories—this addresses the mobile-specific challenge of limited screen space by creating visually and semantically distinct content blocks that are easier to perceive on smaller screens;
    
    \item \textbf{Badge-based categorization}: Each practice card uses compact badges ("Documentation", "Interactive Guide", etc.) to efficiently communicate content type—addressing the mobile constraint of limited screen real estate while maintaining clear information hierarchy;
    
    \item \textbf{Gesture-aware interaction design}: The screen implements appropriate touch target sizes and positioning for gesture-based interaction, addressing MCAG considerations for users with various motor capabilities accessing content via touch interfaces;
    
    \item \textbf{Consistent iconography system}: The implementation uses a coherent visual language with specific icons for each practice category, helping users quickly identify content types—particularly beneficial for users with cognitive disabilities navigating on mobile devices;
    
    \item \textbf{Minimal nesting depth}: The screen maintains a shallow information hierarchy with all main categories accessible from a single scrollable view, reducing the navigation depth required to access content—a crucial consideration for mobile interfaces where deeper navigation can lead to disorientation.
\end{enumerate}

\subsection{Best practices section}
\label{subsec:best-practices-section}

This section provides a formal analysis of the screens within the Best Practices section of \textit{AccessibleHub}. The Best Practices screens serve as educational resources for developers, presenting key accessibility principles, guidelines, and practical implementation techniques. Unlike the Components section which focuses on specific UI elements, the Best Practices section emphasizes overarching principles and approaches to creating accessible mobile experiences.

\subsubsection{WCAG guidelines screen}
\label{subsubsec:guidelines-screen}

The WCAG Guidelines screen serves as a foundational educational resource, introducing the four core principles of the Web Content Accessibility Guidelines: Perceivable, Operable, Understandable, and Robust. This screen provides developers with a clear overview of accessibility fundamentals upon which all implementation practices are built. Figure~\ref{fig:guidelines_screens_sidebyside} shows the main interface of this screen.

\pagebreak

\begin{figure}[ht]
    \centering
    \begin{subfigure}[b]{0.48\textwidth}
        \centering
        \includegraphics[width=\linewidth, alt={First part of the WCAG guidelines screen}]{img/guidelines1.jpg}
        \caption{Guidelines screen - Part 1}
        \label{fig:guidelines-left}
    \end{subfigure}
    \hfill
    \begin{subfigure}[b]{0.48\textwidth}
        \centering
        \includegraphics[width=\linewidth, alt={Second part of the WCAG guidelines screen}]{img/guidelines2.jpg}
        \caption{Guidelines screen - Part 2}
        \label{fig:guidelines-right}
    \end{subfigure}
    \caption{Side-by-side view of the WCAG Guidelines screen sections}
    \label{fig:guidelines_screens_sidebyside}
\end{figure}

\paragraph{Component inventory and WCAG/MCAG/WCAG2Mobile mapping}

Table~\ref{tab:guidelines_component_mapping} provides a formal mapping between the UI components, their semantic roles, the specific WCAG 2.2 criteria they address, their WCAG2Mobile considerations, and their React Native implementation properties.

\begin{longtable}[c]{|C{2.5cm}|C{2cm}|C{2.8cm}|C{3.2cm}|C{4.7cm}|}
\caption{Guidelines screen component-criteria mapping with WCAG2Mobile considerations}
\label{tab:guidelines_component_mapping}\\
\hline
\textbf{Component and Location} & \textbf{Semantic Role} & \textbf{WCAG 2.2 Criteria} & \textbf{WCAG2Mobile Considerations} & \textbf{Implementation Properties} \\
\hline
\endfirsthead
\multicolumn{5}{c}%
{{\bfseries Table \thetable\ -- continued from previous page}} \\
\hline
\textbf{Component and Location} & \textbf{Semantic Role} & \textbf{WCAG 2.2 Criteria} & \textbf{WCAG2Mobile Considerations} & \textbf{Implementation Properties} \\
\hline
\endhead
\hline
\multicolumn{5}{r}{{Continued on next page}} \\
\endfoot
\hline
\endlastfoot
Hero Title (top: "WCAG 2.2 Guidelines") & heading & 1.4.3 Contrast (AA)\newline 2.4.6 Headings (AA)\newline 2.4.10 Section Headings (AAA) & Text readability across variable screen sizes; Header role in screen context rather than web page & \texttt{accessibility \ Role="header"} \\
\hline
Principle Cards (sections: Perceivable, Operable, etc.) & none & 1.3.1 Info and Relationships (A)\newline 1.4.3 Contrast (AA) & Grouping related information in screen context; Mobile-specific content association patterns & Inherited from parent container the proper structural context \\
\hline
Principle Icons (eye, hand icons within cards) & none & 1.1.1 Non-text Content (A) & Reduction of unnecessary focus stops to optimize touch navigation; Efficiency in screen reader navigation sequence & \texttt{accessibility \ ElementsHidden=true},\newline \texttt{importantFor \ Accessibility= \ "no-hide-descendants"} \\
\hline
Principle Title (blue headings: "Perceivable", "Operable") & text & 2.4.6 Headings and Labels (AA) & Clear section identification within mobile screen context & Text styling with semantic meaning \\
\hline
Principle Description (explanatory text under titles) & text & 1.3.1 Info and Relationships (A) & Descriptive content with relation to screen context rather than web page context & Proper text styling with semantic connection to title \\
\hline
Checklist Items (bulleted lists within cards) & text & 1.3.1 Info and Relationships (A)\newline 1.3.2 Meaningful Sequence (A) & Logical grouping for mobile screen reading sequence; Touch-based navigation considerations & All related information inherited from parent \\
\hline
Checkmark Icons (green checks in lists) & none & 1.1.1 Non-text Content (A) & Reduction of unnecessary focus stops for more efficient touch-based screen reader navigation & \texttt{accessibility \ ElementsHidden=true},\newline \texttt{importantFor \ Accessibility= \ "no-hide-descendants"} \\
\hline
Card Container (overall card structure) & none & 2.5.5 Target Size (Enhanced) (AAA)\newline 3.2.5 Change on Request (AAA) & Touch target sizing exceeding WCAG2Mobile recommendations for minimum size; Mobile-specific touch interaction feedback & Container with adequate spacing and consistent interactive behavior \\
\hline
\end{longtable}

\FloatBarrier

\paragraph{Technical implementation analysis}

The code sample in Listing~\ref{lst:guidelines-screen-accessibility} demonstrates the key accessibility properties implemented in the WCAG guidelines screen.

\begin{lstlisting}[
  style=ReactNativeStyle,
  caption={Annotated code sample demonstrating guidelines screen accessibility properties},
  label={lst:guidelines-screen-accessibility},
  basicstyle=\ttfamily\footnotesize,
  numbers=left,
]
{/* 1. Guideline card with accessibility considerations */}
<View key={index} style={themedStyles.guidelineCard}>
  {/* 2. Card header with icon properly hidden from screen readers */}
  <View style={themedStyles.cardHeader}>
    <View style={themedStyles.iconContainer}>
      <Ionicons
        name={guideline.icon}
        size={28}
        color="#0055CC"
        accessibilityElementsHidden={true}
        importantForAccessibility="no-hide-descendants"
      />
    </View>
    <Text style={themedStyles.cardTitle}>{guideline.title}</Text>
  </View>

  {/* 3. Description text with proper semantic connection to title */}
  <Text style={themedStyles.cardDescription}>
    {guideline.description}
  </Text>

  {/* 4. Checklist items with proper grouping and hidden decorative icons */}
  <View style={themedStyles.checkList}>
    {guideline.checkItems.map((item, itemIndex) => (
      <View key={itemIndex} style={themedStyles.checkItemRow}>
        <Ionicons
          name="checkmark-circle"
          size={20}
          color="#28A745"
          style={themedStyles.checkIcon}
          accessibilityElementsHidden={true}
          importantForAccessibility="no-hide-descendants"
        />
        <Text style={themedStyles.checkItemText}>{item}</Text>
      </View>
    ))}
  </View>
</View>
\end{lstlisting}

\FloatBarrier

The implementation of the Guidelines screen addresses several important accessibility considerations:

\begin{enumerate}
    \item \textbf{Proper hiding of decorative elements}: All decorative icons (principle icons, checkmarks) are properly hidden from screen readers using both \\ \texttt{accessibilityElementsHidden=true} and \\ \texttt{importantForAccessibility="no-hide-descendants"}, \\ eliminating unnecessary swipes;
    
    \item \textbf{Semantic structure}: The implementation creates a clear hierarchical structure with the title at the top, followed by descriptions and related checklist items, ensuring proper comprehension of content relationships;
    
    \item \textbf{Grouped related content}: Each principle card groups related information together, associating the title, description, and checklist items as a single conceptual unit;
    
    \item \textbf{Color contrast implementation}: Text elements maintain proper contrast ratios against their backgrounds, with semantic meaning reinforced through visual styling;
    
    \item \textbf{Enhanced target sizing (AAA)}: The implementation provides adequately sized touch targets that exceed the minimum requirements, satisfying the enhanced target size criterion 2.5.5 (AAA);
    
    \item \textbf{Change on request (AAA)}: Content changes and interactions only occur in response to explicit user actions, conforming to criterion 3.2.5 (AAA).
\end{enumerate}

\paragraph{Screen reader support analysis}

Table~\ref{tab:guidelines_screen_reader_analysis} presents results from systematic testing of the Guidelines screen with screen readers on both iOS and Android platforms, with specific WCAG2Mobile considerations noted.

\begin{longtable}[c]{|C{2.8cm}|C{3.5cm}|C{3.5cm}|C{5cm}|}
\caption{Guidelines screen screen reader testing results with WCAG2Mobile considerations}
\label{tab:guidelines_screen_reader_analysis}\\
\hline
\textbf{Test Case} & \textbf{VoiceOver (iOS 16)} & \textbf{TalkBack (Android 14-15)} & \textbf{WCAG/WCAG2Mobile Criteria} \\
\hline
\endfirsthead
\multicolumn{4}{c}%
{{\bfseries Table \thetable\ -- continued from previous page}} \\
\hline
\textbf{Test Case} & \textbf{VoiceOver (iOS 16)} & \textbf{TalkBack (Android 14-15)} & \textbf{WCAG/WCAG2 \ Mobile Criteria} \\
\hline
\endhead
\hline
\multicolumn{4}{r}{{Continued on next page}} \\
\endfoot
\hline
\endlastfoot
Hero Title & \ding{51} Announces ``WCAG 2.2 Guidelines, heading'' & \ding{51} Announces ``WCAG 2.2 Guidelines, heading'' & 1.3.1 - Info and Relationships (Level A), 2.4.6 - Headings and Labels (Level AA) with WCAG2Mobile interpretation for screen context vs. web page \\
\hline
Principle Title & \ding{51} Announces principle title & \ding{51} Announces principle title & 1.3.1 Info and Relationships (Level A), 2.4.6 Headings and Labels (Level AA) with WCAG2Mobile guidance on screen landmark equivalents \\
\hline
Principle Description & \ding{51} Announces full description & \ding{51} Announces full description & 1.3.1 Info and Relationships (Level A) with WCAG2Mobile interpretation for mobile content structure \\
\hline
Checklist Items & \ding{51} Announces each item individually & \ding{51} Announces each item individually & 1.3.1 Info and Relationships (Level A), 1.3.2 Meaningful Sequence (Level A) with WCAG2Mobile guidance on touch-based navigation patterns \\
\hline
Decorative Icons & \ding{51} Not announced or focused & \ding{51} Not announced or focused & 1.1.1 Non-text Content (Level A) with WCAG2Mobile emphasis on swipe reduction for mobile navigation \\
\hline
Navigation Between Principles & \ding{51} Clear sequential navigation & \ding{51} Clear sequential navigation & 2.4.3 Focus Order (Level A) with WCAG2Mobile's screen-specific interpretation, 3.2.5 Change on Request (Level AAA) \\
\hline
\end{longtable}
\FloatBarrier

\paragraph{Implementation overhead analysis}

Table~\ref{tab:guidelines_implementation_overhead} quantifies the additional code required to implement accessibility features in the Guidelines screen.

\begin{longtable}[c]{|C{3.8cm}|C{2.3cm}|C{2.8cm}|C{2.8cm}|}
\caption{Guidelines screen accessibility implementation overhead}
\label{tab:guidelines_implementation_overhead}\\
\hline
\textbf{Accessibility Feature} & \textbf{Lines of Code} & \textbf{Percentage of Total} & \textbf{Complexity Impact} \\
\hline
\endfirsthead
\multicolumn{4}{c}%
{{\bfseries Table \thetable\ -- continued from previous page}} \\
\hline
\textbf{Accessibility Feature} & \textbf{Lines of Code} & \textbf{Percentage of Total} & \textbf{Complexity Impact} \\
\hline
\endhead
\hline
\multicolumn{4}{r}{{Continued on next page}} \\
\endfoot
\hline
\endlastfoot
Semantic Roles & 4 LOC & 0.7\% & Low \\
\hline
Element Hiding & 28 LOC & 5.1\% & Low \\
\hline
Focus Management & 2 LOC & 0.4\% & Low \\
\hline
Contrast Handling & 14 LOC & 2.5\% & Medium \\
\hline
Enhanced Target Size & 8 LOC & 1.4\% & Low \\
\hline
\textbf{Total} & \textbf{56 LOC} & \textbf{10.1\%} & \textbf{Low} \\
\hline
\end{longtable}
\FloatBarrier

This analysis reveals that implementing accessibility for the Guidelines screen adds approximately 10.1\% to the code base, which is notably lower than other screens. This is primarily because the Guidelines screen is largely informative and makes extensive use of static text elements with minimal interactive components. The largest contributor to accessibility overhead is the element hiding implementation to prevent screen readers from announcing decorative elements.

\paragraph{Mobile-specific considerations}

The Guidelines screen implementation addresses several mobile-specific accessibility considerations informed by WCAG2Mobile guidance:

\begin{enumerate}
    \item \textbf{Screen context vs. web page context}: The implementation applies WCAG2Mobile's interpretation of "web page" as "screen," ensuring that success criteria are appropriately adapted to the mobile context;
    
    \item \textbf{Touch-friendly card elevation}: Each principle card utilizes elevation effects (shadows) and appropriate spacing to create a clear visual hierarchy and delineation between content sections, improving touch accuracy and visual clarity as recommended by WCAG2Mobile's interpretation of SC 2.5.8;
    
    \item \textbf{Swipe efficiency optimization}: The implementation carefully eliminates "garbage interactions" by hiding decorative elements from screen readers, reducing the number of swipes required to navigate through the content—a critical consideration aligned with WCAG2Mobile's guidance on SC 4.1.2 for mobile screen reader users;
    
    \item \textbf{Consistent visual language}: The use of consistent iconography and color coding across principles creates a clear visual language that helps users quickly identify different sections, implementing WCAG2Mobile's guidance on SC 3.2.4 for mobile interfaces;
    
    \item \textbf{Proper landmark roles}: The implementation uses appropriate accessibility roles that align with WCAG2Mobile's guidance on adapting web landmark roles to mobile screen contexts.
\end{enumerate}

\subsubsection{Gestures tutorial screen}
\label{subsubsec:gestures-tutorial}

The Gestures tutorial screen provides an interactive educational experience for learning about essential touch gestures and how they translate to screen reader interactions. It enables developers to understand and test the difference between standard touch interactions and screen reader gestures. Figure~\ref{fig:gestures_screens_sidebyside} shows the main interface of this screen.

\begin{figure}[ht]
    \centering
    \begin{subfigure}[b]{0.48\textwidth}
        \centering
        \includegraphics[width=\linewidth, alt={First part of the Gestures tutorial screen}]{img/gestures1.jpg}
        \caption{Gestures tutorial screen - Part 1}
        \label{fig:gestures-left}
    \end{subfigure}
    \hfill
    \begin{subfigure}[b]{0.48\textwidth}
        \centering
        \includegraphics[width=\linewidth, alt={Second part of the Gestures tutorial screen}]{img/gestures2.jpg}
        \caption{Gestures tutorial screen - Part 2}
        \label{fig:gestures-right}
    \end{subfigure}
    \caption{Side-by-side view of the Gestures Tutorial screen sections}
    \label{fig:gestures_screens_sidebyside}
\end{figure}

\FloatBarrier

\paragraph{Component inventory and WCAG/MCAG/WCAG2Mobile mapping}

Table~\ref{tab:gestures_component_mapping} provides a formal mapping between the UI components, their semantic roles, the specific WCAG 2.2 criteria they address, their WCAG2Mobile considerations, and their React Native implementation properties.

\begin{longtable}[c]{|C{2.5cm}|C{2cm}|C{2.8cm}|C{3.5cm}|C{4.7cm}|}
\caption{Gestures tutorial screen component-criteria mapping with WCAG2Mobile considerations}
\label{tab:gestures_component_mapping}\\
\hline
\textbf{Component and Location} & \textbf{Semantic Role} & \textbf{WCAG 2.2 Criteria} & \textbf{WCAG2Mobile Considerations} & \textbf{Implementation Properties} \\
\hline
\endfirsthead
\multicolumn{5}{c}%
{{\bfseries Table \thetable\ -- continued from previous page}} \\
\hline
\textbf{Component and Location} & \textbf{Semantic Role} & \textbf{WCAG 2.2 Criteria} & \textbf{WCAG2Mobile Considerations} & \textbf{Implementation Properties} \\
\hline
\endhead
\hline
\multicolumn{5}{r}{{Continued on next page}} \\
\endfoot
\hline
\endlastfoot
Hero Title (top: "Gestures Tutorial") & heading & 1.4.3 Contrast (AA)\newline 2.4.6 Headings (AA) & Text readability on variable screen sizes; Header role in screen context per WCAG2Mobile interpretation & \texttt{accessibility \ Role="header"} \\
\hline
Practice Cards (gesture sections: Single Tap, Double Tap, Long Press) & none & 1.3.1 Info and Relationships (A) & Logical grouping of related gesture content in screen context rather than web page & Container with clear visual boundaries \\
\hline
Practice Title (section headers: "Single Tap", "Double Tap", etc.) & text & 2.4.6 Headings and Labels (AA) & Clear gesture type identification in mobile screen context & Text styling with appropriate weight and size \\
\hline
Practice Buttons (blue buttons: "Tap me!", "Double Tap me!", etc.) & button & 2.5.1 Pointer Gestures (A)\newline 2.5.2 Pointer Cancellation (A)\newline 4.1.2 Name, Role, Value (A)\newline 2.5.5 Target Size (Enhanced) (AAA) & WCAG2Mobile's emphasis on single-pointer alternatives for gestures; Touch target size exceeds minimum requirements for mobile; Gesture information provided for screen readers & \texttt{accessibility \ Role="button"},\newline \texttt{accessibility \ Label},\newline \texttt{accessibility \ Actions} \\
\hline
Success Feedback (green "successful" messages) & text & 4.1.3 Status Messages (AA) & Non-visual feedback for actions per WCAG2Mobile's guidance on mobile status messages & \texttt{accessibility \ LiveRegion="polite"} \\
\hline
Info Text (instructional text below buttons) & text & 3.3.2 Labels or Instructions (A)\newline 3.2.5 Change on Request (AAA) & Platform-specific gesture guidance addressing WCAG2Mobile's focus on platform accessibility services & Descriptive text with context-aware content \\
\hline
\end{longtable}
\FloatBarrier

\paragraph{Technical implementation analysis}

What makes the Gestures Tutorial screen particularly notable is its sophisticated handling of both standard touch interactions and screen reader interactions. The implementation detects when a screen reader is enabled and adapts the gesture behavior accordingly. Listing~\ref{lst:gestures-screen-accessibility} highlights the key implementation aspects.

\begin{lstlisting}[
  style=ReactNativeStyle,
  caption={Key implementation for accessible gesture detection with screen reader adaptation},
  label={lst:gestures-screen-accessibility},
  basicstyle=\ttfamily\footnotesize,
  numbers=left,
]
// Check if a screen reader is enabled
const [screenReaderEnabled, setScreenReaderEnabled] = useState(false);
useEffect(() => {
  AccessibilityInfo.isScreenReaderEnabled().then((enabled) => {
    setScreenReaderEnabled(enabled);
  });
  const listener = AccessibilityInfo.addEventListener('change', (enabled) => {
    setScreenReaderEnabled(enabled);
  });
  return () => listener.remove();
}, []);

// Double tap handler with screen reader adaptation
const handleDoubleTap = () => {
  if (screenReaderEnabled) {
    // If screen reader is active, show success immediately
    setShowDoubleTapSuccess(true);
    AccessibilityInfo.announceForAccessibility(
      'Double tap gesture completed successfully with screen reader'
    );
    setTimeout(() => setShowDoubleTapSuccess(false), 1500);
    return;
  }

  // Standard double tap detection for users without screen readers
  const now = Date.now();
  if (lastTap && now - lastTap < DOUBLE_TAP_DELAY) {
    setShowDoubleTapSuccess(true);
    AccessibilityInfo.announceForAccessibility(
      'Double tap gesture completed successfully'
    );
    setTimeout(() => setShowDoubleTapSuccess(false), 1500);
    setLastTap(0); // reset
  } else {
    setLastTap(now);
  }
};
\end{lstlisting}
\FloatBarrier

Another key aspect shown by Listing~\ref{lst:gestures-actions-accessibility} of the implementation is the addition of accessibility actions that enable screen reader users to simulate gestures that would otherwise be difficult to perform with a screen reader enabled:

\begin{lstlisting}[
  style=ReactNativeStyle,
  caption={Implementation of accessibility actions for gesture simulation},
  label={lst:gestures-actions-accessibility},
  basicstyle=\ttfamily\footnotesize,
  numbers=left,
]
<TouchableOpacity
  style={themedStyles.practiceButton}
  onLongPress={handleLongPress}
  accessibilityRole="button"
  accessibilityLabel="Practice long press"
  accessibilityActions={[
    { name: 'activate', label: 'Activate long press' },
    { name: 'longpress', label: 'Simulate long press' }
  ]}
  onAccessibilityAction={(event) => {
    if (event.nativeEvent.actionName === 'activate' ||
        event.nativeEvent.actionName === 'longpress') {
      handleLongPress();
    }
  }}
  accessibilityState={{
    disabled: false,
    busy: showLongPressSuccess
  }}
>
  <Text style={themedStyles.practiceButtonText}>Long Press me!</Text>
</TouchableOpacity>
\end{lstlisting}
\FloatBarrier

The implementation addresses several critical accessibility considerations:

\begin{enumerate}
    \item \textbf{Screen reader detection and adaptation}: The code actively detects when a screen reader is enabled and modifies its behavior to accommodate screen reader users, providing an equivalent experience through alternative interaction methods;
    
    \item \textbf{Accessibility actions for gesture simulation}: Custom accessibility actions are provided to allow screen reader users to simulate gestures that would otherwise be difficult to perform while using a screen reader;
    
    \item \textbf{Context-sensitive instructions}: The information text dynamically changes based on whether a screen reader is enabled, providing relevant guidance based on the user's current assistive technology setup;
    
    \item \textbf{Status announcements}: All gesture completions are explicitly announced via \\\texttt{AccessibilityInfo.announceForAccessibility}, ensuring non-visual feedback;
    
    \item \textbf{Visual feedback with accessibility considerations}: Success messages are displayed visually but also properly marked with \texttt{accessibilityLiveRegion="polite"} to ensure screen readers announce them appropriately;
    
    \item \textbf{Enhanced target size (AAA)}: All interactive elements exceed the enhanced target size requirements of 2.5.5 (AAA), with adequate padding and touch area.
\end{enumerate}

\paragraph{Screen reader support analysis}

Table~\ref{tab:gestures_screen_reader_analysis} presents results from systematic testing of the Gestures tutorial screen with screen readers on both iOS and Android platforms, specifically noting WCAG2Mobile interpretations.

\begin{longtable}[c]{|C{2.8cm}|C{3.5cm}|C{3.5cm}|C{5cm}|}
\caption{Gestures tutorial screen screen reader testing results with WCAG2Mobile considerations}
\label{tab:gestures_screen_reader_analysis}\\
\hline
\textbf{Test Case} & \textbf{VoiceOver (iOS 16)} & \textbf{TalkBack (Android 14-15)} & \textbf{WCAG/WCAG2Mobile Criteria} \\
\hline
\endfirsthead
\multicolumn{4}{c}%
{{\bfseries Table \thetable\ -- continued from previous page}} \\
\hline
\textbf{Test Case} & \textbf{VoiceOver (iOS 16)} & \textbf{TalkBack (Android 14-15)} & \textbf{WCAG/WCAG2Mobile Criteria} \\
\hline
\endhead
\hline
\multicolumn{4}{r}{{Continued on next page}} \\
\endfoot
\hline
\endlastfoot
Single Tap Button & \ding{51} Announces label & \ding{51} Announces label & 4.1.2 Name, Role, Value (A) with WCAG2Mobile mobile platform adaptation \\
\hline
Double Tap Button with SR & \ding{51} Simulates double tap with single activation & \ding{51} Simulates double tap with single activation & 2.5.1 Pointer Gestures (A) - WCAG2Mobile's emphasis on single-pointer alternatives for complex gestures \\
\hline
Long Press Button with SR & \ding{51} Provides custom action for long press & \ding{51} Provides custom action for long press & 2.5.1 Pointer Gestures (A) - WCAG2Mobile's requirement for gesture alternatives in mobile contexts \\
\hline
Success Feedback & \ding{51} Announces success messages & \ding{51} Announces success messages & 4.1.3 Status Messages (AA) with WCAG2Mobile's mobile notification context \\
\hline
Context-Sensitive Instructions & \ding{51} Provides SR-specific instructions & \ding{51} Provides SR-specific instructions & 3.3.2 Labels or Instructions (A) with WCAG2Mobile platform-specific adaptation, 3.2.5 Change on Request (AAA) \\
\hline
\end{longtable}
\FloatBarrier

\paragraph{Implementation overhead analysis}

Table~\ref{tab:gestures_implementation_overhead} quantifies the additional code required to implement accessibility features in the Gestures tutorial screen.

\begin{longtable}[c]{|C{3.8cm}|C{2.3cm}|C{2.8cm}|C{2.8cm}|}
\caption{Gestures tutorial screen accessibility implementation overhead}
\label{tab:gestures_implementation_overhead}\\
\hline
\textbf{Accessibility Feature} & \textbf{Lines of Code} & \textbf{Percentage of Total} & \textbf{Complexity Impact} \\
\hline
\endfirsthead
\multicolumn{4}{c}%
{{\bfseries Table \thetable\ -- continued from previous page}} \\
\hline
\textbf{Accessibility Feature} & \textbf{Lines of Code} & \textbf{Percentage of Total} & \textbf{Complexity Impact} \\
\hline
\endhead
\hline
\multicolumn{4}{r}{{Continued on next page}} \\
\endfoot
\hline
\endlastfoot
Screen Reader Detection & 12 LOC & 2.8\% & Medium \\
\hline
Semantic Roles & 6 LOC & 1.4\% & Low \\
\hline
Accessibility Actions & 22 LOC & 5.2\% & High \\
\hline
Descriptive Labels & 12 LOC & 2.8\% & Low \\
\hline
Status Announcements & 18 LOC & 4.2\% & Medium \\
\hline
Adaptive Logic & 34 LOC & 8.0\% & High \\
\hline
\textbf{Total} & \textbf{104 LOC} & \textbf{24.4\%} & \textbf{Medium-High} \\
\hline
\end{longtable}
\FloatBarrier

This analysis reveals that implementing comprehensive accessibility for the Gestures tutorial screen adds approximately 24.4\% to the code base, which is relatively high compared to other screens. This reflects the complex nature of making gesture interactions accessible, particularly the need for adaptive behavior based on screen reader status and the addition of alternative interaction mechanisms.

\paragraph{Mobile-specific considerations}

The Gestures tutorial screen addresses several critical mobile-specific accessibility considerations that are particularly relevant to touch-based platforms and align with WCAG2Mobile guidance:

\begin{enumerate}
    \item \textbf{Alternative input methods}: The implementation provides multiple ways to perform each gesture, accommodating different user capabilities and assistive technologies—directly implementing WCAG2Mobile Success Criterion 2.5.1 which requires single-pointer alternatives for complex gestures;
    
    \item \textbf{Educational comparison}: By explicitly showing the difference between standard gestures and screen reader gestures, the screen serves an important educational function, helping developers understand the distinction between these interaction models as emphasized in WCAG2Mobile's interpretation of touch-based interactions;
    
    \item \textbf{Device adaptation}: The implementation detects the current device state (screen reader enabled/disabled) and adapts its behavior and instructions accordingly, implementing WCAG2Mobile's guidance on adapting to platform accessibility services;
    
    \item \textbf{Custom actions for complex gestures}: The addition of custom accessibility actions enables screen reader users to simulate complex gestures, implementing WCAG2Mobile's emphasis on alternative interaction patterns for gesture-heavy mobile interfaces;
    
    \item \textbf{Platform-specific screen reader support}: The implementation accommodates both VoiceOver and TalkBack, recognizing the platform-specific differences highlighted in WCAG2Mobile's approach to mobile accessibility.
\end{enumerate}

\subsubsection{Logical navigation screen}
\label{subsubsec:logical-navigation}

The Logical navigation screen demonstrates techniques for implementing accessible navigation patterns, particularly the "Skip to Main Content" pattern that allows users to bypass repetitive navigation elements. This pattern is particularly important for screen reader users on mobile devices, where navigating through repetitive content can be especially time-consuming. Figure~\ref{fig:logical_screens_sidebyside} shows the main interface of this screen.

\begin{figure}[ht]
    \centering
    \begin{subfigure}[b]{0.48\textwidth}
        \centering
        \includegraphics[width=\linewidth, alt={First part of the Logical navigation screen}]{img/logical1.jpg}
        \caption{Logical navigation screen - Part 1}
        \label{fig:logical-left}
    \end{subfigure}
    \hfill
    \begin{subfigure}[b]{0.48\textwidth}
        \centering
        \includegraphics[width=\linewidth, alt={Second part of the Logical navigation screen}]{img/logical2.jpg}
        \caption{Logical navigation screen - Part 2}
        \label{fig:logical-right}
    \end{subfigure}
    \caption{Side-by-side view of the Logical navigation screen sections}
    \label{fig:logical_screens_sidebyside}
\end{figure}

\FloatBarrier

\paragraph{Component inventory and WCAG/MCAG/WCAG2Mobile mapping}

Table~\ref{tab:navigation_component_mapping} provides a mapping between the UI components, their semantic roles, the specific WCAG 2.2 criteria they address, their WCAG2Mobile considerations, and their React Native implementation properties.

\begin{longtable}[c]{|C{2.5cm}|C{2cm}|C{2.8cm}|C{3.5cm}|C{4.7cm}|}
\caption{Logical navigation screen component-criteria mapping with WCAG2Mobile considerations}
\label{tab:navigation_component_mapping}\\
\hline
\textbf{Component and Location} & \textbf{Semantic Role} & \textbf{WCAG 2.2 Criteria} & \textbf{WCAG2Mobile Considerations} & \textbf{Implementation Properties} \\
\hline
\endfirsthead
\multicolumn{5}{c}%
{{\bfseries Table \thetable\ -- continued from previous page}} \\
\hline
\textbf{Component and Location} & \textbf{Semantic Role} & \textbf{WCAG 2.2 Criteria} & \textbf{WCAG2Mobile Considerations} & \textbf{Implementation Properties} \\
\hline
\endhead
\hline
\multicolumn{5}{r}{{Continued on next page}} \\
\endfoot
\hline
\endlastfoot
Hero Title (top: "Logical Focus Order") & heading & 1.4.3 Contrast (AA)\newline 2.4.6 Headings (AA) & Text readability across variable screen sizes; Header role in screen context per WCAG2Mobile interpretation & \texttt{accessibility \ Role="header"} \\
\hline
Skip Link Button (blue button: "Skip to Main Content") & button & 2.4.1 Bypass Blocks (A)\newline 2.5.8 Target Size (AA) & Implementation of "bypass blocks" specific to mobile screen context per WCAG2Mobile; Touch target sized for mobile interaction & \texttt{accessibility \ Role="button"},\newline \texttt{accessibility \ Label="Skip to main content"} \\
\hline
Info Cards (sections explaining focus order) & none & 1.3.1 Info and Relationships (A) & Logical grouping of content in screen context rather than web page per WCAG2Mobile & Container with clear visual boundaries \\
\hline
Main Content Container (lower section: "Main Content") & summary & 1.3.1 Info and Relationships (A)\newline 2.4.1 Bypass Blocks (A) & Proper destination for skip links in mobile context per WCAG2Mobile; Screen-based landmark equivalent & \texttt{accessibility \ Role="summary"},\newline \texttt{accessibility \ Label="Main Content Section"} \\
\hline
Form Inputs (text field: "Enter feedback") & none & 3.3.2 Labels or Instructions (A)\newline 4.1.2 Name, Role, Value (A) & Mobile-specific input labeling per WCAG2Mobile; Touch-friendly input implementation & \texttt{accessibility \ Label="Feedback Input"} \\
\hline
Interactive Buttons (blue buttons: "Focusable Button 1", "Submit Feedback") & button & 2.5.8 Target Size (AA)\newline 4.1.2 Name, Role, Value (A) & Touch target sizing exceeding WCAG2Mobile recommendations; Mobile-specific button implementation & \texttt{accessibility \ Role="button"},\newline \texttt{accessibility \ Label} \\
\hline
\end{longtable}
\FloatBarrier

\paragraph{Technical implementation analysis}

The most significant accessibility feature in this screen is the implementation of the "Skip to Main Content" pattern. This pattern allows users, particularly those using screen readers, to bypass repetitive content and navigate directly to the main content area. Listing~\ref{lst:logical-screen-accessibility} highlights the key implementation aspects.

\begin{lstlisting}[
  style=ReactNativeStyle,
  caption={Implementation of Skip to Main Content pattern},
  label={lst:logical-screen-accessibility},
  basicstyle=\ttfamily\footnotesize,
  numbers=left,
]
// References for focus management
const scrollViewRef = useRef<ScrollView>(null);
const mainContentRef = useRef<View>(null);
const [mainContentY, setMainContentY] = useState(0);

// Capture the y-offset of the main content after layout
const handleMainContentLayout = (e: NativeSyntheticEvent<LayoutChangeEvent>) => {
  const { y } = e.nativeEvent.layout;
  setMainContentY(y);
};

// "Skip to main content" logic
const skipToMainContent = () => {
  // 1. Scroll to the main content
  scrollViewRef.current?.scrollTo({
    y: mainContentY,
    animated: true,
  });

  // 2. After a short delay, set accessibility focus to the main content container
  setTimeout(() => {
    if (mainContentRef.current && Platform.OS !== 'web') {
      const reactTag = findNodeHandle(mainContentRef.current);
      if (reactTag) {
        AccessibilityInfo.setAccessibilityFocus(reactTag);
      }
    }
  }, 500);
};
\end{lstlisting}
\FloatBarrier

This involves several key steps:

\begin{enumerate}
    \item \textbf{Tracking content position}: The code tracks the vertical position of the main content area using the \texttt{onLayout} event;
    
    \item \textbf{Programmatic scrolling}: When the skip link is activated, the screen scrolls programmatically to the main content area;
    
    \item \textbf{Focus management}: After scrolling, the code explicitly sets the accessibility focus to the main content area using \texttt{AccessibilityInfo.setAccessibilityFocus}, ensuring screen reader users are properly positioned after skipping;
    
    \item \textbf{Platform adaptation}: The implementation accounts for platform differences, ensuring the pattern works on both iOS and Android devices.
\end{enumerate}

\paragraph{Screen reader support analysis}

Table~\ref{tab:navigation_screen_reader_analysis} presents results from systematic testing of the Logical navigation screen with screen readers on both iOS and Android platforms, with specific WCAG2Mobile considerations noted.

\begin{longtable}[c]{|C{2.8cm}|C{3.5cm}|C{3.5cm}|C{5cm}|}
\caption{Logical navigation screen screen reader testing results with WCAG2Mobile considerations}
\label{tab:navigation_screen_reader_analysis}\\
\hline
\textbf{Test Case} & \textbf{VoiceOver (iOS 16)} & \textbf{TalkBack (Android 14-15)} & \textbf{WCAG/WCAG2Mobile Criteria} \\
\hline
\endfirsthead
\multicolumn{4}{c}%
{{\bfseries Table \thetable\ -- continued from previous page}} \\
\hline
\textbf{Test Case} & \textbf{VoiceOver (iOS 16)} & \textbf{TalkBack (Android 14-15)} & \textbf{WCAG/WCAG2Mobile Criteria} \\
\hline
\endhead
\hline
\multicolumn{4}{r}{{Continued on next page}} \\
\endfoot
\hline
\endlastfoot
Skip Link Button & \ding{51} Announces "Skip to main content, button" & \ding{51} Announces "Skip to main content, button" & 2.4.1 Bypass Blocks (A) with WCAG2Mobile's screen-specific interpretation \\
\hline
Skip Link Activation & \ding{51} Moves focus to main content & \ding{51} Moves focus to main content & 2.4.3 Focus Order (A) with WCAG2Mobile's mobile focus management considerations \\
\hline
Main Content Container & \ding{51} Receives focus properly after skip & \ding{51} Receives focus properly after skip & 4.1.2 Name, Role, Value (A) with WCAG2Mobile's guidance on landmark role equivalents \\
\hline
Form Input Focus & \ding{51} Clear focus indication & \ding{51} Clear focus indication & 2.4.7 Focus Visible (AA) with WCAG2Mobile's mobile platform adaptation \\
\hline
Button Activation & \ding{51} Announces action results & \ding{51} Announces action results & 4.1.3 Status Messages (AA) with WCAG2Mobile's mobile notification context \\
\hline
\end{longtable}
\FloatBarrier

\paragraph{Implementation overhead analysis}

Table~\ref{tab:navigation_implementation_overhead} quantifies the additional code required to implement accessibility features in the Logical navigation screen, including WCAG2Mobile-specific adaptations.

\begin{longtable}[c]{|C{3.8cm}|C{2.3cm}|C{2.8cm}|C{2.8cm}|}
\caption{Logical navigation screen accessibility implementation overhead with WCAG2Mobile considerations}
\label{tab:navigation_implementation_overhead}\\
\hline
\textbf{Accessibility Feature} & \textbf{Lines of Code} & \textbf{Percentage of Total} & \textbf{Complexity Impact} \\
\hline
\endfirsthead
\multicolumn{4}{c}%
{{\bfseries Table \thetable\ -- continued from previous page}} \\
\hline
\textbf{Accessibility Feature} & \textbf{Lines of Code} & \textbf{Percentage of Total} & \textbf{Complexity Impact} \\
\hline
\endhead
\hline
\multicolumn{4}{r}{{Continued on next page}} \\
\endfoot
\hline
\endlastfoot
Skip Link Implementation & 32 LOC & 8.1\% & High \\
\hline
Focus Management & 24 LOC & 6.1\% & High \\
\hline
Semantic Roles & 8 LOC & 2.0\% & Low \\
\hline
Accessibility Labels & 12 LOC & 3.0\% & Low \\
\hline
Platform-Specific Adaptations & 6 LOC & 1.5\% & Medium \\
\hline
WCAG2Mobile-Specific Adaptations & 10 LOC & 2.5\% & Medium \\
\hline
\textbf{Total} & \textbf{92 LOC} & \textbf{23.2\%} & \textbf{Medium-High} \\
\hline
\end{longtable}
\FloatBarrier

This analysis reveals that implementing accessibility for the Logical navigation screen adds approximately 23.2\% to the code base, with the skip link implementation and focus management being the most significant contributors. The WCAG2Mobile-specific adaptations (2.5\%) account for the additional code required to properly implement bypass blocks in a mobile context rather than a web context, including platform-specific focus management techniques.

\paragraph{Mobile-specific considerations}

The Logical navigation screen addresses several mobile-specific accessibility considerations informed by WCAG2Mobile guidance:

\begin{enumerate}
    \item \textbf{Limited viewport management}: Mobile screens have limited viewport space, making it more critical to provide efficient navigation mechanisms that reduce scrolling and swiping—the skip link directly addresses this constraint as emphasized in WCAG2Mobile's guidance on SC 2.4.1;
    
    \item \textbf{Touch-optimized implementation}: The skip link is implemented with adequate touch target size and clear visual feedback, making it usable for touch users with various motor capabilities, aligning with WCAG2Mobile's interpretation of SC 2.5.8;
    
    \item \textbf{Platform-specific focus management}: The implementation accounts for differences in how iOS and Android handle accessibility focus, adhering to WCAG2Mobile's guidance on platform accessibility services;
    
    \item \textbf{Smooth scrolling with focus synchronization}: The implementation coordinates visual scrolling with accessibility focus changes, maintaining a consistent experience that doesn't disorient users—a critical consideration in WCAG2Mobile's approach to mobile navigation;
    
    \item \textbf{Screen-based landmarks}: The implementation uses appropriate accessibility roles that align with WCAG2Mobile's guidance on adapting web landmark roles to mobile screen contexts.
\end{enumerate}

\subsubsection{Screen reader support screen}
\label{subsubsec:screen-reader-support}

The Screen reader support screen provides platform-specific guidance for optimizing applications for VoiceOver (iOS) and TalkBack (Android). It offers developers insight into how screen readers work on mobile platforms and specific gestures users employ to navigate content. Figure~\ref{fig:screen_reader_screens_sidebyside} shows the main interface of this screen.

\begin{figure}[ht]
    \centering
    \begin{subfigure}[b]{0.48\textwidth}
        \centering
        \includegraphics[width=\linewidth, alt={First part of the Screen reader support screen}]{img/screenreader1.jpg}
        \caption{Screen reader support screen - Part 1}
        \label{fig:screen-reader-left}
    \end{subfigure}
    \hfill
    \begin{subfigure}[b]{0.48\textwidth}
        \centering
        \includegraphics[width=\linewidth, alt={Second part of the Screen reader support screen}]{img/screenreader2.jpg}
        \caption{Screen reader support screen - Part 2}
        \label{fig:screen-reader-right}
    \end{subfigure}
    \caption{Side-by-side view of the Screen reader support screen sections}
    \label{fig:screen_reader_screens_sidebyside}
\end{figure}

\FloatBarrier

\paragraph{Component inventory and WCAG/MCAG/WCAG2Mobile mapping}

Table~\ref{tab:screen_reader_component_mapping} provides a mapping between the UI components, their semantic roles, the specific WCAG 2.2 criteria they address, their WCAG2Mobile considerations, and their React Native implementation properties.

\begin{longtable}[c]{|C{2.5cm}|C{2cm}|C{2.8cm}|C{3.5cm}|C{4.7cm}|}
\caption{Screen reader support screen component-criteria mapping with WCAG2Mobile considerations}
\label{tab:screen_reader_component_mapping}\\
\hline
\textbf{Component and Location} & \textbf{Semantic Role} & \textbf{WCAG 2.2 Criteria} & \textbf{WCAG2Mobile Considerations} & \textbf{Implementation Properties} \\
\hline
\endfirsthead
\multicolumn{5}{c}%
{{\bfseries Table \thetable\ -- continued from previous page}} \\
\hline
\textbf{Component and Location} & \textbf{Semantic Role} & \textbf{WCAG 2.2 Criteria} & \textbf{WCAG2Mobile Considerations} & \textbf{Implementation Properties} \\
\hline
\endhead
\hline
\multicolumn{5}{r}{{Continued on next page}} \\
\endfoot
\hline
\endlastfoot
Hero Title (top: "Screen Reader Support") & heading & 1.4.3 Contrast (AA)\newline 2.4.6 Headings (AA)\newline 2.4.10 Section Headings (AAA) & Text readability across variable screen sizes; Header role in screen context per WCAG2Mobile & \texttt{accessibility \ Role="header"} \\
\hline
Platform Toggle Buttons (VoiceOver \ \& TalkBack buttons) & button & 4.1.2 Name, Role, Value (A)\newline 2.5.8 Target Size (AA)\newline 2.5.5 Target Size (Enhanced) (AAA) & Touch target size exceeding WCAG2Mobile recommendations; Platform selection with proper state indication & \texttt{accessibility \ Role="button"},\newline \texttt{accessibility \ State=\{\{selected: ...\}\}} \\
\hline
Platform Icons (Apple/Android icons in buttons) & none & 1.1.1 Non-text Content (A) & Reduction of focus stops for efficient screen reader navigation per WCAG2Mobile & \texttt{accessibility \ ElementsHidden=true},\newline \texttt{importantFor \ Accessibility \ ="no-hide-descendants"} \\
\hline
Gesture Items (Essential Features list: Single tap, Double tap, etc.) & text & 1.3.1 Info and Relationships (A) & Gesture description specific to mobile platform context per WCAG2Mobile & \texttt{accessibility \ Role="text"},\newline \texttt{accessibility \ Label=\`\${item.gesture}: \${item.action}\`} \\
\hline
Implementation Guide Cards (sections: Semantic Structure, Content Descriptions) & none & 1.3.1 Info and Relationships (A)\newline 2.4.10 Section Headings (AAA) & Logical grouping in mobile screen context per WCAG2Mobile & Container with proper visual boundaries \\
\hline
Guide Title (card headers: "Semantic Structure", "Content Descriptions") & text & 2.4.6 Headings and Labels (AA) & Content section identification in screen context per WCAG2Mobile & Semantic text styling with proper hierarchy \\
\hline
Checklist Items (bulleted lists within guide cards) & text & 1.3.1 Info and Relationships (A) & Grouped related information for screen reader navigation per WCAG2Mobile & Parent container with contextual organization \\
\hline
Code Examples (dark code blocks) & text & 1.3.1 Info and Relationships (A) & Code accessibility in mobile context with appropriate labels per WCAG2Mobile & \texttt{accessibility \ Role="text"} \\
\hline
Toggle Code Examples Button ("View Code Examples" links) & button & 3.2.5 Change on Request (AAA) & User control with clear state indication per WCAG2Mobile & \texttt{accessibility \ Role="button"},\newline \texttt{accessibility \ Label} \\
\hline
\end{longtable}
\FloatBarrier

\paragraph{Technical implementation analysis}

A distinguishing feature of this screen is the implementation of platform-specific content that dynamically changes based on the selected platform (iOS or Android). Listing~\ref{lst:screen-reader-screen-accessibility} highlights the key implementation aspects.

\begin{lstlisting}[
  style=ReactNativeStyle,
  caption={Platform toggle implementation with accessibility state},
  label={lst:screen-reader-screen-accessibility},
  basicstyle=\ttfamily\footnotesize,
  numbers=left,
]
{/* Platform toggle buttons with accessibility state */}
<View style={themedStyles.platformToggles}>
  <TouchableOpacity
    style={[
      themedStyles.platformButton,
      activeSection === 'ios' && themedStyles.platformButtonActive,
    ]}
    onPress={() => setActiveSection('ios')}
    accessibilityRole="button"
    accessibilityState={{ selected: activeSection === 'ios' }}
    accessibilityLabel="VoiceOver iOS guide"
  >
    <Ionicons
      name="logo-apple"
      size={24}
      color={activeSection === 'ios' ? colors.background : colors.text}
      style={themedStyles.platformIcon}
      accessibilityElementsHidden={true}
      importantForAccessibility="no-hide-descendants"
    />
    <Text
      style={[
        themedStyles.platformLabel,
        activeSection === 'ios' && themedStyles.platformLabelActive,
      ]}
    >
      VoiceOver (iOS)
    </Text>
  </TouchableOpacity>
  
  {/* Similar implementation for Android toggle */}
  
  {/* Conditional content display */}
  {activeSection && (
    <View style={themedStyles.gestureGuideContainer}>
      <Text style={themedStyles.gestureTitle}>Essential Gestures</Text>
      {platformSpecificGuides[activeSection].map((item, index) => (
        <View
          key={index}
          style={themedStyles.gestureItem}
          accessibilityRole="text"
          accessibilityLabel={`${item.gesture}: ${item.action}`}
        >
          {/* Gesture item content */}
        </View>
      ))}
    </View>
  )}
</View>
\end{lstlisting}
\FloatBarrier

Another notable feature is the implementation of accessible code examples with toggling functionality, as shown in Listing~\ref{lst:screen-reader-code-examples}.

\begin{lstlisting}[
  style=ReactNativeStyle,
  caption={Accessible code examples with toggle functionality},
  label={lst:screen-reader-code-examples},
  basicstyle=\ttfamily\footnotesize,
  numbers=left,
]
{expandedSections.semanticStructure && (
  <View style={themedStyles.codeExampleContainer} accessibilityRole="text">
    <Text style={themedStyles.codeText}>
      {codeExamples.semanticStructure}
    </Text>
  </View>
)}

<TouchableOpacity
  style={themedStyles.learnMoreButton}
  accessibilityRole="button"
  accessibilityLabel={expandedSections.semanticStructure ? 
    "Hide semantic structure code examples" : 
    "View semantic structure code examples"}
  onPress={() => toggleSection('semanticStructure')}
>
  <Text style={themedStyles.learnMoreText}>
    {expandedSections.semanticStructure ? "Hide Code Examples" : "View Code Examples"}
  </Text>
  <Ionicons
    name={expandedSections.semanticStructure ? "arrow-up" : "arrow-forward"}
    size={16}
    color={colors.primary}
    accessibilityElementsHidden={true}
    importantForAccessibility="no-hide-descendants"
  />
</TouchableOpacity>
\end{lstlisting}
\FloatBarrier

The implementation addresses several important accessibility considerations:

\begin{enumerate}
    \item \textbf{Selection state communication}: The platform toggle buttons properly communicate their selection state using \texttt{accessibilityState=\{\{selected: activeSection === 'platform'\}\}}, ensuring screen reader users understand which platform is currently active;
    
    \item \textbf{Comprehensive accessibility labels}: Gesture items combine the gesture name and action into a single accessibility label (\texttt{accessibilityLabel=`\${item.gesture}: \${item.action}`}), providing complete context in a single focus stop;
    
    \item \textbf{Hiding decorative icons}: All decorative icons are properly hidden from screen readers while maintaining their visual presence;
    
    \item \textbf{Semantic grouping}: Related information is grouped semantically, ensuring screen reader users understand the relationships between different pieces of content;
    
    \item \textbf{Dynamic accessibility label}: Toggle buttons adjust their accessibility properties based on their current state, ensuring screen reader users receive appropriate context-sensitive information.
\end{enumerate}

\paragraph{Screen reader support analysis}

Table~\ref{tab:screen_reader_sr_analysis} presents results from systematic testing of the Screen reader support screen with screen readers on both iOS and Android platforms, with specific WCAG2Mobile considerations noted.

\begin{longtable}[c]{|C{2.8cm}|C{3.5cm}|C{3.5cm}|C{5cm}|}
\caption{Screen reader support screen testing results with WCAG2Mobile considerations}
\label{tab:screen_reader_sr_analysis}\\
\hline
\textbf{Test Case} & \textbf{VoiceOver (iOS 16)} & \textbf{TalkBack (Android 14-15)} & \textbf{WCAG/WCAG2Mobile Criteria} \\
\hline
\endfirsthead
\multicolumn{4}{c}%
{{\bfseries Table \thetable\ -- continued from previous page}} \\
\hline
\textbf{Test Case} & \textbf{VoiceOver (iOS 16)} & \textbf{TalkBack (Android 14-15)} & \textbf{WCAG/WCAG2Mobile Criteria} \\
\hline
\endhead
\hline
\multicolumn{4}{r}{{Continued on next page}} \\
\endfoot
\hline
\endlastfoot
Platform Tabs & \ding{51} Announces tab and selection state & \ding{51} Announces tab and selection state & 4.1.2 Name, Role, Value (A) with WCAG2Mobile's mobile UI component context \\
\hline
Gesture Item Navigation & \ding{51} Announces combined gesture and action & \ding{51} Announces combined gesture and action & 1.3.1 Info and Relationships (A) with WCAG2Mobile's mobile navigation optimization \\
\hline
Content Toggle Buttons & \ding{51} Announces state changes & \ding{51} Announces state changes & 3.2.5 Change on Request (AAA) with WCAG2Mobile's mobile interaction context \\
\hline
Code Examples & \ding{51} Announces as text with context & \ding{51} Announces as text with context & 1.3.1 Info and Relationships (A) with WCAG2Mobile's mobile content structure guidance \\
\hline
Platform-Specific Content & \ding{51} Displays appropriate content & \ding{51} Displays appropriate content & WCAG2Mobile's platform accessibility services emphasis \\
\hline
\end{longtable}
\FloatBarrier

\paragraph{Implementation overhead analysis}

Table~\ref{tab:screen_reader_implementation_overhead} quantifies the additional code required to implement accessibility features in the Screen reader support screen, including WCAG2Mobile-specific adaptations.

\begin{longtable}[c]{|C{3.8cm}|C{2.3cm}|C{2.8cm}|C{2.8cm}|}
\caption{Screen reader support screen accessibility implementation overhead with WCAG2Mobile considerations}
\label{tab:screen_reader_implementation_overhead}\\
\hline
\textbf{Accessibility Feature} & \textbf{Lines of Code} & \textbf{Percentage of Total} & \textbf{Complexity Impact} \\
\hline
\endfirsthead
\multicolumn{4}{c}%
{{\bfseries Table \thetable\ -- continued from previous page}} \\
\hline
\textbf{Accessibility Feature} & \textbf{Lines of Code} & \textbf{Percentage of Total} & \textbf{Complexity Impact} \\
\hline
\endhead
\hline
\multicolumn{4}{r}{{Continued on next page}} \\
\endfoot
\hline
\endlastfoot
Platform Toggle Selection States & 14 LOC & 2.5\% & Medium \\
\hline
Element Hiding & 22 LOC & 4.0\% & Low \\
\hline
Comprehensive Labels & 18 LOC & 3.3\% & Medium \\
\hline
Platform-Specific Content & 20 LOC & 3.6\% & High \\
\hline
Semantic Structure & 12 LOC & 2.2\% & Low \\
\hline
WCAG2Mobile-Specific Adaptations & 8 LOC & 1.5\% & Medium \\
\hline
\textbf{Total} & \textbf{94 LOC} & \textbf{17.1\%} & \textbf{Medium} \\
\hline
\end{longtable}
\FloatBarrier

This analysis reveals that implementing accessibility for the Screen reader support screen adds approximately 17.1\% to the code base, with platform-specific content being the most complex component. The WCAG2Mobile-specific adaptations (1.5\%) account for the platform-specific considerations required to properly implement accessibility in a mobile context.

\paragraph{Mobile-specific considerations}

The Screen reader support screen addresses several mobile-specific accessibility considerations informed by WCAG2Mobile guidance:

\begin{enumerate}
    \item \textbf{Platform-specific guidance}: By explicitly separating iOS and Android guidance, the screen acknowledges the significant differences between VoiceOver and TalkBack highlighted in WCAG2Mobile's approach to platform accessibility services;
    
    \item \textbf{Gesture documentation}: The screen catalogs the specific gestures used by screen reader users on mobile platforms, implementing WCAG2Mobile's guidance on providing alternatives for touch-based interactions;
    
    \item \textbf{Implementation context}: By providing both gesture information and implementation guidance on the same screen, developers can directly connect user interaction patterns with the code required to support them—an approach that aligns with WCAG2Mobile's emphasis on practical implementation;
    
    \item \textbf{Touch-friendly interface}: The implementation maintains a touch-friendly interface with adequate target sizes and clear visual feedback, implementing WCAG2Mobile's guidance on SC 2.5.8 for mobile contexts;
    
    \item \textbf{Screen-based structure}: The implementation uses appropriate accessibility roles that align with WCAG2Mobile's guidance on adapting web roles to mobile screen contexts.
\end{enumerate}

\subsubsection{Semantic structure screen}
\label{subsubsec:semantic-structure}

The Semantic Structure screen provides guidance on creating meaningful content hierarchies, appropriate heading levels, and landmark roles. This is particularly important for ensuring screen reader users can efficiently navigate and understand content organization. Figure~\ref{fig:semantics_screens_sidebyside} shows the main interface of this screen.

\begin{figure}[ht]
    \centering
    \begin{subfigure}[b]{0.48\textwidth}
        \centering
        \includegraphics[width=\linewidth, alt={First part of the Semantic structure screen}]{img/semantics1.jpg}
        \caption{Semantic structure screen - Part 1}
        \label{fig:semantics-left}
    \end{subfigure}
    \hfill
    \begin{subfigure}[b]{0.48\textwidth}
        \centering
        \includegraphics[width=\linewidth, alt={Second part of the Semantic structure screen}]{img/semantics2.jpg}
        \caption{Semantic structure screen - Part 2}
        \label{fig:semantics-right}
    \end{subfigure}
    \caption{Side-by-side view of the Semantic Structure screen sections}
    \label{fig:semantics_screens_sidebyside}
\end{figure}

\FloatBarrier

\paragraph{Component inventory and WCAG/MCAG mapping}

Table~\ref{tab:semantics_component_mapping} provides a mapping between the UI components, their semantic roles, the specific WCAG 2.2 criteria they address, their WCAG2Mobile considerations, and their React Native implementation properties.

\begin{longtable}[c]{|C{2.5cm}|C{2cm}|C{2.8cm}|C{3.5cm}|C{4.7cm}|}
\caption{Semantic structure screen component-criteria mapping with WCAG2Mobile considerations}
\label{tab:semantics_component_mapping}\\
\hline
\textbf{Component and Location} & \textbf{Semantic Role} & \textbf{WCAG 2.2 Criteria} & \textbf{WCAG2Mobile Considerations} & \textbf{Implementation Properties} \\
\hline
\endfirsthead
\multicolumn{5}{c}%
{{\bfseries Table \thetable\ -- continued from previous page}} \\
\hline
\textbf{Component and Location} & \textbf{Semantic Role} & \textbf{WCAG 2.2 Criteria} & \textbf{WCAG2Mobile Considerations} & \textbf{Implementation Properties} \\
\hline
\endhead
\hline
\multicolumn{5}{r}{{Continued on next page}} \\
\endfoot
\hline
\endlastfoot
Hero Title (top: "Semantic Structure") & heading & 1.4.3 Contrast (AA)\newline 2.4.6 Headings (AA)\newline 2.4.10 Section Headings (AAA) & Text readability across variable screen sizes; Header role in screen context per WCAG2Mobile & \texttt{accessibility \ Role="header"} \\
\hline
Information Cards (sections: Perceivable, Understandable, Robust) & none & 1.3.1 Info and Relationships (A) & Logical grouping of content sections in screen context per WCAG2Mobile & Container with proper visual boundaries \\
\hline
Card Title (card headers: "Perceivable", "Understandable", "Robust") & text & 2.4.6 Headings and Labels (AA)\newline 2.4.10 Section Headings (AAA) & Information category identification in screen context per WCAG2Mobile & Semantic text styling with proper hierarchy \\
\hline
Card Description (explanatory text under titles) & text & 1.3.1 Info and Relationships (A) & Content description in mobile screen context per WCAG2Mobile & Proper text styling with semantic connection to title \\
\hline
Code Examples (dark code blocks with HTML/role examples) & text & 1.3.1 Info and Relationships (A) & Semantic structure in mobile code examples per WCAG2Mobile & \texttt{accessibility \ Role="text"},\newline \texttt{accessibility \ Label="Source code of..."} \\
\hline
Bullet List Items (checkmarked lists within cards) & text & 1.3.1 Info and Relationships (A)\newline 1.3.2 Meaningful Sequence (A) & Grouped related information for screen reader navigation per WCAG2Mobile & Parent container with proper visual structure \\
\hline
Icon Decorations (card icons and checkmarks) & none & 1.1.1 Non-text Content (A) & Reduction of focus stops for efficient screen reader navigation per WCAG2Mobile & \texttt{accessibility \ ElementsHidden=true},\newline \texttt{importantFor \ Accessibility \ ="no-hide-descendants"} \\
\hline
Card Container (overall card structure)  & none & 2.5.5 Target Size (Enhanced) (AAA)\newline 3.2.5 Change on Request (AAA) & Touch target sizing exceeding WCAG2Mobile recommendations; Predictable behavior in mobile context & Container with adequate spacing and consistent interaction model \\
\hline
\end{longtable}
\FloatBarrier

\paragraph{Technical implementation analysis}

A key aspect of the Semantic Structure screen is its handling of code examples. The implementation makes the code examples accessible to screen reader users while maintaining their visual presentation. Listing~\ref{lst:semantics-screen-accessibility} highlights this implementation.

\begin{lstlisting}[
  style=ReactNativeStyle,
  caption={Accessible code with semantic structure implementation},
  label={lst:semantics-screen-accessibility},
  basicstyle=\ttfamily\footnotesize,
  numbers=left,
]
{/* Example of accessible code block */}
<View
  style={themedStyles.codeExample}
  accessible
  accessibilityRole="text"
  accessibilityLabel="Source code of example of multiple heading levels"
>
  <Text
    style={themedStyles.codeText}
    accessibilityElementsHidden
    importantForAccessibility="no-hide-descendants"
  >
{`// Example of multiple heading levels
<View accessibilityRole="header">
  <Text accessibilityRole="heading" /* Level 1 equivalent */>
    Main Title (H1)
  </Text>
</View>

<View accessibilityRole="main">
  <Text accessibilityRole="heading" /* Level 2 equivalent */>
    Section Title (H2)
  </Text>
  <Text>
    Some descriptive content here...
  </Text>
</View>`}
  </Text>
</View>
\end{lstlisting}
\FloatBarrier

The implementation addresses several important accessibility considerations:

\begin{enumerate}
    \item \textbf{Accessible code blocks}: Code examples are wrapped in accessible containers with descriptive labels, allowing screen reader users to access the code content without getting lost in the syntax details;
    
    \item \textbf{Simplified screen reader experience}: The implementation hides the inner text element from individual accessibility focus, providing the entire code block as a single accessible unit with a meaningful label;
    
    \item \textbf{Educational structure}: The screen progressively builds understanding through a logical sequence of concepts, from basic heading structure to more complex landmark roles;
    
    \item \textbf{Practical examples}: Each concept is illustrated with concrete code examples that developers can adapt for their own implementations;
    
    \item \textbf{Section headings implementation}: The screen itself demonstrates the concepts it teaches by implementing proper section headings with appropriate hierarchy.
\end{enumerate}

\paragraph{Screen reader support analysis}

Table~\ref{tab:semantics_screen_reader_analysis} presents results from systematic testing of the Semantic structure screen with screen readers on both iOS and Android platforms, with specific WCAG2Mobile considerations noted.

\begin{longtable}[c]{|C{2.8cm}|C{3.5cm}|C{3.5cm}|C{5cm}|}
\caption{Semantic structure screen screen reader testing results with WCAG2Mobile considerations}
\label{tab:semantics_screen_reader_analysis}\\
\hline
\textbf{Test Case} & \textbf{VoiceOver (iOS 16)} & \textbf{TalkBack (Android 14-15)} & \textbf{WCAG/WCAG2Mobile Criteria} \\
\hline
\endfirsthead
\multicolumn{4}{c}%
{{\bfseries Table \thetable\ -- continued from previous page}} \\
\hline
\textbf{Test Case} & \textbf{VoiceOver (iOS 16)} & \textbf{TalkBack (Android 14-15)} & \textbf{WCAG/WCAG2Mobile Criteria} \\
\hline
\endhead
\hline
\multicolumn{4}{r}{{Continued on next page}} \\
\endfoot
\hline
\endlastfoot
Card Title Navigation & \ding{51} Announces as section heading & \ding{51} Announces as section heading & 2.4.6 Headings and Labels (AA) with WCAG2Mobile's screen context interpretation \\
\hline
Code Example Blocks & \ding{51} Announces as single unit with label & \ding{51} Announces as single unit with label & 1.3.1 Info and Relationships (A) with WCAG2Mobile's guidance on efficient screen reader navigation \\
\hline
Bullet List Items & \ding{51} Announces individual items with context & \ding{51} Announces individual items with context & 1.3.2 Meaningful Sequence (A) with WCAG2Mobile's mobile navigation optimization \\
\hline
Decorative Icons & \ding{51} Skipped by screen reader & \ding{51} Skipped by screen reader & 1.1.1 Non-text Content (A) with WCAG2Mobile's focus on reducing navigation stops \\
\hline
Section Transitions & \ding{51} Clear focus transition between sections & \ding{51} Clear focus transition between sections & 2.4.3 Focus Order (A) with WCAG2Mobile's mobile focus management \\
\hline
\end{longtable}
\FloatBarrier

\paragraph{Implementation overhead analysis}

Table~\ref{tab:semantics_implementation_overhead} quantifies the additional code required to implement accessibility features in the Semantic structure screen, including WCAG2Mobile-specific adaptations.

\begin{longtable}[c]{|C{3.8cm}|C{2.3cm}|C{2.8cm}|C{2.8cm}|}
\caption{Semantic structure screen accessibility implementation overhead with WCAG2Mobile considerations}
\label{tab:semantics_implementation_overhead}\\
\hline
\textbf{Accessibility Feature} & \textbf{Lines of Code} & \textbf{Percentage of Total} & \textbf{Complexity Impact} \\
\hline
\endfirsthead
\multicolumn{4}{c}%
{{\bfseries Table \thetable\ -- continued from previous page}} \\
\hline
\textbf{Accessibility Feature} & \textbf{Lines of Code} & \textbf{Percentage of Total} & \textbf{Complexity Impact} \\
\hline
\endhead
\hline
\multicolumn{4}{r}{{Continued on next page}} \\
\endfoot
\hline
\endlastfoot
Semantic Roles & 8 LOC & 1.5\% & Low \\
\hline
Element Hiding & 24 LOC & 4.4\% & Low \\
\hline
Accessible Code Blocks & 20 LOC & 3.7\% & Medium \\
\hline
Semantic Structure Examples & 14 LOC & 2.6\% & Medium \\
\hline
WCAG2Mobile-Specific Adaptations & 6 LOC & 1.1\% & Low \\
\hline
\textbf{Total} & \textbf{72 LOC} & \textbf{13.3\%} & \textbf{Low-Medium} \\
\hline
\end{longtable}
\FloatBarrier

This analysis reveals that implementing accessibility for the Semantic structure screen adds approximately 13.3\% to the code base, with accessible code blocks and element hiding being the largest contributors. The WCAG2Mobile-specific adaptations (1.1\%) account for the additional considerations required to properly implement semantic structure in a mobile context rather than a web context.

\paragraph{Mobile-specific considerations}

The Semantic structure screen addresses several mobile-specific accessibility considerations informed by WCAG2Mobile guidance:

\begin{enumerate}
    \item \textbf{Adapting web concepts to mobile}: The screen translates traditional web accessibility concepts (headings, landmarks) to the mobile context, directly implementing WCAG2Mobile's approach to adapting web accessibility concepts to mobile screens;
    
    \item \textbf{Limited screen navigation adaptation}: The guidance accounts for the more limited navigation options available to screen reader users on mobile platforms, addressing WCAG2Mobile's emphasis on efficient navigation in mobile contexts;
    
    \item \textbf{Mobile-optimized content hierarchy}: The implementation demonstrates how to create a clear content hierarchy that works well on smaller mobile screens while maintaining accessibility, implementing WCAG2Mobile's guidance on adapting semantic structure to mobile screens;
    
    \item \textbf{Touch-friendly code examples}: The code blocks are implemented in a touch-friendly manner, ensuring they are accessible via both touch and screen reader navigation—a key consideration in WCAG2Mobile's approach to mobile accessibility;
    
    \item \textbf{Platform-specific semantic roles}: The examples demonstrate the proper use of accessibility roles in React Native that align with WCAG2Mobile's guidance on adapting web roles to mobile screen contexts.
\end{enumerate}

\subsubsection{Best practices implementation insights}
\label{subsubsec:best-practices-insights}

The analysis of the Best Practices screens reveals several key insights for developers implementing accessibility in mobile applications:

\begin{enumerate}
    \item \textbf{Framework enables education through implementation}: The Best Practices screens not only explain accessibility concepts but demonstrate them through their own implementation, providing a meta-level educational experience;
    
    \item \textbf{Platform-specific adaptation is essential}: Several screens explicitly address platform differences between iOS and Android, acknowledging that effective mobile accessibility requires platform-specific knowledge and adaptation;
    
    \item \textbf{Implementation complexity varies by concept}: Some accessibility features (like hiding decorative icons) require minimal code additions, while others (like gesture adaptation for screen readers) involve more complex logic and state management;
    
    \item \textbf{Educational progression}: The screens collectively implement a progressive educational structure, starting with fundamental principles (WCAG Guidelines) and building toward more complex implementations (Skip Navigation, Screen Reader Gestures);
    
    \item \textbf{Mobile-specific considerations go beyond WCAG}: Many of the implemented patterns address mobile-specific concerns that extend beyond traditional WCAG criteria, demonstrating the need for mobile-specific accessibility guidance.
\end{enumerate}

\paragraph{Implementation overhead comparison}

Table~\ref{tab:best_practices_comparative_overhead} compares the implementation overhead across Best Practices screens.

\begin{table}[ht]
\caption{Accessibility implementation overhead by best practices screen}
\label{tab:best_practices_comparative_overhead}
\centering
\begin{tabular}[c]{|C{2.5cm}|C{2cm}|C{2.8cm}|C{2.8cm}|C{4.5cm}|}
\hline
\textbf{Best Practices Screen} & \textbf{Lines of Code} & \textbf{Percentage Overhead} & \textbf{Complexity Impact} & \textbf{Primary Contributors} \\
\hline
Guidelines & 56 & 10.1\% & Low & Element Hiding, Enhanced Target Size \\
\hline
Gestures Tutorial & 104 & 24.4\% & Medium-High & Adaptive Logic, Accessibility Actions \\
\hline
Logical Navigation & 72 & 18.3\% & Medium & Focus Management, Skip Link \\
\hline
Screen Reader Support & 68 & 12.4\% & Medium & State Communication, Element Hiding \\
\hline
Semantic Structure & 58 & 10.8\% & Low-Medium & Accessible Code Blocks, Element Hiding \\
\hline
\end{tabular}
\end{table}
\FloatBarrier

This comparison reveals that screens focusing on interactive behaviors (Gestures, Navigation) require significantly more accessibility code than primarily informational screens (Guidelines, Semantic Structure). This pattern aligns with findings from the Components analysis and suggests that developers should allocate more implementation resources to complex interactive features when planning accessibility work.

\paragraph{Key implementation patterns across best practices screens}

Several implementation patterns are consistently applied across all Best Practices screens:

\begin{enumerate}
    \item \textbf{Proper element hiding}: All screens consistently implement proper hiding of decorative elements using both \texttt{accessibilityElementsHidden=true} and \\\texttt{importantForAccessibility="no-hide-descendants"}, demonstrating the importance of reducing "garbage interactions" for screen reader users;
    
    \item \textbf{Semantic grouping}: Related information is consistently grouped together both visually and semantically, creating clear content relationships for all users;
    
    \item \textbf{Educational structure}: Each screen implements a clear pedagogical structure that progressively builds understanding, starting with fundamental concepts and moving toward more complex implementations;
    
    \item \textbf{Platform adaptation}: The screens account for differences between iOS and Android accessibility implementations, often with platform-specific code paths or content;
    
    \item \textbf{Enhanced target sizing}: All screens implement the AAA-level enhanced target size criterion, demonstrating a commitment to high-level accessibility standards.
\end{enumerate}

\paragraph{Future enhancements}

Based on formal analysis and user testing, several potential enhancements have been identified for future versions of the Best Practices screens:

\begin{enumerate}
    \item \textbf{Interactive assessment tools}: Adding interactive tools for developers to test their knowledge and evaluate their implementations against accessibility criteria;
    
    \item \textbf{Custom screen reader simulation}: Implementing a simplified screen reader simulation to help developers understand how their applications would be perceived by screen reader users;
    
    \item \textbf{Comparative framework implementations}: Expanding the platform-specific guidance to include side-by-side comparisons of how accessibility patterns are implemented in React Native versus Flutter;
    
    \item \textbf{User-generated examples}: Adding the ability for developers to contribute their own accessibility implementation examples to create a community resource;
    
    \item \textbf{Consistent section heading implementation}: Extending the 2.4.10 Section Headings (AAA) criterion implementation to all screens for improved navigational consistency;

\end{enumerate}

\subsection{Accessibility tools screen}
\label{subsec:tools-screen}

The Accessibility tools screen serves as a comprehensive resource guide for developers, cataloging essential tools and resources for testing and improving mobile application accessibility. It provides practical, structured information about screen readers, development tools, and testing utilities that developers can leverage throughout their accessibility implementation workflows. Figure~\ref{fig:tools_screens_sidebyside} shows the main interface of this screen.

\begin{figure}[ht]
    \centering
    \begin{subfigure}[b]{0.48\textwidth}
        \centering
        \includegraphics[width=\linewidth]{img/tools1.jpg}
        \caption{Tools screen - Part 1}
        \label{fig:tools-left}
    \end{subfigure}
    \hfill
    \begin{subfigure}[b]{0.48\textwidth}
        \centering
        \includegraphics[width=\linewidth]{img/tools2.jpg}
        \caption{Tools screen - Part 2}
        \label{fig:tools-right}
    \end{subfigure}
    \caption{Side-by-side view of the Tools screen sections}
    \label{fig:tools_screens_sidebyside}
\end{figure}
\FloatBarrier

\subsubsection{Component inventory and WCAG/MCAG mapping}

Table~\ref{tab:tools_component_mapping} provides a formal mapping between the UI components, their semantic roles, the specific WCAG 2.2 criteria they address, the WCAG2Mobile considerations, and their React Native implementation properties.

\begin{longtable}[c]{|C{2.9cm}|C{2cm}|C{2.8cm}|C{3.2cm}|C{4.7cm}|}
\caption{Tools screen component-criteria mapping with WCAG2Mobile considerations}
\label{tab:tools_component_mapping}\\
\hline
\textbf{Component and Location} & \textbf{Semantic Role} & \textbf{WCAG 2.2 Criteria} & \textbf{WCAG2Mobile Considerations} & \textbf{Implementation Properties} \\
\hline
\endfirsthead
\multicolumn{5}{c}%
{{\bfseries Table \thetable\ -- continued from previous page}} \\
\hline
\textbf{Component and Location} & \textbf{Semantic Role} & \textbf{WCAG 2.2 Criteria} & \textbf{WCAG2Mobile Considerations} & \textbf{Implementation Properties} \\
\hline
\endhead
\hline
\multicolumn{5}{r}{{Continued on next page}} \\
\endfoot
\hline
\endlastfoot
ScrollView Container (main screen wrapper) & scrollview & 2.1.1 Keyboard (A)\newline 2.4.3 Focus Order (A) & SC 2.4.3 applied to screen context; Logical focus sequence for touch-based navigation & \texttt{accessibility \ Role="scrollview"},\newline \texttt{accessibility \ Label="Mobile Accessibility Tools Screen"} \\
\hline
Hero Title (top: "Mobile Accessibility Tools") & header & 1.4.3 Contrast (AA)\newline 2.4.6 Headings (AA) & SC 2.4.6 adapted for screens instead of web pages; Screen title differentiation & \texttt{accessibility \ Role="header"} \\
\hline
Section Headers (Development Tools, Testing Checklist, etc.) & header & 2.4.6 Headings (AA)\newline 1.3.1 Info and Relationships (A)\newline 2.4.10 Section Headings (AAA) & SC 1.3.1 applied to mobile screen hierarchy; Logical section organization & \texttt{accessibility \ Role="header"} \\
\hline
Tool Cards (Accessibility Inspector, Contrast Analyzer, etc.) & button & 1.4.3 Contrast (AA)\newline 2.5.8 Target Size (AA)\newline 4.1.2 Name, Role, Value (A) & SC 2.5.8 Target Size applied to touch context; SC 4.1.2 adapted for mobile platform services & \texttt{accessibility \ Role="button"},\newline \texttt{accessibility \ Label="[Tool]. Double tap to expand/collapse"} \\
\hline
Card Icons (tool icons within cards) & none & 1.1.1 Non-text Content (A) & SC 1.1.1 adapted for mobile screen reader efficiency; Reduction of unnecessary focus stops & \texttt{accessibility \ ElementsHidden=true},\newline \texttt{importantFor \ Accessibility="no \ -hide-descendants"} \\
\hline
Expandable Content (tool descriptions and features) & text & 1.3.1 Info and Relationships (A)\newline 4.1.2 Name, Role, Value (A) & SC 1.3.1 applied to view hierarchy; Information relationships in mobile context & \texttt{role="list"}\newline \texttt{role="listitem"} \\
\hline
Documentation Links (blue "Documentation" links) & link & 2.4.4 Link Purpose (A)\newline 4.1.2 Name, Role, Value (A) & SC 2.4.4 adapted for mobile context; Link purpose clear in context & \texttt{accessibility \ Role="link"},\newline \texttt{accessibility \ Label} \\
\hline
Badge Elements (Built-in tags) & none & 1.1.1 Non-text Content (A) & SC 1.1.1 for decorative content in mobile context; Reduction of swipe interactions & \texttt{importantFor \ Accessibility="no"} \newline \texttt{accessibility \ ElementsHidden=true} \\
\hline
Expansion State Icons (chevron arrows) & none & 1.1.1 Non-text Content (A) & SC 1.1.1 for decorative indicators; Reduction of screen reader navigation friction & \texttt{accessibility \ ElementsHidden} \newline \texttt{importantFor \ Accessibility="no \ -hide-descendants"} \\
\end{longtable}
\FloatBarrier

The component mapping reveals a comprehensive integration of both WCAG and WCAG2Mobile considerations across all UI elements. Each component addresses specific WCAG2Mobile interpretations of the success criteria, with particular emphasis on:

\begin{enumerate}
    \item Mobile-specific semantic role assignment following WCAG2Mobile interpretation of SC 4.1.2 (Name, Role, Value);
    
    \item Strategic element hiding to optimize screen reader navigation on touch interfaces, addressing WCAG2Mobile's mobile-specific guidance for SC 1.1.1 (Non-text Content);
    
    \item Comprehensive semantic structure for list content aligning with WCAG2Mobile's view hierarchy interpretation of SC 1.3.1 (Info and Relationships);
    
    \item Contextual labeling for interactive elements that specifically addresses mobile touch interaction patterns as guided by WCAG2Mobile for SC 2.5.8 (Target Size).
\end{enumerate}

\subsubsection{Technical implementation analysis}

The Tools screen implements a comprehensive catalog of accessibility resources with expandable cards, providing both overview information and detailed usage guidance. The implementation follows a consistent pattern of expansion/collapse functionality with full accessibility support. Listing~\ref{lst:tools-screen-accessibility} highlights the key accessibility implementation aspects.

\begin{lstlisting}[
  style=ReactNativeStyle,
  caption={Tool card implementation with accessibility properties},
  label={lst:tools-screen-accessibility},
  basicstyle=\ttfamily\footnotesize,
  numbers=left,
]
<View
  key={tool.id}
  style={[styles.toolCard, { backgroundColor: colors.surface }]}
  accessibilityRole="button"
  accessibilityLabel={`${tool.title}. Double tap to ${
    isOpen ? 'collapse' : 'expand'} details and practical usage.`}
>
  <TouchableOpacity onPress={() => toggleExpand(tool.id)} style={styles.cardHeader}>
    
    <Text style={styles.cardTitle}>{tool.title}</Text>
    
    {tool.badge && (
      <View 
        style={styles.badge} 
        importantForAccessibility="no" 
        accessibilityElementsHidden={true}
      >
        <Text style={styles.badgeText}>{tool.badge}</Text>
      </View>
    )}
  </TouchableOpacity>

  {isOpen && (
    <View style={styles.cardBody}>
      <Text style={styles.toolDescription}>{tool.description}</Text>
      <View role="list">
        {tool.features.map((feature, idx) => (
          <Text 
            key={idx} 
            style={styles.featureItem} 
            role="listitem"
          >
            - {feature}
          </Text>
        ))}
      </View>
      
      {/* Practical Usage Section */}
      <View style={styles.practicalSection}>
        <Text style={styles.practicalHeader}>
          Practical Usage:
        </Text>
        <Text style={styles.practicalUsage}>
          {tool.practicalUsage}
        </Text>
      </View>
    </View>
  )}
</View>
\end{lstlisting}
\FloatBarrier

The implementation addresses several critical accessibility considerations:

\begin{enumerate}
    \item \textbf{Clear expandable card pattern}: Each tool card implements a consistent expandable pattern with appropriate \texttt{accessibilityRole} and state communication, ensuring screen reader users understand the interactive nature of each card;
    
    \item \textbf{Proper element hiding}: Decorative icons are systematically hidden from screen readers using both \texttt{accessibilityElementsHidden} and \\\texttt{importantForAccessibility="no-hide-descendants"}, eliminating unnecessary focus stops;
    
    \item \textbf{Semantic list structure}: Features are properly structured as lists with correct \texttt{role="list"} and \texttt{role="listitem"} attributes, creating a meaningful hierarchy for screen reader navigation;
    
    \item \textbf{Practical usage section}: Each tool includes a dedicated "Practical Usage" section that provides context-specific guidance on real-world application of the tool, going beyond mere feature listings.
\end{enumerate}

The code in Listing~\ref{lst:tools-link-announcements} shows the implementation of link navigation announcements, which specifically addresses WCAG2Mobile's interpretation of SC 4.1.3 (Status Messages) for mobile context.

\begin{lstlisting}[
  style=ReactNativeStyle,
  caption={Documentation link accessibility implementation with mobile-specific announcements},
  label={lst:tools-link-announcements},
  basicstyle=\ttfamily\footnotesize,
  numbers=left,
]
// Link handler with status announcements for screen readers
const handleOpenLink = async (toolId: string, toolTitle: string) => {
  const url = TOOL_LINKS[toolId];
  if (url && (await Linking.canOpenURL(url))) {
    try {
      await Linking.openURL(url);
      // Mobile-specific announcement for context change
      AccessibilityInfo.announceForAccessibility(
        `Opening documentation for ${toolTitle}`
      );
    } catch {
      // Error feedback for screen reader users
      AccessibilityInfo.announceForAccessibility(
        'Failed to open documentation'
      );
    }
  }
};
\end{lstlisting}
\FloatBarrier

This link implementation demonstrates three crucial accessibility patterns:

\begin{enumerate}
    \item \textbf{Explicit accessibilityRole assignment}: The link is properly identified with the "link" role;
    
    \item \textbf{Contextual label and hint}: Both the label and hint provide clear expectations about the link's purpose and behavior;
    
    \item \textbf{Dynamic announcements}: The implementation uses the AccessibilityInfo API to announce link activation outcomes, maintaining context awareness for non-visual users.
\end{enumerate}

This implementation also addresses the AAA criteria 3.2.5 Change on Request by ensuring that all actions are user-initiated and provide appropriate feedback, particularly when transitioning to external browser contexts.

\subsubsection{Screen reader support analysis}

Table~\ref{tab:tools_screen_reader_analysis} presents results from systematic testing of the Tools screen with screen readers on both iOS and Android platforms, highlighting WCAG2Mobile considerations.

\begin{longtable}[c]{|C{2.8cm}|C{3.5cm}|C{3.5cm}|C{4cm}|}
\caption{Tools screen screen reader testing results with WCAG2Mobile considerations}
\label{tab:tools_screen_reader_analysis}\\
\hline
\textbf{Test Case} & \textbf{VoiceOver (iOS 16)} & \textbf{TalkBack (Android 14-15)} & \textbf{WCAG2Mobile Considerations} \\
\hline
\endfirsthead
\multicolumn{4}{c}%
{{\bfseries Table \thetable\ -- continued from previous page}} \\
\hline
\textbf{Test Case} & \textbf{VoiceOver (iOS 16)} & \textbf{TalkBack (Android 14-15)} & \textbf{WCAG/WCAG2 \ Mobile Criteria} \\
\hline
\endhead
\hline
\multicolumn{4}{r}{{Continued on next page}} \\
\endfoot
\hline
\endlastfoot
Hero Title & \ding{51} Announces ``Mobile Accessibility Tools, heading'' & \ding{51} Announces ``Mobile Accessibility Tools, heading'' & SC 2.4.6 applied to mobile screens instead of web pages; Proper screen title identification \\
\hline
Section Headers & \ding{51} Announces section titles as headings & \ding{51} Announces section titles as headings & SC 1.3.1 applied to mobile content hierarchy; Semantic structure for mobile screens \\
\hline
Tool Card (Collapsed) & \ding{51} Announces title and expansion hint & \ding{51} Announces title and expansion hint & SC 4.1.2 for mobile state communication; Mobile touch interaction cues \\
\hline
Tool Card (Expanded) & \ding{51} Announces features as list items & \ding{51} Announces features as list items & SC 1.3.1 applied to mobile view hierarchies; List semantics for mobile content \\
\hline
Badge Elements & \ding{51} Not individually announced & \ding{51} Not individually announced & SC 1.1.1 for mobile-optimized navigation; Screen reader swipe efficiency \\
\hline
Documentation Links & \ding{51} Announces as link with destination & \ding{51} Announces as link with destination & SC 2.4.4 applied to mobile external links; Clear link purpose for mobile context \\
\hline
Practical Usage Section & \ding{51} Announces header followed by content & \ding{51} Announces header followed by content & SC 1.3.1 for mobile structured content; Logical information hierarchy \\
\hline
Link Activation & \ding{51} Announces ``Opening documentation for [tool]'' & \ding{51} Announces ``Opening documentation for [tool]'' & SC 4.1.3 for mobile context changes; Status message communication \\
\end{longtable}
\FloatBarrier

The screen reader analysis reveals several important patterns in the Tools screen implementation:

\begin{enumerate}
    \item \textbf{Consistent cross-platform behavior}: The implementation achieves consistent behavior across both TalkBack and VoiceOver, providing a unified experience across platforms;
    
    \item \textbf{Efficient navigation path}: The element hiding implementation reduces the number of focus stops by approximately 60\% compared to a non-optimized implementation, allowing more efficient navigation through dense information;
    
    \item \textbf{Clear state communication}: Expandable cards properly communicate their state to screen readers, setting clear expectations for interaction outcomes;
    
    \item \textbf{Appropriate heading structure}: All section headers are correctly announced as headings, providing clear navigation landmarks.
\end{enumerate}

The screen reader behavior also demonstrates support for the AAA criteria 2.4.10 Section Headings through the proper heading announcement pattern. The consistent heading structure creates an efficient navigation path for screen reader users, allowing them to quickly locate specific tool categories and bypass unwanted content.

\subsubsection{WCAG conformance by principle}

Table~\ref{tab:tools_wcag_by_principle} provides a detailed analysis of WCAG 2.2 compliance by principle, highlighting specific WCAG2Mobile adaptations for each principle.

\begin{longtable}[c]{|C{2.5cm}|C{3.8cm}|C{3.2cm}|C{5.2cm}|}
\caption{Tools screen WCAG compliance analysis by principle with WCAG2Mobile considerations}
\label{tab:tools_wcag_by_principle}\\
\hline
\textbf{Principle} & \textbf{Description} & \textbf{Implementation Level} & \textbf{Key Success Criteria with WCAG2Mobile Adaptations} \\
\hline
\endfirsthead
\multicolumn{4}{c}%
{{\bfseries Table \thetable\ -- continued from previous page}} \\
\hline
\textbf{Principle} & \textbf{Description} & \textbf{Implementation Level} & \textbf{Key Success Criteria with WCAG2Mobile Adaptations} \\
\hline
\endhead
\hline
\multicolumn{4}{r}{{Continued on next page}} \\
\endfoot
\hline
\endlastfoot
1. Perceivable & Information and UI components must be presentable to users in ways they can perceive & 14/15 (93.3\%) & 1.1.1 Non-text Content (A) - Mobile element hiding optimization\newline 1.3.1 Info and Relationships (A) - Applied to mobile screen hierarchy\newline 1.3.4 Orientation (AA) - Adaptation for mobile screen rotations\newline 1.4.3 Contrast (Minimum) (AA) - Consideration for mobile viewing conditions\newline 1.4.10 Reflow (AA) - Modified for mobile viewport constraints\newline 1.4.11 Non-text Contrast (AA) - Adapted for mobile viewing contexts \\
\hline
2. Operable & UI components and navigation must be operable & 20/22 (90.9\%) & 2.4.3 Focus Order (A) - Adapted for touch navigation sequences\newline 2.4.6 Headings and Labels (AA) - Applied to mobile screen organization\newline 2.5.1 Pointer Gestures (A) - Mobile-specific success criterion\newline 2.5.2 Pointer Cancellation (A) - Mobile-specific success criterion\newline 2.5.8 Target Size (Minimum) (AA) - Mobile-specific success criterion\newline 2.4.10 Section Headings (AAA) - Applied to mobile screen structure \\
\hline
3. Understandable & Information and operation of UI must be understandable & 13/17 (76.5\%) & 3.2.1 On Focus (A) - Adapted for mobile touch interaction\newline 3.2.3 Consistent Navigation (AA) - Applied to mobile screen contexts\newline 3.2.4 Consistent Identification (AA) - Applied to mobile components\newline 3.2.6 Consistent Help (A) - Mobile-specific implementation\newline 3.3.2 Labels or Instructions (A) - Optimized for mobile input\newline 3.2.5 Change on Request (AAA) - Applied to mobile context changes \\
\hline
4. Robust & Content must be robust enough to be interpreted by a wide variety of user agents & 3/3 (100\%) & 4.1.2 Name, Role, Value (A) - Adapted for mobile accessibility services\newline 4.1.3 Status Messages (AA) - Implemented with mobile-specific announcement APIs \\
\hline
\end{longtable}
\FloatBarrier

The Tools screen demonstrates strong WCAG compliance with notable mobile-specific adaptations. Perceivable (93.3\%) and Operable (90.9\%) principles show excellent implementation of mobile touch targets and screen reader optimization. Robust (100\%) reflects perfect integration with platform accessibility services, critical for development tools. Understandable (76.5\%) presents the main challenge, reflecting the inherent difficulty of making complex technical documentation comprehensible on mobile screens. The pattern suggests successful mobile accessibility implementation with room for improvement in content clarity.

This analysis shows a strong implementation of WCAG criteria across all principles, with specific WCAG2Mobile adaptations for the mobile application context. Notable patterns include:

\begin{enumerate}
    \item \textbf{Mobile-specific success criteria}: Full implementation of the mobile-focused success criteria introduced in WCAG 2.1 and 2.2, including 2.5.1 (Pointer Gestures), 2.5.2 (Pointer Cancellation), and 2.5.8 (Target Size);
    
    \item \textbf{Platform accessibility service adaptation}: Implementation of SC 4.1.2 (Name, Role, Value) and SC 4.1.3 (Status Messages) using mobile platform-specific APIs, following WCAG2Mobile guidance;
    
    \item \textbf{View-based interpretation}: Application of web page concepts to mobile screens, following WCAG2Mobile's terminology adaptations for SC 1.3.1 (Info and Relationships), SC 2.4.6 (Headings and Labels), and SC 3.2.3 (Consistent Navigation);
    
    \item \textbf{Touch-optimized interaction}: Implementation of core operable success criteria with specific consideration for touch interface patterns, as guided by WCAG2Mobile.
\end{enumerate}

\subsubsection{Mobile-specific considerations}

The Tools screen implementation addresses several mobile-specific accessibility considerations that extend beyond standard WCAG criteria:

\begin{enumerate}
    \item \textbf{Progressive disclosure pattern}: The expandable card implementation creates a clean, digestible mobile interface that allows users to focus on one tool at a time, reducing cognitive overload on smaller screens;
    
    \item \textbf{Touch-optimized interaction zones}: Cards maintain larger touch targets (especially the header section), improving usability for users with motor impairments on touch devices;
    
    \item \textbf{Platform-specific tool documentation}: The screen explicitly separates tools by platform (iOS vs. Android), addressing the critical mobile consideration that accessibility implementation differs substantially between platforms;
    
    \item \textbf{Practical guidance emphasis}: Each tool includes specific practical usage instructions, recognizing the mobile-specific challenge of implementing accessibility in constrained mobile interfaces;
    
    \item \textbf{External link handling}: Documentation links implement proper accessibility hints that they open external browsers, preparing users for context switches that are particularly disruptive on mobile devices.
\end{enumerate}

The implementation also addresses mobile-specific considerations for screen reader interaction:

\begin{enumerate}
    \item \textbf{Platform-specific screen reader guidance}: TalkBack and VoiceOver sections include platform-specific gesture instructions that reflect actual implementation differences between Android and iOS;
    
    \item \textbf{Swipe efficiency optimization}: The element hiding strategy is optimized for touch-based screen reader navigation, recognizing that excessive swipe counts create a poor mobile experience;
    
    \item \textbf{Appropriate touch target sizing}: All interactive elements maintain minimum dimensions of 44×44dp as required by WCAG 2.5.8, with most targets substantially larger for improved motor accessibility.
\end{enumerate}

This section confirms implementation of the AAA criteria 2.5.5 Target Size (Enhanced), with all interactive elements exceeding the enhanced minimum target size. The card-based design provides substantially larger touch targets than minimally required, enhancing usability for all users but particularly benefiting those with motor impairments.

\subsubsection{Implementation overhead analysis}

Table~\ref{tab:tools_implementation_overhead} quantifies the additional code required to implement accessibility features in the Tools screen.

\begin{longtable}[c]{|C{3.8cm}|C{2.3cm}|C{2.8cm}|C{2.8cm}|}
\caption{Tools screen accessibility implementation overhead}
\label{tab:tools_implementation_overhead}\\
\hline
\textbf{Accessibility Feature} & \textbf{Lines of Code} & \textbf{Percentage of Total} & \textbf{Complexity Impact} \\
\hline
\endfirsthead
\multicolumn{4}{c}%
{{\bfseries Table \thetable\ -- continued from previous page}} \\
\hline
\textbf{Accessibility Feature} & \textbf{Lines of Code} & \textbf{Percentage of Total} & \textbf{Complexity Impact} \\
\hline
\endhead
\hline
\multicolumn{4}{r}{{Continued on next page}} \\
\endfoot
\hline
\endlastfoot
Semantic Roles & 16 LOC & 2.7\% & Low \\
\hline
Descriptive Labels & 24 LOC & 4.1\% & Medium \\
\hline
Element Hiding & 32 LOC & 5.5\% & Low \\
\hline
List Semantics & 10 LOC & 1.7\% & Low \\
\hline
Link Announcements & 12 LOC & 2.1\% & Low \\
\hline
Expansion State Management & 18 LOC & 3.1\% & Medium \\
\hline
\textbf{Total} & \textbf{112 LOC} & \textbf{19.2\%} & \textbf{Medium} \\
\end{longtable}
\FloatBarrier

This analysis reveals that implementing comprehensive accessibility for the Tools screen adds approximately 19.2\% to the code base. The largest contributors are element hiding (5.5\%) and descriptive labels (4.1\%), reflecting the information-rich nature of this screen and the need to create a streamlined experience for screen reader users.

Considering specific implementation techniques:

\begin{enumerate}
    \item \textbf{Element hiding} requires a dual approach of both \texttt{accessibilityElementsHidden} and \texttt{importantForAccessibility} properties to ensure consistency across platforms;
    
    \item \textbf{Descriptive labels} must balance completeness with brevity, providing sufficient context without excessive verbosity that would slow screen reader navigation;
    
    \item \textbf{Expansion state management} represents a more complex accessibility implementation due to the need to maintain accurate labels that reflect current component states.
\end{enumerate}

Despite this overhead, the complexity impact remains "Medium," indicating that these accessibility features can be implemented without introducing significant development challenges or performance concerns.

\subsubsection{Tool categorization analysis}

The Tools screen implements a carefully structured categorization system that organizes accessibility tools into meaningful groups. Table~\ref{tab:tools_categorization} analyzes this structure from an accessibility perspective.

\begin{longtable}[c]{|C{2.5cm}|C{3.8cm}|C{4.6cm}|C{3.2cm}|}
\caption{Tools screen categorization analysis}
\label{tab:tools_categorization}\\
\hline
\textbf{Category} & \textbf{Tools Included} & \textbf{Accessibility Benefit} & \textbf{WCAG Criteria Supported} \\
\hline
\endfirsthead
\multicolumn{4}{c}%
{{\bfseries Table \thetable\ -- continued from previous page}} \\
\hline
\textbf{Category} & \textbf{Tools Included} & \textbf{Accessibility Benefit} & \textbf{WCAG Criteria Supported} \\
\hline
\endhead
\hline
\multicolumn{4}{r}{{Continued on next page}} \\
\endfoot
\hline
\endlastfoot
Screen Readers & TalkBack (Android), VoiceOver (iOS) & Provides direct access to the primary tools used by people with visual impairments & 1.3.1 Info and Relationships (A), 2.1.1 Keyboard (A), 2.4.3 Focus Order (A) \\
\hline
Development Tools & Accessibility Inspector, Contrast Analyzer, Accessibility Linter & Offers tools for early-stage accessibility integration during development & 1.4.3 Contrast (AA), 1.3.1 Info and Relationships (A), 4.1.2 Name, Role, Value (A) \\
\hline
Testing Checklist & Automated Testing, Accessibility Scanner & Supports systematic verification of accessibility implementation & 3.3.3 Error Suggestion (AA), 3.3.4 Error Prevention (AA) \\
\hline
\end{longtable}
\FloatBarrier

This careful categorization creates a progressive learning path for developers, starting with the tools users employ (screen readers), moving to development-time tools, and concluding with testing utilities. This structure reinforces the full accessibility lifecycle and encourages developers to consider accessibility from multiple perspectives.

The categorization system addresses several additional accessibility considerations:

\begin{enumerate}
    \item \textbf{Logical information hierarchy}: The category structure creates a clear, logical information hierarchy that aligns with the development workflow;
    
    \item \textbf{Progressive disclosure}: Categories implement a consistent expansion pattern that reduces cognitive load while providing complete information;
    
    \item \textbf{Practical context}: Each tool includes practical usage guidance that connects abstract accessibility principles to concrete implementation techniques;
    
    \item \textbf{Platform-specific guidance}: Tools are appropriately categorized by platform when platform-specific considerations apply.
\end{enumerate}

The category structure also supports the AAA criteria 3.2.5 Change on Request by organizing information into logical groups that users can expand at their own pace, maintaining user control over content presentation and information density.

\subsection{Instruction and community screen}
\label{subsec:instruction-community}

The Instruction and community screen serves as a collaborative learning hub that extends beyond technical implementation details. It provides developers with opportunities to engage with the broader accessibility community, learn from practical examples, and discover resources for deeper learning. This analysis examines the technical implementation of accessibility features within this educationally-focused screen.
Figure~\ref{fig:instruction_screens_sidebyside} and \ref{fig:instruction_screens_sidebyside2} show the main interfaces of this screen.

\begin{figure}[ht]
    \centering
    \begin{subfigure}[b]{0.48\textwidth}
        \centering
        \includegraphics[width=\linewidth, alt={First part of the Instruction and community screen}]{img/instruction1.jpg}
        \caption{Instruction screen - Part 1}
        \label{fig:instruction-left}
    \end{subfigure}
    \hfill
    \begin{subfigure}[b]{0.48\textwidth}
        \centering
        \includegraphics[width=\linewidth, alt={Second part of the Instruction and community screen}]{img/instruction2.jpg}
        \caption{Instruction screen - Part 2}
        \label{fig:instruction-right}
    \end{subfigure}
    \caption{Side-by-side view of the Instruction and community screen sections}
    \label{fig:instruction_screens_sidebyside}
\end{figure}

\FloatBarrier

\begin{figure}[ht]
    \centering
    \begin{subfigure}[b]{0.48\textwidth}
        \centering
        \includegraphics[width=\linewidth, alt={Third part of the Instruction and community screen}]{img/instruction3.jpg}
        \caption{Instruction screen - Part 3}
        \label{fig:instruction-left2}
    \end{subfigure}
    \hfill
    \begin{subfigure}[b]{0.48\textwidth}
        \centering
        \includegraphics[width=\linewidth, alt={Fourth part of the Instruction and community screen}]{img/instruction4.jpg}
        \caption{Instruction screen - Part 4}
        \label{fig:instruction-right2}
    \end{subfigure}
    \caption{Side-by-side view of additional Instruction and community screen sections}
    \label{fig:instruction_screens_sidebyside2}
\end{figure}

\FloatBarrier

\subsubsection{Component inventory and WCAG/MCAG mapping}
\label{subsubsec:instruction-component-mapping}

Table~\ref{tab:instruction_component_mapping} provides a formal mapping between the UI components used in the instruction and community screen, their semantic roles, the specific WCAG 2.2 criteria they address, WCAG2Mobile considerations, and their React Native implementation properties.

\begin{longtable}[c]{|C{2.5cm}|C{1.8cm}|C{2.8cm}|C{3.5cm}|C{4.2cm}|}
\caption{Instruction screen component-criteria mapping with WCAG2Mobile considerations}
\label{tab:instruction_component_mapping}\\
\hline
\textbf{Component and Location} & \textbf{Semantic Role} & \textbf{WCAG 2.2 Criteria} & \textbf{WCAG2Mobile Considerations} & \textbf{Implementation Properties} \\
\hline
\endfirsthead
\multicolumn{5}{c}%
{{\bfseries Table \thetable\ -- continued from previous page}} \\
\hline
\textbf{Component and Location} & \textbf{Semantic Role} & \textbf{WCAG 2.2 Criteria} & \textbf{WCAG2Mobile Considerations} & \textbf{Implementation Properties} \\
\hline
\endhead
\hline
\multicolumn{5}{r}{{Continued on next page}} \\
\endfoot
\hline
\endlastfoot
ScrollView Container (main screen wrapper) & scrollview & 2.1.1 Keyboard (A)\newline 2.4.3 Focus Order (A) & Mobile screen-based content organization; Focus management in scrollable content & \texttt{accessibility \ Role="scrollview"},\newline \texttt{accessibility \ Label="Accessibility Community Screen"} \\
\hline
Hero Card (top: "Join the A11y Community") & none & 1.4.3 Contrast (AA)\newline 1.4.6 Contrast (Enhanced) (AAA) & Proper contrast for varying mobile lighting conditions & Container with visual boundaries and appropriate contrast for mobile viewing \\
\hline
Hero Title (card header: "Join the A11y Community") & header & 1.3.1 Info and Relationships (A)\newline 2.4.6 Headings (AA)\newline 2.4.10 Section Headings (AAA) & Screen title differentiation in mobile context & \texttt{accessibility \ Role="header"} \\
\hline
Call-to-Action Button (blue button: "Explore Open Issues") & button & 2.4.4 Link Purpose (A)\newline 2.5.5 Target Size (Enhanced) (AAA)\newline 2.5.8 Target Size (AA) & Touch target size optimization for mobile interaction & \texttt{accessibility \ Role="button"},\newline \texttt{accessibilityLabel},\newline \texttt{minHeight: 48} \\
\hline
Project Cards (ESLint A11y Plugin, React Native Testing Library) & button & 1.4.3 Contrast (AA)\newline 2.5.8 Target Size (AA)\newline 4.1.2 Name, Role, Value (A) & Mobile-optimized touch targets; Comprehensive card labeling & \texttt{accessibility \ Role="button"},\newline \texttt{accessibilityLabel},\newline \texttt{minHeight: 44} \\
\hline
Project Icons (code icons within project cards) & none & 1.1.1 Non-text Content (A) & Reduction of unnecessary focus stops in mobile screen reader navigation & \texttt{accessibility \ Elements \ Hidden=true} \\
\hline
Tag Pills (linting, static-analysis, open-source tags) & none & 1.3.1 Info and Relationships (A) & Content categorization in mobile views & Part of parent's \texttt{accessibilityLabel} to reduce navigation burden \\
\hline
Collapsible Preview (expandable sections: "Show Details") & button & 4.1.2 Name, Role, Value (A)\newline 4.1.3 Status Messages (AA) & Mobile-specific state announcements for expandable content & \texttt{accessibility \ Role="button"},\newline \texttt{accessibilityLabel},\newline \texttt{AccessibilityInfo. \ announceFor \ Accessibility} \\
\hline
Code Snippet (dark code blocks in expanded sections) & text & 1.3.1 Info and Relationships (A)\newline 1.4.3 Contrast (AA) & Code presentation optimization for mobile screens & Platform-specific monospace font with optimized contrast \\
\hline
Community Channel Cards (A11y Stack Exchange, Accessibility Twitter) & button & 2.4.4 Link Purpose (A)\newline 4.1.2 Name, Role, Value (A) & External resource navigation in mobile context & \texttt{accessibility \ Role="button"},\newline \texttt{accessibilityLabel},\newline Mobile-specific link handling \\
\hline
Toolkit Cards (iOS Guidelines, Android Guidelines) & button & 2.4.4 Link Purpose (A)\newline 4.1.2 Name, Role, Value (A)\newline 2.5.8 Target Size (AA) & Mobile-optimized resource links with adequate touch targets & \texttt{accessibility \ Role="button"},\newline \texttt{accessibilityLabel},\newline Generous sizing (48dp) \\
\end{longtable}
\FloatBarrier

The integration of WCAG2Mobile considerations into the component mapping provides mobile-specific context for each accessibility implementation. For example, the Collapsible Preview component incorporates WCAG2Mobile's interpretation of SC 4.1.3 (Status Messages) by ensuring state changes are announced to screen readers in a mobile-specific context where visual indicators may be more easily missed on smaller screens.

\subsubsection{Technical implementation analysis}
\label{subsubsec:instruction-implementation-analysis}

The instruction and community screen implements several innovative accessibility patterns, particularly in its handling of expandable content and code snippets. Listing~\ref{lst:collapsible_preview_implementation} highlights the implementation of collapsible previews with proper accessibility announcements.

\begin{lstlisting}[
  style=ReactNativeStyle,
  caption={Collapsible preview implementation with accessibility announcements},
  label={lst:collapsible_preview_implementation},
  basicstyle=\ttfamily\footnotesize,
  numbers=left,
]
<TouchableOpacity
  onPress={() => {
    setExpandedStoryId(isExpanded ? null : story.id);
    AccessibilityInfo.announceForAccessibility(
      isExpanded 
        ? `${story.title} collapsed` 
        : `${story.title} expanded`
    );
  }}
  accessibilityRole="button"
  accessibilityLabel={`${title}. ${excerpt}`}
  style={{ marginBottom: 16 }}
>
  <View style={{ flexDirection: 'row', justifyContent: 'space-between', marginBottom: 6 }}>
    <Text style={{ fontWeight: '600', fontSize: 16, color: colors.text }}>{title}</Text>
    <Ionicons
      name={isExpanded ? 'chevron-up' : 'chevron-down'}
      size={18}
      color={colors.primary}
      accessibilityElementsHidden={true}
    />
  </View>
  <Text style={{ color: colors.textSecondary, fontSize: 14, lineHeight: 20 }}>{excerpt}</Text>

  {isExpanded && snippet && (
    <View style={{
      backgroundColor: colors.isDarkMode ? '#2c2c2e' : '#f8f8f8',
      borderRadius: 8,
      padding: 8,
      marginTop: 8
    }}>
      {snippet.split('\n').map((line, idx) => (
        <Text
          key={idx}
          style={{
            fontFamily: Platform.OS === 'ios' ? 'Menlo' : 'monospace',
            fontSize: 12,
            color: colors.isDarkMode ? '#e0e0e0' : '#333',
          }}
        >
          {line || ' '}
        </Text>
      ))}
    </View>
  )}

  <Text style={{ marginTop: 6, fontWeight: '600', color: colors.primary }}>
    {isExpanded ? 'Hide Details' : 'Show Details'}
  </Text>
</TouchableOpacity>
\end{lstlisting}
\FloatBarrier

The implementation addresses several key accessibility requirements:

\begin{enumerate}
    \item \textbf{Expansion state announcements}: The code explicitly announces changes in the expansion state of collapsible sections using \texttt{AccessibilityInfo.announceForAccessibility}, ensuring screen reader users are aware when content expands or collapses;
    
    \item \textbf{Comprehensive accessibility labels}: Collapsible previews combine the title and excerpt in their \texttt{accessibilityLabel}, ensuring screen reader users receive full context about the content before deciding to expand it;
    
    \item \textbf{Dynamic color adaptation}: The code snippet background and text colors adjust based on the dark mode state, maintaining appropriate contrast in all viewing modes;
    
    \item \textbf{Accessible code snippets}: Code snippets maintain proper line formatting with platform-appropriate monospace fonts and adequate contrast, making them readable for all users including those with low vision;
    
    \item \textbf{Redundant state indicators}: Beyond the icon change, the component includes a text label that changes between "Show Details" and "Hide Details", providing multiple indicators of the current state.
\end{enumerate}

\subsubsection{Enhanced community channel implementation}
\label{subsubsec:instruction-channel-implementation}

The community channel cards implement additional accessibility properties to ensure proper screen reader navigation and informative context:

\begin{lstlisting}[
  style=ReactNativeStyle,
  caption={Community channel card implementation with accessibility properties},
  label={lst:channel_card_implementation},
  basicstyle=\ttfamily\footnotesize,
  numbers=left,
]
<TouchableOpacity
  style={{
    backgroundColor: colors.surface,
    borderRadius: 16,
    padding: 16,
  }}
  onPress={onPress}
  accessibilityRole="button"
  accessibilityLabel={`${channel.name} with ${channel.members} members`}
>
  <View
    style={{
      width: 48,
      height: 48,
      borderRadius: 24,
      backgroundColor: colors.primary + '20',
      alignItems: 'center',
      justifyContent: 'center',
      marginRight: 12,
    }}
  >
  </View>
  <View style={{ flex: 1 }}>
    <Text style={{ fontSize: textSizes.medium, fontWeight: '600', color: colors.text, marginBottom: 4 }}>
      {channel.name}
    </Text>
    <Text style={{ fontSize: textSizes.small, color: colors.textSecondary, marginBottom: 4 }}>
      {channel.members} members
    </Text>
    <Text style={{ fontSize: textSizes.small, color: colors.text, lineHeight: 18 }}>
      {channel.description}
    </Text>
  </View>
  <Ionicons 
    name="chevron-forward" 
    size={20} 
    color={colors.primary}
    accessibilityElementsHidden={true}
    importantForAccessibility="no-hide-descendants"
  />
</TouchableOpacity>
\end{lstlisting}
\FloatBarrier

Key accessibility features of this implementation include:

\begin{enumerate}
    \item \textbf{Thorough icon hiding}: Both \texttt{accessibilityElementsHidden} and \texttt{importantForAccessibility="no-hide-descendants"} are applied to icons, ensuring they never create unnecessary focus stops;
    
    \item \textbf{Semantic grouping}: Related text elements (name, member count, description) are grouped within a single container view, creating a logical hierarchy for screen reader navigation;
    
    \item \textbf{Comprehensive labeling}: The \texttt{accessibilityLabel} combines key identifying information about the channel, including both name and member count;
    
    \item \textbf{Text size adaptation}: All text elements reference \texttt{textSizes} from the theme context, ensuring they adjust appropriately when the user enables large text mode.
\end{enumerate}

\subsubsection{Link handling with accessibility announcements}
\label{subsubsec:instruction-link-handling}

The implementation includes enhanced link handling to manage context changes when users navigate to external resources:

\begin{lstlisting}[
  style=ReactNativeStyle,
  caption={Enhanced link handling implementation},
  label={lst:enhanced_link_handling},
  basicstyle=\ttfamily\footnotesize,
  numbers=left,
]
const openLink = async (url) => {
  if (await Linking.canOpenURL(url)) {
    await Linking.openURL(url);
    AccessibilityInfo.announceForAccessibility('Opening link');
  }
};
\end{lstlisting}
\FloatBarrier

This implementation:

\begin{enumerate}
    \item \textbf{Checks link compatibility}: The function first verifies that the URL can be opened before attempting to launch it, preventing potential errors;
    
    \item \textbf{Provides transition announcement}: When a link is activated, the function explicitly announces "Opening link" to inform screen reader users about the context change;
    
    \item \textbf{Uses async/await pattern}: The implementation properly handles asynchronous operations, ensuring that the announcement occurs after the link opens, maintaining the correct sequence of events for users.
\end{enumerate}

\subsubsection{Screen reader support analysis}
\label{subsubsec:instruction-screen-reader-analysis}

Table~\ref{tab:instruction_screen_reader_analysis} presents results from systematic testing of the instruction and community screen with screen readers on both iOS and Android platforms, with specific focus on WCAG2Mobile criteria.

\begin{longtable}[c]{|C{2.8cm}|C{3.5cm}|C{3.5cm}|C{4cm}|}
\caption{Instruction screen screen reader testing with WCAG2Mobile focus}
\label{tab:instruction_screen_reader_analysis}\\
\hline
\textbf{Test Case} & \textbf{VoiceOver (iOS 16)} & \textbf{TalkBack (Android 14-15)} & \textbf{WCAG2Mobile Criteria} \\
\hline
\endfirsthead
\multicolumn{4}{c}%
{{\bfseries Table \thetable\ -- continued from previous page}} \\
\hline
\textbf{Test Case} & \textbf{VoiceOver (iOS 16)} & \textbf{TalkBack (Android 14-15)} & \textbf{WCAG2Mobile Criteria} \\
\hline
\endhead
\hline
\multicolumn{4}{r}{{Continued on next page}} \\
\endfoot
\hline
\endlastfoot
Hero Title & \ding{51} Announces ``Join the A11y Community, heading'' & \ding{51} Announces ``Join the A11y Community, heading'' & SC 2.4.6 interpreted for screens instead of web pages \\
\hline
CTA Button & \ding{51} Announces label and action & \ding{51} Announces label and action & SC 2.4.4 and SC 4.1.2 in mobile touch context \\
\hline
Project Cards & \ding{51} Announces project name and description & \ding{51} Announces project name and description & SC 4.1.2 requiring comprehensive labels for mobile navigation decisions \\
\hline
Tag Pills & \ding{51} Not individually focused & \ding{51} Not individually focused & SC 1.3.1 applied to reduce navigation friction on small screens \\
\hline
Collapsible Content & \ding{51} Announces expanded/collapsed state & \ding{51} Announces expanded/collapsed state & SC 4.1.3 applied to ensure users understand state changes in mobile context \\
\hline
Code Snippets & \ding{51} Reads code with platform-specific formatting & \ding{51} Reads code with platform-specific formatting & SC 1.3.1 with platform adaptation for mobile \\
\hline
External Links & \ding{51} Announces purpose and opens browser with announcement & \ding{51} Announces purpose and opens browser with announcement & SC 3.2.1 and SC 3.2.2 applied to handle mobile app-to-app transitions \\
\hline
Touch Target Size & \ding{51} All targets easily activated & \ding{51} All targets easily activated & SC 2.5.8 applied with mobile-specific 48dp minimum sizing \\
\hline
\end{longtable}
\FloatBarrier

The implementation addresses several key MCAG and WCAG2Mobile considerations specific to mobile platforms:

\begin{enumerate}
    \item \textbf{Touch target optimization}: All interactive elements exceed the minimum recommendation of 44×44dp, implementing WCAG2Mobile's interpretation of SC 2.5.8 (Target Size) to accommodate users with motor control limitations on touch screens;
    
    \item \textbf{Swipe efficiency}: Decorative elements are marked with \texttt{accessibilityElementsHidden}, implementing WCAG2Mobile's guidance for reducing unnecessary focus stops to improve navigation efficiency on mobile devices;
    
    \item \textbf{Platform-specific announcements}: The implementation includes platform-aware state announcements, addressing WCAG2Mobile's guidance on SC 4.1.3 (Status Messages) to ensure consistent experiences across iOS and Android;
    
    \item \textbf{Context transition announcements}: External link activations include explicit announcements of context changes, implementing WCAG2Mobile's guidance on SC 3.2.5 (Change on Request) for mobile-specific app transitions;
    
    \item \textbf{Semantic structure}: The screen implements proper heading hierarchy with \texttt{accessibilityRole="header"}, addressing WCAG2Mobile's interpretation of SC 1.3.1 (Info and Relationships) and SC 2.4.6 (Headings and Labels) in a screen-based rather than web page-based context.
\end{enumerate}

\subsubsection{Implementation overhead analysis}
\label{subsubsec:instruction-implementation-overhead}

Table~\ref{tab:instruction_implementation_overhead_appendix} quantifies the additional code required to implement accessibility features in the instruction and community screen.

\begin{longtable}[c]{|C{3.8cm}|C{2.3cm}|C{2.8cm}|C{2.8cm}|}
\caption{Instruction screen accessibility implementation overhead}
\label{tab:instruction_implementation_overhead_appendix}\\
\hline
\textbf{Accessibility Feature} & \textbf{Lines of Code} & \textbf{Percentage of Total} & \textbf{Complexity Impact} \\
\hline
\endfirsthead
\multicolumn{4}{c}%
{{\bfseries Table \thetable\ -- continued from previous page}} \\
\hline
\textbf{Accessibility Feature} & \textbf{Lines of Code} & \textbf{Percentage of Total} & \textbf{Complexity Impact} \\
\hline
\endhead
\hline
\multicolumn{4}{r}{{Continued on next page}} \\
\endfoot
\hline
\endlastfoot
Semantic Roles & 24 LOC & 3.1\% & Low \\
\hline
Descriptive Labels & 32 LOC & 4.2\% & Medium \\
\hline
Element Hiding & 18 LOC & 2.3\% & Low \\
\hline
Status Announcements & 16 LOC & 2.1\% & Medium \\
\hline
Link Handling & 14 LOC & 1.8\% & Low \\
\hline
Collapsible Content Management & 28 LOC & 3.6\% & High \\
\hline
Code Snippet Presentation & 24 LOC & 3.1\% & Medium \\
\hline
\textbf{Total} & \textbf{156 LOC} & \textbf{20.2\%} & \textbf{Medium} \\
\end{longtable}
\FloatBarrier

This analysis reveals that implementing comprehensive accessibility for the instruction and community screen adds approximately 20.2\% to the code base. The most significant contributors are descriptive labels (4.2\%) and collapsible content management (3.6\%), reflecting the information-rich and interactive nature of this screen.

\subsubsection{WCAG conformance by principle}
\label{subsubsec:instruction-wcag-principle}

Table~\ref{tab:instruction_wcag_by_principle} provides a detailed analysis of WCAG 2.2 compliance by principle with WCAG2Mobile considerations:

\begin{longtable}[c]{|C{2.5cm}|C{3.8cm}|C{3.2cm}|C{5.2cm}|}
\caption{Instruction screen WCAG compliance analysis by principle with WCAG2Mobile focus}
\label{tab:instruction_wcag_by_principle}\\
\hline
\textbf{Principle} & \textbf{Description} & \textbf{Implementation Level} & \textbf{Key Success Criteria with Mobile Context} \\
\hline
\endfirsthead
\multicolumn{4}{c}%
{{\bfseries Table \thetable\ -- continued from previous page}} \\
\hline
\textbf{Principle} & \textbf{Description} & \textbf{Implementation Level} & \textbf{Key Success Criteria with Mobile Context} \\
\hline
\endhead
\hline
\multicolumn{4}{r}{{Continued on next page}} \\
\endfoot
\hline
\endlastfoot
1. Perceivable & Information and UI components must be presentable to users in ways they can perceive & 13/13 (100\%) & 1.1.1 Non-text Content (A) - Mobile decorative elements\newline 1.3.1 Info and Relationships (A) - Screen structure\newline 1.3.4 Orientation (AA) - Mobile-specific\newline 1.4.3 Contrast (AA) - Mobile viewing conditions\newline 1.4.10 Reflow (AA) - Mobile-specific \\
\hline
2. Operable & UI components and navigation must be operable & 17/17 (100\%) & 2.4.3 Focus Order (A) - Mobile swipe sequence\newline 2.4.6 Headings (AA) - Screen contexts\newline 2.5.1 Pointer Gestures (A) - Mobile-specific\newline 2.5.2 Pointer Cancellation (A) - Mobile-specific\newline 2.5.8 Target Size (AA) - Mobile-specific \\
\hline
3. Understandable & Information and operation of UI must be understandable & 10/10 (100\%) & 3.2.1 On Focus (A) - Mobile context transitions\newline 3.2.2 On Input (A) - Touch interactions\newline 3.2.3 Consistent Navigation (AA) - Mobile views\newline 3.2.5 Change on Request (AAA) - Mobile app transitions\newline 3.3.2 Labels or Instructions (A) - Mobile input context \\
\hline
4. Robust & Content must be robust enough to be interpreted by a wide variety of user agents & 3/3 (100\%) & 4.1.2 Name, Role, Value (A) - Mobile platform adaptation\newline 4.1.3 Status Messages (AA) - Mobile notification context \\
\hline
\end{longtable}
\FloatBarrier

The instruction and community screen achieves 100\% compliance across all four WCAG principles, with WCAG2Mobile considerations integrated throughout. The implementation particularly excels in addressing mobile-specific criteria such as SC 2.5.8 (Target Size), SC 1.3.4 (Orientation), and SC 4.1.3 (Status Messages), demonstrating how WCAG2Mobile provides valuable guidance for optimizing accessibility in mobile contexts.

\subsubsection{Mobile-specific accessibility implementations}
\label{subsubsec:instruction-mobile-specific}

The instruction and community screen addresses several mobile-specific accessibility challenges:

\begin{longtable}[c]{|C{3.5cm}|C{5.5cm}|C{6cm}|}
\caption{Mobile-specific accessibility implementations}
\label{tab:instruction_mobile_specific}\\
\hline
\textbf{Mobile Challenge} & \textbf{Implementation} & \textbf{Code Evidence} \\
\hline
\endfirsthead
\multicolumn{3}{c}%
{{\bfseries Table \thetable\ -- continued from previous page}} \\
\hline
\textbf{Mobile Challenge} & \textbf{Implementation} & \textbf{Code Evidence} \\
\hline
\endhead
\hline
\multicolumn{3}{r}{{Continued on next page}} \\
\endfoot
\hline
\endlastfoot
Variable screen dimensions & Responsive layout using flexbox and relative measurements & \texttt{const \{width\} = useWindowDimensions()} and flexible layouts \\
\hline
Battery considerations & Gradient backgrounds with platform-optimized implementations & \texttt{<LinearGradient colors=\{gradientColors\} style=\{styles.container\}>} \\
\hline
Touch precision limitations & Generous spacing and large interactive areas & Card components with padding and margins exceeding minimum requirements \\
\hline
External resource handling & Controlled browser launching with accessibility announcements & \texttt{AccessibilityInfo.announce \ ForAccessibility('Opening link')} \\
\hline
Platform-specific styling & Conditional styling based on platform detection & \texttt{fontFamily: Platform.OS === 'ios' ? 'Menlo' : 'monospace'} \\
\end{longtable}
\FloatBarrier

\subsection{Settings screen}
\label{subsec:settings-screen}

The Settings screen serves as a comprehensive control center for adjusting accessibility and display preferences in the \textit{AccessibleHub} application. It offers users fine-grained control over visual appearance, text size, motion effects, and interaction modes. By providing these adjustments directly within the application, the Settings screen exemplifies an embedded accessibility approach where adaptation is treated as a core feature rather than an afterthought. Figure~\ref{fig:settings_screen_main} shows the main interface of this screen.

\begin{figure}[ht]
    \centering
    \includegraphics[width=0.48\textwidth, alt={Settings screen showing accessibility options}]{img/settings_normal.jpg}
    \caption{The Settings screen with various accessibility options}
    \label{fig:settings_screen_main}
\end{figure}

\FloatBarrier

\subsubsection{Component inventory and WCAG/MCAG mapping}

Table~\ref{tab:settings_component_mapping} provides a formal mapping between the UI components, their semantic roles, the specific WCAG 2.2 criteria they address, the WCAG2Mobile considerations, and their React Native implementation properties.

\begin{longtable}[c]{|C{2.5cm}|C{1.8cm}|C{2.8cm}|C{3.2cm}|C{4.5cm}|}
\caption{Settings screen component-criteria mapping with WCAG2Mobile considerations}
\label{tab:settings_component_mapping}\\
\hline
\textbf{Component and Location} & \textbf{Semantic Role} & \textbf{WCAG 2.2 Criteria} & \textbf{WCAG2Mobile Considerations} & \textbf{Implementation Properties} \\
\hline
\endfirsthead
\multicolumn{5}{c}%
{{\bfseries Table \thetable\ -- continued from previous page}} \\
\hline
\textbf{Component and Location} & \textbf{Semantic Role} & \textbf{WCAG 2.2 Criteria} & \textbf{WCAG2Mobile Considerations} & \textbf{Implementation Properties} \\
\hline
\endhead
\hline
\multicolumn{5}{r}{{Continued on next page}} \\
\endfoot
\hline
\endlastfoot
Section Headers (Visual Settings, Readability Enhancements, etc.) & heading & 2.4.6 Headings (AA) & Header announcement in screen context; Semantic significance in mobile navigation & \texttt{accessibility \ Role="header"} \\
\hline
Setting Card (grouped sections containing related settings) & none & 1.3.1 Info and Relationships (A)\newline 1.4.3 Contrast (AA) & Logical grouping on screen; Mobile layout containment; Visual boundaries & Container with proper styling and contrast ratios \\
\hline
Setting Row (individual setting items: Dark Mode, High Contrast, etc.) & none & 1.3.1 Info and Relationships (A)\newline 2.5.8 Target Size (AA) & Touch target optimization; Minimum tap area of 44×44dp; Adaptive sizing & Layout with adaptive \texttt{padding \ Vertical} based on preference \\
\hline
Setting Icon (moon, contrast, text size icons) & none & 1.1.1 Non-text Content (A) & Reduction of unnecessary swipes in screen reader navigation; Mobile efficiency & \texttt{accessibility \ Elements \ Hidden},\newline \texttt{important \ For \ Accessibility="no- \ hide- \ descendants"} \\
\hline
Setting Title (Dark Mode, High Contrast Mode, Large Text, etc.) & text & 2.4.6 Headings and Labels (AA) & Label clarity on small screens; Content identification & Text with proper styling and semantic structure \\
\hline
Setting Description (explanatory text under titles) & text & 1.3.1 Info and Relationships (A)\newline 3.3.2 Labels or Instructions (A) & Compact descriptive context; Information density adaptation & Proper text styling with semantic connection to title \\
\hline
Switch Control (toggle switches on right side) & switch & 4.1.2 Name, Role, Value (A)\newline 3.3.5 Help (AAA)\newline 4.1.3 Status Messages (AA) & Platform-specific switch behavior; State announcements; Touch-optimized interaction & \texttt{accessibility \ Role="switch"},\newline \texttt{accessibility \ Label="...[state included]"},\newline \texttt{accessibility \ Hint} \\
\hline
Divider (section separators) & none & 1.3.1 Info and Relationships (A) & Visual separation with minimal screen reader impact & \texttt{important \ For \ Accessibility="no"},\newline \texttt{accessibility \ Elements \ Hidden=true} \\
\hline
Status Toast (notification when settings change) & status & 4.1.3 Status Messages (AA) & Mobile-specific notification pattern; Multi-channel feedback & \texttt{Accessibility \ Info. \ announce \ For \ Accessibility}, Platform-specific feedback \\
\end{longtable}

\FloatBarrier

\subsubsection{Dynamic accessibility features}

A key aspect of the Settings screen is its implementation of direct accessibility customization options. Figure~\ref{fig:settings_modes} illustrates the application in different accessibility modes.

\begin{figure}[ht]
    \centering
    \begin{subfigure}[b]{0.48\textwidth}
        \centering
        \includegraphics[width=\linewidth, alt={Settings screen with dark mode enabled}]{img/settings1.jpg}
        \caption{Dark mode enabled}
        \label{fig:settings-dark}
    \end{subfigure}
    \hfill
    \begin{subfigure}[b]{0.48\textwidth}
        \centering
        \includegraphics[width=\linewidth, alt={Settings screen with high contrast mode enabled}]{img/settings4.jpg}
        \caption{High contrast mode enabled}
        \label{fig:settings-contrast}
    \end{subfigure}
    \caption{Settings screen with different accessibility modes enabled}
    \label{fig:settings_modes}
\end{figure}

\pagebreak

The accessibility modes implemented in the Settings screen directly address several core WCAG principles:

\begin{enumerate}
    \item \textbf{Dark mode}: Addresses WCAG 1.4.8 Visual Presentation (AAA) by allowing users to adjust color preferences;
    
    \item \textbf{High contrast mode}: Implements WCAG 1.4.3 Contrast (Minimum) (AA) and 1.4.6 Contrast (Enhanced) (AAA) by increasing the contrast ratio between text and background;
    
    \item \textbf{Large text}: Addresses WCAG 1.4.4 Resize Text (AA) by providing text scaling options;
    
    \item \textbf{Reduce motion}: Implements WCAG 2.3.3 Animation from Interactions (AAA) by allowing users to minimize animation effects;
    
    \item \textbf{Color filter}: Addresses WCAG 1.4.8 Visual Presentation (AAA) by providing alternative color schemes for users with color vision deficiencies;
    
    \item \textbf{Large touch targets}: Exceeds WCAG 2.5.8 Target Size (AA) by increasing the interactive area of elements beyond the minimum required dimensions.
\end{enumerate}

Figure~\ref{fig:settings_notifications} demonstrates the visual feedback mechanisms when settings are toggled.

\begin{figure}[ht]
    \centering
    \begin{subfigure}[b]{0.48\textwidth}
        \centering
        \includegraphics[width=\linewidth, alt={Settings screen with large text option enabled notification}]{img/settings3.jpg}
        \caption{Large text enabled notification}
        \label{fig:settings-text-notification}
    \end{subfigure}
    \hfill
    \begin{subfigure}[b]{0.48\textwidth}
        \centering
        \includegraphics[width=\linewidth, alt={Settings screen with color filter enabled notification}]{img/settings2.jpg}
        \caption{Color filter enabled notification}
        \label{fig:settings-filter-notification}
    \end{subfigure}
    \caption{Visual notifications when accessibility settings are toggled}
    \label{fig:settings_notifications}
\end{figure}

\pagebreak

\subsubsection{Technical implementation analysis}

The Settings screen implements a robust approach to accessibility through a combination of semantic structure, proper labeling, and multimodal feedback. Listing~\ref{lst:setting-row-implementation} demonstrates the implementation of a reusable setting row component with comprehensive accessibility properties.

\begin{lstlisting}[
  style=ReactNativeStyle,
  caption={Setting row implementation with accessibility properties},
  label={lst:setting-row-implementation},
  basicstyle=\ttfamily\footnotesize,
  numbers=left,
]
const SettingRow = ({
  icon,
  title,
  description,
  value,
  onToggle,
}) => (
  <View style={themedStyles.settingRow}>
    <View style={themedStyles.settingIcon}>
      <Ionicons
        name={icon}
        size={24}
        color={colors.primary}
        accessibilityElementsHidden
        importantForAccessibility="no-hide-descendants"
      />
    </View>
    <View style={themedStyles.settingContent}>
      <Text style={[themedStyles.settingTitle, { fontSize: textSizes.medium }]}>
        {title}
      </Text>
      <Text style={[themedStyles.settingDescription, { fontSize: textSizes.small }]}>
        {description}
      </Text>
    </View>
    <Switch
      value={value}
      onValueChange={() => {
        onToggle();
        const newValue = !value;
        const message = `${title} ${newValue ? 'enabled' : 'disabled'}`;
        AccessibilityInfo.announceForAccessibility(message);
        if (Platform.OS === 'android') {
          ToastAndroid.show(message, ToastAndroid.SHORT);
          Vibration.vibrate(50);
        }
      }}
      trackColor={{ false: '#767577', true: colors.primary }}
      // Comprehensive accessibility label combining context and state
      accessibilityLabel={`${title}. ${description}. Switch is ${value ? 'on' : 'off'}.`}
      accessibilityRole="switch"
      accessibilityHint="Double tap to toggle setting"
    />
  </View>
);
\end{lstlisting}

Several key accessibility considerations are implemented in this component:

\begin{enumerate}
    \item \textbf{Comprehensive labeling}: The switch control combines title, description, and current state in its \texttt{accessibilityLabel}, ensuring screen reader users receive complete context about the setting;
    
    \item \textbf{Hidden decorative elements}: Icons are properly hidden from screen readers using both \texttt{accessibilityElementsHidden} and \\ \texttt{importantForAccessibility="no-hide-descendants"}, eliminating unnecessary focus stops;
    
    \item \textbf{Multimodal feedback}: When a setting is toggled, the implementation provides feedback through multiple channels: visual (toggle animation), auditory (screen reader announcement), and in the case of Android, haptic feedback (vibration);
    
    \item \textbf{Proper semantic roles}: The switch control has an explicit \texttt{accessibilityRole="switch"}, ensuring its purpose is clearly communicated to assistive technologies;
    
    \item \textbf{Action guidance}: The implementation includes an \texttt{accessibilityHint="Double tap to toggle setting"}, providing additional context on how to interact with the control.
\end{enumerate}

The implementation of section headers, shown in Listing~\ref{lst:section-headers-implementation}, further demonstrates the application's commitment to semantic structure.

\begin{lstlisting}[
  style=ReactNativeStyle,
  caption={Section headers implementation with proper semantic role},
  label={lst:section-headers-implementation},
  basicstyle=\ttfamily\footnotesize,
  numbers=left,
]
{/* VISUAL SETTINGS */}
<View style={themedStyles.section}>
  <Text style={themedStyles.sectionHeader} accessibilityRole="header">
    Visual Settings
  </Text>
  <View style={themedStyles.card}>
    {/* Setting rows */}
  </View>
</View>

{/* READABILITY ENHANCEMENTS */}
<View style={themedStyles.section}>
  <Text style={themedStyles.sectionHeader} accessibilityRole="header">
    Readability Enhancements
  </Text>
  <View style={themedStyles.card}>
    {/* Setting rows */}
  </View>
</View>
\end{lstlisting}

\subsubsection{Screen reader support analysis}

Table~\ref{tab:settings_screen_reader_analysis} presents results from systematic testing of the Settings screen with screen readers on both iOS and Android platforms, with specific focus on WCAG2Mobile's mobile platform interpretation.

\begin{longtable}[c]{|C{2.8cm}|C{3.5cm}|C{3.5cm}|C{4.6cm}|}
\caption{Settings screen screen reader testing results with WCAG2Mobile considerations}
\label{tab:settings_screen_reader_analysis}\\
\hline
\textbf{Test Case} & \textbf{VoiceOver (iOS 16)} & \textbf{TalkBack (Android 14-15)} & \textbf{WCAG/WCAG2Mobile Criteria} \\
\hline
\endfirsthead
\multicolumn{4}{c}%
{{\bfseries Table \thetable\ -- continued from previous page}} \\
\hline
\textbf{Test Case} & \textbf{VoiceOver (iOS 16)} & \textbf{TalkBack (Android 14-15)} & \textbf{WCAG/WCAG2Mobile Criteria} \\
\hline
\endhead
\hline
\multicolumn{4}{r}{{Continued on next page}} \\
\endfoot
\hline
\endlastfoot
Section Headers & \ding{51} Announces ``Visual Settings, heading'' & \ding{51} Announces ``Visual Settings, heading'' & 1.3.1 - Info and Relationships (A), 2.4.6 - Headings and Labels (AA) with WCAG2Mobile's screen-specific interpretation \\
\hline
Switch Controls & \ding{51} Announces complete label with title, description, and state & \ding{51} Announces complete label with title, description, and state & 4.1.2 - Name, Role, Value (A), WCAG2Mobile mobile platform interpretation for switch controls \\
\hline
Switch Toggle & \ding{51} Announces new state after toggling & \ding{51} Announces new state and provides haptic feedback & 4.1.3 - Status Messages (AA), WCAG2Mobile's multi-channel feedback for mobile context \\
\hline
Dividers & \ding{51} Not announced & \ding{51} Not announced & 1.3.1 - Info and Relationships (A), WCAG2Mobile guidance on reducing unnecessary swipes \\
\hline
Setting Cards & \ding{51} Proper grouping of related settings & \ding{51} Proper grouping of related settings & 1.3.1 - Info and Relationships (A), WCAG2Mobile interpretation for logical screen structure \\
\hline
Icons & \ding{51} Not announced & \ding{51} Not announced & 1.1.1 - Non-text Content (A), WCAG2Mobile optimization for screen reader navigation efficiency \\
\hline
Toast Notifications & \ding{51} Announces setting changes & \ding{51} Announces setting changes with visual toast & 4.1.3 - Status Messages (AA), WCAG2Mobile platform-specific notification methods \\
\hline
Large Touch Targets & \ding{51} Easily toggles with reduced precision & \ding{51} Easily toggles with reduced precision & 2.5.8 - Target Size (AA), WCAG2Mobile enhanced touch target sizing for mobile interaction \\
\end{longtable}

\FloatBarrier

The implementation addresses several key MCAG and WCAG2Mobile several key mobile-specific considerations:

\begin{enumerate}
    \item \textbf{Platform-specific adaptations}: The code adjusts feedback mechanisms based on platform capabilities, using \texttt{ToastAndroid} for visual feedback and \texttt{Vibration} for haptic feedback on Android devices;
    
    \item \textbf{Touch-optimized layout}: The setting rows implement larger touch targets when the \texttt{isLargeTouchTargets} option is enabled, as shown by the conditional padding in the style: \texttt{paddingVertical: isLargeTouchTargets ? 20 : 16};
    
    \item \textbf{Multi-sensory feedback}: The implementation provides feedback through multiple channels (visual, auditory, haptic), ensuring users with different sensory capabilities can perceive setting changes;
    
    \item \textbf{Structured grouping}: Related settings are grouped into logical categories with clear headers, helping users with cognitive disabilities understand the organization of settings on a small screen.
\end{enumerate}

\subsubsection{Implementation overhead analysis}

Table~\ref{tab:settings_implementation_overhead} quantifies the additional code required to implement accessibility features in the Settings screen.

\begin{longtable}[c]{|C{3.8cm}|C{2.3cm}|C{2.8cm}|C{2.8cm}|}
\caption{Settings screen accessibility implementation overhead}
\label{tab:settings_implementation_overhead}\\
\hline
\textbf{Accessibility Feature} & \textbf{Lines of Code} & \textbf{Percentage of Total} & \textbf{Complexity Impact} \\
\hline
\endfirsthead
\multicolumn{4}{c}%
{{\bfseries Table \thetable\ -- continued from previous page}} \\
\hline
\textbf{Accessibility Feature} & \textbf{Lines of Code} & \textbf{Percentage of Total} & \textbf{Complexity Impact} \\
\hline
\endhead
\hline
\multicolumn{4}{r}{{Continued on next page}} \\
\endfoot
\hline
\endlastfoot
Semantic Roles & 12 LOC & 2.1\% & Low \\
\hline
Comprehensive Labels & 16 LOC & 2.8\% & Medium \\
\hline
Element Hiding & 18 LOC & 3.2\% & Low \\
\hline
Status Announcements & 14 LOC & 2.5\% & Medium \\
\hline
Platform-specific Feedback & 12 LOC & 2.1\% & Medium \\
\hline
Dynamic Styling & 22 LOC & 3.9\% & Medium \\
\hline
Accessibility State & 8 LOC & 1.4\% & Low \\
\hline
\textbf{Total} & \textbf{102 LOC} & \textbf{18.0\%} & \textbf{Medium} \\
\end{longtable}

This analysis reveals that implementing accessibility for the Settings screen adds approximately 18.0\% to the code base. The most significant contributors are dynamic styling (3.9\%) and element hiding (3.2\%), reflecting the need to adjust visual presentation based on user preferences and to streamline screen reader navigation.

\subsubsection{WCAG conformance by principle}

Table~\ref{tab:settings_wcag_by_principle} provides a detailed analysis of WCAG 2.2 compliance by principle, with specific WCAG2Mobile mobile context interpretations:

\begin{longtable}[c]{|C{2.5cm}|C{3cm}|C{3.2cm}|C{5.2cm}|}
\caption{Settings screen WCAG compliance analysis by principle with WCAG2Mobile considerations}
\label{tab:settings_wcag_by_principle}\\
\hline
\textbf{Principle} & \textbf{Description} & \textbf{Implementation Level} & \textbf{Key Success Criteria with WCAG2Mobile Context} \\
\hline
\endfirsthead
\multicolumn{4}{c}%
{{\bfseries Table \thetable\ -- continued from previous page}} \\
\hline
\textbf{Principle} & \textbf{Description} & \textbf{Implementation Level} & \textbf{Key Success Criteria with WCAG2Mobile Context} \\
\hline
\endhead
\hline
\multicolumn{4}{r}{{Continued on next page}} \\
\endfoot
\hline
\endlastfoot
1. Perceivable & Information and UI components must be presentable to users in ways they can perceive & 14/15 (93.3\%) & 1.1.1 Non-text Content (A) - Mobile-specific content hiding\newline 1.3.1 Info and Relationships (A) - Screen context structure\newline 1.3.4 Orientation (AA) - Mobile orientation adaptation\newline 1.4.3 Contrast (Minimum) (AA) - Mobile viewing conditions\newline 1.4.4 Resize Text (AA) - Mobile text resizing\newline 1.4.8 Visual Presentation (AAA) - Screen adaptation \\
\hline
2. Operable & UI components and navigation must be operable & 16/18 (88.9\%) & 2.3.3 Animation from Interactions (AAA) - Mobile motion sensitivity\newline 2.4.6 Headings and Labels (AA) - Mobile screen context\newline 2.5.1 Pointer Gestures (A) - Touch alternatives\newline 2.5.2 Pointer Cancellation (A) - Mobile touch interaction\newline 2.5.8 Target Size (AA) - Touch-optimized targets \\
\hline
3. Understandable & Information and operation of UI must be understandable & 11/11 (100\%) & 3.2.1 On Focus (A) - Mobile context changes\newline 3.2.2 On Input (A) - Touch interaction results\newline 3.3.2 Labels or Instructions (A) - Mobile context guidance\newline 3.3.5 Help (AAA) - Mobile-specific assistance\newline 3.3.7 Redundant Entry (A) - Mobile form optimization \\
\hline
4. Robust & Content must be robust enough to be interpreted by a wide variety of user agents & 3/3 (100\%) & 4.1.2 Name, Role, Value (A) - Mobile accessibility API\newline 4.1.3 Status Messages (AA) - Mobile notifications \\
\end{longtable}

\FloatBarrier

The Settings screen achieves 100\% compliance with the Perceivable, Understandable, and Robust principles, reflecting its central role in providing accessibility adjustments. The slightly lower compliance with the Operable principle (88\%) is due to the absence of specific keyboard navigation optimizations, which are less relevant in the predominantly touch-based mobile context.

\subsubsection{Mobile-specific considerations}

The Settings screen implementation addresses several mobile-specific accessibility considerations beyond standard WCAG requirements:

\begin{enumerate}
    \item \textbf{Battery-aware implementation}: The screen considers the impact of accessibility features like high contrast and dark mode on battery consumption, which is particularly important for mobile users who may need these features all day;
    
    \item \textbf{Touch ergonomics}: The implementation of larger touch targets addresses the specific challenges of touch interaction for users with motor impairments, exceeding the minimum WCAG requirements to provide a more comfortable experience on smaller screens;
    
    \item \textbf{Multi-device adaptation}: The settings options are implemented with responsive layouts that adapt to different screen sizes and orientations, ensuring consistency across the diverse range of mobile devices;
    
    \item \textbf{Platform convention alignment}: The implementation follows platform-specific visual and interaction patterns, using familiar switch controls and feedback mechanisms that align with user expectations on each platform;
    
    \item \textbf{Haptic feedback integration}: The implementation adds haptic feedback (vibration) when settings are changed on Android devices, providing an additional sensory channel that is particularly valuable in mobile contexts where visual attention may be limited.
\end{enumerate}

\subsection{Framework comparison screen}

To provide a comprehensive, empirical foundation for comparing implementation patterns, the \textit{AccessibleHub} includes a dedicated Framework comparison screen that offers a formal, evidence-based analysis tool for evaluating accessibility implementation across mobile development frameworks. This screen transforms the theoretical architectural differences discussed above into quantifiable metrics and side-by-side comparisons.

The Framework comparison screen's formal methodology extends the Master Thesis' analysis through visual representation and interactive exploration of metrics. Unlike other screens in the \textit{AccessibleHub} application that focus primarily on educational content or component examples, this screen implements a structured, academically-grounded system for comparing React Native and Flutter using transparent metrics, formal methodology, and verifiable data. Figure~\ref{fig:framework_comparison_main} shows the main interface of this screen.

\begin{figure}[ht]
    \centering
    \begin{subfigure}[b]{0.48\textwidth}
        \centering
        \includegraphics[width=\linewidth, alt={Framework comparison screen with React Native selected}]{img/overview1.jpg}
        \caption{React Native framework view}
        \label{fig:framework-comparison-reactnative}
    \end{subfigure}
    \hfill
    \begin{subfigure}[b]{0.48\textwidth}
        \centering
        \includegraphics[width=\linewidth, alt={Framework comparison screen with Flutter selected}]{img/overview2.jpg}
        \caption{Flutter framework view}
        \label{fig:framework-comparison-flutter}
    \end{subfigure}
    \caption{Framework comparison screen showing overview information for both frameworks}
    \label{fig:framework_comparison_main}
\end{figure}

\FloatBarrier

\subsubsection{Component inventory and WCAG/MCAG/WCAG2Mobile mapping}

Table~\ref{tab:framework_comparison_mapping} provides a formal mapping between the UI components, their semantic roles, the specific WCAG 2.2 criteria they address, their WCAG2Mobile considerations, and their React Native implementation properties.

\begin{longtable}[c]{|C{2.5cm}|C{2cm}|C{2.8cm}|C{2.8cm}|C{4.7cm}|}
\caption{Framework comparison screen component-criteria mapping with WCAG2Mobile considerations}
\label{tab:framework_comparison_mapping}\\
\hline
\textbf{Component and Location} & \textbf{Semantic Role} & \textbf{WCAG 2.2 Criteria} & \textbf{WCAG2Mobile Considerations} & \textbf{Implementation Properties} \\
\hline
\endfirsthead
\multicolumn{5}{c}%
{{\bfseries Table \thetable\ -- continued from previous page}} \\
\hline
\textbf{Component and Location} & \textbf{Semantic Role} & \textbf{WCAG 2.2 Criteria} & \textbf{WCAG2Mobile Considerations} & \textbf{Implementation Properties} \\
\hline
\endhead
\hline
\multicolumn{5}{r}{{Continued on next page}} \\
\endfoot
\hline
\endlastfoot
Hero Title (top: "Framework Comparison") & heading & 1.4.3 Contrast (AA)\newline 2.4.6 Headings (AA) & SC 2.4.6 applied to screens rather than pages; Mobile-specific view context & \texttt{accessibility \ Role="header"} \\
\hline
Hero Subtitle (description text under title) & text & 1.4.3 Contrast (AA) & Content description in mobile screen context & Text styling with semantic connection to title \\
\hline
Framework Selection Buttons (React Native/Flutter toggle buttons) & button & 1.4.3 Contrast (AA)\newline 2.5.8 Target Size (AA)\newline 4.1.2 Name, Role, Value (A) & SC 2.5.8 for mobile touch targets; SC 4.1.2 adapted for mobile platform accessibility services & \texttt{accessibility \ Role="button"},\newline \texttt{accessibility \ Label},\newline \texttt{accessibility \ State=\{\{selected: ...\}\}} \\
\hline
Category Tabs (Overview, Accessibility tabs) & tab & 1.4.3 Contrast (AA)\newline 4.1.2 Name, Role, Value (A) & Mobile navigation patterns; Screen-specific tab patterns & \texttt{accessibility \ Role="tab"},\newline \texttt{accessibility \ State=\{\{selected: ...\}\}} \\
\hline
Tab Icons (eye, info icons in tabs) & none & 1.1.1 Non-text Content (A) & SC 1.1.1 applied to reduce unnecessary focus stops in mobile screen reader navigation & \texttt{importantFor \ Accessibility="no-hide \ descendants"} \\
\hline
Framework Info Card (main content card showing framework details) & none & 1.3.1 Info and Relationships (A)\newline 1.4.3 Contrast (AA) & SC 1.3.1 adapted for view-specific information grouping & Semantic container with proper styling \\
\hline
Statistic Cards (feature comparison boxes) & button & 1.4.3 Contrast (AA)\newline 4.1.2 Name, Role, Value (A) & Information presentation in mobile context; SC 2.5.8 for mobile touch targets & \texttt{accessibility \ Role="button"},\newline \texttt{accessibility \ Label} \\
\hline
Rating Bar (progress indicators for features) & progressbar & 1.4.3 Contrast (AA)\newline 4.1.2 Name, Role, Value (A) & Progress visualization in mobile context & \texttt{accessibility \ Role="progressbar"},\newline \texttt{accessibility \ Label} \\
\hline
Info Button (question mark buttons) & button & 1.4.3 Contrast (AA)\newline 4.1.2 Name, Role, Value (A) & SC 4.1.2 applied to mobile touch affordances & \texttt{accessibility \ Role="button"},\newline \texttt{accessibility \ Label} \\
\hline
Modal Dialog (detailed comparison popups) & dialog & 2.4.3 Focus Order (A)\newline 4.1.2 Name, Role, Value (A) & SC 2.4.3 adapted for mobile screen navigation; Mobile view modal context & \texttt{accessibility \ ViewIsModal},\newline Focus management implementation \\
\hline
Modal Tabs (tabs within modal dialogs) & tablist & 2.4.7 Focus Visible (AA)\newline 4.1.2 Name, Role, Value (A) & Mobile-specific tab interface patterns & \texttt{accessibility \ Role="tablist"},\newline \texttt{accessibility \ State=\{\{selected: isActive\}\}} \\
\hline
\end{longtable}

\FloatBarrier

\subsubsection{Formal methodology system implementation}

The Framework comparison screen implements a formal, academically rigorous methodology system that establishes a systematic approach to framework evaluation. Unlike other screens in the application that focus on practical implementation examples, this screen incorporates a formal methodological framework evidenced in Figure~\ref{fig:methodology_screens_sidebyside}.

\begin{figure}[ht]
    \centering
    \begin{subfigure}[b]{0.48\textwidth}
        \centering
        \includegraphics[width=\linewidth, alt={Methodology tab of the Framework Comparison screen}]{img/methodology1.jpg}
        \caption{First part of Methodology tab}
        \label{fig:methodology-1}
    \end{subfigure}
    \hfill
    \begin{subfigure}[b]{0.48\textwidth}
        \centering
        \includegraphics[width=\linewidth, alt={Framework comparison screen with Flutter selected}]{img/methodology2.jpg}
        \caption{Second part of Methodology tab}
        \label{fig:methodology-2}
    \end{subfigure}
    \caption{Methodology tabs of Framework comparison screen}
    \label{fig:methodology_screens_sidebyside}
\end{figure}

\FloatBarrier

The formal methodology system implements several critical characteristics of academic accessibility research:

\begin{enumerate}
    \item \textbf{Explicit methodology declaration}: The screen clearly states the research methodology used, including both empirical testing and documentation analysis;
    
    \item \textbf{Transparent testing protocol}: The methodology specifically documents testing with screen readers on precisely identified devices (VoiceOver iOS 16, TalkBack Android 13) and references WCAG 2.2 compliance verification;
    
    \item \textbf{Source citation}: The methodology includes proper citation of academic sources, establishing an evidence-based foundation;
    
    \item \textbf{Device specification}: The methodology includes explicit hardware specifications (iPhone 14, Pixel 7), creating reproducibility for the evaluation.
\end{enumerate}

This methodology implementation exemplifies the application's commitment to rigorous, evidence-based accessibility evaluation that moves beyond subjective assessments to create verifiable, reproducible comparisons.

\subsubsection{Academic reference implementation}

A distinctive feature of the Framework comparison screen is its comprehensive implementation of academic references. This feature directly connects framework evaluation to peer-reviewed research and formal documentation, creating an evidence-based foundation for accessibility comparisons. Figure~\ref{fig:academic_references} shows the academic references card and its expanded modal dialog.

\begin{figure}[ht]
    \centering
    \begin{subfigure}[b]{0.48\textwidth}
        \centering
        \includegraphics[width=\linewidth, alt={Academic References card displaying an overview of sources}]{img/methodology-academic-overview.jpg}
        \caption{Academic references overview}
        \label{fig:academic-references-card}
    \end{subfigure}
    \hfill
    \begin{subfigure}[b]{0.48\textwidth}
        \centering
        \includegraphics[width=\linewidth, alt={Academic References modal dialog showing detailed citations}]{img/methodology-academic-references.jpg}
        \caption{References modal with formal citations}
        \label{fig:academic-references-modal}
    \end{subfigure}
    \caption{Academic references implementation with formal citation structure}
    \label{fig:academic_references}
\end{figure}

\FloatBarrier

The academic reference system implements several key features:

\begin{enumerate}
    \item \textbf{Formal citation structure}: Each reference includes complete bibliographic information including authors, publication venue, year, and DOI identifiers where applicable;
    
    \item \textbf{Multi-category reference system}: References are categorized by type (research paper, official documentation), creating a clear hierarchy of evidence;
    
    \item \textbf{Modal dialog organization}: References are presented in a structured, tabbed modal dialog with overview, details, and references tabs, providing progressive disclosure of information;
    
    \item \textbf{Systematic literature inclusion}: The reference system integrates both primary research by Gaggi and Perinello \cite{perinello2024accessibility} and systematic reviews by Palmieri \cite{palmieri2022accessibility}, creating a comprehensive evidence base.
\end{enumerate}

Figure~\ref{fig:academic_references_detail} shows the extended References tab with complete citation information including access dates and URLs for official documentation.

\begin{figure}[ht]
    \centering
    \includegraphics[width=0.4\linewidth, alt={Reference tab showing detailed citation information}]{img/references.jpg}
    \caption{References tab showing formatted citations with complete bibliographic information}
    \label{fig:academic_references_detail}
\end{figure}

\FloatBarrier

This academic reference implementation exemplifies how accessibility evaluation can be grounded in formal, verifiable sources, creating accountability and reproducibility in framework comparison.

\subsubsection{Framework data structure}

The Framework comparison screen implements a comparative analysis through a structured data repository for both React Native and Flutter. This structured approach is visualized through framework selection buttons and framework-specific information cards shown in Figure~\ref{fig:framework_selection}.

\begin{figure}[ht]
    \centering
    \begin{subfigure}[b]{0.48\textwidth}
        \centering
        \includegraphics[width=\linewidth, alt={React Native framework card showing details}]{img/overview1.jpg}
        \caption{React Native framework details}
        \label{fig:react-native-details}
    \end{subfigure}
    \hfill
    \begin{subfigure}[b]{0.48\textwidth}
        \centering
        \includegraphics[width=\linewidth, alt={Flutter framework card showing details}]{img/overview2.jpg}
        \caption{Flutter framework details}
        \label{fig:flutter-details}
    \end{subfigure}
    \caption{Framework selection interface showing structured framework data}
    \label{fig:framework_selection}
\end{figure}

The framework data structure implements several key features:

\begin{enumerate}
    \item \textbf{Consistent metadata structure}: Each framework includes consistent metadata fields (name, company, version, description), enabling direct comparison;
    
    \item \textbf{Visual state indication}: The selected framework is visually indicated through background color changes and maintains this state for screen readers via \texttt{accessibilityState};
    
    \item \textbf{Feature categorization}: Framework features are consistently categorized into core attributes (\texttt{Language, Learning Curve, Hot Reload}), creating a systematic comparison structure;
    
    \item \textbf{Quantitative representation}: Features include both qualitative descriptions (e.g., "Moderate" learning curve) and quantitative indicators where applicable, enabling objective comparison.
\end{enumerate}

This structured data approach transforms subjective framework comparisons into a systematic, consistent evaluation framework that enables developers to make evidence-based decisions about framework selection based on accessibility considerations.

\subsubsection{Implementation complexity analysis}

A key contribution of the Framework comparison screen is its formal analysis of implementation complexity across frameworks. Unlike simple feature comparisons, this screen implements a detailed complexity analysis using multiple metrics. Figure~\ref{fig:implementation_complexity} shows both the complexity analysis card and its expanded modal view.

\begin{figure}[ht]
    \centering
    \begin{subfigure}[b]{0.48\textwidth}
        \centering
        \includegraphics[width=\linewidth, alt={Implementation Complexity Analysis card}]{img/implementation-methodology.jpg}
        \caption{Implementation complexity overview}
        \label{fig:implementation-complexity-card}
    \end{subfigure}
    \hfill
    \begin{subfigure}[b]{0.48\textwidth}
        \centering
        \includegraphics[width=\linewidth, alt={Implementation Details modal showing comparison metrics}]{img/implementation-calculation.jpg}
        \caption{Implementation details comparison}
        \label{fig:implementation-details-modal}
    \end{subfigure}
    \caption{Implementation complexity analysis with detailed metrics}
    \label{fig:implementation_complexity}
\end{figure}

\FloatBarrier

The implementation complexity analysis incorporates multiple evaluation dimensions:

\begin{enumerate}
    \item \textbf{Lines of code (LOC) metric}: The analysis quantifies implementation complexity through precise LOC counts for each accessibility feature, providing an objective measure of implementation effort;
    
    \item \textbf{Qualitative complexity assessment}: Features are categorized using a standardized Low/Medium/High complexity scale, with color coding (green/yellow/red) for visual differentiation;
    
    \item \textbf{Framework knowledge requirement}: The analysis considers the required knowledge of framework-specific concepts, addressing the learning curve aspect of accessibility implementation;
    
    \item \textbf{Real-world testing verification}: Complexity assessments are validated through testing on real devices with actual screen readers, ensuring practical relevance.
\end{enumerate}

Figure~\ref{fig:implementation_details} shows the detailed implementation comparison for specific accessibility features across both frameworks.

\begin{figure}[ht]
    \centering
    \begin{subfigure}[b]{0.48\textwidth}
        \centering
        \includegraphics[width=\linewidth, alt={Implementation Details tab showing feature comparisons}]{img/implementation2.jpg}
        \caption{Feature implementation comparison}
        \label{fig:implementation-feature-comparison}
    \end{subfigure}
    \hfill
    \begin{subfigure}[b]{0.48\textwidth}
        \centering
        \includegraphics[width=\linewidth, alt={Implementation Details tab showing code examples}]{img/implementation1.jpg}
        \caption{Implementation code examples}
        \label{fig:implementation-code-examples}
    \end{subfigure}
    \caption{Implementation details showing feature-level comparison and code examples}
    \label{fig:implementation_details}
\end{figure}

\FloatBarrier

The feature-level implementation analysis in Figure~\ref{fig:implementation_details} demonstrates significant differences between frameworks:

\begin{enumerate}
    \item \textbf{Heading element implementation}: React Native requires 7 LOC with Low complexity, while Flutter requires 11 LOC with Medium complexity;
    
    \item \textbf{Language declaration}: The contrast is more pronounced for language declaration, with React Native requiring 7 LOC and Flutter requiring 21 LOC;
    
    \item \textbf{Default accessibility status}: The analysis shows React Native has 1/3 features accessible by default, while Flutter has 0/3, providing a clear metric for out-of-the-box accessibility;
    
    \item \textbf{Total implementation overhead}: React Native requires 21 total LOC for implementation, while Flutter requires 46 LOC, quantifying the overall implementation effort difference.
\end{enumerate}

This detailed implementation comparison provides developers with concrete, evidence-based metrics for understanding the accessibility implementation effort required for each framework.

\subsubsection{Specific accessibility feature comparison}

The Framework comparison screen implements detailed comparisons of specific accessibility features, providing both implementation attributes and code examples. Figure~\ref{fig:feature_implementation_sidebyside} shows implementation details for language declaration and text abbreviations in React Native.

\begin{figure}[ht]
    \centering
    \begin{subfigure}[b]{0.48\textwidth}
        \centering
        \includegraphics[width=\linewidth, alt={First part of Language in Overview tab of the Framework comparison screen}]{img/language-information.jpg}
        \caption{First part of Language tab of Overview}
        \label{fig:language-1}
    \end{subfigure}
    \hfill
    \begin{subfigure}[b]{0.48\textwidth}
        \centering
        \includegraphics[width=\linewidth, alt={Second part of Language in Overview tab of the Framework comparison screen}]{img/language-calculation.jpg}
        \caption{Second part of Language tab of Overview}
        \label{fig:language-2}
    \end{subfigure}
    \caption{Language modals of Framework comparison screen}
    \label{fig:feature_implementation_sidebyside}
\end{figure}

\FloatBarrier

The feature implementation comparison includes several key elements:

\begin{enumerate}
    \item \textbf{WCAG success criteria mapping}: Each feature is explicitly mapped to relevant WCAG criteria (e.g., Text Abbreviations maps to 3.1.4), creating a clear connection between implementation and compliance;
    
    \item \textbf{Default accessibility status}: The comparison explicitly indicates whether each feature is accessible by default (Yes/No), highlighting areas requiring developer intervention;
    
    \item \textbf{Implementation notes}: Each feature includes specific implementation notes explaining the required approach (e.g., "Requires adding accessibilityLabel property");
    
    \item \textbf{Concrete code examples}: The comparison provides complete, executable code examples that developers can directly reference for implementation.
\end{enumerate}

This feature-level comparison transforms abstract accessibility requirements into concrete implementation guidance with direct reference to standards compliance, helping developers understand not just what to implement but why and how.

\subsubsection{Modal dialog accessibility implementation}

The Framework comparison screen implements several modal dialogs that incorporate comprehensive accessibility features. Figure~\ref{fig:modal_accessibility} shows the implementation details modal with properly structured tabs.

\begin{figure}[ht]
    \centering
    \begin{subfigure}[b]{0.48\textwidth}
        \centering
        \includegraphics[width=\linewidth, alt={Modal dialog showing implementation details}]{img/implementation-methodology.jpg}
        \caption{Implementation details modal}
        \label{fig:implementation-details-dialog}
    \end{subfigure}
    \hfill
    \begin{subfigure}[b]{0.48\textwidth}
        \centering
        \includegraphics[width=\linewidth, alt={Modal dialog showing academic references for Implementation tab section}]{img/implementation-references.jpg}
        \caption{References of Implementation modal}
        \label{fig:academic-references-dialog}
    \end{subfigure}
    \caption{Modal dialogs for the Implementation Tab}
    \label{fig:modal_accessibility}
\end{figure}

\FloatBarrier

The modal dialog implementation addresses several critical accessibility requirements:

\begin{enumerate}
    \item \textbf{Clear semantic role}: Modals are properly identified with \texttt{accessibilityViewIsModal}, ensuring screen readers understand their role;
    
    \item \textbf{Tab role assignment}: Tab navigation properly implements \texttt{accessibilityRole="tab"} and \texttt{accessibilityState} to communicate selection state;
    
    \item \textbf{Focus management}: When modals open, focus moves to the modal header and returns to the triggering element when closed, maintaining context;
    
    \item \textbf{Hierarchical content structure}: Content within modals maintains proper heading structure and semantic relationships, ensuring screen reader users can navigate efficiently.
\end{enumerate}

This accessible modal implementation ensures that complex information like methodology details and academic references remains accessible to all users, including those using screen readers.

\subsubsection{Screen reader support analysis}

Table~\ref{tab:framework_comparison_screen_reader} presents results from systematic testing of the Framework comparison screen with screen readers on both iOS and Android platforms, addressing WCAG2Mobile's emphasis on platform-specific accessibility services.

\begin{longtable}[c]{|C{2.8cm}|C{3.5cm}|C{3.5cm}|C{4cm}|}
\caption{Framework comparison screen screen reader testing with WCAG2Mobile considerations}
\label{tab:framework_comparison_screen_reader}\\
\hline
\textbf{Test Case} & \textbf{VoiceOver (iOS 16)} & \textbf{TalkBack (Android 14-15)} & \textbf{WCAG2Mobile Considerations} \\
\hline
\endfirsthead
\multicolumn{4}{c}%
{{\bfseries Table \thetable\ -- continued from previous page}} \\
\hline
\textbf{Test Case} & \textbf{VoiceOver (iOS 16)} & \textbf{TalkBack (Android 14-15)} & \textbf{WCAG2Mobile Considerations} \\
\hline
\endhead
\hline
\multicolumn{4}{r}{{Continued on next page}} \\
\endfoot
\hline
\endlastfoot
Hero Title & {\ding{51}} Announces ``Framework Comparison, heading'' & {\ding{51}} Announces ``Framework Comparison, heading'' & SC 2.4.6 applied to screens rather than pages; Proper heading role in mobile context \\
\hline
Framework Selection & {\ding{51}} Announces framework name and selection state & {\ding{51}} Announces framework name and selection state & SC 4.1.2 adapted for mobile platform accessibility services; State changes properly announced \\
\hline
Category Tabs & {\ding{51}} Announces tab name and selection state & {\ding{51}} Announces tab name and selection state & Mobile-specific tab navigation patterns; Platform accessibility service integration \\
\hline
Framework Info & {\ding{51}} Announces framework details in logical order & {\ding{51}} Announces framework details in logical order & SC 1.3.1 and 1.3.2 adapted for mobile view structure and reading sequence \\
\hline
Statistic Cards & {\ding{51}} Announces label and value & {\ding{51}} Announces label and value & Mobile-specific information presentation; Compact data communication \\
\hline
Rating Bars & {\ding{51}} Announces value and range & {\ding{51}} Announces value and range & SC 4.1.2 implemented for mobile progress visualization; Platform-specific value reporting \\
\hline
Info Buttons & {\ding{51}} Announces purpose and action & {\ding{51}} Announces purpose and action & SC 2.4.4 applied to mobile touch affordances; Platform-specific action indication \\
\hline
Modal Dialog Opening & {\ding{51}} Focus moves to dialog title & {\ding{51}} Focus moves to dialog title & SC 2.4.3 adapted for mobile screen navigation; Mobile view modal context \\
\hline
Modal Tab Navigation & {\ding{51}} Announces tab selection state & {\ding{51}} Announces tab selection state & Mobile-specific tab interface patterns; Platform-specific control behavior \\
\hline
Implementation Examples & {\ding{51}} Announces code examples as text blocks & {\ding{51}} Announces code examples as text blocks & SC 1.3.1 adapted for mobile code presentation; Platform-specific text handling \\
\hline
Color-coded Complexity & {\ding{51}} Announces complexity level, not just color & {\ding{51}} Announces complexity level, not just color & SC 1.4.1 implemented for mobile view contexts; Platform-specific color handling \\
\hline
\end{longtable}

\FloatBarrier

The implementation addresses several key screen reader considerations compliant to MCAG and WCAG2Mobile:

\begin{enumerate}
    \item \textbf{State communication}: Selection states for frameworks and tabs are explicitly communicated via \texttt{accessibilityState}, ensuring screen reader users understand current selection;
    
    \item \textbf{Logical focus order}: Screen elements follow a logical navigation order that matches visual presentation, creating a coherent mental model for screen reader users;
    
    \item \textbf{Code example accessibility}: Code examples are presented in accessible text blocks with proper structure, rather than as images, ensuring screen reader users can access implementation details;
    
    \item \textbf{Non-reliance on color}: Information conveyed through color (complexity indicators) is redundantly provided through explicit text labels, ensuring color-blind users and screen reader users receive the same information.
\end{enumerate}

\subsubsection{Implementation overhead analysis}

Table~\ref{tab:framework_comparison_overhead} quantifies the additional code required to implement accessibility features in the Framework comparison screen.

\begin{longtable}[c]{|C{3.8cm}|C{2.3cm}|C{2.8cm}|C{2.8cm}|}
\caption{Framework comparison screen accessibility implementation overhead}
\label{tab:framework_comparison_overhead}\\
\hline
\textbf{Accessibility Feature} & \textbf{Lines of Code} & \textbf{Percentage of Total Code} & \textbf{Complexity Impact} \\
\hline
\endfirsthead
\multicolumn{4}{c}%
{{\bfseries Table \thetable\ -- continued from previous page}} \\
\hline
\textbf{Accessibility Feature} & \textbf{Lines of Code} & \textbf{Percentage of Total Code} & \textbf{Complexity Impact} \\
\hline
\endhead
\hline
\multicolumn{4}{r}{{Continued on next page}} \\
\endfoot
\hline
\endlastfoot
Semantic Roles & 24 LOC & 2.8\% & Low \\
\hline
Accessibility Labels & 36 LOC & 4.2\% & Medium \\
\hline
Element Hiding & 22 LOC & 2.6\% & Low \\
\hline
Focus Management & 28 LOC & 3.3\% & High \\
\hline
Accessibility Values & 18 LOC & 2.1\% & Medium \\
\hline
Status Announcements & 16 LOC & 1.9\% & Medium \\
\hline
Modal Accessibility & 32 LOC & 3.7\% & High \\
\hline
Tab Role Assignment & 12 LOC & 1.4\% & Medium \\
\hline
Accessibility State & 20 LOC & 2.3\% & Medium \\
\hline
Rating Bar Accessibility & 22 LOC & 2.6\% & Medium \\
\hline
\textbf{Total} & \textbf{230 LOC} & \textbf{26.9\%} & \textbf{Medium-High} \\
\end{longtable}

\FloatBarrier

This analysis reveals that implementing comprehensive accessibility for the Framework comparison screen adds approximately 26.9\% to the code base. The most significant contributors are accessibility labels (4.2\%) and modal accessibility (3.7\%), reflecting the information-rich nature of this screen and the complex interaction patterns with modals and tabs. The implementation overhead is justified by the improved user experience for people with disabilities and the educational value demonstrated through the formal framework comparison.

\subsubsection{WCAG conformance by principle}

Table~\ref{tab:framework_comparison_wcag2mobile} provides a detailed analysis of WCAG 2.2 compliance by principle for the Framework comparison screen, with specific attention to WCAG2Mobile interpretations:

\begin{longtable}[c]{|C{2.5cm}|C{3.8cm}|C{3.2cm}|C{5.2cm}|}
\caption{Framework comparison screen WCAG compliance with WCAG2Mobile interpretations}
\label{tab:framework_comparison_wcag2mobile}\\
\hline
\textbf{Principle} & \textbf{Description} & \textbf{Implementation Level} & \textbf{Key Success Criteria with Mobile Context} \\
\hline
\endfirsthead
\multicolumn{4}{c}%
{{\bfseries Table \thetable\ -- continued from previous page}} \\
\hline
\textbf{Principle} & \textbf{Description} & \textbf{Implementation Level} & \textbf{Key WCAG2Mobile Interpretations} \\
\hline
\endhead
\hline
\multicolumn{4}{r}{{Continued on next page}} \\
\endfoot
\hline
\endlastfoot
1. Perceivable & Information and UI components must be presentable to users in ways they can perceive & 12/13 (92\%) & SC 1.1.1 applied to reduce unnecessary focus stops\newline SC 1.3.1 adapted for view-specific information grouping\newline SC 1.4.1 implemented for mobile view contexts\newline SC 1.4.3 applied to mobile viewing conditions \\
\hline
2. Operable & UI components and navigation must be operable & 16/17 (94\%) & SC 2.4.3 adapted for mobile screen navigation\newline SC 2.4.6 applied to screens rather than pages\newline SC 2.5.8 implemented for mobile touch targets\newline SC 2.4.7 applied to mobile focus indication \\
\hline
3. Understandable & Information and operation of UI must be understandable & 9/10 (90\%) & SC 3.1.1 and 3.1.2 adapted for mobile views\newline SC 3.2.1 and 3.2.2 applied to mobile input methods\newline SC 3.2.4 implemented for consistent mobile patterns\newline SC 3.3.2 adapted for mobile input contexts \\
\hline
4. Robust & Content must be robust enough to be interpreted by a wide variety of user agents & 3/3 (100\%) & SC 4.1.2 adapted for mobile platform accessibility services\newline SC 4.1.3 implemented for mobile status notifications \\
\hline
\end{longtable}

\FloatBarrier

The WCAG compliance analysis in Table~\ref{tab:framework_comparison_wcag2mobile} demonstrates exceptional conformance across all principles, with particular strength in the Robust principle (100\% compliance). The implementation of multiple AAA criteria, including 1.4.8 (Visual Presentation), 2.4.8 (Location), 2.4.9 (Link Purpose - Link Only), and 3.2.5 (Change on Request) exceeds standard requirements to create an enhanced user experience for people with disabilities. 
The Framework comparison screen achieves high compliance across all WCAG principles, with particular strength in the Perceivable, Operable, and Robust principles.

\subsubsection{Mobile-specific considerations}

The Framework comparison screen addresses several mobile-specific accessibility considerations beyond standard WCAG requirements:

\begin{enumerate}
    \item \textbf{Framework-specific adaptation}: The comparison explicitly evaluates each framework's support for mobile-specific accessibility features, addressing the challenges of cross-platform development;
    
    \item \textbf{Screen reader gesture support}: The evaluation includes specific assessment of each framework's support for screen reader gestures, a mobile-specific consideration not fully captured in WCAG;
    
    \item \textbf{Touch target optimization}: Interactive elements implement generous touch targets exceeding the minimum size requirements, addressing the specific challenges of touch interaction;
    
    \item \textbf{Responsive layout adaptation}: The interface adapts to different screen orientations and device types, maintaining accessibility across the varied landscape of mobile form factors;
    
    \item \textbf{Platform-specific implementation patterns}: The comparison addresses the platform-specific nature of accessibility implementation, recognizing that mobile accessibility often requires different approaches for iOS and Android.
\end{enumerate}

\subsubsection{Screen reader support comparison methodology}

The Framework comparison screen implements a rigorous methodology for evaluating screen reader support across frameworks. This methodology, shown in Figure~\ref{fig:screen_reader_methodology}, quantifies support using multiple dimensions with specific weighting.

\begin{figure}[ht]
    \centering
    \begin{subfigure}[b]{0.48\textwidth}
        \centering
        \includegraphics[width=\linewidth, alt={Screen reader support overview}]{img/screen-reader-overview.jpg}
        \caption{Screen reader overview}
        \label{fig:screen-reader-overview}
    \end{subfigure}
    \hfill
    \begin{subfigure}[b]{0.48\textwidth}
        \centering
        \includegraphics[width=\linewidth, alt={Screen reader support metrics calculation}]{img/screen-reader-methodology.jpg}
        \caption{Screen reader metrics calculation}
        \label{fig:screen-reader-calculation}
    \end{subfigure}
    \caption{Screen reader support comparison methodology and calculation approach}
    \label{fig:screen_reader_methodology}
\end{figure}

The screen reader support analysis implements a formal evaluation methodology with the following characteristics:

\begin{enumerate}
    \item \textbf{Component-based weighting}: Screen reader compatibility receives a 30\% weighting in the overall accessibility score, recognizing its critical importance for blind users;
    
    \item \textbf{Platform-specific assessment}: The methodology explicitly evaluates both VoiceOver (iOS) and TalkBack (Android) support separately, addressing the platform-specific nature of screen reader implementation;
    
    \item \textbf{Multi-dimensional evaluation}: Screen reader support is assessed across multiple dimensions including announcement quality, gesture support, and semantics interpretation;
    
    \item \textbf{Empirical validation}: Ratings are based on actual testing with specific device and operating system configurations, creating reproducible results.
\end{enumerate}

This formal screen reader evaluation methodology establishes a systematic approach to comparing screen reader support that moves beyond subjective assessments to create quantifiable, verifiable metrics.

\subsubsection{Implementation complexity calculation methodology}

The Framework comparison screen implements a formal methodology for calculating implementation complexity, as was shown by Figure~\ref{fig:implementation_complexity}.

The implementation complexity calculation methodology includes several key components:

\begin{enumerate}
    \item \textbf{Explicit mathematical formula}: The methodology includes a formal mathematical formula for calculating implementation scores based on lines of code: \\\texttt{implementationScore = Math.max(0, Math.min(5, 5 - (totalLinesOfCode / 10)))};
    
    \item \textbf{Normalization mechanism}: Scores are explicitly normalized to a 0-5 scale to enable consistent comparison;
    
    \item \textbf{LOC-based quantification}: The formula establishes an inverse relationship between lines of code and implementation score, recognizing that lower implementation overhead (fewer LOC) represents better accessibility support;
    
    \item \textbf{Source citation}: The methodology cites specific sources including static code analysis and WCAG 2.2 implementation examples.
\end{enumerate}

This formal calculation methodology creates a transparent, reproducible approach to quantifying implementation complexity that enables objective framework comparison.

\subsubsection{Feature-specific implementation comparison}

The Framework comparison screen implements detailed feature-by-feature comparison of accessibility implementations across frameworks. Figure~\ref{fig:screen_reader_methodology} shows the implementation details for specific accessibility features.

The feature-specific comparison implements several key analytical elements:

\begin{enumerate}
    \item \textbf{Consistent metric application}: Each feature is evaluated using the same metrics (LOC, complexity) across frameworks, enabling direct comparison;
    
    \item \textbf{Complexity visualization}: Complexity ratings are visually coded (green for Low, orange for Medium) for quick comprehension while maintaining accessibility through text labels;
    
    \item \textbf{WCAG criteria mapping}: Features are explicitly mapped to relevant WCAG success criteria (e.g., Heading Elements to 1.3.1, 2.4.6), creating a standards-based evaluation framework;
    
    \item \textbf{Code example comparison}: Implementation examples show directly comparable code implementations for identical features, highlighting the practical differences in syntax and structure.
\end{enumerate}

This feature-level comparison transforms abstract accessibility requirements into concrete, comparable implementations that developers can directly reference for their own work.

\chapter{Beyond WCAG - Extended accessibility principles in \textit{AccessibleHub}}
\label{chap:beyond-wcag}

\section{Introduction}
\label{sec:beyond-wcag-intro}

While the Web Content Accessibility Guidelines (WCAG) provide a robust framework for ensuring digital accessibility, they were primarily developed for web content and do not fully address the unique challenges of mobile interfaces. Mobile applications present distinct accessibility concerns related to touch interaction, limited screen real estate, variable usage contexts, and platform-specific implementations. Additionally, WCAG focuses primarily on technical compliance rather than the broader educational, social, and developmental aspects of creating accessible applications.

AccessibleHub extends beyond standard WCAG criteria in several important ways:

\begin{enumerate}
    \item \textbf{Implementation focus}: Where WCAG defines \textit{what} should be accessible, AccessibleHub demonstrates \textit{how} to implement accessibility in practical code patterns;
    
    \item \textbf{Quantitative measurement}: AccessibleHub introduces formal metrics for measuring accessibility implementation overhead and compliance levels, making abstract guidelines concrete and measurable;
    
    \item \textbf{Educational framework}: AccessibleHub embeds pedagogical principles that facilitate developer learning, extending accessibility from a compliance exercise to an educational journey;
    
    \item \textbf{Mobile-specific adaptations}: The application addresses mobile-specific challenges that fall outside traditional WCAG criteria, such as touch target optimization, swipe efficiency, and battery considerations;
    
    \item \textbf{Social learning integration}: AccessibleHub recognizes that accessibility implementation is both a technical and social process, connecting developers to communities of practice and collaborative learning resources.
\end{enumerate}

The following sections analyze each screen of AccessibleHub through this extended lens, identifying principles that go beyond standard WCAG criteria while contributing to more accessible, inclusive mobile experiences. These principles collectively form an extended theory of mobile accessibility that bridges technical compliance with practical implementation, educational effectiveness, and social engagement.

By systematically documenting these extended principles, we aim to advance accessibility theory beyond compliance-focused approaches toward a more holistic understanding of how developers learn, implement, and socialize accessibility practices. These insights can guide future accessibility tool development, educational resources, and implementation methodologies.

\section{Screen-specific accessibility guidelines}
\label{sec:screen-specific-principles}

\subsection{Home: Metrics-driven}

The Home screen implementation highlights several accessibility principles that extend beyond standard WCAG requirements, specifically addressing quantitative accessibility evaluation in mobile applications:

\begin{enumerate}
    \item \textbf{Comprehensive metrics visualization}: Accessibility compliance are quantified and presented in a transparent, understandable format. The Home screen implements this through dedicated metric cards with clear visual indicators of implementation status, moving beyond binary compliance to represent different degrees of accessibility achievement;
    
    \item \textbf{Multi-dimensional evaluation framework}: Accessibility assessment consider multiple dimensions including component implementation, standards compliance, and empirical testing. This weighted approach, implemented in the metrics calculation system, recognizes that true accessibility extends beyond technical conformance to include real-world usability;
    
    \item \textbf{Transparency in methodology}: Applications should provide clear documentation of accessibility evaluation methodology including test devices, standards versions, and measurement approaches. The modal details system implements this principle by exposing the entire evaluation framework to users, creating accountability in accessibility claims;
    
    \item \textbf{Academic grounding principle}: Accessibility implementations benefit from explicit connection to peer-reviewed research and formal standards. The References tab implements this by connecting implementation practices to specific academic papers and standards documentation;
    
    \item \textbf{Progressive disclosure of complexity}: Technical accessibility details should be organized in layers of increasing complexity, allowing users to access the appropriate level of detail for their needs. The tabbed modal system implements this by separating overview information from detailed implementation specifics.
\end{enumerate}

\subsection{Framework comparison: Evidence-based evaluation}

The Framework comparison screen embodies several principles that extend beyond standard WCAG requirements, particularly focusing on evidence-based evaluation and formal methodology:

\begin{enumerate}
    \item \textbf{Academic grounding principle}: Accessibility evaluations are grounded in peer-reviewed research and formal methodologies. The screen implements this through explicit citations to academic papers and clearly defined evaluation protocols;
    
    \item \textbf{Quantitative metric transparency}: Framework evaluations use explicit, quantitative metrics with clearly defined calculation methodologies. The implementation demonstrates this through LOC counts and explicit complexity ratings;
    
    \item \textbf{Implementation complexity consideration}: Accessibility evaluation assess not just feature presence but implementation complexity. The screen implements this through multi-dimensional complexity assessment (LOC, qualitative rating, knowledge requirements);
    
    \item \textbf{Empirical testing validation}: Accessibility claims are validated through documented testing on specific devices and platforms. The implementation includes explicit references to test devices and operating system versions;
    
    \item \textbf{Comparative analysis principle}: Accessibility features are evaluated through direct side-by-side comparison using consistent metrics. The screen implements this through structured comparison of identical features across frameworks.
\end{enumerate}

\subsection{Best practices main screen: Pedagogical accessibility guidelines}

The Best practices screen defines several educational principles that extend beyond standard WCAG requirements, addressing how accessibility knowledge is structured and presented to developers:

\begin{enumerate}
    \item \textbf{Multi-modal learning principle}: Accessibility education combines different learning modalities (documentation, code examples, interactive guides) to accommodate diverse learning styles. The Best practices screen implements this through explicit categorization of each practice with appropriate badges (Documentation, Code Examples, Interactive Guide) that indicate the learning approach;
    
    \item \textbf{Conceptual categorization}: Accessibility practices are organized by conceptual domain (guidelines, structure, gestures, screen readers, navigation) rather than by technical implementation details. This organization recognizes that developers approach accessibility from different conceptual entry points based on their specific challenges and interests;
    
    \item \textbf{Visual encoding of content types}: Different types of accessibility guidance are visually differentiated through consistent color coding and iconography. The Best practices screen implements this through a formal color system that assigns specific colors to each practice category, reinforcing the conceptual boundaries between different accessibility domains;
    
    \item \textbf{Feature-level accessibility indication}: Each practice area explicitly indicates the specific accessibility features it addresses. The implementation of feature lists with focused icons and labels ensures developers can quickly identify relevant guidelines for particular accessibility challenges;
    
    \item \textbf{Platform-specific guidance principle}: Accessibility education explicitly acknowledges platform differences where relevant (e.g., for screen readers). The Screen Reader Support practice category explicitly indicates its platform-specific nature, recognizing that some accessibility implementations must adapt to platform constraints.
\end{enumerate}

\subsection{Best practices screens: Domain-specific patterns}
\label{subsec:best-practices-screens-beyond-wcag}

The Best Practices screens extend beyond standard WCAG requirements through specialized educational and implementation patterns tailored to specific accessibility domains.

\subsubsection{WCAG guidelines screen: Knowledge scaffolding}

The WCAG Guidelines screen establishes several patterns that extend beyond the guidelines themselves:

\begin{enumerate}
    \item \textbf{Principle-based organization}: Guidelines are organized around the four fundamental WCAG principles rather than success criteria numbers, creating a conceptual structure that emphasizes understanding over compliance checklists;
    
    \item \textbf{Visual principle encoding}: Each WCAG principle is assigned a distinct visual identity through color coding and iconography, reinforcing conceptual boundaries and aiding visual learners;
    
    \item \textbf{Concrete implementation examples}: Abstract success criteria are paired with concrete mobile implementation patterns, bridging the gap between theory and practice;
    
    \item \textbf{Progressive disclosure}: Guidelines are presented with layered complexity, starting with core concepts before revealing detailed requirements;
    
    \item \textbf{Framework-specific adaptations}: Guidelines acknowledge differences in implementation between React Native and Flutter, recognizing that accessibility requirements may have different technical expressions across frameworks.
\end{enumerate}

\subsubsection{Gestures tutorial screen: Interaction equivalence}

The Gestures Tutorial screen demonstrates principles that specifically address the challenges of touch-based interaction:

\begin{enumerate}
    \item \textbf{Adaptive interaction patterns}: Applications detect the user's interaction mode (standard touch vs. screen reader) and adapt their behavior accordingly, ensuring equivalent functionality regardless of input method;
    
    \item \textbf{Gesture alternative principle}: Every gesture-based interaction provides programmatic alternatives through accessibility actions, ensuring screen reader users can access all functionality;
    
    \item \textbf{Context-sensitive instruction}: Guidance changes based on the user's current interaction mode, providing relevant information without overwhelming with unnecessary details;
    
    \item \textbf{Multi-sensory feedback}: Gesture completion confirmation is provided through multiple sensory channels (visual, auditory, haptic), ensuring users receive feedback regardless of sensory capabilities;
    
    \item \textbf{Educational comparison}: Standard gestures and screen reader gestures are explicitly compared, helping developers understand the relationship between these interaction modes.
\end{enumerate}

\subsubsection{Semantic structure screen: Hierarchical organization}

The Semantic Structure screen extends accessibility concepts beyond visual presentation to information architecture:

\begin{enumerate}
    \item \textbf{Self-demonstrating implementation}: The screen itself implements the semantic structures it teaches, providing a meta-level educational experience where the medium demonstrates the message;
    
    \item \textbf{Translation from web to mobile}: Web semantic concepts (headings, landmarks) are explicitly adapted to mobile contexts, acknowledging the differences in platform capabilities;
    
    \item \textbf{Hierarchical navigation principle}: Content is organized in a clear hierarchy with appropriate heading levels and landmark roles, creating an efficient navigation structure for assistive technology users;
    
    \item \textbf{Code-output relationship visualization}: Semantic structures are shown both as code and rendered output, helping developers connect implementation choices with user experience outcomes;
    
    \item \textbf{Progressive semantic development}: Semantic concepts are introduced incrementally, starting with basic heading structure before introducing more complex landmark roles.
\end{enumerate}

\subsubsection{Logical navigation screen: Focus management}

The Logical Navigation screen addresses navigation patterns critical for non-visual users:

\begin{enumerate}
    \item \textbf{Bypass block implementation}: Skip links are implemented to allow users to bypass repetitive content, demonstrating a pattern rarely applied in mobile applications;
    
    \item \textbf{Synchronized visual-focus relationship}: Visual scrolling and accessibility focus are coordinated to maintain a consistent experience for all users;
    
    \item \textbf{Landmark-based navigation}: Content is organized using landmark roles that create navigation shortcuts for assistive technology users;
    
    \item \textbf{Focus persistence}: Focus position is maintained across view changes and dynamic content updates, preventing disorientation during navigation;
    
    \item \textbf{Focus restoration}: When temporary UI elements (dialogs, menus) are dismissed, focus returns to the triggering element, maintaining user context.
\end{enumerate}

\subsubsection{Screen reader support screen: Adaptive guidance}

The Screen Reader Support screen demonstrates platform-specific accessibility patterns:

\begin{enumerate}
    \item \textbf{Platform-tailored guidance}: Accessibility instructions are adapted to the specific capabilities and limitations of each platform's screen reader;
    
    \item \textbf{Gesture dictionary visualization}: Screen reader gestures are presented in a visual format that helps developers understand the non-visual navigation experience;
    
    \item \textbf{Code-gesture relationship}: Implementation patterns are explicitly connected to the screen reader gestures they support, creating a clear relationship between code and user experience;
    
    \item \textbf{Adaptive content presentation}: The screen adapts its content based on the selected platform, showing only relevant information for the current context;
    
    \item \textbf{Unified cross-platform patterns}: Despite platform differences, common patterns are identified that work consistently across platforms, helping developers create coherent cross-platform experiences.
\end{enumerate}

These domain-specific patterns collectively extend accessibility beyond technical compliance, addressing the educational, interaction, structural, navigational, and platform-specific challenges that developers face when creating truly accessible mobile applications.

\subsection{Accessible components main screen: Content categorization}

The Components screen defines several accessibility principles specifically focused on organizing and categorizing interface elements to promote systematic accessibility implementation:

\begin{enumerate}
    \item \textbf{Feature-oriented grouping}: Accessibility features are grouped by functional similarity rather than WCAG criteria, creating more intuitive implementation pathways. The Components screen implements this by organizing related controls together (e.g., "Buttons \& Touchables") regardless of which specific WCAG criteria they address;
    
    \item \textbf{Progressive implementation pathway}: Components are organized in a sequence that builds accessibility knowledge progressively, beginning with fundamental elements before introducing more complex patterns. The Components screen implements this through its hierarchical organization from basic elements (buttons) to complex patterns (dialogs, navigation);
    
    \item \textbf{Cross-cutting feature indication}: Key accessibility features that apply across multiple component types should be visually highlighted to reinforce their importance. The feature icons within each component card implement this by consistently identifying common accessibility considerations (e.g., touch target sizing, focus management);
    
    \item \textbf{Transition announcement principle}: Navigation between component categories should be explicitly announced to assist screen reader users in maintaining context. The Components screen implements this through the \texttt{announceForAccessibility} announcements during navigation;
    
    \item \textbf{Contextual documentation integration}: Each component implementation include direct references to relevant accessibility guidelines, helping developers understand not just how to implement accessibility features but why they matter. The code samples in each component detail screen implement this by including explanatory comments that connect implementation choices to specific WCAG success criteria.
\end{enumerate}

\subsection{Accessible components screens: Hierarchical complexity implementation}
\label{subsec:accessible-components-beyond-wcag}

The Accessible Components section of \textit{AccessibleHub} extends traditional WCAG requirements through practical implementation patterns across common mobile interface elements. 

\subsubsection{Buttons and touchables screen: Fundamental interaction accessibility}
\label{subsubsec:buttons-touchables-beyond-wcag}

The Buttons and touchables screen extends beyond standard WCAG requirements through several key implementation patterns:

\begin{enumerate}
    \item \textbf{Action-outcome oriented labeling}: Labels communicate not just the element identity but its purpose and outcome, enhancing user understanding beyond basic identification;
    
    \item \textbf{Multi-channel feedback}: Interaction feedback is provided through multiple channels (visual, auditory via screen reader announcement, and potentially haptic), ensuring perceivability regardless of user capabilities;
    
    \item \textbf{Visible state persistence}: Active and focus states remain visible longer than minimum requirements, addressing the challenge of transient touch interactions on mobile devices;
    
    \item \textbf{Enhanced target spacing}: Interactive elements are not only individually sized appropriately but spaced to prevent accidental activation of adjacent controls, extending beyond individual target size requirements;
    
    \item \textbf{Educational code-output relationship}: Implementation demonstrates the direct relationship between code properties and user experience outcomes, creating a self-teaching component.
\end{enumerate}

\subsubsection{Form screen: Complex input accessibility beyond technical compliance}
\label{subsubsec:forms-beyond-wcag}

The Form screen demonstrates accessibility principles that extend significantly beyond WCAG's input field requirements:

\begin{enumerate}
    \item \textbf{Semantic field grouping}: Related inputs are explicitly grouped with appropriate container roles, creating logical navigation units for screen reader users beyond basic labeling;
    
    \item \textbf{Contextual validation feedback}: Error messages are not just visually distinguished but programmatically linked to specific fields with appropriate roles and timing;
    
    \item \textbf{Progressive disclosure of complexity}: Form validation complexity is revealed progressively, with immediate feedback for critical errors but delayed validation for less critical fields;
    
    \item \textbf{Platform-optimized input methods}: The implementation leverages platform-native input methods where they provide superior accessibility, demonstrating a hybrid approach beyond framework-agnostic guidelines;
    
    \item \textbf{Comprehensive AAA-level implementation}: The form implements multiple AAA-level criteria including Help (3.3.5) and Error Prevention (3.3.6) through contextual guidance and validation.
\end{enumerate}

\subsubsection{Dialog screen: Focus management beyond basic modality}
\label{subsubsec:dialogs-beyond-wcag}

The Dialog screen implements accessibility principles that address the complex challenges of modal interaction contexts:

\begin{enumerate}
    \item \textbf{Focus restoration mechanism}: The implementation not only traps focus during dialog display but explicitly returns it to the triggering element upon dismissal, maintaining user context beyond basic modal guidelines;
    
    \item \textbf{Context announcement}: Dialog appearance and dismissal are explicitly announced to screen reader users, providing orientation beyond visual cues;
    
    \item \textbf{Multi-path dismissal}: Multiple interaction methods for closing dialogs are provided (explicit button, backdrop press, hardware back button), ensuring operability through different input methods;
    
    \item \textbf{Visual-semantic relationship}: Visual modal presentation is precisely aligned with semantic modal behavior, ensuring consistency between visual and programmatic experiences;
    
    \item \textbf{AAA-level user control}: All modal state changes occur only on explicit user request, implementing AAA criterion 3.2.5 (Change on Request).
\end{enumerate}

\subsubsection{Media screen: Beyond alternative text}
\label{subsubsec:media-beyond-wcag}

The Media screen extends basic alternative text requirements into a comprehensive framework for non-text content accessibility:

\begin{enumerate}
    \item \textbf{Context-aware alternative text}: Alternative descriptions adapt based on the image's context and purpose rather than providing generic descriptions;
    
    \item \textbf{Position indication}: Gallery navigation explicitly communicates the user's position within a sequence (e.g., "Image 2 of 5"), providing orientation beyond basic descriptions;
    
    \item \textbf{Educational transparency}: The implementation explicitly reveals alternative text visually as an educational feature, bridging the gap between visual and non-visual experiences;
    
    \item \textbf{Multi-modal controls}: Media navigation incorporates redundant control mechanisms optimized for different interaction methods;
    
    \item \textbf{User-controlled progression}: All media content changes occur only on explicit user action, implementing AAA criterion 3.2.5 (Change on Request).
\end{enumerate}

\subsubsection{Advanced components screen: Complex interaction patterns}
\label{subsubsec:advanced-beyond-wcag}

The Advanced components screen demonstrates accessibility principles for complex interface patterns that extend significantly beyond basic WCAG requirements:

\begin{enumerate}
    \item \textbf{Alternative interaction pathways}: Inherently visual controls like sliders offer multiple interaction methods optimized for different user capabilities;
    
    \item \textbf{State relationship communication}: Tab interfaces explicitly communicate the relationship between selectors and their associated content through programmatic connections;
    
    \item \textbf{Hierarchy-aware notification}: Alert components are implemented with appropriate live region properties based on their urgency and relationship to other content;
    
    \item \textbf{Value communication precision}: Progress indicators convey exact percentage values rather than approximations, providing precision beyond visual representations;
    
    \item \textbf{Comprehensive AAA compliance}: Controls implement multiple AAA-level criteria including Enhanced Target Size (2.5.5) and Change on Request (3.2.5).
\end{enumerate}

\subsection{Tools: Development-focused accessibility}

While WCAG provides a solid foundation for accessibility requirements, our analysis of the Tools screen highlights several additional guidelines specifically relevant to mobile development workflows:

\begin{enumerate}
    \item \textbf{Tool integration guideline}: Accessibility tools are presented with clear integration paths into existing development workflows, not as standalone solutions. The Tools screen implements this by including explicit "Practical Usage" sections that explain integration;
    
    \item \textbf{Platform-specific guidance principle}: Due to the substantial differences between platform accessibility implementations, tools and guidance are explicitly organized by platform when platform-specific considerations apply. The Tools screen implements this by separating iOS and Android tools;
    
    \item \textbf{Development stage appropriateness}: Accessibility tools are categorized by the development stage in which they are most effective (design, development, testing). This helps developers integrate accessibility throughout the development lifecycle rather than treating it as a single checkbox activity;
    
    \item \textbf{Tool complexity indicator}: Accessibility tools vary significantly in complexity and learning curve. Providing clear indicators of tool complexity (like the "Built-in" badge) helps developers choose appropriate tools based on their experience level;
    
    \item \textbf{Contextual documentation principle}: Links to external resources are contextually relevant and provide appropriate expectations about content (e.g., official documentation vs. community resources). This reduces the cognitive load of finding appropriate resources for specific accessibility challenges.
\end{enumerate}

\subsection{Instruction and Community: Community-centered}

The Instruction and Community screen defines several accessibility principles that extend beyond standard WCAG requirements, focusing on the social and community aspects of accessibility implementation:

\begin{enumerate}
    \item \textbf{Community of practice principle}: Accessibility implementation benefits significantly from social learning and community support. The screen implements this by connecting developers to established community channels and platforms where accessibility knowledge is shared;
    
    \item \textbf{Real-world example guideline}: Illustrating accessibility principles with real-world code samples and success stories enhances understanding and implementation. The screen addresses this through collapsible code examples that demonstrate practical solutions to common challenges;
    
    \item \textbf{Contribution pathway}: Effective accessibility ecosystems provide clear pathways for developers to contribute to open source accessibility projects. The screen implements this by highlighting projects seeking contributors with specific tags that indicate required skills;
    
    \item \textbf{Multi-format learning principle}: Accessibility concepts are presented in multiple formats (text, code, examples) to accommodate different learning styles and reinforce understanding. The screen addresses this through varied content presentation methods;
    
    \item \textbf{Platform-specific ecosystem guidance}: Resources are grouped by platform ecosystem (iOS, Android) to help developers navigate the platform-specific nature of accessibility implementation. The screen implements this in the Developer Toolkit section with platform-specific resource cards.
\end{enumerate}

\subsection{Settings: Self-adapting interface}

The Settings screen defines several accessibility principles that extend beyond standard WCAG requirements, particularly focusing on the ability of interfaces to adapt to user needs:

\begin{enumerate}
    \item \textbf{Embedded customization principle}: Accessibility adjustments are directly embedded within the application rather than relying solely on system-level settings. The Settings screen implements this by providing in-app controls for text size, contrast, and other visual preferences;
    
    \item \textbf{Multi-sensory feedback guideline}: Changes to accessibility settings provide feedback through multiple sensory channels. The implementation combines visual cues (toggle animation), auditory feedback (screen reader announcements), and haptic feedback (vibration) to ensure changes are perceivable regardless of user abilities;
    
    \item \textbf{Contextual help principle}: Setting controls should provide context-specific guidance on their purpose and effect. The implementation combines descriptive labels with specific hints to help users understand the impact of each setting;
    
    \item \textbf{Setting persistence}: User preferences for accessibility features persist across application sessions. The implementation stores accessibility settings persistently, ensuring users don't need to reconfigure their preferences with each use;
    
    \item \textbf{Complementary settings grouping}: Related accessibility settings are grouped together to help users understand their relationships and combined effects. The implementation organizes settings into logical categories (Visual, Readability, Color \& Touch) that reflect how features work together to create accessible experiences.
\end{enumerate}

\subsection{Framework comparison: quality-over-quantity}

The Framework comparison screen not only serves as an analytical tool for current accessibility implementation patterns but also demonstrates several forward-looking principles that extend beyond standard WCAG requirements. This section explores how the screen incorporates emerging accessibility concepts and measurement methodologies that anticipate future extensions to formal accessibility guidelines.

\subsubsection{Quantitative accessibility measurement models}

While WCAG provides a comprehensive qualitative framework for evaluating accessibility, it offers limited guidance on quantitative measurement. These quantitative models extend beyond standard WCAG evaluation approaches in several important ways:

\begin{enumerate}
    \item \textbf{Implementation effort quantification}: The screen formalizes the concept of accessibility implementation effort through lines of code (LOC) metrics, creating a reproducible measurement system for comparing framework approaches;
    
    \item \textbf{Complexity impact rating}: Through the explicit formulaic approach shown in the calculation methodology, the screen implements a structured system for converting qualitative complexity assessments to quantitative scores;
    
    \item \textbf{Weighted component scoring}: The implementation demonstrates a formal approach to weighting different accessibility components based on their relative importance, something not addressed in current WCAG criteria;
    
    \item \textbf{Comparative metrics visualization}: Through rating bars and visual indicators, the screen provides intuitive representation of comparative accessibility metrics that go beyond binary compliance assessment;
    
    \item \textbf{Implementation overhead tracking}: The formal tracking of accessibility implementation overhead represents a novel approach to evaluating the development cost of accessibility features.
\end{enumerate}

These quantitative models anticipate a potential future direction for accessibility standards that could incorporate more specific guidance on implementation effort and complexity measurement, helping development teams make more informed decisions about accessibility approaches.

\begin{table}[ht]
\caption{Implementation complexity quantification model}
\label{tab:implementation_complexity_model}
\centering
\begin{tabular}{|C{3cm}|C{3.5cm}|C{3.5cm}|C{4cm}|}
\hline
\textbf{Complexity Level} & \textbf{Quantitative Score} & \textbf{Defining Characteristics} & \textbf{Development Impact} \\
\hline
Low & 5 & Minimal additional code, simple property additions & Minimal learning curve, straightforward implementation \\
\hline
Medium & 3 & Moderate code addition, some structural changes & Requires framework-specific knowledge, moderate learning curve \\
\hline
High & 1 & Significant code overhead, complex structural changes & Substantial learning curve, significant development overhead \\
\hline
\end{tabular}
\end{table}

\FloatBarrier

\subsubsection{Framework transparency principles}
\label{subsubsec:framework-transparency-principles}

The Framework comparison screen implements several transparency principles that extend beyond current WCAG requirements but align with emerging trends in ethical accessibility implementation:

\begin{enumerate}
    \item \textbf{Methodological transparency}: The explicit presentation of evaluation methodologies creates accountability in accessibility assessment that goes beyond standard compliance checklists;
    
    \item \textbf{Implementation overhead disclosure}: By quantifying and visualizing the development overhead required for accessibility features, the screen addresses an often-overlooked aspect of accessibility implementation—the resource requirements;
    
    \item \textbf{Academic reference integration}: The implementation of formal academic citations and methodology references (shown in the References tab) establishes a scientific foundation for accessibility evaluation;
    
    \item \textbf{Platform-specific limitation disclosure}: Through explicit platform testing notes and version-specific results, the screen acknowledges the reality of platform differences in accessibility implementation.
\end{enumerate}

These transparency principles anticipate future accessibility guidelines that may require more explicit disclosure of evaluation methodologies and implementation limitations, helping users and developers make more informed decisions about accessibility approaches.

\subsubsection{Future WCAG implementation anticipation}
\label{subsubsec:future-wcag-anticipation}

The Framework comparison screen implements several features that align with anticipated future directions for WCAG and mobile accessibility guidelines:

\begin{enumerate}
    \item \textbf{Quantitative implementation metrics}: As accessibility standards evolve, quantitative implementation metrics similar to those in the Framework comparison screen may become formalized in future guidelines;
    
    \item \textbf{Framework-specific guidance}: Future WCAG versions may provide more framework-specific implementation guidance, similar to the comparative approach demonstrated in the screen;
    
    \item \textbf{Developer experience consideration}: As accessibility becomes increasingly integrated into development workflows, future guidelines may address developer experience factors more explicitly;
    
    \item \textbf{Cross-platform consistency requirements}: Future mobile accessibility guidelines may formalize requirements for cross-platform consistency, building on the comparative approach demonstrated in the screen;
    
    \item \textbf{Implementation overhead disclosure}: Transparency about accessibility implementation overhead may become a formal requirement in future guidelines, particularly for enterprise and government applications.
\end{enumerate}

Through these forward-looking implementations, the Framework comparison screen not only provides valuable current analysis but also demonstrates a potential future direction for accessibility evaluation that integrates quantitative metrics, developer experience considerations, and transparent methodology.

\begin{table}[ht]
\caption{Potential future WCAG extensions anticipated by Framework comparison screen}
\label{tab:future_wcag_extensions}
\centering
\begin{tabular}{|C{3.5cm}|C{3.5cm}|C{7cm}|}
\hline
\textbf{Potential Extension} & \textbf{Current Implementation} & \textbf{Future Standards Implication} \\
\hline
Implementation Metrics & LOC and complexity quantification & Standardized metrics for implementation effort \\
\hline
Developer Experience & Learning curve and overhead assessment & Formal consideration of developer factors in accessibility \\
\hline
Cross-platform Consistency & Platform variance metrics & Requirements for consistent behavior across platforms \\
\hline
Empirical Validation & Device-specific testing protocols & Standardized testing methodologies and reproducibility requirements \\
\hline
Educational Requirements & Pattern visualization and explanation & Formal requirements for accessibility knowledge transfer \\
\hline
\end{tabular}
\end{table}

\newpage