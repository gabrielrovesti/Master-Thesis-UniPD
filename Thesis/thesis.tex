\documentclass[
    12pt,
    oneside,
    a4paper,
    english
]{book}

\setcounter{secnumdepth}{5}

% Definizione delle variabili direttamente nel documento principale
\newcommand{\myUni}{University of Padua}
\newcommand{\myDepartment}{Department of Mathematics ``Tullio Levi-Civita''}
\newcommand{\myFaculty}{Master Degree in Computer Science}
\newcommand{\myTitle}{Designing an accessibility learning toolkit: bridging the gap between guidelines and implementation}
\newcommand{\myDegree}{Master's Thesis}
\newcommand{\profTitle}{Prof.}
\newcommand{\myProf}{Ombretta Gaggi}
\newcommand{\graduateTitle}{Candidate}
\newcommand{\myName}{Gabriel Rovesti}
\newcommand{\myStudentID}{ID Number: 2103389}
\newcommand{\myAA}{2024-2025}
\newcommand{\myLocation}{Padova}
\newcommand{\myTime}{July 2025}

% Caricamento dei pacchetti
% ===========================================================
% Pacchetti di base e compliance PDF/A
% ===========================================================
\PassOptionsToPackage{dvipsnames}{xcolor} % opzioni passate prima di tutto
\usepackage{colorprofiles}
\usepackage[a-1a,mathxmp]{pdfx}[2018/12/22] % crea PDF/A-1a

% ===========================================================
% Set di caratteri e lingua
% ===========================================================
\usepackage[T1]{fontenc}   % output font encoding
\usepackage[utf8]{inputenc}% input UTF-8
\usepackage[english]{babel}
\usepackage{lmodern}       % Latin Modern font

% ===========================================================
% Colori, grafica, caption, subfigure
% ===========================================================
\usepackage[dvipsnames]{xcolor} % caricato ora con opzione passata sopra
\usepackage{graphicx}
\usepackage{caption}
\usepackage{subcaption}

% ===========================================================
% Layout di pagina
% ===========================================================
\usepackage[
  top=2.5cm,
  bottom=2.5cm,
  right=2.0cm,
  left=2.5cm,
  includehead,
  includefoot
]{geometry}
\usepackage{fancyhdr}
\usepackage{setspace}
\usepackage{emptypage}
\usepackage{indentfirst}

% ===========================================================
% Sezioni, titoli, citazioni
% ===========================================================
\usepackage{titlesec}
\usepackage{sectsty}
\usepackage{quoting}[font=small]
\usepackage{csquotes}

% ===========================================================
% Tabelle & figura: longtable + patch centratura
% ===========================================================
\usepackage{array}     % estende col spec
\newcolumntype{C}[1]{>{\centering\arraybackslash}p{#1}}


\usepackage{longtable}
\usepackage{etoolbox}

% -- centra automaticamente *ogni* longtable --
\makeatletter
\pretocmd\LT@start{%
  \setlength{\LTleft}{\fill}%
  \setlength{\LTright}{\fill}%
}{}{}
\makeatother

\usepackage{tabularx}
\usepackage{tabulary}
\usepackage{adjustbox}

% Landscape (reset centraggio dentro landscape)
\usepackage{pdflscape}
\AtBeginEnvironment{landscape}{%
  \setlength{\LTleft}{0pt}%
  \setlength{\LTright}{0pt}%
}

% Float & utilità
\usepackage{placeins}   % \FloatBarrier
\usepackage{tikz}

% ===========================================================
% Struttura documento, riferimenti interni
% ===========================================================
\usepackage{bookmark}
\usepackage{nameref}
\usepackage[italian]{varioref}
\usepackage{zref-totpages}

% ===========================================================
% Contenuto extra
% ===========================================================
\usepackage{comment}
\usepackage{epigraph}
\usepackage{lipsum}
\usepackage{listings}

% ===========================================================
% Bibliografia & glossari
% ===========================================================
\usepackage[backend=biber, backref, style=numeric]{biblatex}
\usepackage[toc, acronym]{glossaries}

% ===========================================================
% Altro
% ===========================================================
\usepackage{mparhack,relsize}
\usepackage{chngpage, calc}
\usepackage[bottom]{footmisc}
\usepackage{pifont}

% ===========================================================
% Hyperref *sempre* per ultimo
% ===========================================================
\usepackage{hyperref}

% ===========================================================
% Fine pacchetti
% ===========================================================


% Caricamento della configurazione della tesi
% Load variables
\newcommand{\myUni}{University of Padua}
\newcommand{\myDepartment}{Department of Mathematics ``Tullio Levi-Civita''}
\newcommand{\myFaculty}{Master Degree in Computer Science}
\newcommand{\myTitle}{AccessibleHub - Extended screen analysis}
\newcommand{\myDegree}{Master's Thesis}
\newcommand{\profTitle}{Prof.}
\newcommand{\myProf}{Ombretta Gaggi}
\newcommand{\graduateTitle}{Candidate}
\newcommand{\myName}{Gabriel Rovesti}
\newcommand{\myStudentID}{ID Number: 2103389}
\newcommand{\myAA}{2024-2025}
\newcommand{\myLocation}{Padova}
\newcommand{\myTime}{July 2025}

% Define custom colors
\definecolor{hyperColor}{HTML}{0B3EE3}
\definecolor{tableGray}{HTML}{F5F5F7}
\definecolor{veryPeri}{HTML}{6667ab}

\chapterfont{\fontsize{24}{22}\selectfont}  % Reduces chapter font size

% Set line height
\linespread{1.5}

% Custom hyphenation rules
\hyphenation {
    data-base
    al-go-rithms
    soft-ware
}

\begin{filecontents*}{\jobname.xmpdata}
  \Title{Lorem Ipsum}
  \Author{Gabriel Rovesti} 
  \Language{en-EN}
  \Subject{Lorem ipsum}
  \Keywords{X \sep Y \sep Z}
\end{filecontents*} 

% Images path
\graphicspath{{img/}}

% Force page color, as some editors set a grayish color as default
\pagecolor{white}

% Better spacing for footnotes
\setlength{\skip\footins}{5mm}
\setlength{\footnotesep}{5mm}

\setlength{\headheight}{14.5pt}
\addtolength{\topmargin}{-2.45pt}

% Add a subscript G to a glossary entry
\newcommand{\glox}{\textsubscript{\textbf{\textit{G}}}}

% Improvements the paragraph command
\titleformat{\paragraph}
{\normalfont\normalsize\bfseries}{\theparagraph}{1em}{}
\titlespacing*{\paragraph}
{0pt}{3.25ex plus 1ex minus .2ex}{1.5ex plus .2ex}

% Apply fancy styling to pages
\pagestyle{fancy}
\fancyhf{}
\fancyhead[L]{\leftmark} % Places Chapter N. Chatper title on the top left
\fancyfoot[C]{\thepage} % Page number in the center of the footer

% Adds a blank page while increasing the page number
\newcommand\blankpage{ 
\clearpage
    \begingroup
    \null
    \thispagestyle{empty}
    \hypersetup{pageanchor=false}
    \clearpage
\endgroup
}

% Adds a blank page while increasing the page number
\newcommand\blankpagewithnumber{ 
  \clearpage
  \thispagestyle{plain} % Use plain page style to keep the page number
  \null
  \clearpage
}

% Increase page numbering
\newcommand\increasepagenumbering{
    \addtocounter{page}{+1}
}

% Acronyms
\newacronym{ui}{UI}{User Interface}
\newacronym{ux}{UX}{User Experience}
\newacronym{w3c}{W3C}{World Wide Web Consortium}
\newacronym{aria}{ARIA}{Accessible Rich Internet Applications}

% Glossary
\newglossaryentry{uig}{
    name={User Interface},
    text={User Interface},
    sort=ui,
    description={The User Interface refers to the space where interactions between humans and machines occur. It includes the design and arrangement of graphical elements (such as buttons, icons, and menus) that enable users to interact with software or hardware systems. The goal of a UI is to make the user's interaction simple and efficient in accomplishing tasks within a system.}
}

\newglossaryentry{uxg}{
    name={User Experience},
    text={User Experience},
    sort=ux,
    description={User Experience encompasses the overall experience a user has while interacting with a product or service. It includes not only usability and interface design but also the emotional response, satisfaction, and ease of use a person feels while using a system. UX design focuses on optimizing a product’s interaction to provide meaningful and relevant experiences to users, ensuring that the system is intuitive, efficient, and enjoyable to use.}
}

\newglossaryentry{grayliteraturereview}{
    name={Gray Literature Review},
    text={Gray Literature Review},
    sort=grayliteraturereview,
    description={A structured method of collecting and analyzing non-traditional published literature, much of which is published outside conventional academic channels. This research methodology concerns conducting a review of gray literature, such as technical reports, blog postings, professional forums, and industry documentation, to gain insight from practical experience. Gray literature reviews apply most to software engineering research as they represent real practices, challenges, and solutions that have taken place during implementation that may not have been captured or documented in the academic literature. This methodology acts like a bridge that closes the gap between theoretical research and its industry application.}
}

\newglossaryentry{wcag}{ 
 name={WCAG},
 text={WCAG},
 sort=wcag,
 description={The Web Content Accessibility Guidelines (WCAG) are a set of recommendations for making web content more accessible to people with disabilities. They provide a wide range of recommendations for making web content more accessible, including guidelines for text, images, sound, and more.}
}

\newglossaryentry{ariag}{ 
 name={ARIA},
 text={ARIA},
 sort=aria,
 description={Accessible Rich Internet Applications (ARIA) is a set of attributes that define ways to make web content and web applications more accessible to people with disabilities. ARIA roles, states, and properties help assistive technologies understand and interact with dynamic content and complex user interface controls.}
}

\newglossaryentry{voiceover}{ 
 name={VoiceOver},
 text={VoiceOver},
 sort=voiceover,
 description={VoiceOver is a screen reader built into Apple's macOS and iOS operating systems. It provides spoken descriptions of on-screen elements and allows users to navigate and interact with their devices using gestures and keyboard commands.}
}

\newglossaryentry{talkback}{ 
 name={TalkBack},
 text={TalkBack},
 sort=talkback,
 description={TalkBack is a screen reader developed by Google for Android devices. It provides spoken feedback and vibration to help visually impaired users navigate their devices and interact with apps.}
}
% Make glossaries and bibliography
\makeglossaries
% Redefine the format for the glossary entries to be italic
\renewcommand*{\glstextformat}[1]{\textit{#1}\glox}
\glsaddall

\bibliography{references/bibliography}
\defbibheading{bibliography} {
    \cleardoublepage
    \phantomsection
    \addcontentsline{toc}{chapter}{\bibname}
    \chapter*{\bibname\markboth{\bibname}{\bibname}}
}

% Set up hyperlinks
\hypersetup{
    colorlinks=true,
    linktocpage=true,
    pdfstartpage=1,
    pdfstartview=,
    breaklinks=true,
    pdfpagemode=UseNone,
    pageanchor=true,
    pdfpagemode=UseOutlines,
    plainpages=false,
    bookmarksnumbered,
    bookmarksopen=true,
    bookmarksopenlevel=1,
    hypertexnames=true,
    pdfhighlight=/O,
    allcolors = hyperColor
}

% Set up captions
\captionsetup{
    tableposition=top,
    figureposition=bottom,
    font=small,
    format=hang,
    labelfont=bf
}

\date{}
\hypersetup{pdfstartview=}

\begin{document}
    \frontmatter
    \input{preface/1_title_page}
    \increasepagenumbering
    \clearpage
\phantomsection
\thispagestyle{empty}
\hfill
\vfill

{\small\noindent\textcopyright\ \myName, \myTime. All rights reserved. \myDegree: ``\textit{\myTitle}'', \myUni, \myDepartment.}
    \cleardoublepage
\phantomsection
\pdfbookmark{Acknowledgements}{Acknowledgements}

\begin{flushright}{
    \slshape
    ``We don't read and write poetry because it's cute. We read and write poetry because we are members of the human race. And the human race is filled with passion. And medicine, law, business, engineering, these are noble pursuits and necessary to sustain life. But poetry, beauty, romance, love, these are what we stay alive for.''} \\
    \medskip
    --- N.H. Kleinbaum, Dead Poets Society
\end{flushright}

\begingroup
\let\clearpage\relax
\let\cleardoublepage\relax
\let\cleardoublepage\relax

\chapter*{Acknowledgements}

\noindent First and foremost, I would like to express my gratitude to Prof. Gaggi, given her support throughout two paths of thesis, both in bachelor and master degree, for valuable knowledge and support throughout these academic years, both humanly and academically. 

\vspace{0.35cm}

\noindent I would like to thank my mum, the only person who supported me practically throughout these years and my girlfriend Erica, who was not only a real friend, but also a great human being, a person I would always like to have in my life, given her unconditional support and help throughout these years.

\vspace{0.35cm}

\noindent A special thank you to the few real friends I have, because they helped me do many things up until now and I wouldn't live a day without them.

\vspace{0.75cm}

\noindent{\myLocation, \myTime}
\hfill \textit{\myName}

\endgroup

    
    \cleardoublepage
\phantomsection
\pdfbookmark{Abstract}{Abstract}
\begingroup
\let\clearpage\relax
\let\cleardoublepage\relax
\chapter*{Abstract}

The following thesis aims to conduct a comparative analysis of accessibility features and guidelines for mobile application development, mainly using the React Native framework and comparing it with Flutter, upon building with previous research. This study extends the investigation conducted on the Flutter framework's accessibility capabilities to React Native, providing a comprehensive examination of both frameworks' approaches, while creating accessible mobile user interfaces.

The research present in this thesis involves the creation of an application with a focus on implementing accessibility features, seeking throughout this process, the identification of similarities, differences and potential improvements in accessibility implementation of equivalent features between the frameworks. It includes a thorough literature review to establish the current state of mobile accessibility research, an in-depth analysis of React Native's approach to accessibility tools and guidelines, while having a comparison between code samples of both frameworks demonstrating accessibility features. \\

Accessibility remains one of the goals and one of the main foundations of a good user experience, but also a good developer mindset in order to consider and expand the usage of a product to different users and categories, regardless of their capabilities, while guaranteeing a seamless interaction with different components and devices. For this reason, WCAG guidelines will be implemented as the success criteria, serving as a comprehensive framework for making content across the web more accessible, while being equally relevant and applicable to mobile app development. As mobile devices become increasingly prevalent, implementing such guidelines will be important in understanding as technology evolves, how accessibility features will adapt, considering future trends and potential. 

\endgroup
\vfill

    \input{preface/5_table_of_contents}
    \printglossary[type=\acronymtype, title=Acronyms and abbreviations, toctitle=Acronyms and abbreviations]
    \printglossary[type=main, title=Glossary, toctitle=Glossary]
    
    \mainmatter
    \chapter{Introduction}
\label{chap:intro}

% \begin{figure}[H]
%     \centering
%     \includegraphics[alt={Testo alternativo dell'immagine}, width=1\columnwidth]{img/quantum_entanglement.jpeg}
%     \caption{Lorem}
%     \label{fig:entanglement}
% \end{figure}
% 
% Introduzione al contesto applicativo.
% 
% Lorem Figure \ref{fig:entanglement}
% 
% Esempio di utilizzo di un termine nel glossario \gls{api}.
% 
% Esempio di citazione direttamente nel testo \cite{site:agile-manifesto}.
% 
% Esempio di citazione nel piè di pagina \footcite{womak:lean-thinking}.
% Introduzione all'idea dello stage\footcite{article:spooky}.

\section{Background and motivations}
\label{chap:intro-background}

In an era where digital technology permeates every aspect of our lives, mobile devices have emerged as the primary gateway to the digital world, allowing a lot of new people to be connected at any given time, no matter the condition. An estimated number of circa 7 billions \cite{article:number-of-users}, representing a dramatic increase from just one billion users in 2013. This explosive growth has not only changed how we communicate and access information but has also created a massive market for different needs and introduced new categories of users, with different habits and cultures into a truly global market.

As mobile applications become increasingly central to daily life, ensuring their accessibility to all users, regardless of their abilities or disabilities, has become a critical imperative, since not only technology should be able to connect, but also to unite seamlessly people with different capabilities. Accessibility refers to the design and development practices enabling all users, regardless of their abilities or disabilities, to perceive, understand and navigate with digital content effectively. Not only the quantity of media increased, but also the quantity of different media which allow to access information definitely increased; finding appropriate measurements to establish a good level of understanding and usability is important and finding appropriate levels of measurements is non-trivial. \\

An estimated portion of over one billion people lives globally with some forms of disability \cite{article:who-disability}. Inaccessible mobile applications can, therefore, present considerable barriers to participation in that large and growing part of modern life that involves education, employment, social interaction, and even basic services. Accessibility is not about a majority giving special dispensation to a minority but rather about providing equal access and opportunities to very big and diverse user bases.

This encompasses a wide range of considerations to be made on the actual products design and the user classes, including but not limited to:
\begin{enumerate}
    \item \textit{Visual accessibility}: supporting users who are blind or have low vision, requiring alternative description, a clear language and screen readers support;

    \item \textit{Auditory accessibility}: providing alternatives for users who are deaf or hard of hearing, offering clear controls and alternative  visuals for audio content, ensuring compatibility with assistive devices and giving feedback to specific actions done by users;

    \item \textit{Motor accessibility}: accommodating users with limited dexterity or mobility, providing alternative input navigation, create a design so to help avoiding complex gestures, customize the interactions and gestures, reducing precision and accommodating errors;

    \item \textit{Cognitive accessibility}: ensuring content is understandable for users with different cognitive abilities, creating a consistent and predictable navigation, visual support on UI so to support focus and attention and enhancing comprehension of interface components.
    
\end{enumerate}

In the mobile environment, such considerations is important, since there is a complex web of interactions to be considered, mainly focusing on two aspects:

\begin{enumerate}
    \item Device diversity and integration - accommodating different gestures, interfaces and interaction modalities
        \begin{itemize}
            \item Traditional mobile devices (smartphones, tablets)
            \item Emerging form factors (foldables, dual-screen devices)
            \item Wearable technology (smartwatches, fitness trackers)
            \item Embedded systems (vehicle interfaces, smart home controls)
            \item IoT devices with mobile interfaces
        \end{itemize}
    \item Usage context variations - may influence the overload of information and the cognitive load perceived by the user
        \begin{itemize}
            \item Environmental conditions (lighting, noise, movement)
            \item User posture and mobility situations
            \item Attention availability and cognitive load
            \item Physical constraints and limitations
            \item Social and cultural contexts
        \end{itemize}
\end{enumerate}

These considerations are important since they impact how accessibility features should go above and beyond, carefully considering the design in terms of \gls{uig} (\acrshort{ui}), considering the design and layout of interactive elements and the actual experience in using the product, considering the degree of satisfaction and meeting of user's needs, so the \gls{uxg} (\acrshort{ux}).  

\section{Related works}
\label{chap:intro-related-works}

\section{Thesis structure}
\label{chap:intro-structure} 

% \begin{description}
%     \item[{\hyperref[chap:second-chapter]{The second chapter}}] describes ...
%     
%     \item[{\hyperref[chap:third-chapter]{The third chapter}}] describes ...
%     
%     \item[{\hyperref[chap:fourth-chapter]{The fourth chapter}}] describes ...
%     
%     \item[{\hyperref[chap:fifth-chapter]{The fifth chapter}}] describes ...
%     
%     \item[{\hyperref[chap:sixth-chapter]{The sixth chapter}}] describes ...
%     
%     \item[{\hyperref[chap:conclusions]{The last chapter}}] describes ...
% \end{description}
% 
% Regarding the text composition, the following typographical conventions have been adopted for this document:
% \begin{itemize}
% 	\item acronyms, abbreviations, and ambiguous or uncommon terms mentioned are defined in the glossary, located at the end of this document;
% 	\item For the first occurrence of terms listed in the glossary, the following nomenclature is used: \gls{apig};
% 	\item foreign language terms or those belonging to technical jargon are highlighted in \textit{italic} characters.
% \end{itemize}

\newpage
    \chapter{Mobile accessibility guidelines and standards}
\label{chap:accessibility}
Introduction to chapter

\section{Section}
\label{sec:second-section}

\newpage
    \chapter{AccessibleHub: Transforming mobile accessibility guidelines into code}
\label{chap:accessibility-toolkit}

\chapterintroline{
    This chapter presents an accessibility-focused learning toolkit, which is an all-encompassing guide to mobile application developers. It extends Gaggi's research implemented by Budai's work into Flutter accessibility and gives a more focused approach to orient the developers themselves in how to actually implement an accessible mobile application. Here \textit{AccessibleHub} is introduced, an interactive learning toolkit built using \gls{reactnative}, which aims to enhance accessibility implementation through hands-on examples, component-level guidance, and comparative insights between React Native and \gls{flutter}. By providing a structured educational approach grounded in \gls{wcagg} principles and mobile-specific considerations, AccessibleHub empowers developers to bridge the gap between accessibility guidelines and real-world implementation. 
}

\section{Introduction}
\label{sec:intro}

\subsection{Challenges in implementing accessibility guidelines}

The importance of mobile app accessibility extends beyond mere compliance with legal regulations. Ensuring equal access to digital content and services is not only an ethical obligation but also a smart business decision. By prioritizing accessibility, app developers and companies can tap into a larger user base, improve user satisfaction, and demonstrate their commitment to social responsibility.
Despite the clear benefits and moral imperatives of mobile app accessibility, many developers still struggle to effectively implement accessibility guidelines in their projects. The \acrshort{wcagacr}, developed by the \gls{w3cg}, serve as the international standard for digital accessibility. However, translating these guidelines into practical implementation can be a challenging task, particularly starting from pure formal guidelines into everyday code. \\

One of the primary challenges lies in the complexity of the guidelines themselves. WCAG encompasses a wide range of \textit{success criteria}, organized under four main general \textit{principles}: perceivable, operable, understandable, and robust. Each principle contains multiple guidelines, and each guideline has several success criteria at different levels of \textit{conformance} (A, AA, AAA). Navigating this intricate web of requirements and understanding how to apply them to specific mobile app components can be overwhelming for developers, especially those new to accessibility. 
Moreover, the practical implementation of accessibility guidelines often varies across different platforms and frameworks. \textit{iOS} and \textit{Android}, the two dominant mobile operating systems, have their own unique accessibility \textit{API}s, tools, and best practices. Cross-platform frameworks like React Native and Flutter add another layer of complexity, as developers must ensure that their accessibility implementations are compatible with the underlying platform-specific mechanisms. \\ 

Furthermore, there is often a lack of clear, practical examples and guidance on how to implement accessibility features in real-world mobile app projects. While the \textit{WCAG} provides a solid foundation, it is primarily focused on web content and may not always directly address the unique challenges and interaction patterns of mobile apps. Developers often struggle to bridge the gap between the theoretical guidelines and the specific implementation details required for their projects.

\subsection{The need for practical developer education}

To address these challenges and bridge the gap between accessibility guidelines and practical implementation, there is a pressing need for developer education resources that focus on real-world, hands-on learning experiences. Traditional documentation and guidelines, while valuable, often fall short in providing the level of detail and interactivity needed to effectively guide developers through the accessibility implementation process.
This is where the concept of an \textit{accessibility learning toolkit} comes into play. An accessibility toolkit is designed to serve as a comprehensive, interactive resource that empowers developers to create accessible mobile applications by providing:

\begin{enumerate}
    \item Clear explanations of \acrshort{wcagacr} guidelines and their applicability to mobile apps;
    
    \item Step-by-step implementation guidance for common mobile app components and interaction patterns;
    
    \item Practical code examples and tutorials that demonstrate best practices;
    
    \item Hands-on exercises and challenges to reinforce learning and build confidence;
    
    \item Tools and techniques for testing and validating the accessibility of mobile apps.
\end{enumerate}

The primary goal of an accessibility learning toolkit is to bridge the gap between the theoretical knowledge of accessibility guidelines and the practical skills needed to implement them effectively in real-world projects. 
The toolkit should cater to developers at various levels of expertise, from beginners who are new to accessibility concepts to experienced professionals seeking to deepen their knowledge and stay up-to-date with the latest best practices. By providing a comprehensive, hands-on learning resource, the accessibility toolkit can play a crucial role in promoting a culture of inclusive design and development within the mobile app industry. \\

Current research, including Budai's work on Flutter accessibility testing, has primarily focused on end-user validation and testing methodologies. However, developers need practical, implementation-focused guidance that bridges multiple frameworks and platforms.
Despite widespread accessibility guidelines and standard, mobile application developers face significant challenges in translating theoretical requirements into practical implementations. This gap between guidelines and implementation is particularly evident in mobile development, where different platforms, screen sizes, and interaction models add complexity to accessibility implementation. Some of the most common challenges include:

\begin{itemize}
    \item Complex testing requirements - developers must validate across multiple devices, \gls{screenreaderg}, and interaction modes;
    
    \item Framework-specific implementations - each platform has unique accessibility \gls{apig}s and requirements;
    
    \item Limited practical examples - most documentation focuses on theoretical guidelines rather than concrete implementation patterns;
    
    \item Performance considerations - accessibility features must be implemented without compromising app performance.
\end{itemize}

Effective developer education in accessibility requires a solid grounding in learning theories that emphasize hands-on, interactive approaches. By integrating established learning theories with technical education principles, it's possible to justify the interactive and practical approach adopted in this toolkit. In doing so, we draw on constructivist and experiential learning models, which have been widely recognized as effective frameworks in technical and developer education.

Constructivist learning theories, pioneered by Piaget \cite{piaget1970science} and Vygotsky \cite{vygotsky1978mind}, posit that learning is an active process in which individuals construct knowledge based on their prior experiences and interactions with the environment. In the context of developer education, this suggests that hands-on learning is more effective than passive instruction \cite{savery2006overview}. By engaging with real-world accessibility challenges and actively experimenting with code implementations, developers can build a deeper understanding of accessibility guidelines and best practices, by having a tool at their disposal easy to use and to navigate. \\

Kolb's \textit{Experiential Learning Theory} \cite{kolb1984experiential} further supports this approach by describing learning as a four-stage cycle: concrete experience, reflective observation, abstract conceptualization, and active experimentation. For developers learning about accessibility, this cycle might involve encountering accessibility issues in their projects, analyzing existing solutions and guidelines, synthesizing their understanding of \acrshort{wcagacr} principles, and applying these principles to their own code. \textit{AccessibleHub} facilitates this learning cycle by providing a structured, interactive environment for developers to engage with accessibility concepts and implementations being organized into different core sections. By aligning with these proven pedagogical approaches, \textit{AccessibleHub} aims to provide an effective and engaging learning experience for developers. Moreover, by fostering a community of practice around accessibility while providing easier access to learning resources, this project encourages ongoing learning and knowledge sharing among developers, promoting the continuous improvement and dissemination of accessibility best practices.

\subsection{Research objectives and methodology}

Building upon previous research into mobile accessibility, this work aims to provide a comprehensive understanding of accessibility implementation across major cross-platform frameworks. While existing research indeed set grounds for both guidelines on accessibility and testing methodologies, there is a critical need to understand how these guidelines translate into practice for developers. 

This research addresses three fundamental questions about accessibility implementation in mobile development frameworks (referring to these ones as \textit{research questions}, following the work in \cite{perinello2024accessibility}:

\begin{itemize}
    \item First, we investigate whether components and widgets provided by frameworks are \textit{accessible by default}, without requiring additional developer intervention. This analysis is crucial for understanding the baseline accessibility support provided by each framework and identifying areas where additional implementation effort may be required;
    
    \item Second, we examine the \textit{feasibility of making non-accessible components accessible} through additional development effort. This involves analyzing the technical capabilities of each framework and identifying the necessary modifications to achieve accessibility compliance;
    
    \item Third, we quantify the \textit{development overhead required to implement accessibility features} when they are not provided by default. This includes measuring additional code requirements, analyzing complexity increases, and evaluating the impact on development workflows.
\end{itemize}

These questions is addressed via the usage of a systematic methodology aiming to address in detail accessibility support in React Native and Flutter, focusing on component implementation patterns and native platform integration. The implementation is comparative, allowing developers to directly implement accessible code examples with different degrees of implementation complexity measured quantitatively (including lines of code, required properties, and additional components needed for accessibility support). Comprehensive testing of implementations is also done using screen readers and other assistive technologies to verify accessibility compliance.

The \textit{goal} is to create an accessible application that serves three key purposes:
\begin{enumerate}
    \item To provide developers with practical, interactive examples of accessibility implementation, able to be copied easily and ported inside of other projects;
    
    \item To compare and contrast accessibility approaches between the main cross-development mobile frameworks in the current mobile landscape;
    
    \item To establish a reusable pattern library that demonstrates engine architecture, widget systems, and native platform integration, while ensuring compliance with current accessibility guidelines and legal requirements.
\end{enumerate}

The following sections will detail the development of \textit{AccessibleHub}, an application developed in React Native designed to serve as a practical manual for implementing accessibility features. While the technical aspects of cross-platform frameworks will be discussed later, the focus remains on providing developers with actionable implementation patterns and comparative insights for building accessible applications.

\section{React Native Overview}
\label{sec:reactnative-overview}

\gls{reactnative} is an open-source framework developed by Meta that enables developers to build mobile applications using JavaScript and the React paradigm (\cite{site:reactnative}). It employs a declarative, component-based approach through the use of \textit{JSX}, which is an XML-like syntax that allows developers to intermix JavaScript logic with markup. This combination not only improves code readability but also enhances modularity and facilitates code reuse.

\begin{figure}[ht]
    \centering
    \includegraphics[width=0.4\textwidth, alt={React Native Logo}]{img/react-native-logo.png}
    \caption{React Native Logo}
\label{fig:reactnative-logo}
\end{figure}

\subsection{Core architecture and features}
\begin{itemize}
    \item \textit{Component-based architecture:}  
    The entire user interface in React Native is built from reusable components. Each component encapsulates its own logic and presentation, which greatly aids in the maintainability and scalability of complex applications;
    
    \item \textit{JSX syntax:}  
    Developers write the \acrshort{ui} using \textit{JSX}, a syntax extension similar to \textit{HTML}. This blending of code and layout simplifies the development process and enables a more intuitive understanding of the component structure;
    
    \item \textit{Bridging mechanism:}  
    React Native’s bridge enables asynchronous communication between the JavaScript layer and native modules. This means that while the application is written in JavaScript, performance-critical tasks can be executed using native code (e.g., Objective-C, Swift, or Java), ensuring a native look and feel without sacrificing performance;
    
    \item \textit{Hot reloading:}  
    One of the standout features present in this framework, which allows developers to see changes in real time without restarting the entire application. This accelerates the development cycle and aids in rapid prototyping;
    
    \item \textit{Unified codebase:}  
    React Native enables the development of applications for both iOS and Android using a single codebase. This unified approach reduces development time and effort compared to maintaining separate codebases for each platform.
\end{itemize}

\subsection{Accessibility in React Native}
React Native provides a robust set of accessibility features that are deeply integrated into its component model. This allows developers to create inclusive applications without relying on external libraries or writing platform-specific code (following what's present into \cite{site:reactnativeaccess}). Here are the key accessibility features in React Native:

\begin{itemize}
    \item \textbf{Accessibility properties}: React Native components can be enhanced with a variety of accessibility properties that provide semantic meaning and context for assistive technologies. These properties include:
    \begin{itemize}
        \item \texttt{accessibilityLabel}: A concise, descriptive string that identifies the component for screen reader users;
        \item \texttt{accessibilityRole}: Defines the component's semantic role (e.g., \texttt{"button"}, \texttt{"header"}), helping assistive technologies interpret its purpose correctly;
        \item \texttt{accessibilityHint}: Provides additional context about a component's function or the result of interacting with it;
        \item \texttt{accessibilityState}: Describes the current state of a component (e.g., \texttt{selected}, \texttt{disabled}), which is essential for conveying dynamic changes.
    \end{itemize}
    
    \item \textbf{Accessibility actions}: React Native allows developers to define custom accessibility actions for components, enabling advanced interactions beyond the default gestures. For example, a custom \texttt{accessibilityAction} could be added to a component to trigger a specific behavior when activated by an assistive technology;
    
    \item \textbf{Accessibility focus}: React Native manages accessibility focus automatically, ensuring that the correct component receives focus when navigating with assistive technologies. Developers can also programmatically control focus using the \texttt{accessibilityElementsHidden} and \\\texttt{importantForAccessibility} properties;
    
    \item \textbf{Accessibility events}: React Native provides accessibility events that notify assistive technologies when important changes occur in the application. These events include:
    \begin{itemize}
        \item \texttt{onAccessibilityTap}: Called when a user double-taps a component while using an assistive technology;
        \item \texttt{onMagicTap}: Called when a user performs the "magic tap" gesture (a double-tap with two fingers) to activate a component;
        \item \texttt{onAccessibilityFocus}: Called when a component receives accessibility focus;
        \item \texttt{onAccessibilityBlur}: Called when a component loses accessibility focus.
    \end{itemize}
\end{itemize}

By leveraging these built-in accessibility features, developers can create React Native applications that are inclusive and accessible to users with diverse needs and abilities. The tight integration of accessibility into the core component model ensures that developers can create accessible apps without sacrificing performance or maintainability.

\subsection{Advantages and developer benefits}

Using React Native offers several benefits for developers, briefly listed here:
\begin{itemize}
    \item \textit{Rapid development:}  
    Thanks to hot reloading and a vast ecosystem of reusable components, developers can iterate quickly and efficiently;
    
    \item \textit{Cross-platform consistency:}  
    With a unified codebase for both iOS and Android, developers can ensure a consistent user experience without duplicating effort;
    
    \item \textit{Integrated accessibility:}  
    React Native’s direct integration of accessibility properties allows developers to implement accessible features without having to rely on external tools or write platform-specific code;
    
    \item \textit{Community and support:}  
    A large and active community means extensive documentation, a wealth of third-party libraries, and a robust support network for troubleshooting and enhancements;
    
    \item \textit{Seamless transition for web developers:}  
    Developers familiar with React for web applications will find the transition to React Native smooth, as the core concepts and \textit{JSX} syntax remain consistent.
\end{itemize}

\subsection{Differences from native iOS/Android and web development}
\begin{itemize}
    \item \textit{Native iOS/Android:}  
    In native development, accessibility is handled through platform-specific \gls{apig}: \gls{voiceover} on \textit{iOS} and \gls{talkback} on \textit{Android}, which require different tools and approaches. React Native provides a unified \textit{API}s, streamlining the implementation of accessibility features across both platforms.
    
    \item \textit{Web development:}  
    Whereas web accessibility is achieved by adding \gls{ariag} attributes to \textit{HTML}, React Native integrates accessibility directly within its component structure. This intrinsic approach treats accessibility as a core attribute of each component, rather than an external addition.
\end{itemize}

In summary, React Native offers a modern, efficient, and developer-friendly environment that not only simplifies cross-platform mobile development but also incorporates accessibility into its core design. This makes it an ideal choice for creating inclusive applications, and it forms the foundational platform upon which the \textit{AccessibleHub} toolkit is built.

\section{AccessibleHub: An Interactive Learning Toolkit}
\label{sec:accessiblehub}

\subsection{Core architecture and design principles}

\textit{AccessibleHub} is a React Native application designed to serve as an interactive manual for implementing accessibility features in mobile development. Unlike traditional documentation or testing frameworks, the application provides developers with hands-on examples and implementation patterns that can be directly applied to their projects.

The application is structured around four conceptual main sections:
\begin{enumerate}
    \item \textit{Component examples}: Interactive demonstrations of common \acrshort{ui} elements with proper accessibility implementations, including buttons, forms, media content, and navigation patterns. This allows developers to clearly see the implementation of an accessible component and easily copy the code to their convenience;
    
    \item \textit{Framework comparison}: A detailed analysis of accessibility implementation approaches between React Native and Flutter, highlighting differences in component structure, properties, and required code;
    
    \item \textit{Testing tools}: Built-in utilities for validating accessibility features, allowing developers to understand how screen readers and other assistive technologies interact with their implementations;
    
    \item \textit{Implementation guidelines}: Technical documentation that connects WCAG requirements to practical code examples, providing clear paths for meeting accessibility standards.
\end{enumerate}

Each component presented serves dual purposes: demonstrating proper accessibility implementation while providing reusable code patterns. The application emphasizes practical implementation over theoretical guidelines, showing developers not just what to implement effectively. By focusing on developer experience, \textit{AccessibleHub} bridges the gap between accessibility requirements and actual implementation, providing a resource that can be directly integrated into the development workflow. \\

The \textit{design} philosophy of \textit{AccessibleHub} is founded on principles that bridge theoretical accessibility guidelines with practical implementation needs. While analyzing the current landscape of mobile development frameworks and accessibility implementation presented in \ref{chap:accessibility-literature}, a clear pattern emerges: developers need more practical, implementation-focused guidance that directly addresses the complexity of building accessible applications. To address this need, \textit{AccessibleHub} adopts three fundamental architectural principles:

\begin{enumerate}
    \item The usage of a \textit{component-first architecture}, where each UI element exists as an independent, self-contained unit demonstrating both implementation patterns and accessibility features. In other words, each one of them is being constructed within an \textit{accessibility-first} experience which ensures that usage of screen readers and other assistive technologies is kept as a priority. This modular approach provides two advantages: it first allows developers to comprehend and apply accessibility features in isolation, hence reducing cognitive load and implementation complexity, and enables systematic testing and validation of accessibility features of every component. Also, this means accessibility patterns can be studied, implemented, and verified in isolation from added complexity brought in by interactions among those components;

    \item \textit{Progressive enhancement} as a core design methodology. Instead of presenting accessibility as big challenge from the start, components are structured in increasing levels of complexity. This starts with basic elements like buttons and text inputs where basic accessibility patterns can be established. As developers master these foundational components, the application introduces more complex patterns such as forms, navigation systems, and gesture-based interactions. This helps into guiding the development towards more complicated scenarios;

    \item Focus on \textit{framework-agnostic patterns}, not depending on a specific framework while providing concrete code implementations. Even though \textit{AccessibleHub} has been implemented in React Native, all the patterns and principles explained are designed to transcend into specific framework implementations. The approach wants to give importance to the compatibility and reusability in the framework on the mobile development side. It will compare the implementations, mainly between React Native and Flutter, to show how developers can port accessibility patterns across different frameworks and understand core accessibility concepts in an easy-to-implement manner within professional projects. 
    
\end{enumerate}

Through these principles, \textit{AccessibleHub} aims to transform accessibility from an afterthought into an \textit{accessibility-by-design}. The application serves not just as a reference implementation, but as an educational tool that guides developers through the process of building truly accessible applications. This approach recognizes that effective accessibility implementation requires both theoretical understanding and practical experience, providing developers with the tools they need to create more inclusive mobile applications.

\subsection{Educational framework design}

\textit{AccessibleHub}'s educational framework is designed to provide a structured, incremental learning experience that progressively builds accessibility knowledge and skills. The content is organized into different \textit{learning modules}, each focusing on a key aspect of mobile accessibility. This is structured incrementally, so to help a developer gather a general idea on what needs to be implemented following a practical roadmap of steps: this allows to focus on different aspects of mobile accessibility, selecting each time the most relevant ones.

The core of the application is divided into different main screens, following:

\begin{enumerate}
    \item \textbf{Home} - The entry point for the \textit{AccessibleHub} application (\ref{fig:homescreen}). It provides an overview of the main sections and guides users on where to start their accessibility learning journey. The Home screen is designed to be intuitive and user-friendly, with clear call-to-action towards the accessible components section, allowing a developer or a user navigate to the desired section from the Home screen, comprehensive of comparison between the main mobile frameworks, learn about best practices in mobile accessibility and access testing tools documentation. There is also present a compliance dashboard provides an overview of an app's accessibility compliance status, based on the \acrshort{wcagacr} and \acrshort{mcagacr} guidelines. Developers can use this information to prioritize their accessibility efforts and focus on the areas that need the most attention;

    \begin{figure}[ht]
    \centering
    \includegraphics[width=0.4\linewidth, alt={Screenshot of the Home Screen of AccessibleHub}]{img/homescreen.jpg}
    \caption{The Home Screen of \textit{AccessibleHub}}\label{fig:homescreen}
    \end{figure}

\pagebreak

    \item \textbf{Accessible Components} - Developers can learn how to implement accessible UI components in their mobile applications (\ref{fig:components}). This section is divided into four subscreens, each focusing on a specific category of components:

    \begin{itemize}
        \item \textit{Buttons and Touchables}: It covers the implementation of accessible buttons and touchable elements. It provides code examples and best practices for ensuring that these interactive elements are perceivable, operable, and understandable by all users, including those with disabilities;

        \item \textit{Forms}: The subscreen focuses on creating accessible input forms, including text fields, checkboxes, radio buttons, and date/time pickers. It demonstrates how to properly label form elements, provide instructions and feedback, and ensure that forms can be navigated and completed using various input methods, such as keyboards and screen readers;

        \item \textit{Media}: In the Media subscreen, developers learn how to make media content, such as images, videos, and audio, accessible to users with visual or auditory impairments. This includes providing alternative text for images, captions for videos, and transcripts for audio content;

        \item \textit{Dialogs}: It covers the creation of accessible modal dialogs, popups, and alerts. It provides guidance on how to ensure that these elements are properly announced by screen readers, can be easily dismissed, and do not interfere with the user's ability to navigate the application, maintaining focus management and ensuring clear exit strategies;

       \item \textit{Advanced}: This particular subscreen covers elements like alerts, sliders, progress bars and tab navigation, analyzing how accessibility may regard different animated or interactive components for more complex gesture interactions used everyday by users.
    \end{itemize}

Throughout the Components section, code implementations are shared as examples, which developers can easily copy to their clipboard and integrate into their own projects. This hands-on approach allows developers to quickly apply the accessibility principles they learn and see the results in action.

\begin{figure}[ht]
\centering
\includegraphics[width=0.4\linewidth, alt={Screenshot of the Components Screen of AccessibleHub}]{img/components.jpg}
\caption{The Components Screen of \textit{AccessibleHub}}\label{fig:components}
\end{figure}

\pagebreak

\item \textbf{Best Practices} - Designed to give developers a general understanding of the overarching principles and guidelines for creating accessible mobile applications (\ref{fig:best-practices}). It is divided into five subscreens, each addressing a key aspect of mobile accessibility:

    \begin{itemize}
        \item \textit{Gestures Tutorial}: This subscreen provides an overview of the various gesture interactions used in mobile applications and how to make them accessible to users with motor impairments or those relying on assistive technologies. It covers best practices for implementing alternative input methods and providing clear instructions and feedback. These gestures are general, tested to be used universally, both by everyday users and screen reader ones;

        \item \textit{Semantics Structure}: Here, developers learn about the importance of using semantic \textit{HTML} and \gls{ariag} roles to convey the structure and meaning of the application's content. This helps screen readers and other assistive technologies better understand and navigate the application;

        \item \textit{Navigation}: This one focuses on creating accessible navigation patterns, such as menus, tabs, and breadcrumbs. It provides guidance on how to ensure that navigation elements are properly labeled, can be operated using various input methods, and provide clear feedback to the user, jumping directly to the main context of a screen and bringing the attention to an element on-screen without distracting him from the action to be completed;

        \item \textit{Screen Reader Support}: This subscreen covers the specific considerations for making mobile applications compatible with screen readers, such as \gls{voiceover} on \textit{iOS} and \gls{talkback} on \textit{Android}. It includes best practices for labeling elements, providing alternative text, and ensuring that the application's content and functionality can be fully accessed and understood using a screen reader;

        \item \textit{Accessibility Guidelines}: The Accessibility Guidelines subscreen provides an overview of the key accessibility standards to be followed and a general list of principles to incorporate into a project, seeing how they apply to mobile application development. It helps developers understand the different levels of conformance and how to assess their application's accessibility against these guidelines.
    \end{itemize}

\begin{figure}[ht]
\centering
\includegraphics[width=0.4\linewidth, alt={Screenshot of the Best Practices Screen of AccessibleHub}]{img/best-practices.jpg}
\caption{The Best Practices Screen of \textit{AccessibleHub}}\label{fig:best-practices}
\end{figure}

\pagebreak

\item \textbf{Framework Comparison} - It provides a side-by-side comparison of the accessibility features and implementation differences between popular mobile development frameworks, such as React Native and Flutter (\ref{fig:frameworks-comparison}). This section helps developers understand how accessibility is handled in each framework and provides guidance on leveraging the specific accessibility APIs and tools available in each one. This is divided into different categories, offering a practical and formal overview on how such frameworks are compared with each other. By highlighting the similarities and differences between frameworks, developers can make informed decisions about which framework to use for their accessibility needs and how to optimize their implementations for each platform;

\begin{figure}[ht]
\centering
\includegraphics[width=0.4\linewidth, alt={Screenshot of the Frameworks Comparison Screen of AccessibleHub}]{img/frameworks-comparison.jpg}
\caption{The Frameworks Comparison Screen of \textit{AccessibleHub}}\label{fig:frameworks-comparison}
\end{figure}

\pagebreak

\item \textbf{Tools} - It serves as a central hub for accessing various accessibility-related tools and resources (\ref{fig:tools}). This includes links to official documentation, such as the React Native Accessibility \acrshort{api} reference and the \textit{Flutter Accessibility package} documentation. It also provides quick access to popular accessibility testing tools, such as \textit{Accessibility Scanner} for \textit{Android} and \textit{Accessibility Inspector} for \textit{iOS}. By consolidating these resources in one place, the Tools screen makes it easy for developers to find the information and tools they need to ensure their applications are fully accessible; 

\begin{figure}[ht]
\centering
\includegraphics[width=0.4\linewidth, alt={Screenshot of the Tools Screen of AccessibleHub}]{img/tools.jpg}
\caption{The Tools Screen of \textit{AccessibleHub}}\label{fig:tools}
\end{figure}

\pagebreak

\item \textbf{Settings} - Allows users to customize various aspects of the \textit{AccessibleHub} application to suit their individual learning needs and preferences (\ref{fig:settings}). This includes options for adjusting the font size, color contrast (including options for gray scale and dark mode), reduced motion settings and others to help users and ensure the application itself is accessible to a wide range of users. It also provides information on how to configure the accessibility settings on the user's device, such as enabling screen readers or adjusting the display settings. By offering these customization options and guidance, the page reinforces the importance of accessibility as an everyday tool, meant to accompany practical user needs in an easy and quick way;

\begin{figure}[ht]
\centering
\includegraphics[width=0.4\linewidth, alt={Screenshot of the Settings Screen of AccessibleHub}]{img/settings.jpg}
\caption{The Settings Screen of \textit{AccessibleHub}}\label{fig:settings}
\end{figure}

\pagebreak

\item \textbf{Instruction and Community} - It provides a collaborative learning environment that extends beyond technical implementation (\ref{fig:instruction-community}). This section offers developers an opportunity to dive deeper into accessibility knowledge through curated resources and community engagement allowing for easier exploration towards other online resources. This provides an overview of currently open projects in the field of accessibility, provides advices on specific plugins and offers community examples of interest for a developers to be motivated into the creation of other accessible projects. By providing a platform for continuous learning and collaboration, this screen reinforces the importance of accessibility as a collective effort and a fundamental aspect of modern mobile application development.

\begin{figure}[ht]
\centering
\includegraphics[width=0.4\linewidth, alt={Screenshot of the Instruction and Community Screen of AccessibleHub}]{img/instruction-community.jpg}
\caption{The Instruction and Community Screen of \textit{AccessibleHub}}\label{fig:instruction-community}
\end{figure}
    
\end{enumerate}

\pagebreak

\subsection{From guidelines to implementation: a screen-based methodology}

Accessibility guidelines and standards - most notably the \acrshort{wcagacr}  and related mobile-specific considerations—establish the formal foundation for inclusive digital design, as discussed in \ref{sec:accessibility-guidelines}. These criteria are essential but inherently abstract and can be challenging to implement directly in code. Building on Perinello and Gaggi's approach focusing solely on post-implementation testing, the methodology presented embeds accessibility into the development process. We do this by analyzing each screen of the application through a structured framework that connects theoretical requirements with practical implementation strategies. The approach to be considered is built following these layers:

\begin{enumerate}
    \item \textit{Theoretical foundation} – This layer encompasses the abstract principles and success criteria defined by \textit{WCAG}/\acrshort{mcagacr}. For example, \textit{WCAG}’s four core principles require that content be presented in ways users can perceive, interact with, and understand. These criteria serve as the benchmark for our analysis;

    \item \textit{Implementation pattern} – Here, we translate the abstract requirements into concrete code structures within a mobile development context. In \textit{AccessibleHub}, this involves the systematic use of React Native properties (such as \textit{accessibilityLabel}, \textit{accessibilityRole}, etc.) to ensure that \acrshort{ui} components satisfy the established guidelines;

    \item \textit{User interaction flow} – Finally, we consider how end users interact with these components. This includes the behavior of assistive technologies (like screen readers), proper focus management, and the overall usability of the component within its real-world context.
\end{enumerate}

To illustrate this methodology in practice, we first map the \textit{UI} elements of a representative screen to their corresponding semantic roles. Next, we link each component to the relevant \acrshort{wcagacr} and \acrshort{mcagacr} criteria presented in the previous subsections, noting both the minimum compliance requirements and potential enhancements. Finally, we describe the technical solution—specifically, how React Native accessibility code properties are applied to meet and exceed these standards. This structured approach not only bridges the gap between abstract guidelines and real-world coding tasks but also sets the stage for the more detailed, screen-by-screen analyses presented in the next section.

\section{Accessibility Implementation Guidelines}
\label{sec:implementation-guidelines}

Having established the overall architecture of \textit{AccessibleHub} and the guiding principles from both \textit{WCAG} and \textit{MCAG}, we now present a screen-by-screen analysis. Each subsection highlights the key \textit{success criteria} addressed, references relevant \textit{mobile-specific considerations}, and demonstrates practical solutions in React Native. Where applicable, we contrast these with Flutter's approach, building upon the insights from Budai \cite{budai2024mobile} in Flutter - following guidelines and then giving advice into introducing new ones.

\subsection{Home screen}

The Home Screen serves as the primary entry point of the \textit{AccessibleHub} application. It provides key metrics on accessibility compliance (e.g., number of accessible components, \acrshort{wcagacr} conformance level) and direct navigation to core sections: \textit{Accessible Components} (Quick Start), \textit{Best Practices}, \textit{Testing Tools}, and the \textit{Framework Comparison}. An example of the interface is shown in Figure~\ref{fig:home_screens_sidebyside}.

\begin{figure}[ht]
    \centering
    \begin{subfigure}[b]{0.48\textwidth}
        \centering
        \includegraphics[width=\linewidth, alt={First part of the Home Screen}]{img/home1.png}
        \caption{Home screen - Part 1}
        \label{fig:home-left}
    \end{subfigure}
    \hfill
    \begin{subfigure}[b]{0.48\textwidth}
        \centering
        \includegraphics[width=\linewidth, alt={Second part of the Home Screen}]{img/home2.png}
        \caption{Home screen - Part 2}
        \label{fig:home-right}
    \end{subfigure}
    \caption{Side-by-side view of the two Home sections, with metrics and navigation buttons}
    \label{fig:home_screens_sidebyside}
\end{figure}

\subsubsection{Component inventory and WCAG/MCAG mapping}

Table~\ref{tab:component_criteria_mapping} provides a formal mapping between the UI components, their semantic roles, the specific WCAG 2.2 and MCAG criteria they address, and their React Native implementation properties.

\begin{longtable}{|p{2.5cm}|p{2cm}|p{2.8cm}|p{2.8cm}|p{4.3cm}|}
\caption{Home screen component-criteria mapping}
\label{tab:component_criteria_mapping}\\
\hline
\textbf{Component} & \textbf{Semantic Role} & \textbf{WCAG 2.2 Criteria} & \textbf{MCAG Considerations} & \textbf{Implementation Properties} \\
\hline
\endfirsthead
\multicolumn{5}{c}%
{{\bfseries Table \thetable\ -- continued from previous page}} \\
\hline
\textbf{Component} & \textbf{Semantic Role} & \textbf{WCAG 2.2 Criteria} & \textbf{MCAG Considerations} & \textbf{Implementation Properties} \\
\hline
\endhead
\hline
\multicolumn{5}{r}{{Continued on next page}} \\
\endfoot
\hline
\endlastfoot
Hero Title & heading & 1.4.3 Contrast (AA)\newline 2.4.6 Headings (AA) & Text readability on variable screen sizes & \texttt{accessibilityRole \ ="header"} \\
\hline
Stats Cards & button & 1.4.3 Contrast (AA)\newline 2.5.8 Target Size (AA)\newline 4.1.2 Name, Role, Value (A) & Touch target size\newline Non-essential information & \texttt{accessibilityRole \ ="button"}\newline \texttt{accessibilityLabel="\$\{value\}\% \$\{type\}, tap for details"}\newline \texttt{accessibilityHint="Shows \$\{type\} details"} \\
\hline
Decorative Icons & none & 1.1.1 Non-text Content (A) & Reduction of unnecessary focus stops & \texttt{accessibility \ ElementsHidden \ =true}\newline \texttt{important \ ForAccessibility="no"} \\
\hline
Quick Start Button & button & 1.4.3 Contrast (AA)\newline 2.5.8 Target Size (AA)\newline 2.5.2 Pointer Cancellation (A) & One-handed operation & \texttt{accessibilityRole \ ="button"}\newline \texttt{minHeight: 48}\newline \texttt{minWidth: 150} \\
\hline
Feature Cards & button & 1.3.1 Info and Relationships (A)\newline 1.4.3 Contrast (AA)\newline 2.5.8 Target Size (AA) & Logical grouping & \texttt{accessibilityRole \ ="button"}\newline \texttt{accessibilityLabel \ ="\$\{title\}"}\newline \texttt{accessibilityHint \ ="\$\{hint\}"} \\
\hline
Modal Dialog & dialog & 2.4.3 Focus Order (A)\newline 4.1.2 Name, Role, Value (A) & Keyboard trap prevention & \texttt{accessibilityRole \ ="dialog"}\newline Focus management implementation \\
\hline
Modal Tabs & tablist & 2.4.7 Focus Visible (AA)\newline 4.1.2 Name, Role, Value (A) & Touch interaction & \texttt{accessibilityRole \ ="tablist"}\newline \texttt{accessibility \ State= \ \{\{ selected: isActive \}\}} \\
\end{longtable}

\subsubsection{Formal metrics calculation methodology}

The Home Screen displays three key metrics that provide quantitative measurements of the application's accessibility. These metrics are not arbitrary but are calculated using a formal methodology defined in the \texttt{calculateAccessibilityScore} function within \texttt{index.tsx}. 

\begin{figure}[ht]
    \centering
    \begin{subfigure}[b]{0.48\textwidth}
        \centering
        \includegraphics[width=\linewidth]{img/wcag-compliance.jpg}
        \caption{WCAG compliance overview}
        \label{fig:wcag-compliance-modal}
    \end{subfigure}
    \hfill
    \begin{subfigure}[b]{0.48\textwidth}
        \centering
        \includegraphics[width=\linewidth]{img/wcag-compliance-details.jpg}
        \caption{WCAG compliance details}
        \label{fig:wcag-details-modal}
    \end{subfigure}
    \caption{Modal dialogs showing WCAG compliance metrics}
    \label{fig:wcag_modal_pair}
\end{figure}

\begin{figure}[ht]
    \centering
    \begin{subfigure}[b]{0.48\textwidth}
        \centering
        \includegraphics[width=\linewidth]{img/component-modal.jpg}
        \caption{Component metrics overview}
        \label{fig:component-overview-modal}
    \end{subfigure}
    \hfill
    \begin{subfigure}[b]{0.48\textwidth}
        \centering
        \includegraphics[width=\linewidth]{img/component-details.jpg}
        \caption{Component metrics details}
        \label{fig:component-details-modal}
    \end{subfigure}
    \caption{Modal dialogs showing component accessibility metrics}
    \label{fig:component_modal_pair}
\end{figure}

\begin{figure}[ht]
    \centering
    \begin{subfigure}[b]{0.48\textwidth}
        \centering
        \includegraphics[width=\linewidth]{img/screen-reader-modal.jpg}
        \caption{Screen reader testing overview}
        \label{fig:screen-reader-overview}
    \end{subfigure}
    \hfill
    \begin{subfigure}[b]{0.48\textwidth}
        \centering
        \includegraphics[width=\linewidth]{img/screen-reader-details.jpg}
        \caption{Screen reader testing details}
        \label{fig:screen-reader-details}
    \end{subfigure}
    \caption{Modal dialogs showing screen reader testing metrics}
    \label{fig:screen_reader_modals}
\end{figure}

\begin{figure}[ht]
    \centering
    \begin{subfigure}[b]{0.48\textwidth}
        \centering
        \includegraphics[width=\linewidth]{img/methodology.jpg}
        \caption{Methodology explanation}
        \label{fig:methodology-modal}
    \end{subfigure}
    \hfill
    \begin{subfigure}[b]{0.48\textwidth}
        \centering
        \includegraphics[width=\linewidth]{img/references.jpg}
        \caption{References documentation}
        \label{fig:references-modal}
    \end{subfigure}
    \caption{Modal dialogs showing methodology and references}
    \label{fig:methodology_references}
\end{figure}

\paragraph{Component accessibility score}

The Component Accessibility Score is calculated using the following formula:
\begin{equation}
\text{ComponentScore} 
= \left(\frac{\text{AccessibleComponents}}{\text{TotalComponents}}\right) \times 100
\end{equation}

Where:
\begin{itemize}
    \item \texttt{AccessibleComponents} = Number of components with properly implemented accessibility attributes (18)
    \item \texttt{TotalComponents} = Total number of UI components used in the application (20)
\end{itemize}

The implementation in \texttt{index.tsx} maintains a formal registry of all UI components:

\begin{lstlisting}[
  style=ReactNativeStyle,
  caption={Component registry and calculation},
  label={lst:component-registry},
  basicstyle=\ttfamily\footnotesize,
  numbers=left,
]
// Component registry with accessibility status tracking
const componentsRegistry = {
  'button': { implemented: true, accessible: true, screens: ['home', 'gestures'] },
  'text': { implemented: true, accessible: true, screens: ['home', 'guidelines'] },
  // ... other components
  'tooltip': { implemented: true, accessible: false, screens: [] },
  // Total: 20 components, 18 fully accessible
};

// Component calculation
const componentsTotal = Object.keys(componentsRegistry).length;
const accessibleComponents = Object.values(componentsRegistry)
  .filter(c => c.implemented && c.accessible).length;
const componentScore = Math.round((accessibleComponents / componentsTotal) * 100);
\end{lstlisting}

\paragraph{WCAG compliance score}

The WCAG Compliance Score represents the percentage of implemented WCAG 2.2 success criteria across four principles:

\begin{equation}
\text{WCAGCompliance} 
= \left(\frac{\text{CriteriaLevelAMet} + \text{CriteriaLevelAAMet}}{\text{TotalCriteria}}\right) \times 100
\end{equation}

Where:
\begin{itemize}
    \item \texttt{CriteriaLevelAMet} = Number of Level A success criteria implemented (25)
    \item \texttt{CriteriaLevelAAMet} = Number of Level AA success criteria implemented (13)
    \item \texttt{TotalCriteria} = Total applicable WCAG criteria (43)
\end{itemize}

The implementation maintains a comprehensive tracking system for WCAG criteria:

\begin{lstlisting}[
  style=ReactNativeStyle,
  caption={WCAG criteria tracking and calculation},
  label={lst:wcag-tracking},
  basicstyle=\ttfamily\footnotesize,
  numbers=left,
]
// WCAG criteria tracking with implementation status
const wcagCriteria = {
  '1.1.1': { level: 'A', implemented: true, name: "Non-text Content" },
  '1.3.1': { level: 'A', implemented: true, name: "Info and Relationships" },
  // ... other criteria
  '4.1.3': { level: 'AA', implemented: true, name: "Status Messages" },
};

// WCAG compliance calculation
const criteriaValues = Object.values(wcagCriteria);
const totalCriteria = criteriaValues.length;
const levelACriteriaMet = criteriaValues
  .filter(c => c.level === 'A' && c.implemented).length;
const levelAACriteriaMet = criteriaValues
  .filter(c => c.level === 'AA' && c.implemented).length;
const wcagCompliance = Math.round(
  ((levelACriteriaMet + levelAACriteriaMet) / totalCriteria) * 100
);
\end{lstlisting}

\paragraph{Screen reader testing score}

The Screen Reader Testing Score represents empirical testing with VoiceOver (iOS) and TalkBack (Android):

\begin{equation}
\text{TestingScore} 
= \left(\frac{\text{VoiceOverAvg} + \text{TalkBackAvg}}{2}\right) \times 20
\end{equation}

Where:
\begin{itemize}
    \item \texttt{VoiceOverAvg} = Average score from VoiceOver testing across categories (4.34/5)
    \item \texttt{TalkBackAvg} = Average score from TalkBack testing across categories (4.18/5)
\end{itemize}

The scores are based on structured testing of five key aspects:

\begin{lstlisting}[
  style=ReactNativeStyle,
  caption={Screen reader testing results and calculation},
  label={lst:screen-reader-testing},
  basicstyle=\ttfamily\footnotesize,
  numbers=left,
]
// Screen reader test results from empirical testing
const screenReaderTests = {
  voiceOver: { // iOS
    navigation: 4.5, // Logical navigation flow
    gestures: 4.0,   // Gesture recognition
    labels: 4.5,     // Label clarity and completeness
    forms: 4.2,      // Form control accessibility
    alerts: 4.5      // Alert and dialog accessibility
  },
  talkBack: { // Android
    navigation: 4.3,
    gestures: 4.2,
    labels: 4.4,
    forms: 4.0,
    alerts: 4.0
  }
};

// Testing score calculation
const voiceOverScores = Object.values(screenReaderTests.voiceOver);
const talkBackScores = Object.values(screenReaderTests.talkBack);
const voiceOverAvg = voiceOverScores.reduce((sum, score) => 
  sum + score, 0) / voiceOverScores.length;
const talkBackAvg = talkBackScores.reduce((sum, score) => 
  sum + score, 0) / talkBackScores.length;
const testingScore = Math.round(((voiceOverAvg + talkBackAvg) / 2) * 20);
\end{lstlisting}

\paragraph{Overall accessibility score}

The overall accessibility score is calculated using weighted components:
\begin{equation}
\text{OverallScore} 
= (\text{ComponentScore} \times 0.4) 
+ (\text{WCAGCompliance} \times 0.4) 
+ (\text{TestingScore} \times 0.2)
\end{equation}

This weighting system gives equal importance to component implementation and standards compliance (40\% each), with empirical testing contributing 20\% to the final score.

\subsubsection{Technical implementation analysis}

The following annotated code sample demonstrates the key accessibility properties implemented in the Home Screen:

\begin{lstlisting}[
  style=ReactNativeStyle,
  caption={Annotated code sample demonstrating Home Screen accessibility properties},
  label={lst:home-screen-accessibility},
  basicstyle=\ttfamily\footnotesize,
  numbers=left,
]
// 1. ScrollView container with proper role and label
<ScrollView
  accessibilityRole="scrollview"
  accessibilityLabel="AccessibleHub Home Screen"
>
  {/* 2. Hero section with semantic heading */}
  <View style={themedStyles.heroCard}>
    <Text style={themedStyles.heroTitle} accessibilityRole="header">
      The ultimate accessibility-driven toolkit for developers
    </Text>

    {/* 3. Stats section with interactive metrics */}
    <View style={themedStyles.statsContainer}>
      <View style={themedStyles.statCard}>
        <TouchableOpacity
          style={themedStyles.touchableStat}
          onPress={() => openMetricDetails('component')}
          accessible
          accessibilityRole="button"
        >
          {/* 4. Content with accessibilityElementsHidden to prevent redundant 
              announcements */}
          <Text style={themedStyles.statNumber} accessibilityElementsHidden>
            {accessibilityMetrics.componentCount}
          </Text>
          <Text style={themedStyles.statLabel} accessibilityElementsHidden>
            Components
          </Text>
        </TouchableOpacity>
      </View>
  </View>

  {/* 6. Quick Start button with appropriate sizing for touch targets */}
  <TouchableOpacity
    style={themedStyles.quickStartCard}
    onPress={() => router.push('/components')}
    accessibilityRole="button"
    accessibilityLabel="Quick start with component examples"
    accessibilityHint="Navigate to components section"
  >
    <View style={themedStyles.cardText}>
      <Text style={themedStyles.cardTitle}>Quick Start</Text>
      <Text style={themedStyles.cardDescription}>
        Explore accessible component examples
      </Text>
    </View>
  </TouchableOpacity>
</ScrollView>
\end{lstlisting}

\subsubsection{Contrast and color analysis}

Table~\ref{tab:contrast_analysis} presents the formal contrast analysis for UI elements on the Home Screen. All elements meet at least WCAG Level~AA requirements (4.5:1 for normal text).

\begin{longtable}{|p{2.8cm}|p{2.8cm}|p{2.8cm}|p{2.4cm}|p{2.4cm}|}
\caption{Home screen contrast analysis}
\label{tab:contrast_analysis}\\
\hline
\textbf{UI Element} & \textbf{Foreground Color} & \textbf{Background Color} & \textbf{Contrast Ratio} & \textbf{WCAG Compliance} \\
\hline
\endfirsthead
\multicolumn{5}{c}%
{{\bfseries Table \thetable\ -- continued from previous page}} \\
\hline
\textbf{UI Element} & \textbf{Foreground Color} & \textbf{Background Color} & \textbf{Contrast Ratio} & \textbf{WCAG Compliance} \\
\hline
\endhead
\hline
\multicolumn{5}{r}{{Continued on next page}} \\
\endfoot
\hline
\endlastfoot
Hero Title & \#000000 (Light)\newline \#FFFFFF (Dark) & \#FFFFFF (Light)\newline \#121212 (Dark) & 21:1 (Light)\newline 21:1 (Dark) & AAA ($\ge7{:}1$) \\
\hline
Subtitle & \#6B7280 (Light)\newline \#A0AEC0 (Dark) & \#FFFFFF (Light)\newline \#121212 (Dark) & 4.6:1 (Light)\newline 5.2:1 (Dark) & AA ($\ge4.5{:}1$) \\
\hline
Stat Numbers & \#0066CC (Light)\newline \#3B82F6 (Dark) & \#FFFFFF (Light)\newline \#121212 (Dark) & 4.7:1 (Light)\newline 5.1:1 (Dark) & AA ($\ge4.5{:}1$) \\
\hline
Quick Start Button & \#FFFFFF & \#0066CC & 4.8:1 & AA ($\ge4.5{:}1$) \\
\hline
Feature Card Titles & \#000000 (Light)\newline \#FFFFFF (Dark) & \#FFFFFF (Light)\newline \#1E293B (Dark) & 21:1 (Light)\newline 16:1 (Dark) & AAA ($\ge7{:}1$) \\
\end{longtable}

\subsubsection{Screen reader support analysis}

Table~\ref{tab:screen_reader_analysis} presents results from systematic testing of the Home Screen with screen readers on both iOS and Android platforms.

\begin{longtable}{|p{2.8cm}|p{3.5cm}|p{3.5cm}|p{4cm}|}
\caption{Home screen screen reader testing results}
\label{tab:screen_reader_analysis}\\
\hline
\textbf{Test Case} & \textbf{VoiceOver (iOS 16)} & \textbf{TalkBack (Android 14)} & \textbf{WCAG Criteria Addressed} \\
\hline
\endfirsthead
\multicolumn{4}{c}%
{{\bfseries Table \thetable\ -- continued from previous page}} \\
\hline
\textbf{Test Case} & \textbf{VoiceOver (iOS 16)} & \textbf{TalkBack (Android 14)} & \textbf{WCAG Criteria Addressed} \\
\hline
\endhead
\hline
\multicolumn{4}{r}{{Continued on next page}} \\
\endfoot
\hline
\endlastfoot
Hero Title & \ding{51} Announces ``The ultimate accessibility-driven toolkit for developers, heading'' & \checkmark Announces ``The ultimate accessibility-driven toolkit for developers, heading'' & 1.3.1 - Info and Relationships (Level A), 2.4.6 - Headings and Labels (Level AA) \\
\hline
Metrics Cards & \ding{51} Announces full label with metrics and hint & \checkmark Announces full label with metrics and hint & 1.3.1 Info and Relationships (Level A), 4.1.2 Name, Role, Value (Level A) \\
\hline
Quick Start Button & \ding{51} Announces ``Quick start with component examples, button'' & \ding{51} Announces ``Quick start with component examples, button'' & 2.4.4 Link Purpose (In Context) (Level A), 4.1.2 Name, Role, Value (Level A) \\
\hline
Feature Cards & \ding{51} Announces title and hint & \ding{51} Announces title and hint & 2.4.4 Link Purpose (In Context) (Level A), 4.1.2 Name, Role, Value (Level A) \\
\hline
Modal Dialog Opening & \ding{51} Focus moves to dialog title & \checkmark Focus moves to dialog title & 2.4.3 Focus Order (Level A) \\
\hline
Modal Tab Navigation & \ding{51} Announces tab selection state & \checkmark Announces tab selection state & 4.1.2 Name, Role, Value (Level A) \\
\hline
Modal Dialog Closing & \ding{51} Focus returns to triggering element & \ding{54} Occasional focus loss (fixed in v1.0.3) & 2.4.3 Focus Order (Level A) \\
\end{longtable}

The implementation addresses several key MCAG considerations:
\begin{enumerate}
    \item \textbf{Swipe optimization}: Decorative elements are marked with \\ \texttt{importantForAccessibility="no"} to reduce unnecessary swipes, addressing professor feedback about ``garbage interactions.''
    \item \textbf{Clear instructions}: The modal tabs implementation provides clear state announcements, ensuring screen reader users understand the current selection.
    \item \textbf{Platform-specific adaptations}: The implementation accounts for differences between VoiceOver and TalkBack behavior, as evidenced by the test results.
\end{enumerate}

\subsubsection{Implementation overhead analysis}

Table~\ref{tab:implementation_overhead} quantifies the additional code required to implement accessibility features in the Home Screen.

\begin{longtable}{|p{3.8cm}|p{2.3cm}|p{2.8cm}|p{2.8cm}|}
\caption{Accessibility implementation overhead}
\label{tab:implementation_overhead}\\
\hline
\textbf{Accessibility Feature} & \textbf{Lines of Code} & \textbf{Percentage of Total} & \textbf{Complexity Impact} \\
\hline
\endfirsthead
\multicolumn{4}{c}%
{{\bfseries Table \thetable\ -- continued from previous page}} \\
\hline
\textbf{Accessibility Feature} & \textbf{Lines of Code} & \textbf{Percentage of Total} & \textbf{Complexity Impact} \\
\hline
\endhead
\hline
\multicolumn{4}{r}{{Continued on next page}} \\
\endfoot
\hline
\endlastfoot
Semantic Roles & 12 LOC & 2.1\% & Low \\
\hline
Descriptive Labels & 24 LOC & 4.3\% & Medium \\
\hline
Element Hiding & 8 LOC & 1.4\% & Low \\
\hline
Focus Management & 18 LOC & 3.2\% & Medium \\
\hline
Contrast Handling & 16 LOC & 2.9\% & Medium \\
\hline
Metrics Calculation & 78 LOC & 14.1\% & High \\
\hline
\textbf{Total} & \textbf{156 LOC} & \textbf{28.0\%} & \textbf{Medium-High} \\
\end{longtable}

This analysis reveals that implementing comprehensive accessibility adds approximately 28\% to the code base of the Home Screen, with the metrics calculation system representing the most significant component. This overhead is justified by the improved user experience for people with disabilities and the educational value for developers learning to implement accessibility.

\subsubsection{WCAG conformance by principle}

Table~\ref{tab:wcag_by_principle} provides a detailed analysis of WCAG 2.2 compliance by principle:

\begin{longtable}{|p{2.5cm}|p{3.8cm}|p{3.2cm}|p{5.2cm}|}
\caption{WCAG compliance analysis by principle}
\label{tab:wcag_by_principle}\\
\hline
\textbf{Principle} & \textbf{Description} & \textbf{Implementation Level} & \textbf{Key Success Criteria} \\
\hline
\endfirsthead
\multicolumn{4}{c}%
{{\bfseries Table \thetable\ -- continued from previous page}} \\
\hline
\textbf{Principle} & \textbf{Description} & \textbf{Implementation Level} & \textbf{Key Success Criteria} \\
\hline
\endhead
\hline
\multicolumn{4}{r}{{Continued on next page}} \\
\endfoot
\hline
\endlastfoot
1. Perceivable & Information and UI components must be presentable to users in ways they can perceive & 11/13 (85\%) & 1.1.1 Non-text Content (A)\newline 1.3.1 Info and Relationships (A)\newline 1.4.3 Contrast (Minimum) (AA) \\
\hline
2. Operable & UI components and navigation must be operable & 16/17 (94\%) & 2.4.3 Focus Order (A)\newline 2.4.7 Focus Visible (AA)\newline 2.5.8 Target Size (Minimum) (AA) \\
\hline
3. Understandable & Information and operation of UI must be understandable & 8/10 (80\%) & 3.2.1 On Focus (A)\newline 3.2.4 Consistent Identification (AA)\newline 3.3.2 Labels or Instructions (A) \\
\hline
4. Robust & Content must be robust enough to be interpreted by a wide variety of user agents & 3/3 (100\%) & 4.1.1 Parsing (A)\newline 4.1.2 Name, Role, Value (A)\newline 4.1.3 Status Messages (AA) \\
\end{longtable}

\subsubsection{Mobile-specific considerations}

The Home Screen implementation addresses several mobile-specific accessibility considerations beyond standard WCAG requirements:

\begin{enumerate}
    \item \textbf{Touch target sizing}: All interactive elements maintain minimum dimensions of 48$\times$48dp, exceeding the WCAG 2.5.8 requirement of 24$\times$24px and addressing the mobile-specific need for larger touch targets.
    \item \textbf{Reduced motion support}: The implementation respects the device's reduced motion settings and provides an in-app toggle, addressing vestibular disorders that are particularly relevant in mobile contexts.
    \item \textbf{Dark mode support}: The application's theming system adapts to both light and dark modes, addressing the mobile-specific need for readability in various lighting conditions.
    \item \textbf{Screen reader gesture optimization}: The implementation carefully manages focus to ensure efficient navigation with touch gestures, as shown in the screen reader testing results.
    \item \textbf{One-handed operation}: The layout places primary interactive elements within reach of a thumb during one-handed use, a critical mobile accessibility consideration not explicitly covered by WCAG.
\end{enumerate}

\subsubsection{Future enhancements}

Based on the formal analysis, several potential enhancements have been identified for future versions:
\begin{enumerate}
    \item \textbf{Real-time metric updates}: Implementing dynamic updates to accessibility metrics as developers modify their applications, providing immediate feedback on compliance.
    \item \textbf{Enhanced focus visualization}: Further improving focus indicators to ensure they meet the enhanced 3:1 contrast ratio recommended by WCAG~2.2 for user interface components.
    \item \textbf{Focus restoration in TalkBack}: Addressing the occasional focus loss issue in TalkBack when closing modal dialogs.
    \item \textbf{Voice command support}: Adding support for voice activation of primary functions, further enhancing accessibility for users with motor impairments.
    \item \textbf{Automated testing integration}: Expanding the metrics calculation system to include results from automated testing tools.
\end{enumerate}

\subsection{Accessible Components Section}

\subsubsection{Relevant guidelines and success criteria}

\paragraph{WCAG 2.2}

\paragraph{MCAG}

\subsubsection{Implementation details in React Native}

\paragraph{Key observations}

\paragraph{Future enhancements}

\subsection{Accessible Component 1}

\subsubsection{Relevant guidelines and success criteria}

\paragraph{WCAG 2.2}

\paragraph{MCAG}

\subsubsection{Implementation details in React Native}

\paragraph{Key observations}

\paragraph{Future enhancements}

\subsection{Best Practices Section}

\subsubsection{Relevant guidelines and success criteria}

\paragraph{WCAG 2.2}

\paragraph{MCAG}

\subsubsection{Implementation details in React Native}

\paragraph{Key observations}

\paragraph{Future enhancements}

\subsection{Best Practice 1}

\subsubsection{Relevant guidelines and success criteria}

\paragraph{WCAG 2.2}

\paragraph{MCAG}

\subsubsection{Implementation details in React Native}

\paragraph{Key observations}

\paragraph{Future enhancements}

\subsection{Framework Comparison}

\subsubsection{Relevant guidelines and success criteria}

\paragraph{WCAG 2.2}

\paragraph{MCAG}

\subsubsection{Implementation details in React Native}

\paragraph{Key observations}

\paragraph{Tools}

\subsection{Best Practice 1}

\subsubsection{Relevant guidelines and success criteria}

\paragraph{WCAG 2.2}

\paragraph{MCAG}

\subsubsection{Implementation details in React Native}

\paragraph{Key observations}

\paragraph{Future enhancements}

\subsection{Settings}

\subsubsection{Relevant guidelines and success criteria}

\paragraph{WCAG 2.2}

\paragraph{MCAG}

\subsubsection{Implementation details in React Native}

\paragraph{Key observations}

\paragraph{Future enhancements}

\subsection{Instruction and Community}

\subsubsection{Relevant guidelines and success criteria}

\paragraph{WCAG 2.2}

\paragraph{MCAG}

\subsubsection{Implementation details in React Native}

\paragraph{Key observations}

\paragraph{Future enhancements}


\newpage


    \chapter{Accessibility analysis: framework comparison and implementation patterns} 
\label{chap:implementation}

\chapterintroline{
   This chapter offers a systematic, comparative analysis of accessibility implementation in React Native and Flutter. Through empirical evaluation of equivalent components, we address three core questions: the default accessibility of components, the feasibility of implementing accessibility for non-accessible components, and the development effort required for these implementations. Combining quantitative metrics with qualitative assessments of developer experience, this analysis provides practical insights into how each framework facilitates the creation of accessible mobile applications.
}

\section {Research methodology}

\subsection {Research questions and objectives}

\subsection {Testing approach and criteria}

\subsection {Evaluation metrics}

\section {React Native Approach}

\subsection {Component accessibility analysis}

\subsection {Native platform integration}

\subsection {Implementation patterns and solutions}

\section {Flutter Approach}

\subsection{Framework overview}

\subsection {Component accessibility patterns}

\subsection {Platform-specific considerations}

\subsection {Common implementation challenges}

\section {Comparative analysis}

\subsection {Default accessibility support}

\subsection {Implementation complexity}

\subsection {Developer experience}

\subsection {Performance implications}

\section {Framework-specific best practices}

\subsection{React Native optimization patterns}

\subsection{Flutter accessibility patterns}

\subsection{Cross-platform considerations}

\newpage
    \chapter{Conclusions and future research}
\label{chap:conclusions}
\chapterintroline{
    This chapter synthesizes the key findings from our comparative analysis of React Native and Flutter accessibility implementations, presents implications for mobile developers, and outlines directions for future research in cross-platform mobile accessibility. By contextualizing our findings within the broader landscape of accessible mobile development, we bridge theoretical understanding with practical implementation guidance.
}

\section{Summary of key findings}
\label{sec:key-findings}



\section{Implications for mobile developers}
\label{sec:implications}


\section{Conclusion and critical thoughts}
\label{sec:conclusion}



\section{Limitations of the research}
\label{sec:limitations}


\section{Directions for future research}
\label{sec:future-research}

    
    \backmatter
    \chapter{Bibliography}
\label{cap:bibliography}

\nocite{*}

% Books bibliography
\printbibliography[heading=subbibliography, title={Books}, type=book]

% Articles bibliography
\printbibliography[heading=subbibliography, title={Articles and papers}, type=article]

% Websites bibliography
\printbibliography[heading=subbibliography, title={Sites}, type=online]

\end{document}