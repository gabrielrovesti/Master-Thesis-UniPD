\chapter{Mobile accessibility: guidelines, standards and related works}
\label{chap:accessibility}

\chapterintroline{
    This chapter reviews mobile accessibility research and standards. It covers current accessibility legislation, key development guidelines (focusing on practical implementation), and significant studies on user experience, development challenges, and testing methodologies.
}

\section{Accessibility legislative frameworks}
\label{chap:accessibility-history-rules}

The journey towards digital accessibility has been shaped by both legislative frameworks and technological advancements, alongside the evolution of devices and how they integrate into daily life. These developments reflect not just a response to legal requirements, but a fundamental shift in how we approach digital design and development. The goal has evolved from simple compliance to embracing universal design principles - creating products and services that can be used by everyone, regardless of their abilities or circumstances \cite{article:universal-design}.

Universal design in the digital world embodies the principle that technology should be inclusive, since many times it's treated as an afterthought, while it must be considered from the earliest stages of development. This evolution has been particularly significant in the mobile ecosystem, where the constant need of connectivity and the multiple usages of these devices have opened multiple opportunities, but also challenges for both users and content creators. Connectivity, convenience and creativity are one of the main focus and purpose of the online world, where Internet and access to a mobile device has been recognized to be one of the fundamental rights for human beings in general. As evidenced by the multiple ways users interact with mobile platforms, as described in \ref{chap:intro-background}, there are significant challenges in the current state of digital accessibility. These challenges stem from two main factors: the difficulty in addressing user needs and the lack of clear implementation guidelines for developers. \\

To understand the current state of mobile accessibility, it's crucial to examine the legislative landscape that has shaped its development. This progression of laws and regulations demonstrates how accessibility requirements have evolved from broad civil rights protections to specific technical standards for digital interfaces. Several key legislative milestones across different regions have shaped this evolution - we will see the main ones 

\begin{itemize}
    \item In the \textit{United States}, the foundation was built through a number of major pieces of legislation. The \textit{Americans with Disabilities Act (ADA)} of \textit{1990}, while predating modern mobile technology, established a number of critical precedents regarding the rights of disabled citizens. Initially targeted at physical accessibility, interpretations of the ADA have expanded to include digital spaces, both mobile applications and websites. At the same time, OSes like Windows implemented accessibility features pre-loaded within the system itself in \textit{1995}, instead of having them available as add-ons or plug-ins. This is further reinforced by the \textit{Section 508 Amendment} in \textit{1998} \cite{eo14028} to the Rehabilitation Act, addressing digital accessibility requirements relative to federal agencies and their contractors for websites alike. Shortly after, between \textit{2002} and \textit{2005}, Apple introduced both Universal Access and VoiceOver, both with the goal of increasing accessibility within options and controls present inside of their devices;
    
    \item \textit{Italy} has developed its own robust framework for digital accessibility, building upon and extending European requirements. \textit{Legge Stanca (Law 4/2004)}, updated in \textit{2010}, established comprehensive accessibility requirements for public administration websites and applications. This was further enhanced by the creation of \textit{AGID (Agenzia per l'Italia Digitale)} in \textit{2012}, which provides detailed technical guidelines and ensures compliance across public and private sectors;
    
    \item The \textit{European Union} has moved to more modern legislation concerning digital accessibility in recent times. The \textit{European Accessibility Act}, passed in \textit{2019}, contains broad requirements with specific coverage of modern digital technologies. This is further codified in the \textit{Directive (EU) 2016/2102 on the accessibility of websites and mobile applications of public sector bodies} \cite{eu2016directive}, which explicitly mandates WCAG 2.1 AA compliance for all public sector mobile applications This is different from earlier legislation, as legislation like the explicit inclusion of mobile applications as central in modern digital interaction by the EAA, is complemented by standard \textit{EN 301 549} that provides detailed technical specifications aligned with international accessibility guidelines.
\end{itemize}

These legislative frameworks are supported by international technical standards, especially the \textit{Web Content Accessibility Guidelines}, created by the \acrshort{w3c}. WCAG has evolved from its first version in 1999 to this year's WCAG 2.2 (came out in 2023), reflecting increased sophistication in digital interfaces and interaction patterns. In each iteration, more scope and detail about the requirements have been added; recent versions place particular emphasis on mobile and touch interfaces. WCAG serves as the primary technical foundation for digital accessibility implementation worldwide, providing specific, testable criteria for making content accessible to people with disabilities, serving as one of the main foundations for developers and content creators to be used as standard of reference. The guidelines implement three levels of conformance (A, AA, and AAA), providing increasingly stringent accessibility requirements. These will be explored in depth and used as main reference for the work present inside of this research, to establish clear degrees of success criteria to be met by the frameworks relative implementations.

\section{Accessibility standard guidelines}
\label{sec:accessibility-guidelines}

Accessibility guidelines and standards form the foundation upon which inclusive mobile app development practices are built. They provide a shared framework for understanding and addressing the diverse needs of users with disabilities, ensuring that mobile apps are perceivable, operable, understandable, and robust. This section explores the key accessibility guidelines and standards relevant to mobile app development, describing them briefly before seeing how they apply to the concrete use case of this thesis' application, following the principles presented here.

\subsection{Web Content Accessibility Guidelines (WCAG)}

The \gls{wcagg}, developed by the \gls{w3cg}, serve as the international standard for digital accessibility (\cite{site:wcag}). Although originally designed for web content, the WCAG principles and guidelines are equally applicable to mobile app development. The WCAG is organized around four main principles:

\begin{itemize}
    \item \textit{Perceivable}: Information and user interface components must be presentable to users in ways they can perceive. This includes providing text alternatives for non-text content, creating content that can be presented in different ways without losing meaning, and making it easier for users content;
    
    \item \textit{Operable}: User interface components and navigation must be operable. This means that all functionality should be available also from a keyboard, users should have enough time to read and use the content, and content should not cause seizures or physical reactions;
    
    \item \textit{Understandable}: Information and the interactions provided by the user interface must be understandable. This involves making text content readable and understandable, making content appear and operate in predictable ways, and helping users avoid and correct mistakes;
    
    \item \textit{Robust}: Content must be robust enough that it can be interpreted by a wide variety of user agents, including assistive technologies. This requires maximizing compatibility with current and future user agents.
\end{itemize}

Under each principle, the WCAG provides specific guidelines and success criteria at three levels of conformance (A, AA, and AAA). These success criteria are testable statements that help developers determine whether their app meets the accessibility requirements. By understanding and applying the WCAG principles and guidelines, mobile app developers can create more inclusive and accessible experiences for their users.

\subsection{Mobile Content Accessibility Guidelines (MCAG)}

While \acrshort{wcagacr} offers a comprehensive foundation, mobile platforms introduce additional complexities that may not be fully addressed by web-centric guidelines. The \gls{mcagg} (\cite{site:mcag}) build upon the previous ones by focusing on the specific interaction patterns, form factors, and environmental contexts unique to mobile devices. For example, MCAG emphasizes:

\begin{itemize}
    \item \textit{Touch interaction and gestures}: Ensuring that tap targets, swipe gestures, and multi-finger interactions are usable for individuals with varying motor skills;
    
    \item \textit{Limited screen real estate}: Designing content that remains clear and functional on smaller displays, including proper zooming and reflow behavior;

    \item \textit{Diverse mardware and os versions}: Accounting for a wide range of device capabilities, operating system versions, and hardware configurations that can affect accessibility;

    \item \textit{Contextual usage scenarios}: Recognizing that mobile apps are often used in changing lighting conditions, noisy environments, or while users are on the move.
    
\end{itemize}

In practice, \acrshort{mcagacr} complements \acrshort{wcagacr} by providing more granular, mobile-oriented guidance considering specific factors. Developers who follow these guidelines in addition to \acrshort{wcagacr} are better equipped to deliver an inclusive experience that accounts for real-world mobile usage. 

\subsection{Mobile-specific accessibility considerations}

While the previous guidelines provide by themselves a solid foundation for digital accessibility, mobile apps present unique challenges and considerations that require additional attention. Some of the key mobile-specific accessibility factors include:

\begin{itemize}
    \item \textit{Touch interaction}: Mobile devices rely heavily on touch-based interactions, such as tapping, swiping, and multi-finger gestures. Developers must ensure that all interactive elements are large enough to be easily tapped, provide alternative input methods for complex gestures, and offer appropriate haptic and visual feedback;
    
    \item \textit{Small screens}: The limited screen real estate on mobile devices can pose challenges for users with visual impairments. Developers should provide sufficient contrast, use clear and legible fonts, and ensure that content can be easily zoomed or resized without losing functionality;
    
    \item \textit{Screen reader compatibility}: Mobile screen readers, such as VoiceOver on iOS and TalkBack on Android, require proper labeling and semantic structure to effectively convey content and functionality to users with visual impairments. Developers must use appropriate accessibility APIs and ensure that all elements are properly labeled and navigable;
    
    \item \textit{Device fragmentation}: The wide range of mobile devices, screen sizes, and operating system versions can complicate accessibility testing and implementation. Developers should test their apps on a diverse range of devices and ensure that accessibility features function consistently across different configurations;
    
    \item \textit{Mobile context}: Mobile apps are often used in a variety of contexts, such as outdoors, in low-light conditions, or in noisy environments. Developers should consider these contexts and provide appropriate accommodations, such as high-contrast modes or subtitles for audio content.
\end{itemize}

By understanding and addressing these mobile-specific accessibility considerations, developers can create apps that are more inclusive and usable for a wider range of users.

\section{State of research and literature review}
\label{chap:accessibility-literature}

Having established the regulatory frameworks and technical standards that govern mobile accessibility, it becomes crucial to understand how these requirements translate into practical implementation, both of research and applications. 
Research in mobile accessibility spans multiple areas, from user interaction studies to framework-specific analyses. This section outlines the relevant work, organized by key research themes, that informs the presented approach in comparing frameworks. Various studies will be reviewed on how people, with and without impairments, interact with mobile devices. Such studies typically report on accessibility barriers and present insights into the effectiveness of general guidelines on accessibility. This literature review focuses a great deal on research related to challenges faced by users with disabilities and the implementation of accessibility features in mobile development frameworks, discussing the practical importance of the presented work.

\subsection{Users and developers accessibility studies}

In exploring accessibility solutions for mobile applications, a notable contribution comes from Zaina et al. \cite{zaina2022preventing}, who conducted extensive research into accessibility barriers that arise when using design patterns for building mobile user interfaces. The authors recognize that several user interface design patterns are present inside of libraries, but do not attach significant importance to accessibility features, which are already present in language. This study tried to adopt a \gls{grayliteraturereview} approach, gathering insights and capture real practitioners' experiences and challenges in implementing UI patterns, done by investigating professional forums or blogs. This approach proved valuable, since this was recognized as a source of practical knowledge and evidence a comprehensive catalog documenting 9 different user interface design patterns, along with descriptions of accessibility barriers present for each one and specific guidelines for prevention, for example inside of Input and Data components but also animated parts. The study's validation phase involved 60 participants, highlighting the fact participants saw value in the guidelines not just for implementing accessibility features, but also for improving their overall understanding of accessible design principles. These comprehensive results demonstrated both the practical applicability of the guidelines in real development scenarios and their effectiveness as an educational tool for raising awareness about accessibility concerns among developers.\\

Another significant contribution to report here was conducted by Vendome et al. \cite{vendome2019can} and analyzed the implementation of accessibility features inside of Android applications both quantitatively and qualitatively, with the main goal of understanding accessibility practices among developers and identify common implementation patterns through a systematic approach, while mining the web to look for data. The methodology of the research contained two major parts: first, they did a mining-based analysis of 13,817 Android applications from GitHub that had at least one follower, star, or fork to avoid abandoned projects. They have done a static analysis on the usage of accessibility APIs and the presence of assistive content description in GUI components. A second component was a qualitative review of 366 Stack Overflow discussions related to accessibility, which were formally coded following an open-coding process with multi-author agreement. \\

The key results of the mining study were that while half of the apps supported assistive content descriptions for all GUI components, only 2.08\% used accessibility APIs. The Stack Overflow analysis revealed that support for visually impaired users dominated the discussions - 43\% of the questions-and remarkably enough, 36\% of the accessibility API-related questions were about using these APIs for non-accessibility purposes. The study identified several critical barriers to accessibility implementation: lack of developer knowledge about accessibility features, limited automated support and insufficient guidance for screen readers, while having a notable gap between accessibility guidelines and implementation practices. \\

Another paper reporting notable findings is the one from Pandey et al. \cite{pandey2022accessibility}, an analytical work of 96 mailing list threads combined with 18 interviews carried out with programmers with visual impairments. The authors investigate how frameworks shape programming experiences and collaboration with sighted developers. As expected, it concluded that accessibility problems are difficult to be reduced either to programming tool UI frameworks alone: they result from interactions between multiple software components including IDEs, browser developer tools, UI frameworks, operating systems, and screen readers, a topic of this thesis and research. Results showed that, although UI frameworks have the potential to enable relatively independent creation of user interfaces that reduce reliance on sighted assistance, many of those frameworks claimed themselves to be accessible out-of-the-box, but only partially lived up to this promise. Indeed, their results showed that various accessibility barriers in programming tools and UI frameworks complicate writing UI code, debugging, and testing, and even collaboration with sighted colleagues. 

\subsection{User categories and development approaches}

In recent studies addressing accessibility in mobile applications, various user categories are analyzed to determine their unique needs and challenges, resulting in a range of development approaches tailored to specific user groups.
A good example is the systematic mapping carried out by Oliveira et al. \cite{oliveira2018elderly} about mobile accessibility for elderly users. The mapping underlined that this group faces physical and cognitive constraints, such as problems with small text, intricate navigation, and complex touch interactions. The authors suggest that, in order for content and functions to be more accessible and user-friendly even for those users whose limitations are a consequence of age, applications targeting elderly users should embed font adjustments, use of simpler language, and larger interactive elements. This paper does not only point to overcoming already present barriers but also supports and pleads for the development of age-inclusive mobile designs that would raise the level of usability and engagement for elderly users. \\

In the field of cognitive disability, the authors Jaramillo-Alcázar et al. \cite{jaramillo2017cognitive} introduce a study on the accessibility of mobile serious games, a recent developing area in both education and therapy. Their study underlines the fact that for serious games, the integration of cognitive accessibility features such as adjustable speeds, simplified instructions, and interactive elements with distinct visual appearances is crucial to help users with cognitive impairments. By discussing the features of serious games that pertain to cognitive accessibility, categorized by implementation complexity and user impact, the authors created an assessment framework. The authors identify that defining which features potentially benefit users with cognitive impairments sets the call for a normal model to guide developers in creating game interfaces accessible to the users' cognitive abilities and learning needs, with the aim of improving inclusiveness and educational potentials of mobile games.

\subsection{Testing methodologies and evaluation frameworks}

Testing and evaluating mobile accessibility presents a complex challenges, often requiring a multi-faceted approach, combining both automated tools and manual evaluation. While automated testing tools have evolved significantly, research consistently shows that no single approach can comprehensively assess all aspects of mobile accessibility.
Silva et al. \cite{silva2018survey} conducted an analysis by comparing the efficiency of automated testing tools against guidelines from the WCAG and platform-specific requirements. Silva's study researched ten different automated testing platforms, evaluating their capabilities for various accessibility criteria. Their results indicated critical limitations in the way automated tools approached accessibility testing, especially regarding mobile contexts. 
While these tools demonstrated strong capabilities in identifying technical violations, such as missing alternative text, insufficient color and improper usage of hierarchies, they consistently struggled with more nuanced aspects of accessibility, like giving meaningful description of images or verify the logical content organization when writing headings. Tools can identify the presence of error messages but cannot see if these messages are helpful and provide clear guidance for corrections; the same holds for automated tests for touch targets sizing, which cannot be evaluated in their placement makes sense from a user perspective. \\

This understanding is further reinforced by a comprehensive study led by Alshayban et al., \cite{alshayban2020accessibility} where over 1,000 Android applications in the Google Play Store were analyzed. Their work examined both the technical accessibility features and user feedback, showing that different testing methodologies often identify different kinds of accessibility issues. They also reported that automated tools could identify as many as 57\% of the technical accessibility violations but missed many issues with significant user experience impacts. Their study seems to indicate that the most effective approach to testing accessibility combines a number of different methodologies. The research identifies three key components for effective accessibility testing:

\begin{itemize}
    \item \textit{Automated testing tools}: These tools are good at systematic checking of technical requirements through programmatic analysis. They provide continuous monitoring of accessibility violations during development, while being particularly effective at regression testing and performing both static and dynamic analysis of code for common accessibility patterns;
    
    \item \textit{Manual expert evaluation}: This involves detailed assessment of contextual appropriateness by accessibility experts. They can validate semantic relationships between interface elements, evaluate complex interaction patterns, and assess error handling mechanisms in ways that automated tools cannot;
    
    \item \textit{User testing}: Provides insights through real-world usage scenarios with diverse user groups, including structured feedback from users with disabilities and testing with various assistive technologies. This often reveals issues that neither automated tools nor expert evaluation can identify, particularly regarding practical usability.
\end{itemize}

It's important to consider guidelines which can be precisely implemented for testing mobile components and ensure their accessibility across different platforms and user needs. As demonstrated by the research, neither automated tools or human testing alone can guarantee complete accessibility coverage. This underscores the critical importance of having standardized guidelines working as a general guidance framework for both automated testing tools and human evaluators. Such guidelines provide measurable success criteria that can be systematically tested while also offering the context and depth needed for manual evaluation. By following these established standards, developers can ensure a more comprehensive approach to accessibility implementation, one that benefits from both automated efficiency and human insight.

\subsection{Framework implementation approaches}

While previous research has extensively documented accessibility challenges and user needs, less attention has been paid to practical implementation comparisons across frameworks. Most comparative studies between Flutter and React Native have focused primarily on performance metrics and testing capabilities. For instance, Abu Zahra and Zein \cite{zahra2022systematic} conducted a systematic comparison between the two frameworks from an automation testing perspective, analyzing aspects such as reusability, integration, and compatibility across different devices. Their findings showed that React Native outperformed Flutter in terms of reusability and compatibility, though both frameworks demonstrated similar capabilities in terms of integration. 

However, when it comes to accessibility-specific comparisons, the research landscape is more limited. 
A research discussing and comparing the two frameworks addressing accessibility issues, which this thesis wants to base upon, is the research by Gaggi and Perinello \cite{perinello2024accessibility}, investigating three main questions: whether components are accessible by default, if non-accessible components can be made accessible, and the development cost in terms of additional code required. The study examines a set of UI elements against WCAG criteria and proposes solutions when official documentation is insufficient. \\

\subsection{Accessibility tools and extensions}

Accessibility tools and extensions development has been instrumental in bridging the gap between theory and practice. These tools have also allowed developers to efficiently include accessibility in their applications while meeting standards. For instance, Chen et al. \cite{chen2023automated} presented \textit{AccuBot}, a publicly available automated testing tool for mobile applications. The tool is integrated with continuous integration pipelines for the detection of WCAG 2.2 criteria violations, such as insufficient contrast ratios and missing \acrshort{aria} labels. In their evaluation of $500$ mobile apps, they reported that \textit{AccuBot} reduces manual testing efforts by $40$\% while sustaining high precision in identifying technical accessibility barriers.\\

Another contribution worth mentioning is the screen-reader simulation toolkit, \textit{ScreenMate}, which has been proposed by Lee et al.\cite{lee2021screenmate}. It allows developers to simulate how their mobile interfaces would behave under popular screen readers such as VoiceOver and TalkBack. By simulating user interactions for visually impaired users, \textit{ScreenMate} helps to early detect navigation inconsistencies and poorly labeled components during the development cycle. The authors have validated the toolkit in a case study with $15$ development teams, showing a $30$\% reduction in post-release bug reports about accessibility.\\

In the context of framework-specific support, Nguyen et al.\cite{nguyen2022flutter} implemented \textit{AccessiFlutter}, a plugin for Flutter guiding developers to implement widgets in an accessibility-friendly manner. It provides real-time feedback on component properties, such as using semantic labels for icons or the validation of touch target size. A comparative analysis showed that apps implemented with \textit{AccessiFlutter} attained $95$\% compliance with WCAG AA criteria, compared to manual implementation. Similarly, \cite{singh2023react} developed \textit{A11yReact}, a React Native library providing accessible pre-built components and automated auditing. Their study showed that the developers using \textit{A11yReact} needed $50$\% fewer code changes to achieve accessibility compared to regular React Native development processes.\\

These tools highlight the critical role of embedding accessibility into the development process from the outset. By leveraging automation, simulation, and framework-specific support, they tackle both technical and usability challenges, promoting inclusive design practices while maintaining development efficiency. This proactive approach ensures that accessibility is not an afterthought but a fundamental aspect of the development lifecycle, ultimately leading to more inclusive and user-friendly applications. \\

The current body of research reveals a significant gap in practical accessibility comparisons between mobile frameworks. While numerous studies examine accessibility barriers, user experiences, and support tools, there remains a lack of systematic comparative analysis of accessibility implementation between Flutter and React Native. This thesis aims to bridge this gap by expanding on Budai's research on Flutter, conducting an in-depth evaluation of both frameworks from a developer perspective. The objective is to provide practical, mobile-specific guidelines that enable developers to make informed decisions about accessibility implementation, offering detailed analysis of components and widgets across both frameworks.

\newpage