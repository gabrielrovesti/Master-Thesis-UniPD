\cleardoublepage
\phantomsection
\pdfbookmark{Abstract}{Abstract}
\begingroup
\let\clearpage\relax
\let\cleardoublepage\relax
\chapter*{Abstract}
This thesis presents a systematic comparative analysis of accessibility implementation approaches in mobile application development frameworks, specifically React Native and Flutter. Building upon previous research focused on Flutter's accessibility capabilities, this study extends the investigation to provide a comprehensive examination of how both frameworks enable developers to create accessible mobile user interfaces.

The research methodology encompasses the development of \textit{AccessibleHub}—an educational toolkit application that serves as both a research vehicle and practical resource for developers. Through this implementation, the study identifies specific patterns, similarities, differences, and potential improvements in accessibility implementation between the frameworks. The analysis examines implementation complexity, developer experience, and compliance with Web Content Accessibility Guidelines (WCAG) 2.2 standards, providing quantitative metrics for framework comparison. \\

A core contribution of this work is the translation of abstract accessibility guidelines into concrete implementation patterns, bridging the gap between theoretical requirements and practical code. The research demonstrates that while both frameworks can achieve equivalent accessibility outcomes, they differ significantly in their architectural approaches and implementation overhead. React Native employs a property-based model that typically requires less code but offers less explicit semantic control, while Flutter's widget-based semantic system provides more granular accessibility management at the cost of increased implementation complexity. 

\pagebreak

Accessibility represents not merely a compliance requirement but a fundamental aspect of both user experience and developer mindset. By expanding product usability across diverse user populations regardless of capabilities, properly implemented accessibility features ensure seamless interaction across components and devices. This research provides evidence-based guidance for framework selection and implementation strategies, contributing to the advancement of accessible mobile application development as technology continues to evolve.
\endgroup
\vfill