\cleardoublepage
\phantomsection
\pdfbookmark{Abstract}{Abstract}
\begingroup
\let\clearpage\relax
\let\cleardoublepage\relax
\chapter*{Abstract}

The following thesis aims to conduct a comparative analysis of accessibility features and guidelines for mobile application development, mainly using the React Native framework and comparing it with Flutter, upon building with previous research. This study extends the investigation conducted on the Flutter framework's accessibility capabilities to React Native, providing a comprehensive examination of both frameworks' approaches, while creating accessible mobile user interfaces.

The research present in this thesis involves the creation of an application with a focus on implementing accessibility features, seeking throughout this process, the identification of similarities, differences and potential improvements in accessibility implementation of equivalent features between the frameworks. It includes a thorough literature review to establish the current state of mobile accessibility research, an in-depth analysis of React Native's approach to accessibility tools and guidelines, while having a comparison between code samples of both frameworks demonstrating accessibility features. \\

Accessibility remains one of the goals and one of the main foundations of a good user experience, but also a good developer mindset in order to consider and expand the usage of a product to different users and categories, regardless of their capabilities, while guaranteeing a seamless interaction with different components and devices. For this reason, WCAG guidelines will be implemented as the success criteria, serving as a comprehensive framework for making content across the web more accessible, while being equally relevant and applicable to mobile app development. As mobile devices become increasingly prevalent, implementing such guidelines will be important in understanding as technology evolves, how accessibility features will adapt, considering future trends and potential. 

\endgroup
\vfill
