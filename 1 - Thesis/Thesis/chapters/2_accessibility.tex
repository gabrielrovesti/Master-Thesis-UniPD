\chapter{Mobile accessibility guidelines and standards}
\label{chap:accessibility}

\section{State of research and literature review}
\label{chap:accessibility-overview}

Research in mobile accessibility spans multiple areas, from user interaction studies to framework-specific analyses. This section outlines the relevant work, organized by key research themes, that informs our approach to comparing frameworks. We further review various studies on how people, with and without disabilities, interact with mobile devices. Such studies typically report on accessibility barriers and present insights into the effectiveness of general guidelines on accessibility. Our literature review focuses a great deal on research related to challenges faced by users with disabilities and the implementation of accessibility features in mobile development frameworks. \\

In exploring accessibility solutions for mobile applications, a notable contribution comes from Zaina et al. \cite{zaina2022preventing}, who conducted extensive research into accessibility barriers that arise when using design patterns for building mobile user interfaces. The authors recognize that several user interface design patterns are present inside of libraries, but do not attach significant importance to accessibility features, which are already present in language. This study tried to adopt a \gls{grayliteraturereview} approach, gathering insights and capture real practitioners' experiences and challenges in implementing UI patterns, done by investigating professional forums or blogs. This approach proved valuable, since this was recognized as a source of practical knowledge and evidence a comprehensive catalog documenting 9 different user interface design patterns, along with descriptions of accessibility barriers present for each one and specific guidelines for prevention, for example inside of Input and Data components but also animated parts. The study's validation phase involved 60 participants, highlighting the fact participants saw value in the guidelines not just for implementing accessibility features, but also for improving their overall understanding of accessible design principles. These comprehensive results demonstrated both the practical applicability of the guidelines in real development scenarios and their effectiveness as an educational tool for raising awareness about accessibility concerns among developers.\\

Another significant contribution to report here was conducted by Vendome et al. \cite{vendome2019can} analyzed the implementation of accessibility features inside of Android applications both quantitatively and qualitatively, with the main goal of understanding accessibility practices among developers and identify common implementation patterns through a systematic approach, while mining the web to look for data. The methodology of the research contained two major parts: first, they did a mining-based analysis of 13,817 Android applications from GitHub that had at least one follower, star, or fork to avoid abandoned projects. They have done a static analysis on the usage of accessibility APIs and the presence of assistive content description in GUI components. A second component was a qualitative review of 366 Stack Overflow discussions related to accessibility, which were formally coded following an open-coding process with multi-author agreement. \\

The key results of the mining study were that while half of the apps supported assistive content descriptions for all GUI components, only 2.08\% used accessibility APIs. The Stack Overflow analysis revealed that support for visually impaired users dominated the discussions-43\% of the questions-and remarkably enough, 36\% of the accessibility API-related questions were about using these APIs for non-accessibility purposes.
The study identified several critical barriers to accessibility implementation: lack of developer knowledge about accessibility features, limited automated support and insufficient guidance for screen readers, while having a notable gap between accessibility guidelines and implementation practices. \\

Another paper reporting notable findings is the one from Pandey et al. \cite{pandey2022accessibility}, an analytical work of 96 mailing list threads combined with 18 interviews carried out with programmers with visual impairments. The authors investigate how frameworks shape programming experiences and collaboration with sighted developers. As expected, it concluded that accessibility problems are difficult to be reduced either to programming tool UI frameworks alone: they result from interactions between multiple software components including IDEs, browser developer tools, UI frameworks, operating systems, and screen readers, a topic of this thesis and research. Results showed that, although UI frameworks have the potential to enable relatively independent creation of user interfaces that reduce reliance on sighted assistance, many of those frameworks claimed themselves to be accessible out-of-the-box, but only partially lived up to this promise. Indeed, their results showed that various accessibility barriers in programming tools and UI frameworks complicate writing UI code, debugging, and testing, and even collaboration with sighted colleagues. \\

A research discussing and comparing the two frameworks addressing accessibility issues, which this thesis wants to base upon, is the research by Gaggi and Perinello \cite{perinello2024accessibility}, investigating three main questions: whether components are accessible by default, if non-accessible components can be made accessible, and the development cost in terms of additional code required. The study examines a set of UI elements against WCAG criteria and proposes solutions when official documentation is insufficient. This thesis wants to expand previous research conducted by Budai \cite{budai2024mobile} on Flutter's accessibility features, proposing mobile-specific guidelines when necessary, proposing connection and a deeper evaluation of both frameworks and proposing a more practical and developer-focused approach. The actual goal is to provide a practical resource that helps developers making informed decisions about accessibility implementation in their mobile apps, while analyzing and comparing in detail components between Flutter and React Native.

\newpage